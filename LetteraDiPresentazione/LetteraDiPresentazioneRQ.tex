% In questo file vengono definiti comandi (macro) utili in tutti i documenti

\newcommand{\gruppo}{\textit{Sirius}} % nome gruppo
\newcommand{\progetto}{\textit{Sequenziatore}} % nome progetto
\newcommand{\capitolato}{\href{http://www.math.unipd.it/~tullio/IS-1/2013/Progetto/C4p.pdf}{http://www.math.unipd.it/~tullio/IS-1/2013/Progetto/C4p.pdf}} 

% GLOSSARIO
\newcommand{\lastversionG}{1.0.0}
\newcommand{\Glossario}{\textit{Glossario\_{}v\lastversionG{}.pdf}}
\newcommand{\doctitleG}{Glossario}
\newcommand{\infoG}{\doctitleG\ v\lastversionG}

% NORME DI PROGETTO
\newcommand{\lastversionNDP}{1.0.0}
\newcommand{\NormeDiProgetto}{\textit{NormeDiProgetto\_{}v\lastversionNDP{}.pdf}}
\newcommand{\doctitleNDP}{Norme di progetto}
\newcommand{\infoNDP}{\doctitleNDP\ v\lastversionNDP}

% ANALISI DEI REQUISITI
\newcommand{\lastversionAR}{1.0.0}
\newcommand{\AnalisiDeiRequisiti}{\textit{AnalisiDeiRequisiti\_{}v\lastversionAR{}.pdf}}
\newcommand{\doctitleAR}{Analisi dei requisiti}
\newcommand{\infoAR}{\doctitleAR\ v\lastversionAR}

% Studio di fattibilita
\newcommand{\lastversionSDF}{1.0.0}
\newcommand{\StudioDiFattibilita}{\textit{StudioDiFattibilita\_{}v\lastversionSDF{}.pdf}}
\newcommand{\doctitleSDF}{Studio Di Fattibilità}
\newcommand{\infoSDF}{\doctitleSDF\ v\lastversionSDF}

% Piano di qualifica
\newcommand{\lastversionPDQ}{1.0.0}
\newcommand{\PianoDiQualifica}{\textit{PianoDiQualifica\_{}v\lastversionPDQ{}.pdf}}
\newcommand{\doctitlePDQ}{Piano Di Qualifica}
\newcommand{\infoPDQ}{\doctitlePDQ\ v\lastversionPDQ}

% VERBALE 2014-02-03
\newcommand{\lastversionVb}{1.0.0}
\newcommand{\VerbaleB}{\textit{Verbale2014-02-03\_{}v\lastversionVb{}.pdf}}
\newcommand{\doctitleVb}{Verbale 2014-02-03}
\newcommand{\infoVb}{\doctitleVb\ v\lastversionVb}

%Piano di progetto
\newcommand{\lastversionPDP}{1.0.0}
\newcommand{\PianoDiProgetto}{\textit{PianoDiProgetto\_{}v\lastversionPDP{}.pdf}}
\newcommand{\doctitlePDP}{Piano Di Progetto}
\newcommand{\infoPDP}{\doctitlePDP\ v\lastversionPDP}
%Specifica Tecnica
\newcommand{\lastversionST}{1.0.0}
\newcommand{\SpecificaTecnica}{\textit{SpecificaTecnica\_{}v\lastversionST{}.pdf}}
\newcommand{\doctitleST}{Specifica Tecnica}
\newcommand{\infoST}{\doctitleST\ v\lastversionST} %macro

% NB: lastversionXX, doctitle XX, infoXX, con XX sigla del documento, devono essere definiti nel file modello/macro.tex
\newcommand{\lastversion}{C}%versione del documento
\newcommand{\doctitle}{B}%nome documento
\newcommand{\info}{A}%nome e versione documento
\documentclass[11pt,a4paper]{article}
\usepackage[a4paper,portrait,top=3.5cm,bottom=3.5cm,left=3cm,right=3cm,bindingoffset=5mm]{geometry}


\usepackage[italian]{babel}
\usepackage{ucs} %unicode sistema gli accenti
\usepackage[utf8x]{inputenc} %unicode sistema gli accenti
\usepackage{fancyhdr}
%\usepackage{subfigure} % per figure affiancate
\usepackage{hyperref}
%\usepackage{float} % per far bene le figures
\usepackage{indentfirst}
\usepackage{color}
\usepackage{colortbl}
\usepackage{rotating}
\usepackage[table]{xcolor}
\usepackage{wrapfig}
\usepackage{array}
\usepackage{eurosym}
\usepackage{graphicx}
\usepackage{breakurl}
\usepackage{lastpage} % total page count
\usepackage{chngpage}
\usepackage{amsfonts}
\usepackage{listings}
\usepackage{enumitem}
\usepackage{booktabs}% funzionalità aggiuntive per tabelle
\usepackage{bookmark}% consente la creazione di segnalibri
\usepackage{caption}% funzionalità aggiuntive didascalie
\usepackage{tabularx}% funzionalità aggiuntive per tabelle
\usepackage{longtable}% funzionalità aggiuntive per tabelle
\usepackage{ltxtable}% funzionalità aggiuntive per tabelle

\pagestyle{fancy}
\newcommand{\groupname}{Sirius - Sequenziatore}

\newcommand{\subscript}[1]{\raisebox{-0.6ex}{\scriptsize #1}}


\fancypagestyle{plain}{%
	\chead{}
	\lfoot{\info}
	\cfoot{}
	\rfoot{\thepage\ / \pageref{LastPage}}
	\renewcommand{\headrulewidth}{0.3pt}
	\renewcommand{\footrulewidth}{0.3pt}
}
	\lhead{\setlength{\unitlength}{1mm}
        \begin{picture}(0,0)
                \put(5,0){\includegraphics[scale=0.1]{../modello/img/sirius.png}}
        \end{picture}}
	\rhead{\groupname}
	\chead{}
	%\lfoot{\info}
	\cfoot{}
%	\rfoot{\thepage}
	%\rfoot{\thepage\ / \pageref{LastPage}}
	\renewcommand{\headrulewidth}{0.3pt}
	\renewcommand{\footrulewidth}{0.3pt}
\linespread{1.2}	% valore interlinea


\fancypagestyle{romano}{
	\lhead{\setlength{\unitlength}{1mm}
        \begin{picture}(0,0)
                \put(5,0){\includegraphics[scale=0.03]{../modello/img/sirius.png}}
        \end{picture}}
	\chead{}
	\rhead{\groupname}
	\lfoot{\info}
	\cfoot{}
	%\rfoot{\thepage}
	\renewcommand{\headrulewidth}{0.3pt}
	\renewcommand{\footrulewidth}{0.3pt}
}



\hypersetup{
    colorlinks=true,linkcolor=[rgb]{0.11,0.55,0.83
    },          % colore link interni
    urlcolor=cyan           %colore link esterni
}
\definecolor{err}{rgb}{0.9,0.1,0.1}
\definecolor{rt}{rgb}{0.1,0.6,0.8}
\definecolor{grey}{rgb}{0.4,0.3,0.4}
\definecolor{mycolor}{rgb}{0.67,1,0.18}

\bibliographystyle{plain_ita}%bibliografia stile italiano

%\pagenumbering{Roman}
\setlength\parindent{0pt} % sempre senza indentatura
% fine layout

\begin{document}
%\pagenumbering{arabic}%numeri di pagina  arabic
\begin{flushright}
Alla cortese attenzione del Committente:\\
Prof. Vardanega Tullio\\
Prof. Cardin Riccardo\\
Università degli Studi di Padova\\
Dipartimento di Matematica\\
Via Trieste 63\\
35121, Padova\\
\bigskip
Padova, 28 Giugno 2014\\
\end{flushright}
\textit{Responsabile di Progetto} \gruppo\\
\medskip
\textbf{Oggetto}: Presentazione gruppo \gruppo.\\
Egregio Professore Vardanega Tullio,\\
Con la presente, il gruppo \gruppo ~intende comunicarLe ufficialmente l'intenzione di partecipare alla prossima Revisione di Qualifica per il prodotto da Lei commissionato, denominato:\\
\begin{center}
\textbf{\progetto} \textit{Software} di gestione di processi sequenziali con esecuzione da \textit{smartphone}\\
\end{center}
Viene di seguito riportata la tabella dei componenti del gruppo \gruppo{} e dei principali ruoli ricoperti durante il periodo di Qualifica.\\
\\
\begin{tabular}{l l l}
\hline
\textbf{Nominativo} & \textbf{Matricola} & \textbf{Ruolo}\\
\hline
Botter Marco & 561940 & Analista\\ %\includegraphics[scale=0.9]{Pics/luca.png}\\
\hline
Giachin Vanni & 1005519 & Analista\\ %\includegraphics[scale=0.9]{Pics/antonio.png}\\
\hline
Marcomin Gabriele & 1008916 & Analista\\ %\includegraphics[scale=0.9]{Pics/emanuele.png}\\
\hline
Quaglio Davide & 1026451 &  Responsabile\\ %\includegraphics[scale=0.9]{Pics/dq.png}\\
\hline
Santangelo Davide & 1004037 & Amministratore\\ %\includegraphics[scale=0.9]{Pics/jorge.png}\\
\hline
Seresin Davide & 611100 & Amministratore\\ %\includegraphics[scale=0.9]{Pics/fabio.png}\\
\hline
\\
\end{tabular}\\
La richiesta è corredata dai seguenti documenti:
\begin{itemize}
\item \infoAR ~(\AnalisiDeiRequisiti);
\item \infoPDP ~(\PianoDiProgetto);
\item \infoPDQ ~(\PianoDiQualifica);
\item \infoNDP ~(\NormeDiProgetto);
\item \infoDP ~(\DefinizioneDiProdotto);
\item \infoMPO ~(\ManualePO);
\item \infoMU ~(\ManualeUtente);
\item \infoG ~(\Glossario);
\item \infoST ~(\SpecificaTecnica). 
\end{itemize}

Rimango a Sua completa disposizione per ogni ulteriore chiarimento.
La ringrazio per l'attenzione e con l'occasione porgiamo\\
\begin{flushright}
Cordiali Saluti.\\
Quaglio Davide
\end{flushright}
\end{document}