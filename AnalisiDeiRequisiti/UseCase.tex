\section{Casi d'uso}
In questa sezione sono riportati i casi d'uso che necessari per identificare i requisiti funzionali.
Le convenzioni utilizzate per l'identificazione e la definizione dei casi d'uso sono riportate nel documento \NormeDiProgetto{}.

\subsection{Attori coinvolti}
Gli attori che possono interagire con il sistema sono i seguenti:\\

\begin{figure}[H]
\centering
\begin{minipage}[b]{0.45\textwidth}
\centering
\includegraphics[scale=0.61]%
{./grafici/UtenteGenerico}
\caption{Utente generico}
\end{minipage}
\begin{minipage}[b]{0.45\textwidth}
\centering
\includegraphics[scale=0.61]%
{./grafici/UtenteAutenticato}
\caption{Utente autenticato}
\end{minipage}
\end{figure}

\begin{itemize}
\item \textit{Utente:} utente generico che può registrarsi e autenticarsi al sistema diventando \textit{Utente autenticato} o \textit{Process owner\ped{G}};
\item \textit{Utente autenticato:} utente che può iscriversi, eseguire i processi disponibili ed effettuare il \textit{logout} diventando \textit{Utente}.
\end{itemize}

\begin{figure}[H]
\centering
\includegraphics[scale=0.60]%
{./grafici/Amministratore}
\caption{Process owner}
\end{figure}

\begin{itemize}
\item \textit{Process owner:} utente che può creare processi e monitorarne l'esecuzione.
\end{itemize}