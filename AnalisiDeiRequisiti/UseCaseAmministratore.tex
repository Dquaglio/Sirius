\subsection{Ambito amministratore}

\iffalse  % commento multilinea
\paragraph{UCA 1:}
\begin{itemize}
\item \textbf{Attori:} amministratore;
\item \textbf{Descrizione:} 
\item \textbf{Precondizione:} 
\item \textbf{Scenario principale:} 
\begin{itemize}
\item 
\end{itemize}
%\item \textbf{Estensioni:} 
%\item \textbf{Inclusioni:} 
%\item \textbf{Scenario alternativo:} 
\item \textbf{Postcondizione:}
\end{itemize}
\fi

\subsubsection{UCA 0: Caso base amministratore}
\begin{itemize}
\item \textbf{Attori:} amministratore;
\item \textbf{Descrizione:} l'amministratore può creare dei nuovi processi e gestire i processi creati;
\item \textbf{Precondizione:} il sistema è attivo e l'amministratore ha iniziato ad interfacciarcisi;
\item \textbf{Scenario principale:} 
\begin{itemize}
\item l'amministratore può creare un nuovo processo;
\item l'amministratore può gestire i processi creati.
\end{itemize}
%\item \textbf{Estensioni:} 
%\item \textbf{Inclusioni:} 
%\item \textbf{Scenario alternativo:} 
\item \textbf{Postcondizione:} il sistema ha eseguito e le operazioni effettuate dall'amministratore e ha salvato le modifiche sui processi.
\end{itemize}

\subsubsection{UCA 1: Creazione di un nuovo processo}
\begin{itemize}
\item \textbf{Attori:}
amministratore;
\item \textbf{Descrizione:} 
l'amministratore può creare un nuovo processo, gestirne i passi e definirne le condizioni di terminazione;
\item \textbf{Precondizione:} l'amministratore ha richiesto la creazione di un nuovo processo;
\item \textbf{Flusso principale degli eventi:} 
\begin{enumerate}
\item l'amministratore può definire un nuovo processo;
\item l'amministratore può definire i criteri di terminazione del nuovo processo;
\item l'amministratore può gestire i passi del nuovo processo;
\item l'amministratore può avviare il processo.
\end{enumerate}
\item \textbf{Postcondizione:}
il processo creato dall'utente è stato avviato e salvato dal sistema che ritorna allo stato iniziale.
\end{itemize}

\paragraph{UCA 1.1: Definizione di un nuovo processo}
\begin{itemize}
\item \textbf{Attori:} 
amministratore;
\item \textbf{Descrizione:} 
l'amministratore può creare un nuovo processo definendone il nome, che deve essere univoco, e la descrizione;
\item \textbf{Precondizione:} l'amministratore ha richiesto la creazione di un nuovo processo;
\item \textbf{Flusso principale degli eventi:} 
\begin{enumerate}
\item l'amministratore inserisce il nome del processo;
\item l'amministratore inserisce la descrizione del processo.
\end{enumerate}
\item \textbf{Scenario alternativo:}
\begin{itemize}
\item il nome inserito è già stato scelto per un altro processo, perciò l'amministratore viene avvisato e può cambiare i dati immessi;
\end{itemize}
\item \textbf{Postcondizione:} 
la definizione del processo è stata inserita e il sistema continua con la procedura di creazione del processo.
\end{itemize}

\paragraph{UCA 1.1.1: Inserimento del nome del processo}
\begin{itemize}
\item \textbf{Attori:} 
amministratore;
\item \textbf{Descrizione:} 
l'amministratore può inserire un nome per il nuovo processo in creazione; 
\item \textbf{Precondizione:}
l'amministratore ha richiesto la creazione di un nuovo processo e vuole inserire il nome scelto;
\item \textbf{Postcondizione:} 
il nome del processo è stato inserito e il sistema continua la procedura di creazione del nuovo processo.
\end{itemize}

\paragraph{UCA 1.1.2: Inserimento della descrizione del processo}
\begin{itemize}
\item \textbf{Attori:} 
amministratore;
\item \textbf{Descrizione:} 
l'amministratore può inserire una descrizione per il nuovo processo in creazione; 
\item \textbf{Precondizione:}
l'amministratore ha richiesto la creazione di un nuovo processo e vuole inserire la descrizione;
\item \textbf{Postcondizione:} 
la descrizione del processo è stata inserita e il sistema continua la procedura di creazione del nuovo processo.
\end{itemize}

\paragraph{UCA 1.2: Definizione dei criteri di terminazione del processo}
\begin{itemize}
\item \textbf{Attori:}
amministratore;
\item \textbf{Descrizione:}
l'amministratore può scegliere i criteri di terminazione del processo in definizione.
Questi criteri sono il completamento del processo da parte di un certo numero di utenti ed eventualmente una precisa data di scadenza;
\item \textbf{Precondizione:}
l'amministratore vuole definire i criteri di terminazione del processo in creazione;
\item \textbf{Scenario principale:}
\begin{itemize}
\item l'amministratore può scegliere il numero di completamenti che causano la terminazione del processo;
\item l'amministratore può scegliere la data di scadenza del processo.
\end{itemize}
\item \textbf{Postcondizione:}
i criteri di terminazione sono stati inseriti  e il sistema continua con la procedura di creazione del processo.
\end{itemize}

\paragraph{UCA 1.2.1: Scelta del massimo numero possibile di completamenti del processo}
\begin{itemize}
\item \textbf{Attori:}
amministratore;
\item \textbf{Descrizione:}
l'amministratore può scegliere il numero di completamenti del processo da parte degli utenti, necessario e sufficiente a causarne la terminazione;
\item \textbf{Precondizione:}
l'amministratore vuole definire il massimo numero possibile di completamenti del processo in creazione;
\item \textbf{Postcondizione:}
il massimo numero possibile di completamenti del processo in creazione è stato inserito e il sistema continua con la definizione dei criteri di terminazione.
\end{itemize}

\paragraph{UCA 1.2.2: Scelta della data di scadenza del processo}
\begin{itemize}
\item \textbf{Attori:}
amministratore;
\item \textbf{Descrizione:}
l'amministratore può scegliere la data di terminazione del processo;
\item \textbf{Precondizione:}
l'amministratore vuole definire la data di scadenza del processo in creazione;
\item \textbf{Flusso principale degli eventi:}
\begin{enumerate}
\item l'amministratore può scegliere il numero di completamenti che causano la terminazione del processo;
\item l'amministratore può scegliere la data di scadenza del processo.
\end{enumerate}
\item \textbf{Postcondizione:}
la data di scadenza del processo in creazione è stato inserita e il sistema continua con la definizione dei criteri di terminazione.
\end{itemize}

\paragraph{UCA 1.3: Gestione dei passi del processo}
\begin{itemize}
\item \textbf{Attori:} 
amministratore;
\item \textbf{Descrizione:} 
l'amministratore può creare un nuovo passo oppure gestire i passi creati.
\item \textbf{Precondizione:} 
l'amministratore ha definito un nuovo processo e vuole gestirne i passi;
\item \textbf{Scenario principale:} 
\begin{itemize}
\item l'amministratore può creare un nuovo passo;
\item l'amministratore può visualizzare la lista dei passi creati;
\item l'amministratore può modificare un passo creato;
\item l'amministratore può eliminare un passo creato.
\end{itemize}
\item \textbf{Postcondizione:}
il sistema ha eseguito e salvato le operazioni effettuate dall'amministratore sui passi del processo in creazione e il sistema è pronto per avviarlo.
\end{itemize}

\paragraph{UCA 1.3.1: Creazione di un passo}
\begin{itemize}
\item \textbf{Attori:} 
amministratore;
\item \textbf{Descrizione:} 
l'amministratore può aggiungere un nuovo passo al processo in creazione definendone i dati, i vincoli di superamento, e una descrizione.
\item \textbf{Precondizione:} 
l'amministratore vuole aggiungere un nuovo passo al processo in creazione;
\item \textbf{Scenario principale:} 
\begin{itemize}
\item l'amministratore inserisce la descrizione del passo;
\item l'amministratore inserisce uno o più dati al passo;
\item l'amministratore può inserire uno o più criteri di superamento del passo.
\end{itemize}
\item \textbf{Postcondizione:}
Il sistema ha aggiunto il passo definito dall'utente al processo in creazione.
\end{itemize}

\paragraph{UCA 1.3.1.1: Inserimento della descrizione del passo}
\begin{itemize}
\item \textbf{Attori:} 
amministratore;
\item \textbf{Descrizione:} 
l'amministratore può inserire la descrizione del passo in creazione, per definirne lo scopo;
\item \textbf{Precondizione:} 
l'amministratore vuole inserire la descrizione del passo in creazione;
\item \textbf{Postcondizione:}
la descrizione del passo è stata inserita e il sistema prosegue con la creazione del passo.
\end{itemize}

\paragraph{UCA 1.3.1.2: Inserimento dei dati del passo}
\begin{itemize}
\item \textbf{Attori:} 
amministratore;
\item \textbf{Descrizione:} 
l'amministratore può aggiungere uno o più campi dati al passo in creazione, scegliendone il nome e il tipo;
\item \textbf{Precondizione:} 
l'amministratore vuole aggiungere un campo dati al passo in creazione;
\item \textbf{Scenario principale:} 
\begin{itemize}
\item l'amministratore inserisce il nome del dato;
\item l'amministratore seleziona il tipo del dato.
\end{itemize}
\item \textbf{Postcondizione:}
il dato scelto è stato inserito e il sistema prosegue con la creazione del passo.
\end{itemize}

\paragraph{UCA 1.3.1.2.1: Inserimento del nome del dato}
\begin{itemize}
\item \textbf{Attori:} 
amministratore;
\item \textbf{Descrizione:} 
l'amministratore può inserire il nome del campo dati del passo in creazione, per chiarirne il significato;
\item \textbf{Precondizione:} 
l'amministratore vuole aggiungere il nome del campo dati aggiunto al passo in creazione;
\item \textbf{Postcondizione:} 
il nome del dato aggiunto è stato inserito e il sistema prosegue con la definizione del dato del passo in creazione.
\end{itemize}

\paragraph{UCA 1.3.1.2.2: Selezione del tipo del dato}
\begin{itemize}
\item \textbf{Attori:} 
amministratore;
\item \textbf{Descrizione:} 
l'amministratore può scegliere se il tipo di dato da aggiungere deve essere testuale, numerico o un'immagine; 
\item \textbf{Precondizione:} 
l'amministratore vuole aggiungere un campo dati al passo in creazione;
\item \textbf{Postcondizione:} 
il tipo del dato aggiunto è stato inserito e il sistema prosegue con la definizione del dato del passo in creazione.
\end{itemize}

\paragraph{UCA 1.3.1.3: Definizione di un criterio di superamento del passo}
\begin{itemize}
\item \textbf{Attori:} 
amministratore;
\item \textbf{Descrizione:}
l'amministratore può definire uno o più criteri di superamento del passo in creazione.
Per ciascun criterio può stabilire delle condizioni, e il passo eseguibile dopo il loro soddisfacimento.
I vincoli di ciascun criterio devono essere disgiunti per non creare ambiguità sul successivo passo da eseguire: eventuali errori vengono notificati dall'amministratore che potrà correggere i dati immessi; 
\item \textbf{Precondizione:} 
l'amministratore vuole aggiungere un criterio di superamento del passo in creazione;
\item \textbf{Flusso principale degli eventi:} 
\begin{enumerate}
\item l'amministratore definisce le condizioni di avanzamento;
\item l'amministratore definisce il passo eseguibile al soddisfacimento delle condizioni scelte;
\end{enumerate}
\item \textbf{Scenario alternativo:}
\begin{itemize}
\item le condizioni definite non sono disgiunte dalle condizioni precedentemente create, quindi l'utente viene avvisato dell'errore e può correggere i vincoli inseriti;
\end{itemize}
\item \textbf{Postcondizione:}
il criterio di superamento definito dall'amministratore è stato aggiunto e il sistema prosegue con la creazione del passo.
\end{itemize}

\paragraph{UCA 1.3.1.3.1: Inserimento delle condizioni di avanzamento}
\begin{itemize}
\item \textbf{Attori:} 
amministratore;
\item \textbf{Descrizione:} 
l'amministratore può decidere le condizioni che determinano il criterio di superamento in definizione.
Può inserire delle condizioni sull'approvazione del passo, sui dati numerici e sulla data, l'ora e la posizione geografica dell'utente al momento dell'invio dei dati richiesti.
\item \textbf{Precondizione:} 
l'amministratore sta definendo un nuovo criterio di superamento del passo in creazione;
\item \textbf{Scenario principale:} 
\begin{itemize}
\item l'amministratore può definire i vincoli sull'approvazione del passo;
\item l'amministratore può definire i vincoli sulle coordinate;
\item l'amministratore può definire i vincoli temporali;
\item l'amministratore può definire i vincoli sui dati numerici.
\end{itemize}
\item \textbf{Postcondizione:}
le condizioni definite dall'amministratore sono state aggiunte e il sistema prosegue con la definizione del criterio di superamento del passo.
\end{itemize}

\paragraph{UCA 1.3.1.3.1.1: Inserimento dei vincoli sull'approvazione del passo}
\begin{itemize}
\item \textbf{Attori:} 
amministratore;
\item \textbf{Descrizione:} 
l'amministratore può decidere se per soddisfare il criterio di superamento in definizione, i dati inviati dall'utente necessiteranno dell'approvazione dell'amministratore;
\item \textbf{Precondizione:} 
l'amministratore vuole definire i vincoli sull'approvazione del passo in creazione;
\item \textbf{Postcondizione:}
i vincoli sull'approvazione del passo sono stati inseriti e il sistema prosegue con la definizione del criterio di superamento del passo.
\end{itemize}

\paragraph{UCA 1.3.1.3.1.2: Inserimento dei vincoli sulle coordinate}
\begin{itemize}
\item \textbf{Attori:} 
amministratore;
\item \textbf{Descrizione:} 
l'amministratore può inserire i vincoli sulla posizione dell'utente al momento dell'invio dei dati, stabilendo le coordinate del luogo in cui dovrà trovarsi e un'eventuale raggio di tolleranza;
\item \textbf{Precondizione:} 
l'amministratore vuole definire i vincoli sulle coordinate dell'utente al momento dell'invio dei dati;
\item \textbf{Postcondizione:}
i vincoli sulle coordinate dell'utente sono stati inseriti e il sistema prosegue con la definizione del criterio di superamento del passo.
\end{itemize}

\paragraph{UCA 1.3.1.3.1.3: Inserimento dei vincoli temporali}
\begin{itemize}
\item \textbf{Attori:} 
amministratore;
\item \textbf{Descrizione:} 
l'amministratore può inserire uno o più intervalli temporali in cui l'utente può inviare i dati;
\item \textbf{Precondizione:} 
l'amministratore vuole definire i vincoli temporali sull'invio dei dati;
\item \textbf{Postcondizione:}
i vincoli temporali sull'invio dei dati sono stati inseriti e il sistema prosegue con la definizione del criterio di superamento del passo.
\end{itemize}

\paragraph{UCA 1.3.1.3.1.4: Inserimento dei vincoli sui dati numerici}
\begin{itemize}
\item \textbf{Attori:} 
amministratore;
\item \textbf{Descrizione:} 
l'amministratore può inserire i vincoli sui dati numerici che possono essere il numero di cifre, la possibilità di inserire cifre decimali o meno, e un eventuale limite superiore o inferiore;
\item \textbf{Precondizione:} 
esiste almeno un dato numerico nel passo in creazione e l'amministratore vuole definirne i vincoli;
\item \textbf{Postcondizione:}
i vincoli sui dati numerici del passo sono stati inseriti e il sistema prosegue con la definizione del criterio di superamento del passo.
\end{itemize}

\paragraph{UCA 1.3.1.3.2: Definizione passo successivo}
\begin{itemize}
\item \textbf{Attori:} 
amministratore;
\item \textbf{Descrizione:}
l'amministratore può definire il passo eseguibile al soddisfacimento dei vincoli scelti. Tra tutti i passi creati può scegliere solo quelli da cui è impossibile ritornare al passo in creazione, oppure la fine del processo;
\item \textbf{Precondizione:} 
l'amministratore sta definendo un nuovo criterio di avanzamento e vuole stabilire il passo successivo al passo in creazione;
\item \textbf{Postcondizione:}
il passo raggiungibile soddisfacendo i criteri di avanzamento in definizione è stato aggiunto.
\end{itemize}

\paragraph{UCA 1.3.2: Visualizzazione della lista dei passi creati}
\begin{itemize}
\item \textbf{Attori:} 
amministratore;
\item \textbf{Descrizione:}
l'amministratore può visualizzare la lista, eventualmente vuota, dei passi creati;
\item \textbf{Precondizione:} 
l'amministratore sta gestendo i passi del processo in creazione;
\item \textbf{Postcondizione:}
il sistema ha visualizzato la lista dei passi creati.
\end{itemize}

\paragraph{UCA 1.3.3: Modifica di un passo}
\begin{itemize}
\item \textbf{Attori:} 
amministratore;
\item \textbf{Descrizione:} 
l'amministratore può modificare un passo creato. In particolare può modificarne la descrizione, il nome dei dati e i criteri di superamento; 
\item \textbf{Precondizione:}
l'amministratore vuole modificare un passo dalla lista dei passi creati;
\item \textbf{Scenario principale:} 
\begin{itemize}
\item l'amministratore può modificare la descrizione del passo;
\item l'amministratore può modificare la descrizione dei dati del passo;
\item l'amministratore può modificare i criteri di superamento del passo.
\end{itemize}
\item \textbf{Postcondizione:}
le modifiche effettuate dall'amministratore sul passo creato sono state eseguite e il sistema ritorna alla gestione dei passi del processo. 
\end{itemize}

\paragraph{UCA 1.3.3.1: Modifica della descrizione di un passo}

\begin{itemize}
\item \textbf{Attori:} 
amministratore;
\item \textbf{Descrizione:}
l'amministratore può inserire una nuova la descrizione del passo gestito;
\item \textbf{Precondizione:} 
l'amministratore vuole modificare la descrizione di un passo creato;
\item \textbf{Postcondizione:}
la nuova descrizione è stata inserita e il sistema prosegue con la modifica del passo gestito.
\end{itemize}

\paragraph{UCA 1.3.3.2: Modifica della descrizione dei dati di un passo}
\begin{itemize}
\item \textbf{Attori:} 
amministratore;
\item \textbf{Descrizione:}
l'amministratore può inserire una nuova descrizione ai dati del passo gestito;
\item \textbf{Precondizione:} 
l'amministratore vuole modificare la descrizione di un passo creato;
\item \textbf{Postcondizione:}
la nuova descrizione è stata inserita e il sistema prosegue con la modifica del passo gestito.
\end{itemize}

\paragraph{UCA 1.3.3.3: Modifica dei criteri di superamento di un passo}
\begin{itemize}
\item \textbf{Attori:} 
amministratore;
\item \textbf{Descrizione:} 
l'amministratore può modificare uno o più criteri di superamento del passo gestito.
Per ciascun criterio può modificare le condizioni di avanzamento, e il passo eseguibile dopo il loro soddisfacimento.
I vincoli di ciascun criterio, dopo la modifica, devono essere disgiunti per non creare ambiguità sul successivo passo da eseguire: eventuali
errori vengono notificati dall'amministratore che potrà correggere i dati immessi;
\item \textbf{Precondizione:}
l'amministratore vuole modificare un criterio di superamento del passo gestito;
\item \textbf{Scenario principale:} 
\begin{itemize}
\item l'amministratore può modificare le condizioni di avanzamento del passo;
\item l'amministratore può modificare il passo eseguibile al soddisfacimento delle condizioni scelte.
\end{itemize}
\item \textbf{Scenario alternativo:}
\begin{itemize}
\item le condizioni definite non sono disgiunte dalle altre condizioni presenti, quindi l'utente viene avvisato dell'errore e può correggere i vincoli modificati;
\end{itemize}
\item \textbf{Postcondizione:}
le modifiche effettuate dall'amministratore sui criteri di superamento del passo gestito sono state eseguite e il sistema prosegue con la modifica del passo. 
\end{itemize}

\paragraph{UCA 1.3.3.3.1: Modifica delle condizioni di avanzamento}
\begin{itemize}
\item \textbf{Attori:} 
amministratore;
\item \textbf{Descrizione:} 
l'amministratore può modificare le condizioni che determinano il criterio di superamento in definizione.
Può modificare le condizioni sull'approvazione del passo, sui dati numerici e sulla data, l'ora e la posizione geografica dell'utente al momento dell'invio dei dati richiesti.
\item \textbf{Precondizione:}
l'amministratore vuole modificare le condizioni di avanzamento del passo gestito;
\item \textbf{Scenario principale:} 
\begin{itemize}
\item l'amministratore può modificare i vincoli sull'approvazione del passo;
\item l'amministratore può modificare i vincoli sulle coordinate;
\item l'amministratore può modificare i vincoli temporali;
\item l'amministratore può modificare i vincoli sui dati numerici.
\end{itemize}
\item \textbf{Postcondizione:}
le modifiche effettuate dall'amministratore sulle condizioni di avanzamento del passo gestito sono state eseguite e il sistema prosegue con la modifica dei criteri di superamento. 
\end{itemize}

\paragraph{UCA 1.3.3.3.1.1: Modifica dei vincoli sull'approvazione del passo}
\begin{itemize}
\item \textbf{Attori:} 
amministratore;
\item \textbf{Descrizione:} 
l'amministratore può modificare i vincoli sull'approvazione del passo;
\item \textbf{Precondizione:} 
l'amministratore vuole modificare i vincoli sull'approvazione del passo gestito;
\item \textbf{Postcondizione:}
i vincoli sull'approvazione del passo sono stati modificati e il sistema prosegue con la modifica del criterio di superamento del passo.
\end{itemize}

\paragraph{UCA 1.3.3.3.1.2: Modifica dei vincoli sulle coordinate}
\begin{itemize}
\item \textbf{Attori:} 
amministratore;
\item \textbf{Descrizione:} 
l'amministratore può modificare i vincoli sulla posizione dell'utente al momento dell'invio dei dati, stabilendo le coordinate del luogo in cui dovrà trovarsi e un'eventuale raggio di tolleranza;
\item \textbf{Precondizione:} 
l'amministratore vuole modificare i vincoli sulle coordinate dell'utente al momento dell'invio dei dati;
\item \textbf{Postcondizione:}
i vincoli sulle coordinate dell'utente sono stati modificati e il sistema prosegue con la modifica del criterio di superamento del passo.
\end{itemize}

\paragraph{UCA 1.3.3.3.1.3: Modifica dei vincoli temporali}
\begin{itemize}
\item \textbf{Attori:} 
amministratore;
\item \textbf{Descrizione:} 
l'amministratore può modificare gli intervalli temporali in cui l'utente può inviare i dati;
\item \textbf{Precondizione:} 
l'amministratore vuole modificare i vincoli temporali sull'invio dei dati;
\item \textbf{Postcondizione:}
i vincoli temporali sull'invio dei dati sono stati modificati e il sistema prosegue con la modifica del criterio di superamento del passo.
\end{itemize}

\paragraph{UCA 1.3.3.3.1.4: Modifica dei vincoli sui dati numerici}
\begin{itemize}
\item \textbf{Attori:} 
amministratore;
\item \textbf{Descrizione:} 
l'amministratore può modificare i vincoli sui dati numerici del passo gestito;
\item \textbf{Precondizione:} 
esiste almeno un dato numerico nel passo gestito e l'amministratore vuole modificarne i vincoli;
\item \textbf{Postcondizione:}
i vincoli sui dati numerici del passo sono stati modificati e il sistema prosegue con la modifica del criterio di superamento del passo.
\end{itemize}

\paragraph{UCA 1.3.3.3.2: Modifica del passo successivo}
\begin{itemize}
\item \textbf{Attori:} 
amministratore;
\item \textbf{Descrizione:}
l'amministratore può sostituire il passo eseguibile al soddisfacimento dei vincoli scelti con uno nuovo. Tra tutti i passi creati può scegliere solo quelli da cui è impossibile ritornare al passo in creazione, oppure la fine del processo;
\item \textbf{Precondizione:} 
l'amministratore sta modificando un criterio di avanzamento e vuole scegliere il passo successivo al passo in creazione;
\item \textbf{Postcondizione:}
il passo raggiungibile soddisfacendo i criteri di avanzamento in definizione è stato sostituito con un altro scelto dall'amministratore.
\end{itemize}

\paragraph{UCA 1.3.4: Eliminazione di un passo}
\begin{itemize}
\item \textbf{Attori:} 
amministratore;
\item \textbf{Descrizione:} 
l'amministratore può eliminare un passo del processo in creazione; 
\item \textbf{Precondizione:}
l'amministratore vuole eliminare un passo tra quelli nella lista dei passi creati;
\item \textbf{Postcondizione:}
il passo scelto dall'amministratore è stato eliminato e il sistema si porta nello stato di gestione dei passi del processo in creazione, pronto ad eseguire altre operazioni sui passi. 
\end{itemize}

\paragraph{UCA 1.4: Avvio del processo}
\begin{itemize}
\item \textbf{Attori:} 
amministratore;
\item \textbf{Descrizione:} 
l'amministratore può avviare il passo creato aggiungendolo ai processi in gestione; 
\item \textbf{Precondizione:} 
l'amministratore ha definito un processo con almeno un passo;
\item \textbf{Postcondizione:} 
il processo creato dall'utente è stato avviato, il sistema lo ha aggiunto ai passi creati e ritorna allo stato iniziale.
\end{itemize}

\subsubsection{2: Gestione processi creati}
\begin{itemize}
\item \textbf{Attori:} 
amministratore;
\item \textbf{Descrizione:} 
l'amministratore dall'interfaccia del programma server può gestire i processi precedentemente creati e in corso, invitando utenti a parteciparvi, monitorando i risultati e qualora fosse necessario permette ad alcuni utenti di completare dei passi che richiedano il suo intervento (come ad esempio valutare una fotografia);
\item \textbf{Precondizione:} 
Interfaccia programma server aperta dall'amministratore;
\item \textbf{Scenario principale:} 
\begin{itemize}
\item l'amministratore può recuperare informazioni su un processo (UCA 1.1);
\item l'amministratore può selezionare quali utenti possono partecipare ad un processo (UCA 1.2);
\item l'amministratore può permettere l'avanzamento di passi che richiedono il suo diretto intervento (UCA 1.3).
\end{itemize}
\item \textbf{Postcondizione:} 
L'Amministratore ha concluso le operazioni che desiderava fare sui processi esistenti. Il sistema si trova nello stato iniziale dell'interfaccia appena aperta, pronto ad eseguire nuove operazioni.
\end{itemize}

\paragraph{UCA 1.1: Recupera informazioni su un processo}
\begin{itemize}
\item \textbf{Attori:} amministratore;
\item \textbf{Descrizione:} 
l'amministratore può scegliere un processo precedentemente creato e recuperare informazioni sul suo avanzamento e eventuali dati raccolti; 
\item \textbf{Precondizione:} 
L'interfaccia di amministrazione é pronta e si vuole recuperare informazioni su un processo precedentemente creato;
\item \textbf{Flusso principale degli eventi:} 
\begin{enumerate}
\item Consultazione delle informazioni sul processo (UCA 1.1.1).
\end{enumerate}
\item \textbf{Scenari alternativi:}
\begin{enumerate}
\item Non sono presenti processi precedentemente creati, il sistema rimane nello stato precedente.
\end{enumerate}
\item \textbf{Inclusioni:}
\begin{enumerate}
\item Selezione di un processo esistente (UCA 1.4);
\item Conferma selezione del processo (UCA 1.5).
\end{enumerate}
\item \textbf{Postcondizione:} 
l'amministratore ha recuperato e potuto consultare le informazioni su un processo precedentemente creato. Il sistema si trova nello stato iniziale dell'interfaccia appena aperta, pronto ad eseguire nuove operazioni.
\end{itemize}

\paragraph{UCA 1.1.1: Consultazione delle informazioni sul processo}
\begin{itemize}
\item \textbf{Attori:} 
amministratore;
\item \textbf{Descrizione:} 
l'amministratore può consultare le informazioni sul processo precedentemente selezionato recuperate dal sistema; 
\item \textbf{Precondizione:} 
l'amministratore ha selezionato e confermato un processo da cui recuperare le informazioni;
\item \textbf{Postcondizione:} 
l'amministratore ha consultato le informazioni desiderate.
\end{itemize}

\paragraph{UCA 1.2: Selezione di quali utenti possono partecipare ad un processo}
\begin{itemize}
\item \textbf{Attori:} 
amministratore;
\item \textbf{Descrizione:} 
l'amministratore può scegliere un processo precedentemente creato e ad esso permettere la partecipazione di alcuni utenti registrati;
\item \textbf{Precondizione:} 
L'interfaccia di amministrazione é pronta e si vuole permettere la partecipazione di alcuni utenti registrati ad un processo precedentemente creato;
\item \textbf{Flusso principale degli eventi:} 
\begin{enumerate}
\item Selezione degli utenti che possono partecipare al processo (UCA 1.2.1);
\item Conferma selezione degli utenti (UCA 1.2.2).
\end{enumerate}
\item \textbf{Scenari alternativi:}
\begin{enumerate}
\item Non sono presenti processi precedentemente creati, il sistema rimane nello stato precedente;
\item Non sono presenti utenti registrati, il sistema rimane nello stato precedente.
\end{enumerate}
\item \textbf{Inclusioni:}
\begin{enumerate}
\item Selezione di un processo esistente (UCA 1.4);
\item Conferma selezione del processo (UCA 1.5).
\end{enumerate}
\item \textbf{Postcondizione:} 
l'amministratore ha permesso a determinati utenti registrati la partecipazione ad un processo precedentemente creato. Il sistema si trova nello stato iniziale dell'interfaccia appena aperta, pronto ad eseguire nuove operazioni.
\end{itemize}

\paragraph{UCA 1.2.1: Selezione degli utenti che possono partecipare al processo}
\begin{itemize}
\item \textbf{Attori:} 
amministratore;
\item \textbf{Descrizione:} 
l'amministratore può scegliere degli utenti da una lista di utenti registrati; 
\item \textbf{Precondizione:} 
Il sistema presenta una lista di utenti registrarti e resta in attesa che l'amministratore selezioni o deselezioni alcuni di questi per permettergli o meno la partecipazione al processo precedentemente scelto;
\item \textbf{Postcondizione:} 
l'amministratore ha selezionato gli utenti registrati a cui permettere o meno la partecipazione al processo precedentemente selezionato e il sistema resta in attesa della conferma.
\end{itemize}

\paragraph{UCA 1.2.2: Conferma selezione degli utenti}
\begin{itemize}
\item \textbf{Attori:} 
amministratore;
\item \textbf{Descrizione:} 
l'amministratore può confermare la selezione degli utenti registrati a cui permettere o meno la partecipazione al processo precedentemente selezionato; 
\item \textbf{Precondizione:} 
l'amministratore ha selezionato un processo a cui permettere o meno la partecipazione ad alcuni utenti registrati e ha selezionato gli utenti registrati a cui permettere o meno la partecipazione a tale processo;
\item \textbf{Postcondizione:} 
l'amministratore ha confermato la selezione degli utenti registrati a cui permettere o meno la partecipazione al processo precedentemente selezionato.
\end{itemize}

\paragraph{UCA 1.3: Permetti completamento passi che richiedono intervento umano}
\begin{itemize}
\item \textbf{Attori:} 
amministratore;
\item \textbf{Descrizione:} 
l'amministratore può permettere di completare eventuali passi (di un processo in attesa) che richiedono il suo diretto controllo sui dati comunicati dall'utente che sta aspettando che vengano verificati;
\item \textbf{Precondizione:} 
L'interfaccia di amministrazione é pronta e si vuole permettere di completare, o richiederne la ripetizione, di eventuali passi che richiedono il diretto controllo dell'amministratore sui dati comunicati dall'utente che sta aspettando che vengano verificati. Il processo contente il passo da verificare è detto "in attesa";
\item \textbf{Flusso principale degli eventi:} 
\begin{enumerate}
\item Consultazione delle informazioni inserite (UCA 1.3.1);
\item Gestione del processo in attesa (UCA 1.3.2).
\end{enumerate}
\item \textbf{Scenari alternativi:}
\begin{enumerate}
\item Non sono presenti processi precedentemente in attesa, il sistema rimane nello stato precedente.
\end{enumerate}
\item \textbf{Inclusioni:}
\begin{enumerate}
\item Selezione di un processo esistente (UCA 1.4);
\item Conferma selezione del processo (UCA 1.5).
\end{enumerate}
\item \textbf{Postcondizione:} 
l'amministratore ha controllato le informazioni inserite dall'utente relative al passo del processo in attesa e ha voluto permetterne il completamento o richiesto che il passo venga ripetuto. Il sistema si trova nello stato iniziale dell'interfaccia appena aperta, pronto ad eseguire nuove operazioni.
\end{itemize}

\paragraph{UCA 1.3.1: Consultazione delle informazioni inserite}
\begin{itemize}
\item \textbf{Attori:} 
amministratore;
\item \textbf{Descrizione:} 
l'amministratore può consultare le informazioni inserite dall'utente relative al passo i cui dati devono essere verificati manualmente del processo in attesa precedentemente scelto; 
\item \textbf{Precondizione:} 
l'amministratore ha confermato la selezione del processo di cui si vuole permettere l'avanzamento o richiedere la ripetizione del passo i cui dati devono essere verificati manualmente;
\item \textbf{Postcondizione:} 
l'amministratore ha consultato le le informazioni inserite dall'utente relative al passo i cui dati devono essere verificati manualmente del processo in attesa precedentemente scelto.
\end{itemize}

\paragraph{UCA 1.3.2: Gestione del processo in attesa}
\begin{itemize}
\item \textbf{Attori:} 
amministratore;
\item \textbf{Descrizione:} 
l'amministratore può decidere, in base alle informazioni consultate, se permettere il superamento o richiedere la ripetizione del passo i cui dati sono stati verificati manualmente del processo in attesa precedentemente scelto;
\item \textbf{Precondizione:} 
l'amministratore ha consultato le le informazioni inserite dall'utente relative al passo i cui dati devono essere verificati manualmente del processo in attesa precedentemente scelto e può scegliere se può essere superato o deve essere ripetuto;
\item \textbf{Scenario principale:} 
\begin{itemize}
\item Avanzamento concesso (UCA 1.3.4.1);
\item Richiesta ripetizione (UCA 1.3.4.2).
\end{itemize}
\item \textbf{Postcondizione:} 
l'amministratore ha deciso, in base alle informazioni consultate, se permettere il superamento o richiedere la ripetizione del passo i cui dati sono stati verificati manualmente del processo in attesa precedentemente scelto.
\end{itemize}

\paragraph{UCA 1.3.4.1: Avanzamento concesso}
\begin{itemize}
\item \textbf{Attori:} 
amministratore;
\item \textbf{Descrizione:} 
l'amministratore ha scelto , in base alle informazioni consultate, di permettere il superamento del passo i cui dati sono stati verificati manualmente del processo in attesa precedentemente scelto;
\item \textbf{Precondizione:} 
l'amministratore ha consultato le le informazioni inserite dall'utente relative al passo i cui dati devono essere verificati manualmente del processo in attesa precedentemente scelto e ha scelto che può essere superato;
\item \textbf{Postcondizione:} 
Il sistema aggiorna i propri dati considerando il passo superato.
\end{itemize}

\paragraph{UCA 1.3.4.2: Richiesta ripetizione}
\begin{itemize}
\item \textbf{Attori:} 
amministratore;
\item \textbf{Descrizione:} 
l'amministratore ha scelto , in base alle informazioni consultate, di richiedere la ripetizione del passo i cui dati sono stati verificati manualmente del processo in attesa precedentemente scelto;
\item \textbf{Precondizione:} 
l'amministratore ha consultato le le informazioni inserite dall'utente relative al passo i cui dati devono essere verificati manualmente del processo in attesa precedentemente scelto e ha scelto che deve essere ripetuto;
\item \textbf{Postcondizione:} 
Il sistema aggiorna i propri dati richiedendo la ripetizione del passo.
\end{itemize}

\paragraph{UCA 1.4: Selezione di un processo esistente}
\begin{itemize}
\item \textbf{Attori:} 
amministratore;
\item \textbf{Descrizione:} 
l'amministratore può scegliere un processo precedentemente creato da una lista di processi; 
\item \textbf{Precondizione:} 
Il sistema presenta una lista di processi e resta in attesa che l'amministratore selezioni un processo precedentemente creato;
\item \textbf{Postcondizione:} 
l'amministratore ha selezionato un processo precedentemente creato.
\end{itemize}

\paragraph{UCA 1.5: Conferma selezione del processo}
\begin{itemize}
\item \textbf{Attori:} 
amministratore;
\item \textbf{Descrizione:} 
l'amministratore può confermare la selezione del processo precedentemente creato; 
\item \textbf{Precondizione:} 
l'amministratore ha selezionato un processo da precedentemente creato;
\item \textbf{Postcondizione:} 
l'amministratore ha confermato la selezione del processo precedentemente creato.
\end{itemize}