\section{Introduzione}
\subsection{Scopo del Documento}
Lo scopo di questo di questo documento è la presentazione delle funzionalità offerte dal prodotto \progetto{}.\\
In questo documento verranno quindi elencati e descritti dettagliatamente, l'insieme dei requisiti dedotti dall'analisi del capitolato d'appalto e all'incontro con il proponente.
\subsection{Scopo del Prodotto}
Lo scopo del progetto \progetto{}, è di fornire un servizio di gestione di processi definiti da una serie di passi da eseguirsi in sequenza o senza un ordine predefinito, utilizzabile da dispositivi mobili di tipo smaptphone o tablet.
\subsection{Glossario}
Al fine di rendere più leggibile e comprensibile i documenti, i termini tecnici, di dominio, gli acronimi e le parole che necessitano di essere chiarite, sono riportate nel documento \Glossario{}.\\
Ogni occorrenza di vocaboli presenti nel \textit{Glossario} deve essere seguita da una ``G'' maiuscola in pedice.
\subsection{Riferimenti}
\subsubsection{Normativi}
\begin{itemize}
\item \textbf{Capitolato d'appalto C4: }\href{http://www.math.unipd.it/~tullio/IS-1/2013/Progetto/C4.pdf}{Seq: Gestore di processi sequenziali con esecuzione da smartphone}
\item \textbf{Verbali esterni: }
\item \textbf{Norme di Progetto: }\NormeDiProgetto
\end{itemize}
\subsubsection{Informativi}
\begin{itemize}
\item \textbf{Diapositive del corso di Ingegneria del Software }\href{http://www.math.unipd.it/~tullio/IS-1/2013/}{mod A}
\item \textbf{UML}\ped{G}
\end{itemize}