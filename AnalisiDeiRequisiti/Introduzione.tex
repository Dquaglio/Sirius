\section{Introduzione}
\subsection{Scopo del Documento}
Lo scopo di questo di questo documento è la presentazione delle funzionalità offerte dal prodotto Sequenziatore. In questo documento verranno, quindi, elencati e descritti dettagliatamente l'insieme dei requisiti dedotti successivamente all'analisi del capitolato C4 e all'incontro con il proponente.
\subsection{Scopo del Prodotto}
Lo scopo del prodotto è realizzare una applicazione per la gestione di passi, che possono essere sequenziali o casuali, fruibile direttamente dal browser e da tipi diversi di terminale; come smartphone, tablet e personal pc.
\subsection{Glossario}
Al fine di rendere più leggibile e comprensibile i documenti, i termini tecnici, di dominio, gli acronimi e le parole che necessitano di
essere chiarite, sono riportate nel documento Glossario.
Ogni occorrenza di vocaboli presenti nel Glossario è marcata da una “G” maiuscola in
pedice.
\subsection{Riferimenti}
\subsubsection{Normativi}
\begin{itemize}
\item \textbf{Capitolato d'appalto C4: }\href{http://www.math.unipd.it/~tullio/IS-1/2013/Progetto/C4.pdf}{Seq: Gestore di processi sequenziali con esecuzione da smartphone}
\item \textbf{Verbali esterni: }
\item \textbf{Norme di Progetto: }\NormeDiProgetto
\end{itemize}
\subsubsection{Informativi}
\begin{itemize}
\item \textbf{Diapositive del corso di Ingegneria del Software }\href{http://www.math.unipd.it/~tullio/IS-1/2013/}{mod A}
\item \textbf{UML}\ped{G}
\end{itemize}