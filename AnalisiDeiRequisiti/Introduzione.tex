\section{Introduzione}
\subsection{Scopo del Documento}
Lo scopo di questo di questo documento è la presentazione delle funzionalità offerte dal prodotto \progetto{}.\\
In questo documento verranno quindi elencati e descritti dettagliatamente, l'insieme dei requisiti dedotti dall'analisi del capitolato d'appalto e all'incontro con il proponente.
\subsection{Scopo del Prodotto}
Lo scopo del progetto \progetto{}, è di fornire un servizio di gestione di processi definiti da una serie di passi da eseguirsi in sequenza o senza un ordine predefinito, utilizzabile da dispositivi mobili di tipo \textit{smaptphone} o \textit{tablet}.
\subsection{Glossario}
Al fine di rendere più leggibile e comprensibile i documenti, i termini tecnici, di dominio, gli acronimi e le parole che necessitano di essere chiarite, sono riportate nel documento \Glossario{}.\\
Ciascuna occorrenza dei vocaboli presenti nel \textit{Glossario} è seguita da una ``G'' maiuscola in pedice.
\subsection{Riferimenti}
\subsubsection{Normativi}
\begin{itemize}
\item Capitolato d'appalto C4, Seq: Gestore di processi sequenziali con esecuzione da smartphone:
\url{http://www.math.unipd.it/~tullio/IS-1/2013/Progetto/C4.pdf}
\item Norme di Progetto: \NormeDiProgetto{}.
\end{itemize}
\subsubsection{Informativi}
\begin{itemize}
\item Regolamento dei documenti, prof. Vardanega Tullio: \url{http://www.math.unipd.it/~tullio/IS-1/2013/}
\item Dispense di ingegneria del software modulo A, Ingegneria dei requisiti, prof. Vardanega Tullio:
\url{http://www.math.unipd.it/~tullio/IS-1/2013/Dispense/L06.pdf}
\item Dispense di ingegneria del software modulo A, Diagrammi dei casi d'uso, prof. Cardin Riccardo:
\url{http://www.math.unipd.it/~tullio/IS-1/2013/Dispense/E01c.pdf}
\item Dispense di ingegneria del software modulo B, Esercizi sugli errori rilevati in RR, prof. Cardin Riccardo:
\url{http://www.math.unipd.it/~rcardin/pdf/Esercitazione%20-%20Errori%20comuni%20RR_4x4.pdf}
\end{itemize}