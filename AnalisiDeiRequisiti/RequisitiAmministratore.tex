\begin{longtable}{lXp{0.14\textwidth}}
\toprule
\textbf{Requisito} & \textbf{Descrizione} & \textbf{Fonte}\\
\toprule
FOBA 1&Il sistema dovrà permettere all'amministratore la creazione di processi&Capitolato\\
\midrule
FOBA 1.1&L'amministratore dovrà inserire un nome che identifichi univocamente il processo che vuole creare&Interna\\
\midrule
FOBA 1.2&L'amministratore dovrà inserire la descrizione del processo che vuole creare &Interna\\
\midrule
FOBA 1.3&Il sistema dovrà permettere all'amministratore di definire i criteri di terminazione di un processo durante la sua creazione&Capitolato\\
\midrule
FOBA 1.3.1&L'amministratore dovrà inserire il numero massimo di completamenti del processo in creazione &Interna\\
\midrule
FOBA 1.3.2&L'amministratore potrà inserire la data di terminazione del processo in creazione&Interna\\
\midrule
FOBA 1.4&Il sistema dovrà permettere all'amministratore di gestire i passi del processo in creazione&Capitolato\\
\midrule
FOBA 1.4.1&Il sistema dovrà permettere all'amministratore di creare un passo del processo in creazione&Capitolato\\
\midrule
FOBA 1.4.1.1&L'amministratore dovrà inserire la descrizione del passo in creazione&Interna\\
\midrule
FOBA 1.4.1.2&Il sistema dovrà permettere all'amministratore di inserire uno o più dati al passo in creazione&Capitolato\\
\midrule
FOBA 1.4.1.2.1&L'amministratore potrà inserire un nome al dato che vuole aggiungere al passo in creazione&Interna\\
\midrule
FOBA 1.4.1.2.2&L'amministratore dovrà scegliere il tipo del dato che vuole aggiungere al passo in creazione&Interna\\
\midrule
FOBA 1.4.1.2.2.1&L'amministratore potrà scegliere un dato testuale come tipo del dato aggiunto al passo in creazione&Capitolato\\
\midrule
FOBA 1.4.1.2.2.2&L'amministratore potrà scegliere un dato numerico come tipo del dato aggiunto al passo in creazione&Capitolato\\
\midrule
FOBA 1.4.1.2.2.3&L'amministratore potrà scegliere un'immagine come tipo del dato aggiunto al passo in creazione&Capitolato\\
\midrule
FOBA 1.4.1.3&Il sistema dovrà permettere all'amministratore di definire uno o più criteri di superamento del passo in creazione&Interna\\
\midrule
FOBA 1.4.1.3.1&Per ogni criterio di superamento, l'amministratore dovrà definire una o più condizioni di avanzamento&Interna\\
\midrule
FDEA 1.4.1.3.1.1&Per ogni criterio di superamento, l'amministratore potrà scegliere se i dati ricevuti dall'utente richiederanno il suo controllo per concludere il passo in creazione&Verbale 2014-02-03\\
\midrule
FOBA 1.4.1.3.1.2&Per ogni criterio di superamento, l'amministratore potrà inserire un vincolo sulla posizione geografica dell'utente al momento dell'invio dei dati&Capitolato\\
\midrule
FOBA 1.4.1.3.1.2.1&Il sistema dovrà permettere all'amministratore di stabilire una precisa posizione geografica&Capitolato\\
\midrule
FOPA 1.4.1.3.1.2.2&L'amministratore potrà inserire un raggio di tolleranza rispetto alla posizione geografica inserita durante la definizione delle condizioni di avanzamento di un passo&Interna\\
\midrule
FOBA 1.4.1.3.1.3&Per ogni criterio di superamento, l'amministratore potrà stabilire uno o più intervalli temporali in cui l'utente può inviare i dati richiesti&Capitolato\\
\midrule
FDEA 1.4.1.3.1.4&Per ogni criterio di superamento, l'amministratore potrà inserire dei vincoli sui dati numerici presenti nel passo in creazione&Capitolato\\
\midrule
FOPA 1.4.1.3.1.4.1&L'amministratore potrà stabilire un numero minimo e massimo di cifre durante la definizione dei vincoli su un dato numerico&Interna\\
\midrule
FDEA 1.4.1.3.1.4.2&L'amministratore, durante la definizione dei vincoli su un dato numerico, potrà stabilire se tale numero potrà contenere cifre decimali&Interna\\
\midrule
FOPA 1.4.1.3.1.4.3&L'amministratore, durante la definizione dei vincoli su un dato numerico, potrà stabilire un limite superiore e inferiore per tale numero&Interna\\
\midrule
FOPA 1.4.1.3.1.5&L'amministratore potrà stabilire la facoltatività dell'esecuzione di un passo&Capitolato\\
\midrule
FOBA 1.4.1.3.2&Il sistema dovrà permettere all'amministratore di scegliere il passo eseguibile dall'utente una volta soddisfatto il criterio di superamento in definizione&Capitolato\\
\midrule
FOBA 1.4.2&L'amministratore potrà visualizzare la lista dei passi creati durante la creazione di un nuovo processo&Interna\\
\midrule
FDEA 1.4.3&L'amministratore, durante la creazione di un nuovo processo, potrà modificare un passo esistente&Interna\\
\midrule
FDEA 1.4.3.1&Il sistema dovrà permettere all'amministratore di modificare la descrizione di un passo di un processo in creazione&Interna\\
\midrule
FDEA 1.4.3.2&Il sistema dovrà permettere all'amministratore di modificare la descrizione dei dati di un passo di un processo in creazione&Interna\\
\midrule
FDEA 1.4.3.3&Il sistema dovrà permettere all'amministratore di modificare i criteri di superamento dei passi del processo in creazione&Interna\\
\midrule
FDEA 1.4.3.3.1&Il sistema dovrà permettere all'amministratore di modificare le condizioni di avanzamento dei passi del processo in creazione&Interna\\
\midrule
FDEA 1.4.3.3.1.1&Il sistema dovrà permettere all'amministratore di modificare i vincoli sull'approvazione dei passi del processo in creazione&Interna\\
\midrule
FDEA 1.4.3.3.1.2&Il sistema dovrà permettere all'amministratore di modificare i vincoli dei passi del processo in creazione, relativi alla posizione dell'utente al momento dell'invio dei dati&Interna\\
\midrule
FDEA 1.4.3.3.1.3&Il sistema dovrà permettere all'amministratore di modificare gli intervalli temporali in cui l'utente potrà inviare i dati, stabiliti nei passi del processo in creazione&Interna\\
\midrule
FDEA 1.4.3.3.1.4&Il sistema dovrà permettere all'amministratore di modificare i vincoli sui dati numerici dei passi del processo in creazione&Interna\\
\midrule
FOPA 1.4.3.3.1.5&Il sistema dovrà permettere all'amministratore di modificare le impostazioni sulla facoltatività dei passi del processo in creazione&Interna\\
\midrule
FDEA 1.4.3.3.2&Il sistema dovrà permettere all'amministratore di sostituire il passo eseguibile al soddisfacimento dei criteri di superamento dei passi del processo in creazione&Interna\\
\midrule
FDEA 1.4.4&Il sistema dovrà permettere all'amministratore di eliminare un passo del processo in creazione&Interna\\
\midrule
FOBA 1.5&Il sistema dovrà permettere all'amministratore di avviare un processo in creazione che contiene almeno un passo&Capitolato\\
\midrule
FDEA 2&Il sistema dovrà permettere all'amministratore la gestione dei processi creati&Capitolato\\
\midrule
FDEA 2.1&Il sistema dovrà permettere all'amministratore di scegliere un processo avviato&Interna\\
\midrule
FOPA 2.1.2&Il sistema dovrà permettere all'amministratore di ricercare un processo inserendone il nome&Interna\\
\midrule
FDEA 2.1.3&Il sistema dovrà permettere all'amministratore di selezionare un processo da gestire&Interna\\
\midrule
FOPA 2.2&Il sistema dovrà permettere all'amministratore di selezionare gli utenti a cui permettere l'iscrizione al processo gestito&Interna\\
\midrule
FOPA 2.2.1&L'amministratore potrà visualizzare la lista degli utenti registrati al sistema&Interna\\
\midrule
FOPA 2.2.2&Il sistema dovrà permettere all'amministratore di selezionare dalla lista gli utenti a cui consentire l'iscrizione al processo gestito&Interna\\
\midrule
FDEA 2.3&Il sistema dovrà permettere all'amministratore di consultare informazioni sul processo gestito&Interna\\
\midrule
FOPA 2.3.1&Il sistema dovrà permettere all'amministratore di recuperare informazioni sul processo gestito&Interna\\
\midrule
FOPA 2.3.1.1&Il sistema dovrà permettere all'amministratore di visualizzare la descrizione del processo gestito&Interna\\
\midrule
FOPA 2.3.1.2&Il sistema dovrà permettere all'amministratore di visualizzare i criteri di terminazione del processo gestito&Interna\\
\midrule
FOPA 2.3.1.3&Il sistema dovrà permettere all'amministratore di visualizzare i dati dei passi del processo gestito&Interna\\
\midrule
FOPA 2.3.1.4&Il sistema dovrà permettere all'amministratore di visualizzare le condizioni di superamento dei passi del processo gestito&Interna\\
\midrule
FDEA 2.3.2&Il sistema dovrà permettere all'amministratore di visualizzare lo stato dell'esecuzione del processo&Interna\\
\midrule
FDEA 2.3.2.1&Il sistema dovrà permettere all'amministratore di visualizzare il numero di utenti iscritti al processo gestito&Interna\\
\midrule
FDEA 2.3.2.2&Il sistema dovrà permettere all'amministratore di visualizzare il numero di completamenti del processo gestito&Interna\\
\midrule
FDEA 2.3.3&Il sistema dovrà permettere all'amministratore di visualizzare i dati inviati dagli utenti che hanno comportato il superamento di un passo del processo gestito&Interna\\
\midrule
FDEA 2.4&Il sistema dovrà permettere all'amministratore di controllare i dati inviati dagli utenti che richiedono la sua approvazione& Verbale 2014-02-03\\
\midrule
FOBA 2.4.1&Il sistema dovrà permettere all'amministratore di visualizzare i dati inviati dagli utenti che richiedono la sua approvazione&Verbale 2014-02-03\\
\midrule
FDEA 2.4.2&Il sistema dovrà permettere all'amministratore di approvare i dati controllati&Verbale 2014-02-03\\
\midrule
FDEA 2.4.3&Il sistema dovrà permettere all'amministratore di respingere i dati controllati&Verbale 2014-02-03\\
\midrule
FDEA 2.4.4&Il sistema dovrà inviare l'esito del controllo agli utenti che hanno inviato dei dati che richiedono approvazione&Verbale 2014-02-03\\
\midrule
FDEA 2.5&Il sistema dovrà permettere all'amministratore di terminare un processo avviato&Interna\\
\midrule
FDEA 2.6&Il sistema dovrà permettere all'amministratore di eliminare un processo terminato dall'insieme dei processi creati&Interna\\ 
\bottomrule
\caption{Tabella dei requisiti amministratore}
\end{longtable}