\section{Descrizione generale}

\subsection{Contesto d'uso del prodotto}
Lo scopo del progetto \progetto{}, è di fornire un servizio di gestione di processi definiti da una serie di passi da eseguirsi in sequenza o senza un ordine predefinito.\\
Il sistema dovrà consentire la creazione di tali processi, e la loro esecuzione da parte di utenti dotati di dispositivo mobili di tipo \textit{smartphone} o \textit{tablet}, ricevendo da questi la richiesta di completamento di un passo del processo accompagnata da vari dati.\\
Il sistema dovrà essere in grado di determinare se il passo è stato eseguito con successo o meno in base ai dati forniti.

\subsection{Funzionalità del prodotto}
Il prodotto dovrà fornire un'interfaccia utilizzabile dagli utenti atta alla gestione ed esecuzione di processi, ed un'altra dedicata al solo \textit{process owner\ped{G}}, tramite cui potrà creare dei processi e monitorarne l'esecuzione da parte degli utenti.\\
Le principali funzionalità dedicate agli utenti, saranno fruibili ai soli utenti autenticati, perciò dovrà essere consentita la registrazione e l'autenticazione al sistema.\\
Agli utenti autenticati sarà permesso di:

\begin{itemize}
\item Visualizzare informazioni sui processi disponibili;
\item Effettuare l'iscrizione ad un processo disponibile;
\item Gestire i processi a cui sono iscritti:
\begin{itemize}
\item Visualizzare lo stato di avanzamento dei processi;
\item Visualizzare i criteri di superamento dei passi in corso;
\item Inserire eventuali dati richiesti;
\item Inviare i dati richiesti.
\end{itemize}
\end{itemize}

L'utente \textit{process owner\ped{G}}, tramite un'interfaccia dedicata, potrà:

\begin{itemize}
\item Creare nuovi processi:
\begin{itemize}
\item Definire nome e descrizione di nuovo processo;
\item Creare i passi del nuovo processo:
\begin{itemize}
\item Definire i dati che dovranno essere inseriti o raccolti;
\item Definire le condizioni di avanzamento del passo;
\item Definire se il passo può essere saltato.
\end{itemize}
\item Definire i criteri di terminazione dell'intero processo.
\end{itemize}
\item Gestire i processi precedentemente creati:
\begin{itemize}
\item Ottenere informazioni sul loro avanzamento;
\item Concedere il diritto di inscrizione al processo a determinati utenti;
\item Consultare la raccolta dei dati inviati dagli utenti;
\item Far avanzare passi che richiedono intervento umano.
\end{itemize}
\end{itemize}

\subsection{Caratteristiche degli utenti}
Il programma è rivolto principalmente ad utenti con conoscenze di base nell'utilizzo di \textit{smartphone} o \textit{tablet}.\\
L'uso dell'interfaccia richiede competenze informatiche di base, in particolare riguardanti l'utilizzo di un \textit{browser\ped{G}}.

\subsection{Vincoli generali}
Il progetto dovrà rispettare i seguenti vincoli generali:
\begin{itemize}
\item utilizzo di una componente \textit{server\ped{G}} sviluppata con il linguaggio \textit{Java\ped{G}};
\item utilizzo dei linguaggi \textit{HTML5\ped{G}}, \textit{Javascript\ped{G}} e \textit{CSS\ped{G}} per lo sviluppo dell'interfaccia;
\item utilizzo di un \textit{browser\ped{G}} per interagire con l'interfaccia utente;
\item accessibilità dell'interfaccia utente da \textit{smartphone} e \textit{tablet}.
\end{itemize}