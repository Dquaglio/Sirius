\section{Casi d'uso}

\iffalse  % commento multilinea
\paragraph{UC 1:}
\begin{itemize}
\item \textbf{Attori:} utente autenticato;
\item \textbf{Descrizione:} 
\item \textbf{Precondizione:} 
\item \textbf{Scenario principale:} 
\begin{itemize}
\item 
\end{itemize}
%\item \textbf{Estensioni:} 
%\item \textbf{Inclusioni:} 
%\item \textbf{Scenario alternativo:} 
\item \textbf{Postcondizione:}
\end{itemize}
\fi

\subsection{Ambito utente}

\subsubsection{UCU 0: Caso base utente generico}

\begin{itemize}
\item \textbf{Attori:} utente, utente autenticato;
\item \textbf{Descrizione:} l'utente accede alla pagina principale e, per poter accedere alle principali funzionalità, deve autenticarsi.
L'utente non autenticato può registrarsi oppure effettuare il login se è già registrato diventando utente autenticato.
L'utente autenticato può gestire il proprio account, gestire un insieme di processi ed effettuare il logout, ritornando allo stato di utente non autenticato;
\item \textbf{Precondizione:} il sistema è operativo e l'utente ha iniziato ad interfacciarvisi;
\item \textbf{Scenario principale:}
\begin{itemize}
\item l'utente può effettuare la registrazione;
\item l'utente può effettuare il login e diventare autenticato;
\item l'utente autenticato può gestire il proprio account;
\item l'utente autenticato può gestire i processi disponibili;
\item l'utente autenticato può effettuare il logout diventando utente.
\end{itemize}
%\item \textbf{Scenario alternativo:}
%\item \textbf{Inclusioni:}
\item \textbf{Postcondizione:} il sistema ha eseguito le operazioni desiderate dall'utente, salvando eventuali modifiche effettuate all'account e ai processi gestiti.
\end{itemize}

\paragraph{UCU 1: Registrazione}
\begin{itemize}
	\item \textbf{Attori:} utente;
	\item \textbf{Descrizione:} l'utente può registrarsi inserendo i dati richiesti dal sistema;
	\item \textbf{Precondizione:} il sistema é operativo e l'utente ha richiesto la registrazione;
	\item \textbf{Scenario principale:}
	\begin{itemize}
		\item l'utente inserisce lo username scelto;
		\item l'utente inserisce la password scelta;
		\item l'utente inserisce il proprio nome;
		\item l'utente inserisce il proprio cognome;
		\item l'utente inserisce la propria data di nascita;
		%\item l'utente inserisce un proprio numero di telefono.
		\item l'utente inserisce la propria email.
	\end{itemize}
	\item \textbf{Scenario alternativo:}
	\begin{itemize}
		\item Un altro utente possiede lo stesso username scelto dall'utente che viene avvisato dell'errore e può scegliere un username diverso.
	\end{itemize}
	\item \textbf{Postcondizione:} il sistema ha salvato i dati immessi dall'utente che ora può autenticarsi.
\end{itemize}

\subparagraph{UCU 1.1: Inserimento username per registrazione}
\begin{itemize}
	\item \textbf{Attori:} utente;
	\item \textbf{Descrizione:} l'utente può inserire il proprio username;
	\item \textbf{Precondizione:} l'utente ha richiesto la registrazione e vuole inserire lo username scelto;
	\item \textbf{Postcondizione:} lo username è stato inserito e il sistema prosegue con la registrazione.
\end{itemize}

\subparagraph{UCU 1.2: Inserimento password per registrazione}
\begin{itemize}
	\item \textbf{Attori:} utente;
	\item \textbf{Descrizione:} l'utente può inserire la propria password;
	\item \textbf{Precondizione:} l'utente ha richiesto la registrazione e vuole inserire la password scelta;
	\item \textbf{Postcondizione:} la password è stata inserita e il sistema prosegue con la registrazione.
\end{itemize}

\subparagraph{UCU 1.3: Inserimento nome per registrazione}
\begin{itemize}
	\item \textbf{Attori:} utente;
	\item \textbf{Descrizione:} l'utente può inserire il proprio nome;
	\item \textbf{Precondizione:} l'utente ha richiesto la registrazione e vuole inserire il proprio nome;
	\item \textbf{Postcondizione:} il nome è stato inserito e il sistema prosegue con la registrazione.
\end{itemize}

\subparagraph{UCU 1.4: Inserimento cognome per registrazione}
\begin{itemize}
	\item \textbf{Attori:} utente.
	\item \textbf{Descrizione:} l'utente può inserire il proprio cognome;
	\item \textbf{Precondizione:} l'utente ha richiesto la registrazione e vuole inserire il proprio cognome;
	\item \textbf{Scenario principale:}
	\item \textbf{Postcondizione:} il cognome è stato inserito e il sistema prosegue con la registrazione.
\end{itemize}

\subparagraph{UCU 1.5: Inserimento data di nascita per registrazione}
\begin{itemize}
	\item \textbf{Attori:} utente;
	\item \textbf{Descrizione:} l'utente può inserire la propria data di nascita;
	\item \textbf{Precondizione:} l'utente ha richiesto la registrazione e vuole inserire la propria data di nascita;
	\item \textbf{Scenario principale:}
	\item \textbf{Scenario alternativo:}
	\begin{itemize}
    \item l'utente inserisce una data di nascita futura rispetto alla data di registrazione, quindi viene avvisato con un errore e può inserire la data di nascita corretta.
    \end{itemize}
	\item \textbf{Postcondizione:} la data di nascita è stata inserita e il sistema prosegue con la registrazione.
\end{itemize}

\iffalse % ha senso modificare l'username che dovrebbe essere la chiave primaria?
% se si volesse ripristinare queso paragrafo si dovrà anche aggiornare i numeri degli UC.
\subparagraph{UCU 1.6: Inserimento numero di telefono per registrazione}
\begin{itemize}
	\item \textbf{Attori:} utente;
	\item \textbf{Descrizione:} l'utente può inserire il proprio numero di telefono;
	\item \textbf{Precondizione:} l'utente ha richiesto la registrazione e vuole inserire il proprio numero di telefono;
	\item \textbf{Postcondizione:} il numero di telefono è stato inserito e il sistema prosegue con la registrazione utente.
\end{itemize}
\fi

\subparagraph{UCU 1.6: Inserimento email per registrazione}
\begin{itemize}
	\item \textbf{Attori:} utente;
	\item \textbf{Descrizione:} l'utente può inserire una sua email;
	\item \textbf{Precondizione:} l'utente ha richiesto la registrazione e vuole inserire la propria email;
\item \textbf{Postcondizione:} l'email è stata inserita e il sistema prosegue con la registrazione utente.
\end{itemize}

\paragraph{UCU 2: Login}
\begin{itemize}
\item \textbf{Attori:} utente.
\item \textbf{Descrizione:} l'utente può inserire i propri dati d'accesso, ossia lo username e la propria password. Se lo username appartiene all'insieme degli utenti del sistema e la relativa password è corretta, allora l'utente viene autenticato;
\item \textbf{Precondizione:} il sistema è operativo e l'utente non autenticato vuole effettuare il login;
\item \textbf{Scenario principale:}
\begin{itemize}
\item l'utente inserisce il proprio username;
\item l'utente inserisce la propria password.
\end{itemize}
\item \textbf{Scenario alternativo:}
\begin{itemize}
\item le credenziali non risultano essere corrette, viene dunque segnalato l'errore all'utente che può reinserire i dati.
\end{itemize}
\item \textbf{Postcondizione:} l'utente diventa autenticato e il sistema è pronto per consentire la gestione dei processi e dell'account.
\end{itemize}

\subparagraph{UC 2.1: Inserimento username per login}
\begin{itemize}
\item \textbf{Attori:} utente;
\item \textbf{Descrizione:} l'utente può inserire il proprio username;
\item \textbf{Precondizione:} l'utente ha richiesto il login e vuole inserire la propria data di nascita;
\item \textbf{Postcondizione:} lo username è stato inserito e il sistema prosegue con la procedura di autenticazione utente.
\end{itemize}

\subparagraph{UC 2.2: Inserimento password per login}
\begin{itemize}
\item \textbf{Attori:} utente;
\item \textbf{Descrizione:} l'utente può inserire la propria password;
\item \textbf{Precondizione:} l'utente ha richiesto il login e vuole inserire la propria password per il login;
\item \textbf{Postcondizione:} la password è stata inserita e il sistema prosegue con la procedura di autenticazione utente.
\end{itemize}

\subsubsection{UCL 0: Caso base utente autenticato}

\begin{itemize}
\item \textbf{Attori:} utente, utente autenticato;
\item \textbf{Descrizione:} l'utente accede alla pagina principale e, per poter accedere alle principali funzionalità, deve autenticarsi.
L'utente non autenticato può registrarsi oppure effettuare il login se è già registrato diventando utente autenticato.
L'utente autenticato può gestire il proprio account, gestire un insieme di processi ed effettuare il logout, ritornando allo stato di utente non autenticato;
\item \textbf{Precondizione:} il sistema è operativo e l'utente ha iniziato ad interfacciarvisi;
\item \textbf{Scenario principale:}
\begin{itemize}
\item l'utente può effettuare la registrazione;
\item l'utente può effettuare il login e diventare autenticato;
\item l'utente autenticato può gestire il proprio account;
\item l'utente autenticato può gestire i processi disponibili;
\item l'utente autenticato può effettuare il logout diventando utente.
\end{itemize}
%\item \textbf{Scenario alternativo:}
%\item \textbf{Inclusioni:}
\item \textbf{Postcondizione:} il sistema ha eseguito le operazioni desiderate dall'utente, salvando eventuali modifiche effettuate all'account e ai processi gestiti.
\end{itemize}

\paragraph{UCL 1: Gestione dell'account} %opt
\begin{itemize}
	\item \textbf{Attori:} utente autenticato;
	\item \textbf{Descrizione:} l'utente visualizza i dati salvati nel sistema relativi al suo account e può modificarli;
	\item \textbf{Precondizione:} l'utente autenticato vuole gestire i propri dati;
	\item \textbf{Scenario principale:}
	\begin{itemize}
		\item l'utente può visualizzare i propri dati;
		\item l'utente può modificare i propri dati.
	\end{itemize}
	\item \textbf{Postcondizione:} il sistema ha eseguito e salvato le operazioni effettuate dall'utente sui dati del suo account.
\end{itemize}

\subparagraph{UCL 1.1: Visualizzazione dei dati dell'utente}
\begin{itemize}
	\item \textbf{Attori:} utente autenticato.
	\item \textbf{Descrizione:} l'utente può visualizzare i dati salvati nel sistema relativi al suo account;
	\item \textbf{Precondizione:} l'utente vuole gestire i suoi dati;
	\item \textbf{Scenario principale:}
	\begin{itemize}
		\item l'utente visualizza il suo username;
		\item l'utente visualizza il suo nome;
		\item l'utente visualizza il suo cognome;
		\item l'utente visualizza la sua data di nascita;
		\item l'utente visualizza la sua email.
		%\item l'utente visualizza il suo numero di telefono.
	\end{itemize}
	\item \textbf{Postcondizione:} il sistema ha visualizzato i dati dell'account dell'utente.
\end{itemize}

\subparagraph{UCL 1.1.1: Visualizzazione username}
\begin{itemize}
	\item \textbf{Attori:} utente autenticato.
	\item \textbf{Descrizione:} l'utente può visualizzare il suo username;
	\item \textbf{Precondizione:} l'utente vuole visualizzare i suoi dati;
\item \textbf{Postcondizione:} il sistema ha visualizzato lo username dell'utente.
\end{itemize}

\subparagraph{UCL 1.1.2: Visualizzazione nome}
\begin{itemize}
	\item \textbf{Attori:} utente autenticato;
	\item \textbf{Descrizione:} l'utente può visualizzare il suo nome;
	\item \textbf{Precondizione:} l'utente vuole visualizzare i suoi dati;
	\item \textbf{Postcondizione:} il sistema ha visualizzato il nome dell'utente.
\end{itemize}

\subparagraph{UCL 1.1.3: Visualizzazione cognome}
\begin{itemize}
	\item \textbf{Attori:} utente autenticato;
	\item \textbf{Descrizione:} l'utente può visualizzare il suo cognome;
	\item \textbf{Precondizione:} l'utente vuole visualizzare i suoi dati;
	\item \textbf{Postcondizione:} il sistema ha visualizzato il cognome dell'utente.
\end{itemize}

\subparagraph{UCL 1.1.4: Visualizzazione data di nascita}
\begin{itemize}
	\item \textbf{Attori:} utente autenticato;
	\item \textbf{Descrizione:} l'utente può visualizzare la sua data di nascita;
	\item \textbf{Precondizione:} l'utente vuole visualizzare i suoi dati;
	\item \textbf{Postcondizione:} il sistema ha visualizzato la data di nascita dell'utente.
\end{itemize}

\iffalse % ha senso il campo dati telefono?
% se si volesse ripristinare queso paragrafo si dovrà anche aggiornare i numeri degli UC.
\subparagraph{UCL 1.1.5: Visualizzazione numero di telefono}
\begin{itemize}
	\item \textbf{Attori:} utente autenticato;
	\item \textbf{Descrizione:} l'utente può visualizzare il numero di telefono salvato nel sistema;
	\item \textbf{Precondizione:} l'utente vuole visualizzare i suoi dati;
	\item \textbf{Scenario principale:}
	\begin{itemize}
	 \item l'utente visualizza il proprio numero di telefono.
	\end{itemize}
	\item \textbf{Postcondizione:} il sistema ha visualizzato il numero di telefono dell'utente.
\end{itemize}
\fi

\subparagraph{UCL 1.1.5: Visualizzazione email}
\begin{itemize}
	\item \textbf{Attori:} utente autenticato;
	\item \textbf{Descrizione:} l'utente può visualizzare la sua email salvata nel sistema;
	\item \textbf{Precondizione:} l'utente vuole visualizzare i suoi dati;
	\item \textbf{Scenario principale:}
	\begin{itemize}
	 \item l'utente visualizza la propria email.
	\end{itemize}
	\item \textbf{Postcondizione:} il sistema ha visualizzato il numero di telefono dell'utente.
\end{itemize}

 
\subparagraph{UCL 1.2: Modifica dati utente} %opt
\begin{itemize}
	\item \textbf{Attori:} utente autenticato.
	\item \textbf{Descrizione:} l'utente vuole modificare i dati salvati nel sistema relativi al suo account;
	\item \textbf{Precondizione:} l'utente vuole gestire i suoi dati;
	\item \textbf{Scenario principale:}
	\begin{itemize}
		%\item l'utente può modificare il suo username;
		\item l'utente può modificare la sua password;
		%\item l'utente può modificare il suo nome;
		%\item l'utente può modificare il suo cognome;
		%\item l'utente può modificare il suo numero di telefono;
		\item l'utente può modificare la sua email.
	\end{itemize}
\item \textbf{Postcondizione:} il sistema ha salvato le modiche dei dati dell'utente.
\end{itemize}

\iffalse % ha senso modificare l'username che dovrebbe essere la chiave primaria?
% se si volesse ripristinare queso paragrafo, si dovrà anche correggerlo e aggiornari i numeri degli UC.
\subparagraph{UCL 1.2.1: Modifica username}
\begin{itemize}
	\item \textbf{Attori:} utente autenticato.
	\item \textbf{Descrizione:} l'utente inserisce il proprio username, e se lo username risulta non è gia stato utilizzato da un altro utente, allora la modifica verrà salvata dal sistema
	\item \textbf{Precondizione:} l'utente vuole modicare il suo username.
	\item \textbf{Scenario principale:}
	\begin{itemize}
		\item l'utente inserisce il nuovo username scelto.
	\end{itemize}
	\item \textbf{Scenario alternativo:}
	\begin{itemize}
		\item il nuovo username è gia presente nel sistema, ne viene data comunicazione dell'errore e della possibilità di reinserire un username diverso.
	\end{itemize}
\item \textbf{Postcondizione:} il cambiamento di username viene salvato dal sistema.
\end{itemize}
\fi

\subparagraph{UCL 1.2.1: Modifica password}
\begin{itemize}
	\item \textbf{Attori:}  utente autenticato;
	\item \textbf{Descrizione:} per modificare la password d'accesso, l'utente deve inserire la password corrente e la nuova password scelta. Se i dati immessi risultano corretti vengono salvati dal sistema, altrimenti l'errore viene segnalato all'utente che dovrà correggerli;
	\item \textbf{Precondizione:} l'utente vuole modificare la sua password;
	\item \textbf{Scenario principale:}
	\begin{itemize}
		\item l'utente inserisce la password attualmente salvata;
		\item l'utente inserisce la nuova password scelta.
	\end{itemize}
	\item \textbf{Scenario alternativo:}
	\begin{itemize}
		\item la password corrente immessa non è corretta, l'errore viene segnalato all'utente che può reinserire la password.
	\end{itemize}
	\item \textbf{Postcondizione:} il sistema ha salvato la nuova password scelta dall'utente.
\end{itemize}

\subparagraph{UCL 1.2.1.1: Inserimento password corrente}
\begin{itemize}
	\item \textbf{Attori:} utente autenticato;
	\item \textbf{Descrizione:} l'utente può inserire la password corrente;
	\item \textbf{Precondizione:} l'utente vuole modificare la sua password;
	\item \textbf{Postcondizione:} la password corrente è stata inserita e il sistema prosegue con la procedura di modifica della password.
\end{itemize}

\subparagraph{UCL 1.2.1.2 Inserimento nuova password}
\begin{itemize}
	\item \textbf{Attori: } utente autenticato;
	\item \textbf{Descrizione:} l'utente può inserire la nuova password scelta;
	\item \textbf{Precondizione:} l'utente vuole modificare la sua password;
	\item \textbf{Postcondizione:} la nuova password è stata inserita e il sistema prosegue con la procedura di modifica della password.
\end{itemize}

\iffalse % ha senso modificare nome e cognome?
% se si volesse ripristinare quesi paragrafi, si dovrà anche correggerli e aggiornari i numeri degli UC.
\subparagraph{UCL 1.2.2.3 Modifica nome}
\begin{itemize}
	\item \textbf{Attori: } utente autenticato.
	\item \textbf{Descrizione: } l'utente inserisce il nuovo nome, il sistema procede con il salvataggio delle modifiche
	\item \textbf{Precondizione: } l'utente vuole modicare il suo nome salvato nel sistema.
	\item \textbf{Postcondizione:} la modifica del nome viene salvato dal sistema.
\end{itemize}

\subparagraph{UCL 1.2.2.4  Modifica cognome}
\begin{itemize}
	\item \textbf{Attori: }  utente autenticato.
	\item \textbf{Descrizione: } l'utente inserisce il nuovo cognome, il sistema procede con il salvataggio del dato.
	\item \textbf{Precondizione: } l'utente vuole cambiare il suo cognome salvato nel sistema.
	\item \textbf{Postcondizione:} il nuovo cognome viene salvato dal sistema.
\end{itemize}
\fi

\iffalse
\subparagraph{UCL 1.2.2  Modifica numero di telefono}
\begin{itemize}
\item \textbf{Attori: } utente autenticato;
\item \textbf{Descrizione: } l'utente inserisce il nuovo numero di telefono e il sistema procede con la modifica;
\item \textbf{Precondizione: } l'utente vuole modificare il numero di telefono salvato;
\item \textbf{Postcondizione:} il sistema ha salvato il nuovo di numero di telefono scelto dall'utente.
\end{itemize}
\fi

\subparagraph{UCL 1.2.2  Modifica email}
\begin{itemize}
\item \textbf{Attori:} utente autenticato;
\item \textbf{Descrizione:} l'utente inserisce una nuova email per modificare l'email salvata nel sistema;
\item \textbf{Precondizione:} l'utente vuole modificare l'email salvata nel sistema;
\item \textbf{Postcondizione:} il sistema ha salvato la nuova email scelta dall'utente.
\end{itemize}

\paragraph{UCL 3: Logout}
\begin{itemize}
	\item \textbf{Attori:} utente,utente autenticato.
	\item \textbf{Descrizione:} l'utente autenticato richiede di terminare la propria sessione e uscire
	dal sistema diventando nuovamente utente.
	\item \textbf{Precondizione:} l'utente autenticato vuole uscire dal sistema.
	\item \textbf{Scenario principale:}
	\begin{itemize}
		\item l'utente richiede di terminare la propria sessione;
		\item l'utente e uscito dal sistema;
		\item l'utente autenticato e diventato utente.
	\end{itemize}
	\item \textbf{Postcondizione:} l'utente autenticato è uscito dal sistema ridiventando utente.
\end{itemize}