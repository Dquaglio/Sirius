 \section{Casi d'uso}

\iffalse
\begin{itemize}
\item \textbf{Attori:} utente autenticato;
\item \textbf{Descrizione:}
\item \textbf{Precondizione:}
\item \textbf{Scenario principale:}\
\begin{enumerate}
\item
\end{enumerate}
%\item \textbf{Scenario alternativo:}
%\item \textbf{Inclusioni:}
%\item \textbf{Estensioni:}
\item \textbf{Postcondizione:}
\end{itemize}
\fi

\subsection{Ambito utente}

\subsubsection{UCU 0: Caso base utente }

\begin{itemize}
\item \textbf{Attori:} utente, utente autenticato;
\item \textbf{Descrizione:} l'utente accede alla pagina principale e, per poter accedere alle principali funzionalità, deve autenticarsi.
L'utente non autenticato può registrarsi oppure effettuare il login se è già registrato diventando utente autenticato.
L'utente autenticato può gestire il proprio account, gestire un insieme di processi ed effettuare il logout, ritornando allo stato di utente non autenticato;
\item \textbf{Precondizione:} il sistema è attivo e l'utente ha iniziato ad interfacciarcisi;
\item \textbf{Scenario principale:}
\begin{enumerate}
\item l'utente può effettuare la registrazione;
\item l'utente può effettuare il login e diventare autenticato;
\item l'utente autenticato può gestire il proprio account;
\item l'utente autenticato può gestire i processi disponibili;
\item l'utente autenticato può effettuare il logout diventando utente.
\end{enumerate}
%\item \textbf{Scenario alternativo:}
%\item \textbf{Inclusioni:}
\item \textbf{Postcondizione:} il sistema ha eseguito le operazioni desiderate dall'utente, salvando eventuali modifiche effettuate all'account e ai processi gestiti.
\end{itemize}

\subsubsection{UCU 1: Registrazione}
\begin{itemize}
	\item \textbf{Attori:} utente.
	\item \textbf{Descrizione: } l'utente vuole registrarsi nel sistema, inserisce quindi i suoi dati e verrà registrato dopo aver ricevuto conferma avvenuta registrazione da parte del sistema .
	\item \textbf{Precondizione:} il sistema é operativo e l'utente richiede la registrazione.
	\item \textbf{Scenario principale:}
	\begin{itemize}
		\item l'utente inserisce lo username scelto;
		\item l'utente inserisce la password scelta;
		\item l'utente inserisce il proprio nome;
		\item l'utente inserisce il proprio cognome;
		\item l'utente inserisce la data di nascità
		\item l'utente, a sua discrezione, inserisce il proprio numero di telefono;
l'utente viene registrato
	\end{itemize}
	\item \textbf{Scenario alternativo:}
	\begin{itemize}
		\item Un altro utente possiede l'username scelta dall'utente, che verrà avvisato dell'errore e potrà scegliere un username diverso.
		\item La password e la relativa conferma risultano diverse, allorà l'utente verrà avvisato dell'errore e potrà reinserire i dati.
	\end{itemize}
	\item \textbf{Postcondizione:} la registrazione e stata effeffettuata, il sistema ha salvato i dati immessi dall'utente ed ora e al punto iniziale
\end{itemize}

\subsubsection{UCU 1.1: Inserisci username per registrazione}
\begin{itemize}
	\item \textbf{Attori:} utente.
	\item \textbf{Descrizione: } l'utente inserisce il proprio username.
	\item \textbf{Precondizione:} l'utente richiede la registrazione e vuole procedere con 				l'inserimento dell'username scelto.
	\item \textbf{Scenario principale:}
	\begin{itemize}
		\item l'utente inserisce lo username da lui scelto
	\end{itemize}
	\item \textbf{Postcondizione:}  l'username é stato inserito, il sistema procede con la 			procedura di registrazione.
\end{itemize}

\subsubsection{UCU 1.2: Inserisci password per registrazione}
\begin{itemize}
	\item \textbf{Attori: } utente.
	\item \textbf{Descrizione: } l'utente inserisce la propria password.
	\item \textbf{Precondizione: } l'utente richiede la registrazione e vuole procedere con 				l'inserimento della password.
	\item \textbf{Scenario principale:}
	\begin{itemize}
		\item l'utente inserisce la password da lui scelta.
	\end{itemize}
	\item \textbf{Postcondizione: }la password e stata inserita, il sistema procede con la 			procedura di registrazione.
\end{itemize}

\subsubsection{UCU 1.3: Inserisci nome per registrazione}
\begin{itemize}
	\item \textbf{Attori: } utente.
	\item \textbf{Descrizione: } l'utente inserisce il proprio nome.
	\item \textbf{Precondizione: } l'utente richiede la registrazione e vuole procedere con l'inserimento del proprio nome.
	\item \textbf{Scenario principale:}
	\begin{itemize}
		\item l'utente inserisce il proprio nome.
	\end{itemize}
	\item \textbf{Postcondizione: } il nome é stato inserito, il sistema procede con la procedura di registrazione.
\end{itemize}

\subsubsection{UCU 1.4: Inserisci cognome per registrazione}
\begin{itemize}
	\item \textbf{Attori: } utente.
	\item \textbf{Descrizione: } l'utente inserisce il proprio cognome.
	\item \textbf{Precondizione: } l'utente richiede la registrazione e vuole procedere con l'inserimento del proprio cognome.
	\item \textbf{Scenario principale:}
	\begin{itemize}
		\item l'utente inserisce il proprio cognome.
	\end{itemize}
	\item \textbf{Postcondizione: } il cognome é stato inserito, il sistema procede con la procedura di registrazione.
\end{itemize}

\subsubsection{UCU 1.5: Inserisci la data di nascita}
\begin{itemize}
	\item \textbf{Attori: } utente.
	\item \textbf{Descrizione: } l'utente inserisce la propria data di nascita.
	\item \textbf{Precondizione: } l'utente richiede la registrazione e procede con l'inserimento della propria data di nascita.
	\item \textbf{Scenario principale:}
	\begin{itemize}
		\item l'utente inserisce la data di nascita.
	\end{itemize}
	\item \textbf{Scenario alternativo:}
	\begin{itemize}
    \item l'utente inserisce una data di nascita futura, rispetto alla data di registrazione; quindi l'utente verrà avvisato con un errore e potrà inserire la data di nascita corretta
    \end{itemize}
	\item \textbf{Postcondizione: } la data di nascita è stata inserita, il sistema procede con la procedura di registrazione.
\end{itemize}

\subsubsection{UCU 1.6: Inserisci numero di telefono per registrazione}
\begin{itemize}
	\item \textbf{Attori: } utente.
	\item \textbf{Descrizione: } l'utente inserisce il proprio numero di telefono.
	\item \textbf{Precondizione: } l'utente richiede la registrazione e procede con l'inserimento del proprio numero di telefono.
	\item \textbf{Scenario principale:}
	\begin{itemize}
		\item l'utente inserisce il proprio numero di telefono.
	\end{itemize}
\item \textbf{Postcondizione: } il numero di telefono e stato inserito, il sistema procede con la procedura di registrazione.
\end{itemize}

\subsubsection{UCU 2: Login}
\begin{itemize}
\item \textbf{Attori: }utente, utente autenticato.
\item \textbf{Descrizione: } l'utente inserisce i propri dati d'accesso, ossia l'username e la propria password. Se l'username appartiene all'insieme degli utenti del  sistema, e la relativa password è corretta, allora l'utente verrà autenticato al sistema
\item \textbf{Precondizione:} il sistema é funzionante, l'utente non ancora autenticato vuole loggarsi
\item \textbf{Scenario principale:}
\begin{itemize}
\item l'utente inserisce il proprio username (UC 2.1);
\item l'utente inserisce la propria password (UC 2.2);
\item l'utente viene autenticato al sistema;
\end{itemize}
\item \textbf{Scenario alternativo:}
\begin{itemize}
\item le credenziali non risultano essere corrette e viene data la possibilità all'utente di reinserire la propria username e password
\end{itemize}
\item \textbf{Postcondizione:} l'utente non autentificato viene loggato al sistema diventando quindi utente autenticato ed entra. L'utente autentificato ora potrà gestire il proprio account, visualizzare i processi disponibili e selezionarli.
\end{itemize}

\subsubsection{UC 2.1: Inserisci username login}
\begin{itemize}
\item \textbf{Attori: } utente.
\item \textbf{Descrizione: } l'utente inserisce il proprio username.
\item \textbf{Precondizione:} l'utente é pronto ad inserire il proprio username.
\item \textbf{Scenario principale:}
\begin{itemize}
\item L'utente inserisce il proprio username.
\end{itemize}
\item \textbf{Postcondizione:} l'utente ha inserito il proprio username.
\end{itemize}

\subsubsection{UC 2.2: Inserisci password login}
\begin{itemize}
\item \textbf{Attori: } utente.
\item \textbf{Descrizione: } l'utente inserisca la propria password.
\item \textbf{Precondizione:} l'utente é pronto ad inserire la propria password per il login.
\item \textbf{Scenario principale:}
\begin{itemize}
\item l'utente inserisce la propria password.
\end{itemize}
\item \textbf{Postcondizione:} l'utente ha inserito la propria password.
\end{itemize}

\subsubsection{UCU 3: Gestione dell'account} %opt
\begin{itemize}
	\item \textbf{Attori: } primari: utente autenticato.
	\item \textbf{Descrizione: } l'utente visualizza i suoi dati ed, eventualmente, puo modicarli.
	\item \textbf{Precondizione:} il sistema é funzionante, l'utente autenticato vuole visualizzare e/o modicare i propri dati.
	\item \textbf{Scenario principale:}
	\begin{itemize}
		\item l'utente puo visualizzare e/o modicare i suoi dati.
	\end{itemize}
	\item \textbf{Postcondizione:} l'utente ha visualizzato e/o ha modicato i suoi dati, solo dopo aver memorizzato eventuali modiche dei dati dell'utente, il sistema
si ritrova allo stato successivo al login 
\end{itemize}

\subsubsection{UCU 3.1: Visualizzazione dei dati dell'utente}
\begin{itemize}
	\item \textbf{Attori: } primari: utente autenticato.
	\item \textbf{Descrizione: } l'utente visualizza i suoi dati.
	\item \textbf{Precondizione: } l'utente vuole visualizzare i suoi dati.
	\item \textbf{Scenario principale:}
	\begin{itemize}
		\item l'utente visualizza i propri dati personali
	\end{itemize}
	\item \textbf{Postcondizione:} il sistema ha mostrato all'utente i suoi dati personali
\end{itemize}

\subsubsection{UCU 3.1.1: Visualizza username}
\begin{itemize}
	\item \textbf{Attori: } utente autenticato.
	\item \textbf{Descrizione: } l'utente visualizza il suo username.
	\item \textbf{Precondizione: } l'utente vuole visualizzare i propri dati.
	\item \textbf{Scenario principale:}
	\begin{itemize}
		\item l'utente visualizza il proprio username.
	\end{itemize}
\item \textbf{Postcondizione:} il sistema ha mostrato all'utente la sua username.
\end{itemize}

\subsubsection{UCU 3.1.2: Visualizza nome}
\begin{itemize}
	\item \textbf{Attori: } utente autenticato.
	\item \textbf{Descrizione: } l'utente visualizza il proprio nome.
	\item \textbf{Precondizione: } l'utente vuole visualizzare i propri dati.
	\item \textbf{Scenario principale:}
	\begin{itemize}
		\item l'utente visualizza il proprio nome.
	\end{itemize}
	\item \textbf{Postcondizione:} il sistema ha mostrato all'utente il proprio nome.
\end{itemize}

\subsubsection{UCU 3.1.3: Visualizza cognome}
\begin{itemize}
	\item \textbf{Attori: } utente autenticato.
	\item \textbf{Descrizione: } l'utente visualizza il suo cognome.
	\item \textbf{Precondizione: } l'utente vuole visualizzare i propri dati.
	\item \textbf{Scenario principale:}
	\begin{itemize}
		\item l'utente visualizza il proprio cognome.
	\end{itemize}
	\item \textbf{Postcondizione:} il sistema ha mostrato all'utente il suo cognome.
\end{itemize}

\subsubsection{UCU 3.1.4: Visualizza data di nascita}
\begin{itemize}
	\item \textbf{Attori: } utente autenticato.
	\item \textbf{Descrizione: } l'utente visualizza la sua data di nascita
	\item \textbf{Precondizione: } l'utente vuole visualizzare i propri dati.
	\item \textbf{Scenario principale:}
	\begin{itemize}
		\item l'utente visualizza la sua data di nascita
	\end{itemize}
	\item \textbf{Postcondizione:} il sistema ha mostrato all'utente la propria data di nascita
\end{itemize}

\subsubsection{UCU 3.1.5: Visualizza numero di telefono}
\begin{itemize}
	\item \textbf{Attori: } utente autenticato.
	\item \textbf{Descrizione: } l'utente visualizza il suo numero di telefono.
	\item \textbf{Precondizione: } l'utente vuole visualizzare i propri dati.
	\item \textbf{Scenario principale:}
	\begin{itemize}
	 \item l'utente visualizza il proprio numero di telefono.
	\end{itemize}
	\item \textbf{Postcondizione:} il sistema ha mostrato all'utente il proprio numero di telefono, ora il sistema ha finito di visualizzare i dati personali dell'utente
\end{itemize}

 
\subsubsection{UCU 3.2: Modica dati utente} %opt
\begin{itemize}
	\item \textbf{Attori: } utente autenticato.
	\item \textbf{Descrizione: } l'utente vuole modicare i propri dati; se i dati immessi dall'utente sono corretti il sistema salverà le modifiche.
	\item \textbf{Precondizione: } l'utente vuole modicare i propri dati.
	\item \textbf{Scenario principale:}
	\begin{itemize}
		\item l'utente sceglie il dato da modicare;
		\item l'utente dopo aver scelto il dato da modicare, procede con il cambiamento;
il sistema cambia il dato con quello indicato dall'utente.
	\end{itemize}
\item \textbf{Postcondizione:} il sistema ha salvato le modiche dei dati dell'utente.
\end{itemize}

\subsubsection{UCU 3.2.1: Modica username}
\begin{itemize}
	\item \textbf{Attori: } utente autenticato.
	\item \textbf{Descrizione: } l'utente inserisce il proprio username, e se l'username risulta non é gia stato utilizzato da un altro utente, allora la modifica verrà salvata dal sistema
	\item \textbf{Precondizione: } l'utente vuole modicare il suo username.
	\item \textbf{Scenario principale:}
	\begin{itemize}
		\item l'utente inserisce il nuovo username scelto.
	\end{itemize}
	\item \textbf{Scenario alternativo:}
	\begin{itemize}
		\item il nuovo username è gia presente nel sistema, ne viene data comunicazione dell'errore e della possibilità di reinserire un username diverso.
	\end{itemize}
\item \textbf{Postcondizione:} il cambiamento di username viene salvato dal sistema.
\end{itemize}

\subsubsection{UCU 3.2.2: Modica password}
\begin{itemize}
	\item \textbf{Attori: }  utente autenticato.
	\item \textbf{Descrizione: } l'utente inserisce la vecchia password, la nuova password e la conferma della nuova password. Se la vecchia password è corretta ed la nuova password e la conferma sono uguali, allora il sistema salverà le modifiche 
	\item \textbf{Precondizione: } l'utente vuole modicare la sua password.
	\item \textbf{Scenario principale:}
	\begin{itemize}
		\item l'utente inserisce la vecchia password;
		\item l'utente inserisce la nuova password scelta;
		\item l'utente inserisce la conferma della password scelta.
	\end{itemize}
	\item \textbf{Scenario alternativo:}
	\begin{itemize}
		\item la password corrente non corrisponde alla vecchia password , l'utente riceve la notifica dell'errore e puço reinserire il dato.
		\item le due nuove password inserite (la nuova password e la conferma) non combaciano, ne viene data comunicazione all'utente che può reinserire i dati.
	\end{itemize}
	\item \textbf{Postcondizione:} il sistema ha effettuato il salvaggio della nuova password al  posto della vecchia
\end{itemize}

\subsubsection{UCU 3.2.2.1: Inserisci vecchia password}
\begin{itemize}
	\item \textbf{Attori: } utente autenticato.
	\item \textbf{Descrizione: } l'utente inserisce la vecchia password.
	\item \textbf{Precondizione: } l'utente vuole modicare la sua password.
	\item \textbf{Scenario principale:}
	\begin{itemize}
		\item l'utente inserisce la sua vecchia password.
	\end{itemize}
	\item \textbf{Postcondizione:} la vecchia password e stata digitata.
\end{itemize}

\subsubsection{UCU 3.2.2.2 Inserisci nuova password}
\begin{itemize}
	\item \textbf{Attori: } primari: utente autenticato.
	\item \textbf{Descrizione: } l'utente inserisce la nuova password scelta.
	\item \textbf{Precondizione: } l'utente vuole modicare la propria password.
	\item \textbf{Scenario principale:}
	\begin{itemize}
		\item l'utente inserisce la nuova password scelta.
	\end{itemize}
	\item \textbf{Postcondizione:} la nuova password è stata digitata.
\end{itemize}

\subsubsection{UCU 3.2.2.3 Modica nome}
\begin{itemize}
	\item \textbf{Attori: } utente autenticato.
	\item \textbf{Descrizione: } l'utente inserisce il nuovo nome, il sistema procede con il salvataggio delle modifiche
	\item \textbf{Precondizione: } l'utente vuole modicare il suo nome salvato nel sistema.
	\item \textbf{Scenario principale:}
	\begin{itemize}
		\item l'utente inserisce il nuovo nome scelto.
	\end{itemize}
	\item \textbf{Postcondizione:} la modifica del nome viene salvato dal sistema.
\end{itemize}

\subsubsection{UCU 3.2.2.4  Modica cognome}
\begin{itemize}
	\item \textbf{Attori: }  utente autenticato.
	\item \textbf{Descrizione: } l'utente inserisce il nuovo cognome, il sistema procede con il salvataggio del dato.
	\item \textbf{Precondizione: } l'utente vuole cambiare il suo cognome salvato nel sistema.
	\item \textbf{Scenario principale:}
	\begin{itemize}
		\item l'utente inserisce il nuovo cognome scelto.
	\end{itemize}
	\item \textbf{Postcondizione:} il nuovo cognome viene salvato dal sistema.
\end{itemize}

\subsubsection{UCU 3.2.2.5  Modica numero di telefono}
\begin{itemize}
\item \textbf{Attori: } utente autenticato.
\item \textbf{Descrizione: }l'utente inserisce il nuovo numero di telefono, il sistema procede con la modifica.
\item \textbf{Precondizione: } l'utente vuole modicare il suo numero di telefono salvato nel sis-
tema.
\item \textbf{Scenario principale:}
	\begin{itemize}
		\item l'utente inserisce il nuovo numero di telefono.
	\end{itemize}
\item \textbf{Postcondizione:} il nuovo di numero di telefono viene salvato dal sistema.
\end{itemize}

\subsubsection{UCU 4: Gestione dei processi}
\begin{itemize}
\item \textbf{Attori:} utente autenticato;
\item \textbf{Descrizione:} l'utente può scegliere un processo per gestirlo.
Se l'utente non è iscritto al processo può iscrivervisi, altrimenti può eseguirlo o disiscrivervisi.
Scegliendo un processo l'utente può inoltre visualizzarne le informazioni;
\item \textbf{Precondizione:} l'utente autenticato vuole gestire i processi a cui può partecipare;
\item \textbf{Scenario principale:}
\begin{enumerate}
\item l'utente può scegliere un processo;
\item l'utente può visualizzare informazioni su un processo;
\item l'utente può iscriversi a un processo a cui non è già iscritto;
\item l'utente può eseguire un processo a cui è iscritto;
\item l'utente può disiscriversi da un processo a cui è iscritto.
\end{enumerate}
%\item \textbf{Scenario alternativo:}
%\item \textbf{Estensioni:}
\item \textbf{Postcondizione:} il sistema ha eseguito e salvato le operazioni desiderate dall'utente sui processi selezionati.
\end{itemize}

\subsubsection{UCU 4.1: Scelta di un processo}
\begin{itemize}
\item \textbf{Attori:} utente autenticato;
\item \textbf{Descrizione:} l'utente può selezionare un processo da una lista selezionata o da i risultati di una ricerca;
\item \textbf{Precondizione:} l'utente autenticato vuole gestire un processo tra quelli a cui può partecipare;
\item \textbf{Flusso principale degli eventi:}\
\begin{enumerate}
\item l'utente può aprire una lista di processi;
\item l'utente può selezionare un processo.
\end{enumerate}
\item \textbf{Estensioni:} l'utente può effettuare la ricerca di un processo;
%\item \textbf{Inclusioni:}
\item \textbf{Postcondizione:} il sistema ha aperto la pagina di gestione del processo scelto.
%\item \textbf{Scenario alternativo:}
\end{itemize}

\subsubsection{UCU 4.1.1: Apertura di una lista di processi}
\begin{itemize}
\item \textbf{Attori:} utente autenticato;
\item \textbf{Descrizione:} l'utente può scegliere e aprire una lista di processi;
\item \textbf{Precondizione:} l'utente vuole scegliere un processo tra quelli a cui può partecipare;
\item \textbf{Scenario principale:}\
\begin{enumerate}
\item l'utente può aprire la lista dei processi a cui è iscritto;
\item l'utente può aprire la lista dei processi a cui non è iscritto;
\end{enumerate}
%\item \textbf{Scenario alternativo:}
%\item \textbf{Inclusioni:}
%\item \textbf{Estensioni:}
\item \textbf{Postcondizione:} il sistema ha visualizzato la lista dei processi scelta dall'utente.
\end{itemize}

\subsubsection{UCU 4.1.1.1: Apertura lista dei processi in esecuzione}
\begin{itemize}
\item \textbf{Attori:} utente autenticato;
\item \textbf{Descrizione:} l'utente può aprire e visualizzare la lista dei processi in esecuzione, cioè quelli a cui è già iscritto. 
\item \textbf{Precondizione:} l'utente vuole visualizzare una lista di processi;
\item \textbf{Scenario principale:} l'utente può visualizzare la lista dei processi a cui è iscritto;
%\item \textbf{Scenario alternativo:}
%\item \textbf{Inclusioni:}
%\item \textbf{Estensioni:}
\item \textbf{Postcondizione:} il sistema ha visualizzato la lista dei processi a cui l'utente è iscritto.
\end{itemize}

\subsubsection{UCU 4.1.1.2: Apertura lista dei processi disponibili}
\begin{itemize}
\item \textbf{Attori:} utente autenticato;
\item \textbf{Descrizione:} l'utente può aprire e visualizzare la lista dei processi disponibili, cioè quelli a cui non è iscritto. 
\item \textbf{Precondizione:} l'utente vuole visualizzare una lista di processi;
\item \textbf{Scenario principale:} l'utente può visualizzare la lista dei processi a cui non è iscritto;
%\item \textbf{Scenario alternativo:}
%\item \textbf{Inclusioni:}
%\item \textbf{Estensioni:}
\item \textbf{Postcondizione:} il sistema ha visualizzato la lista dei processi a cui l'utente non è iscritto.
\end{itemize}

\subsubsection{UCU 4.1.2: Selezione di un processo}
\begin{itemize}
\item \textbf{Attori:} utente autenticato;
\item \textbf{Descrizione:} l'utente può selezionare un processo dalla lista dei processi precedentemente aperta;
\item \textbf{Precondizione:} l'utente sta visualizzando una lista di uno o più processi;
\item \textbf{Scenario principale:} l'utente può selezionare un processo dalla lista;
%\item \textbf{Scenario alternativo:}
%\item \textbf{Inclusioni:}
%\item \textbf{Estensioni:}
\item \textbf{Postcondizione:} il sistema ha aperto la pagina di gestione del processo scelto.
\end{itemize}

\subsubsection{UCU 4.1.3: Ricerca di un processo}
\begin{itemize}
\item \textbf{Attori:} utente autenticato;
\item \textbf{Descrizione:} l'utente può ricercare uno insieme dei processi tra tutti quelli a cui può partecipare;
\item \textbf{Precondizione:} l'utente vuole cercare un processo tra quelli a cui può partecipare;
\item \textbf{Scenario principale:} l'utente può ricercare un processo inserendone il nome;
%\item \textbf{Inclusioni:}
%\item \textbf{Estensioni:}
\item \textbf{Postcondizione:} il sistema ha visualizzato la lista dei processi che soddisfano i criteri di ricerca.
\end{itemize}

\subsubsection{UCU 4.2: Visualizzazione delle informazioni di un processo}
\begin{itemize}
\item \textbf{Attori:} utente autenticato;
\item \textbf{Descrizione:} l'utente può visualizzare la descrizione del processo, i suoi criteri di terminazione e la sua data di crezione.
\item \textbf{Precondizione:} il sistema ha aperto la pagina di gestione di un processo;
\item \textbf{Scenario principale:}\
\begin{enumerate}
\item l'utente può visualizzare la descrizione del processo;
\item l'utente può visualizzare i criteri di terminazione del processo; UCA??
\item l'utente può visualizzare la data di creazione del processo;
\end{enumerate}
%\item \textbf{Scenario alternativo:}
%\item \textbf{Inclusioni:}
%\item \textbf{Estensioni:}
\item \textbf{Postcondizione:} il sistema ha visualizzato le informazioni del processo gestito.
\end{itemize}

\subsubsection{UCU 4.3: Iscrizione ad un processo}
\begin{itemize}
\item \textbf{Attori:} utente autenticato;
\item \textbf{Descrizione:} l'utente può iscriversi ad un processo a cui non è ancora iscritto;
\item \textbf{Precondizione:} il sistema ha aperto la pagina di gestione di un processo e l'utente non è iscritto ad esso;
\item \textbf{Scenario principale:} l'utente può iscriversi al processo gestito;
%\item \textbf{Scenario alternativo:}
%\item \textbf{Inclusioni:}
%\item \textbf{Estensioni:}
\item \textbf{Postcondizione:} l'utente è iscritto al processo gestito;
\end{itemize}

\subsubsection{UCU 4.4: Esecuzione di un processo}
\begin{itemize}
\item \textbf{Attori:} utente autenticato;
\item \textbf{Descrizione:} l'utente può eseguire un processo sequenziale o non ordinato;
\item \textbf{Precondizione:} il sistema ha aperto la pagina di gestione di un processo e l'utente è iscritto ad esso;
\item \textbf{Scenario principale:}\
\begin{enumerate}
\item l'utente può eseguire un processo sequenziale;
\item l'utento può eseguire un processo non ordinato.
\end{enumerate}
%\item \textbf{Scenario alternativo:}
%\item \textbf{Inclusioni:}
%\item \textbf{Estensioni:}
\item \textbf{Postcondizione:} il sistema ha eseguito e salvato le operazioni effettuate dall'utente sul processo gestito.
\end{itemize}

\subsubsection{UCU 4.4.1: Esecuzione di un processo sequenziale}
\begin{itemize}
\item \textbf{Attori:} utente autenticato;
\item \textbf{Descrizione:} l'utente può visualizzare lo stato di avanzamento del processo e procedere all'esecuzione del passo in corso:
\item \textbf{Precondizione:} il sistema ha aperto la pagina di gestione di un processo sequenziale e l'utente è iscritto ad esso;
\item \textbf{Scenario principale:}
\begin{enumerate}
\item l'utente può visualizzare lo stato dell'esecuzione del processo;
\item l'utente può eseguire il passo in corso.
\end{enumerate}
\item \textbf{Scenario alternativo:} tutti i passi sono stati conclusi e il sistema visualizza una notifica che segnala che il processo è stato terminato; 
%\item \textbf{Inclusioni:}
%\item \textbf{Estensioni:}\end{itemize}
\item \textbf{Postcondizione:} il sistema ha eseguito e salvato le operazioni effettuate dall'utente sul processo gestito.
\end{itemize}

\subsubsection{UCU 4.4.1.1: Visualizzazione dello stato del processo}

\subsubsection{UCU 4.4.1.2: Esecuzione di un passo}
\begin{itemize}
\item \textbf{Attori:} utente autenticato;
\item \textbf{Descrizione:} l'utente può visualizzare lo stato di avanzamento del processo e procedere all'esecuzione del passo in corso:
\item \textbf{Precondizione:} il sistema ha aperto la pagina di gestione di un processo sequenziale, l'utente è iscritto ad esso e non lo ha concluso;
\item \textbf{Scenario principale:}
\begin{enumerate}
\item visualizzazione dei criteri di superamento del passo in corso;
\item inserimento dei dati richiesti;
\item invio dei dati;
\end{enumerate}\begin{description} \textbf{leftmargin=0cm}
%\item \textbf{Scenario alternativo:}
\item \textbf{Inclusioni:} Visualizzazione notifiche;
%\item \textbf{Estensioni:}
\item \textbf{Postcondizione:} il sistema ha eseguito e salvato le operazioni effettuate dall'utente sul processo gestito.
\end{itemize}

\subsubsection{UCU 4.4.1.2.1: Visualizzazione dei criteri di superamento del passo in corso} UCA??

\subsubsection{UCU 4.4.1.2.2: Inserimento dei dati richiesti - tutti i dati possono richiedere l'approvazione?}
\begin{itemize}
\item \textbf{Attori:} utente autenticato;
\item \textbf{Descrizione:} l'utente può inserire i dati richiesti per l'esecuzione del passo in corso. Può essere richiesto il caricamento di una foto scattata dall'utente, l'inserimento di testo e l'inserimento di dati numerici.
\item \textbf{Precondizione:} il sistema sta visualizzando all'utente i dati di esecuzione di un passo, ed esso richiede l'inserimento uno o più dati da parte dell'utente.
\item \textbf{Scenario principale:}
\begin{enumerate}
\item caricamento di una foto;
\item inserimento di testo;
\item inserimento di dati numerici;
\end{enumerate}
%\item \textbf{Scenario alternativo:}
%\item \textbf{Inclusioni:}
%\item \textbf{Estensioni:}
\item \textbf{Postcondizione:} l'utente ha inserito dei dati che sono pronti per essere inviati al sistema.
\end{itemize}

\subsubsection{UCU 4.4.1.2.3: Invio dei dati richiesti}
\begin{itemize}
\item \textbf{Attori:} utente autenticato;
\item \textbf{Descrizione:} l'utente può inviare 
\item \textbf{Precondizione:}
\item \textbf{Scenario principale:}
\begin{enumerate}
\item
\end{enumerate}
%\item \textbf{Scenario alternativo:}
%\item \textbf{Inclusioni:}
%\item \textbf{Estensioni:}
\item \textbf{Postcondizione:}
\end{itemize}

\subsubsection{UCU 4.4.1.2.4: Visualizzazione di notifiche}
\begin{itemize}
\item \textbf{Attori:} utente autenticato;
\item \textbf{Descrizione:}
\item \textbf{Precondizione:}
\item \textbf{Scenario principale:}
\begin{enumerate}
\item
\end{enumerate}
%\item \textbf{Scenario alternativo:}
%\item \textbf{Inclusioni:}
%\item \textbf{Estensioni:}
\item \textbf{Postcondizione:}
\end{itemize}

\subsubsection{UCU 4.4.1.2.5: Salto del passo corrente} %da estendere a 4.4.1.2.1; opt
\begin{itemize}
	\item \textbf{Attori: } primari: utente autenticato
	\item \textbf{Descrizione: }: l'utente autenticato riceve la notifica che è possibile saltare in esecuzione
	\item \textbf{Precondizione:} L'utente autentificato ha selezionato un processo, sta eseguendo il passo e riceve la comunicazione che il passo è saltabile
	\item \textbf{Flusso principale degli eventi:}
	\begin{enumerate}
		\item l'utente autentificato riceve la notifica che il passo può essere saltato
		\item l'utente autentificato conferma di voler saltare il passo
		\item l'utente ha saltato il passo con successo ed è pronto con il successivo
	\end{enumerate}
	\item \textbf{Scenario alternativo}
	\begin{itemize}
		\item l'utente autentificato riceve la notifica e decide di non saltare il passo
	\end{itemize}
	\item \textbf{Postcondizione:}: L'utente autentificato ha saltato con successo il passo corrente ed è pronto a proseguire con il passo successivo
\end{itemize}

\subsubsection{UCU 4.5: Disiscrizione da un processo}
\begin{itemize}
\item \textbf{Attori:} utente autenticato;
\item \textbf{Descrizione:} l'utente può disiscriversi da un processo a cui non è ancora iscritto;
\item \textbf{Precondizione:} il sistema ha aperto la pagina di gestione di un processo e l'utente è iscritto ad esso;
\item \textbf{Scenario principale:} l'utente può disiscriversi dal processo gestito;
%\item \textbf{Scenario alternativo:}
%\item \textbf{Inclusioni:}
%\item \textbf{Estensioni:}
\item \textbf{Postcondizione:} l'utente non è iscritto al processo gestito;
\end{itemize}

\subsubsection{UCU 5: Logout}
\begin{itemize}
	\item \textbf{Attori:} utente,utente autenticato.
	\item \textbf{Descrizione:} l'utente autenticato richiede di terminare la propria sessione e uscire
	dal sistema diventando nuovamente utente.
	\item \textbf{Precondizione:} l'utente autenticato vuole uscire dal sistema.
	\item \textbf{Scenario principale:}
	\begin{itemize}
		\item l'utente richiede di terminare la propria sessione;
		\item l'utente e uscito dal sistema;
		\item l'utente autenticato e diventato utente.
	\end{itemize}
	\item \textbf{Postcondizione:} l'utente autenticato é uscito dal sistema ridiventando utente.
\end{itemize}