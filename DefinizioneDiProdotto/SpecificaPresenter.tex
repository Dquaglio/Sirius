\section{Specifica della componente presenter}
Questa componente consente la gestione della logica principale dell'applicazione \progetto{} .

\subsection{Client}

Il \textit{presenter} consente di gestire la logica delle pagine dell'applicazione.
La inizializzazione delle classi e la gestione degli eventi di cambio pagina, avviene tramite la classe principale \texttt{Router}, che estende la classe \texttt{Backbone.Router} fornita dal \textit{framework\ped{G} Backbone}.
Le altre classi della componente, consentono di renderizzare le viste utilizzando i \textit{template} della componente \textit{view}, di gestire gli eventi generati dagli utenti, e di gestire la comunicazione con il server tramite le classi della componente \textit{model}.
Ogni classe presenter, tranne le classi derivate da BaseDispatcher, è figlia di BasePresenter.

La componente è formata dalle seguenti \textit{classi}:
\begin{itemize}
	\item \hyperref[router]{\logic{}.Rou\fshyp{}ter};
	\item \hyperref[login]{\logic{}.Lo\fshyp{}gin};
	\item \hyperref[mainUser]{\logicUser{}.Ma\fshyp{}in\fshyp{}U\fshyp{}ser};
	\item \hyperref[register]{\logicUser{}.Re\fshyp{}gis\fshyp{}ter};
	\item \hyperref[openProcessU]{\logicUser{}.O\fshyp{}pen\fshyp{}Pro\fshyp{}cess};
	\item \hyperref[managementProcessU]{\logicUser{}.Ma\fshyp{}na\fshyp{}ge\fshyp{}Pro\fshyp{}cess};
	\item \hyperref[ev1]{\logicUser{}.EventDispatcher};
	\item \hyperref[sendData]{\logicUser{}.Send\fshyp{}Da\fshyp{}ta};
	\item \hyperref[printReport]{\logicUser{}.Print\fshyp{}Re\fshyp{}port};
	\item \hyperref[mainProcessOwner]{\logicAdmin{}.Main\fshyp{}Pro\fshyp{}cess\fshyp{}Ow\fshyp{}ner};
	\item \hyperref[newProcess]{\logicAdmin{}.New\fshyp{}Pro\fshyp{}cess};
	\item \hyperref[addStep]{\logicAdmin{}.Add\fshyp{}Step};
	\item \hyperref[openProcess]{\logicAdmin{}.O\fshyp{}pen\fshyp{}Pro\fshyp{}cess};
	\item \hyperref[manageProcess]{\logicAdmin{}.Ma\fshyp{}na\fshyp{}ge\fshyp{}Pro\fshyp{}cess};
	\item \hyperref[ev2]{\logicAdmin{}.EventDispatcher};
	\item \hyperref[checkStep]{\logicAdmin{}.Check\fshyp{}Step};
	\item \hyperref[pd]{\logicAdmin{}.ProcessData}.
	
\end{itemize}


\subsubsection{Package \logic{}}

\paragraph{Router}
\label{router}

\begin{figure}[H] \centering
\includegraphics[trim=0cm 0.8cm 0cm 0cm,clip=true,scale=0.75]%
{./classi/client/presenter/Router.png} \caption{Diagramma classe \textit{Router}}
\end{figure}

\begin{flushleft}
\begin{itemize}
\item \textbf{Descrizione:} Classe che permette di coordinare l'inizializzazione e la renderizzazione delle pagine, gestendo gli eventi e le azioni di cambio pagina;
\item \textbf{Relazioni con altri componenti:}
\begin{sloppypar}
La classe reperisce le informazioni di sessione dalla classe \texttt{\model{}::U\fshyp{}ser\fshyp{}Mo\fshyp{}del} e comunica con le seguenti classi se l'utente dispone dei diritti d'accesso necessari:
\begin{itemize}
\item \texttt{\logic{}.Lo\fshyp{}gin};
\item \texttt{\logicUser{}.Re\fshyp{}gis\fshyp{}ter};
\item \texttt{\logicUser{}.Ma\fshyp{}in\fshyp{}Us\fshyp{}er};
\item \texttt{\logicUser{}.Us\fshyp{}er\fshyp{}Da\fshyp{}ta};
\item \texttt{\logicUser{}.Op\fshyp{}en\fshyp{}Pro\fshyp{}cess\fshyp{}gic};
\item \texttt{\logicUser{}.Ma\fshyp{}nag\fshyp{}ment\fshyp{}Pro\fshyp{}cess};
\item \texttt{\logicAdmin{}.Ma\fshyp{}in\fshyp{}Pro\fshyp{}cess\fshyp{}Ow\fshyp{}ner};
\item \texttt{\logicAdmin{}.O\fshyp{}pen\fshyp{}Pro\fshyp{}cess};
\item \texttt{\logicAdmin{}.New\fshyp{}Pro\fshyp{}cess};
\item \texttt{\logicAdmin{}.Check\fshyp{}Step};
\item \texttt{\logicAdmin{}.Ma\fshyp{}na\fshyp{}ge\fshyp{}Pro\fshyp{}cess};
\end{itemize}
\end{sloppypar}
\item \textbf{Attributi:}
\begin{sloppypar}
\begin{itemize}
\item \texttt{+ UserData session:}\\ oggetto di tipo \texttt{\model{}.U\fshyp{}ser\fshyp{}Da\fshyp{}ta}, che consente di gestire la sessione dell'utente;
\item \texttt{+ Backbone.View[] views:}\\ array che contiene le classi del presenter in esecuzione;
\item \texttt{+ Object routes:}\\ oggetto ridefinito da \texttt{Backbone.Router} che associa ad ogni evento di \textit{routing\ped{G}}, un metodo della classe;
\end{itemize}
\end{sloppypar}
\item \textbf{Metodi:}
\begin{sloppypar}
\begin{itemize}
\item \texttt{+ void home():}\\ gestisce l'evento di \textit{routing\ped{G} home};
\item \texttt{+ void processes():}\\ gestisce l'evento di \textit{routing\ped{G} processes};
\item \texttt{+ void newProcess():}\\ gestisce l'evento di \textit{routing\ped{G} newProcess};
\item \texttt{+ void checkStep():}\\ gestisce l'evento di \textit{routing\ped{G} checkStep};
\item \texttt{+ void process():}\\ gestisce l'evento di \textit{routing\ped{G} process};
\item \texttt{+ void register():}\\ gestisce l'evento di \textit{routing\ped{G} register};
\item \texttt{+ void user():}\\ gestisce l'evento di \textit{routing\ped{G} user};
\item \texttt{+ bool checkSession(String pageId):}\\ ritorna \texttt{true} solo se l'utente è autenticato; in caso contrario crea e renderizza la pagina di \textit{login};
\item \texttt{+ void load(String resource, String pageId):}\\ crea e aggiunge una vista di tipo \textit{resource} al campo dati \texttt{this.views}, all'indice \textit{pageId};
\item \texttt{+ void changePage(String pageId):}\\ imposta la pagina con id \textit{pageId} come attiva, ed esegue la transizione di cambio pagina.
\end{itemize}
\end{sloppypar}
\end{itemize}
\end{flushleft}

\paragraph{Login}
\label{login}

\begin{figure}[H] \centering
\includegraphics[trim=0cm 0.8cm 0cm 0cm,clip=true,scale=0.75]%
{./classi/client/presenter/Login.png} \caption{Diagramma classe  \textit{Login}}
\end{figure}

\begin{flushleft}
\begin{itemize}
\item \textbf{Descrizione:} Classe che ha il compito di gestire le richieste di autenticazione al sistema;
\item \textbf{Relazioni con altri componenti:}
\begin{sloppypar}
La classe gestisce i dati di sessione comunicando con la classe \texttt{\model{}.U\fshyp{}ser\fshyp{}Mo\fshyp{}del} e realizza l'interfaccia grafica utilizzando il \textit{template} \texttt{\view{}Lo\fshyp{}gin}.
\end{sloppypar}
\item \textbf{Attributi:}
\begin{sloppypar}
\begin{itemize}
\item \texttt{+ UserDataModel model:}\\ campo dati di tipo \texttt{\model{}.U\fshyp{}ser\fshyp{}Mo\fshyp{}del} che contiene i dati di sessione dell'utente;
\item \texttt{+ Object template:}\\ oggetto ridefinito da \texttt{Backbone.View}, che contiene il \textit{template HTML\ped{G}} associato alla classe;
\item \texttt{+ Object el:}\\ oggetto ridefinito da \texttt{Backbone.View} che rappresenta l'elemento \textit{HTML\ped{G}} entro cui la classe ascolta eventi generati dagli utenti;
\item \texttt{+ Object events:}\\ oggetto ridefinito da \texttt{Backbone.View} che associa ad ogni evento generato dagli utenti nella pagina \textit{HTML\ped{G}}, un metodo della classe;
\end{itemize}
\end{sloppypar}
\item \textbf{Metodi:}
\begin{sloppypar}
\begin{itemize}
\item \texttt{+ void initialize():}\\ metodo ridefinito da \texttt{Backbone.View}, invocato alla costruzione di ciascun oggetto della classe, che consente di aggiungere una pagina \textit{HTML\ped{G}} associata al componente;
\item \texttt{+ void render():}\\ metodo ridefinito da \texttt{Backbone.View}, che consente di aggiungere alla pagina \textit{HTML\ped{G}} il \textit{template} campo dati della classe;
\item \texttt{+ void login(Event event):}\\ effettua una richiesta di \textit{login}, utilizzando il campo dati \model{} per comunicare con il \textit{server\ped{G}}.
\end{itemize}
\end{sloppypar}
\end{itemize}
\end{flushleft}

\subsubsection{Package \logicUser{}}

\paragraph{EventDispatcher}
\label{ev1}

\begin{figure}[H] \centering
\includegraphics[trim=0cm 0.8cm 0cm 0cm,clip=true,scale=0.75]%
{./classi/client/presenter/po/EventDispatcherB.png} \caption{Diagramma classe  \textit{EventDispatcher}}
\end{figure}

\begin{flushleft}
\begin{itemize}
\item \textbf{Descrizione:} Classe per la gestione della notifica di presenza di passi che richiedono approvazione.
\item \textbf{Relazioni con altri componenti:}
\begin{sloppypar}
La classe estende la classe \texttt{\logic{}.Base\fshyp{}Dis\fshyp{}pat\fshyp{}cher}.
\end{sloppypar}
\item \textbf{Attributi:}
\begin{sloppypar}
\begin{itemize}
	\item \texttt{- int INTERVALMS :} indica l'intervallo di pooling;
  	\item \texttt{+intervalId :} indica l'id dell'intervallo;
\end{itemize}
\end{sloppypar}
\item \textbf{Metodi:}
\begin{sloppypar}
\begin{itemize}  
	\item \texttt{- intervalFunction(collection : Backbone.Collection) : void:}\\ invoca il metodo notify se la collezione contiene dei nuovi passi che richiedono approvazione;
  	\item \texttt{+ startListen() : void :}\\ inizia a monitorare eventuali variazioni nella collezione di passi che richiedono approvazione;
  	\item \texttt{+ stopListen() : void :}\\ interrompe l'esecuzione della funzione periodica con id "intervalId";
  	\item \texttt{+ notify() : void:}\\ notifica gli observer.
\end{itemize}
\end{sloppypar}
\end{itemize}
\end{flushleft}

\paragraph{MainUser}
\label{mainUser}

\begin{figure}[H] \centering
\includegraphics[trim=0cm 0.8cm 0cm 0cm,clip=true,scale=0.75]%
{./classi/client/presenter/user/MainUser.png} \caption{Diagramma classe  \textit{MainUser}}
\end{figure}

\begin{flushleft}
\begin{itemize}
\item \textbf{Descrizione:} Classe che ha il compito della gestione generale della logica delle funzionalità utente;
\item \textbf{Relazioni con altri componenti:}
\begin{sloppypar}
La classe comunica con l'interfaccia \texttt{\viewUser{}.I\fshyp{}Main\fshyp{}U\fshyp{}ser} per la realizzazione dell'interfaccia grafica, e ProcessCollection per il recupero della lista dei processi attivi e da iscrivere.
\end{sloppypar}
\item \textbf{Attributi:}
\begin{sloppypar}
\begin{itemize}
\item \texttt{+ UserDataModel model:}\\ campo dati di tipo \texttt{\model{}.U\fshyp{}ser\fshyp{}Mo\fshyp{}del} che contiene i dati di sessione dell'utente;
\item \texttt{+ ProcessCollection avviableProcess:}\\ campo dati contenente i processi iscrivibili dall'utente;
\item \texttt{+ ProcessCollection runningProcess:}\\ campo dati contenente i processi iscritti dall'utente;
\item \texttt{+ Object template:}\\ oggetto ridefinito da \texttt{Backbone.View}, che contiene il \textit{template HTML\ped{G}} associato alla classe;
\item \texttt{+ Object el:}\\ oggetto ridefinito da \texttt{Backbone.View} che rappresenta l'elemento \textit{HTML\ped{G}} entro cui la classe ascolta eventi generati dagli utenti;
\item \texttt{+ String id:}\\ campo dati ridefinito da \texttt{Backbone.View} contente l'id della classe;
\end{itemize}
\end{sloppypar}
\item \textbf{Metodi:}
\begin{sloppypar}
\begin{itemize}
\item \texttt{+ initialize(options : Object) : void:}\\ metodo ridefinito da \texttt{Backbone.View}, invocato alla costruzione di ciascun oggetto della classe, che consente di aggiungere una pagina \textit{HTML\ped{G}} associata al componente;
\item \texttt{+ render(options : Object) : void:}\\ metodo ridefinito da \texttt{Backbone.View}, che consente di aggiungere alla pagina \textit{HTML\ped{G}} il \textit{template} campo dati della classe;
\item \texttt{+ update() : void) : void:}\\ aggiorna le collezioni di processi recuperando i dati dal server;
\item \texttt{+ notifyApprovedData(collection : ProcessCollection) : void:}\\ metodo per la gestione della notifica della presenta di un nuovo passo approvato/respinto.
\end{itemize}
\end{sloppypar}
\end{itemize}
\end{flushleft}

\paragraph{Register}
\label{register}

\begin{figure}[H] \centering
\includegraphics[trim=0cm 0.8cm 0cm 0cm,clip=true,scale=0.75]%
{./classi/client/presenter/user/Register.png} \caption{Diagramma classe  \textit{Register}}
\end{figure}

\begin{flushleft}
\begin{itemize}
\item \textbf{Descrizione:} Classe che ha il compito di gestire le richieste di registrazione da parte dell'utente;
\item \textbf{Relazioni con altri componenti:}
\begin{sloppypar}
La classe comunica con l'interfaccia \texttt{\viewUser{}.I\fshyp{}Re\fshyp{}gi\fshyp{}ster} per la realizzazione dei \textit{widget} per la registrazione.
\end{sloppypar}
\item \textbf{Attributi:}
\begin{sloppypar}
\begin{itemize}
\item \texttt{+ Object template:}\\ oggetto ridefinito da \texttt{Backbone.View}, che contiene il \textit{template HTML\ped{G}} associato alla classe;
\item \texttt{+ Object el:}\\ oggetto ridefinito da \texttt{Backbone.View} che rappresenta l'elemento \textit{HTML\ped{G}} entro cui la classe ascolta eventi generati dagli utenti;
\item \texttt{+ Object events:}\\ oggetto ridefinito da \texttt{Backbone.View} che associa ad ogni evento generato dagli utenti nella pagina \textit{HTML\ped{G}}, un metodo della classe;
\end{itemize}
\end{sloppypar}
\item \textbf{Metodi:}
\begin{sloppypar}
\begin{itemize}
\item \texttt{+ void initialize():}\\ metodo ridefinito da \texttt{Backbone.View}, invocato alla costruzione di ciascun oggetto della classe, che consente di aggiungere una pagina \textit{HTML\ped{G}} associata al componente;
\item \texttt{+ void render():}\\ metodo ridefinito da \texttt{Backbone.View}, che consente di aggiungere alla pagina \textit{HTML\ped{G}} il \textit{template} campo dati della classe;
\item \texttt{- void register(Event event):}\\ effettua una richiesta di registrazione, utilizzando il campo dati \model{} per comunicare con il \textit{server\ped{G}};
\item \texttt{- printMessage() : void :}\\apre un popup con titolo "title" e contenuto "content";
\item \texttt{- validateData(title : String, content : String) : String :}\\ valida la data;
\item \texttt{- validatePassword(password : String, confirmPassword : String) : String:}\\ valida la password;
\item \texttt{- cancelRegister(event : Event) : void :}\\gestisce l'evento per eliminare la registrazione.
\end{itemize}
\end{sloppypar}
\end{itemize}
\end{flushleft}


\paragraph{OpenProcess}
\label{openProcessU}

\begin{figure}[H] \centering
\includegraphics[trim=0cm 0.8cm 0cm 0cm,clip=true,scale=0.75]%
{./classi/client/presenter/user/OpenProcess.png} \caption{Diagramma classe  \textit{OpenProcess}}
\end{figure}

\begin{flushleft}
\begin{itemize}
\item \textbf{Descrizione:} Classe che ha il compito di selezionare, ricercare e aprire un processo fra quelli eseguibili;
\item \textbf{Relazioni con altri componenti:}
\begin{sloppypar}
La classe realizza e modifica l'opportuno \textit{widget} mediante l'interfaccia \texttt{\viewUser{}.I\fshyp{}Op\fshyp{}en\fshyp{}Pro\fshyp{}cess} e utilizza la classe \texttt{\collection{}.Pro\fshyp{}cess\fshyp{}Col\fshyp{}lec\fshyp{}tion} per gestire e ottenere i dati dal \textit{server\ped{G}}.
\end{sloppypar}
\item \textbf{Attributi:}
\begin{sloppypar}
\begin{itemize}
\item \texttt{+ ProcessCollection collection:}\\ campo dati di tipo \texttt{\collection{}.Pro\fshyp{}cess\fshyp{}Col\fshyp{}lec\fshyp{}tion} che contiene la lista dei processi non terminati o non ancora eliminati dall'utente;
\item \texttt{+ Object template:}\\ oggetto ridefinito da \texttt{Backbone.View}, che contiene il \textit{template HTML\ped{G}} associato alla classe;
\item \texttt{+ Object el:}\\ oggetto ridefinito da \texttt{Backbone.View} che rappresenta l'elemento \textit{HTML\ped{G}} entro cui la classe ascolta eventi generati dagli utenti;
\item \texttt{+ String id:}\\ campo dati ridefinito da \texttt{Backbone.View} contente l'id della classe;
\end{itemize}
\end{sloppypar}
\item \textbf{Metodi:}
\begin{sloppypar}
\begin{itemize}
\item \texttt{+ void initialize():}\\ metodo ridefinito da \texttt{Backbone.View}, invocato alla costruzione di ciascun oggetto della classe, che consente di aggiungere una pagina \textit{HTML\ped{G}} associata al componente;
\item \texttt{+ void render():}\\ metodo ridefinito da \texttt{Backbone.View}, che consente di aggiungere alla pagina \textit{HTML\ped{G}} il \textit{template} campo dati della classe;
\item \texttt{+ void update():}\\ aggiorna il campo dati \texttt{collection} comunicando con il \textit{server\ped{G}}.
\end{itemize}
\end{sloppypar}
\end{itemize}
\end{flushleft}

\paragraph{ManagementProcess}
\label{managementProcessU}
\begin{flushleft}

\begin{figure}[H] \centering
\includegraphics[trim=0cm 0.8cm 0cm 0cm,clip=true,scale=0.75]%
{./classi/client/presenter/user/ManagementProcess.png} \caption{Diagramma classe  \textit{ManagementProcess}}
\end{figure}

\begin{itemize}
\item \textbf{Descrizione:} Classe che ha il compito di gestire e accedere alle informazioni relative allo stato del processo selezionato.;
\item \textbf{Relazioni con altri componenti:}
\begin{sloppypar}
La classe comunica con l'interfaccia \texttt{\viewUser{}.I\fshyp{}Ma\fshyp{}na\fshyp{}gment\fshyp{}Pro\fshyp{}cess} per realizzare il \textit{widget} che permette la gestione del processo selezionato, utilizza la classe \texttt{\model{}.Pro\fshyp{}cess\fshyp{}Mo\fshyp{}del} per gestire e ottenere i dati dal \textit{server\ped{G}}, e provvede ad invocare le seguenti classi in base alle decisioni dell'utente:
\begin{itemize}
\item \texttt{\logicUser{}.Print\fshyp{}Re\fshyp{}port};
\item \texttt{\logicUser{}.Send\fshyp{}Da\fshyp{}ta}.
\end{itemize}
\end{sloppypar}
\item \textbf{Attributi:}
\begin{sloppypar}
\begin{itemize}
\item \texttt{+ ProcessModel process:}\\ campo dati di tipo \texttt{\model{}.Pro\fshyp{}cess\fshyp{}Mo\fshyp{}del} che contiene i dati del processo in gestione;
\item \texttt{+ Object template:}\\ oggetto ridefinito da \texttt{Backbone.View}, che contiene il \textit{template HTML\ped{G}} associato alla classe;
\item \texttt{+ Object el:}\\ oggetto ridefinito da \texttt{Backbone.View} che rappresenta l'elemento \textit{HTML\ped{G}} entro cui la classe ascolta eventi generati dagli utenti;
\item \texttt{+ String id:}\\ campo dati ridefinito da \texttt{Backbone.View} contente l'id della classe;
\end{itemize}
\end{sloppypar}
\item \textbf{Metodi:}
\begin{sloppypar}
\begin{itemize}
\item \texttt{+ void initialize():}\\ metodo ridefinito da \texttt{Backbone.View}, invocato alla costruzione di ciascun oggetto della classe, che consente di aggiungere una pagina \textit{HTML\ped{G}} associata al componente;
\item \texttt{+ void render():}\\ metodo ridefinito da \texttt{Backbone.View}, che consente di aggiungere alla pagina \textit{HTML\ped{G}} il \textit{template} campo dati della classe;
\item \texttt{+ void update():}\\ aggiorna i campi dati \texttt{process} e \texttt{processData} comunicando con il \textit{server\ped{G}};
\item \texttt{+ String getParam(String param):}\\ ritorna il valore del parametro \textit{param} se presente nella \textit{URL\ped{G}}.
\end{itemize}
\end{sloppypar}
\end{itemize}
\end{flushleft}

\paragraph{PrintReport}
\label{printReport}

\begin{figure}[H] \centering
\includegraphics[trim=0cm 0.8cm 0cm 0cm,clip=true,scale=0.75]%
{./classi/client/presenter/user/PrintReport.png} \caption{Diagramma classe  \textit{PrintReport}}
\end{figure}

\begin{flushleft}
\begin{itemize}
\item \textbf{Descrizione:} Classe che ha il compito di gestire la creazione del report di fine processo;
\item \textbf{Relazioni con altri componenti:}
\begin{sloppypar}
La classe comunica con l'interfaccia \texttt{\viewUser{}.I\fshyp{}Print\fshyp{}Re\fshyp{}port} per realizzare il \textit{widget} per creare il report di fine processo, e utilizza la classe \texttt{\collection{}.Pro\fshyp{}cess\fshyp{}Data\fshyp{}Col\fshyp{}lec\fshyp{}tion} per gestire e ottenere i dati dal \textit{server\ped{G}}.
\end{sloppypar}
\item \textbf{Attributi:}
\begin{sloppypar}
\begin{itemize}
\item \texttt{+ ProcessDataCollection processdata:}\\ campo dati di tipo \texttt{\collection{}.Pro\fshyp{}cess\fshyp{}Da\fshyp{}ta\fshyp{}Col\fshyp{}lec\fshyp{}tion} che contiene i dati inviati dall'utente relativi al processo in gestione;
\item \texttt{+ Object template:}\\ oggetto ridefinito da \texttt{Backbone.View}, che contiene il \textit{template HTML\ped{G}} associato alla classe;
\item \texttt{+ Object el:}\\ oggetto ridefinito da \texttt{Backbone.View} che rappresenta l'elemento \textit{HTML\ped{G}} entro cui la classe ascolta eventi generati dagli utenti;
\item \texttt{+ String id:}\\ campo dati ridefinito da \texttt{Backbone.View} contente l'id della classe;
\end{itemize}
\end{sloppypar}
\item \textbf{Metodi:}
\begin{sloppypar}
\begin{itemize}
\item \texttt{+ void initialize():}\\ metodo ridefinito da \texttt{Backbone.View}, invocato alla costruzione di ciascun oggetto della classe, che consente di aggiungere una pagina \textit{HTML\ped{G}} associata al componente;
\item \texttt{+ void render():}\\ metodo ridefinito da \texttt{Backbone.View}, che consente di aggiungere alla pagina \textit{HTML\ped{G}} il \textit{template} campo dati della classe.
\end{itemize}
\end{sloppypar}
\end{itemize}
\end{flushleft}

\paragraph{SendData}
\label{sendData}

\begin{figure}[H] \centering
\includegraphics[trim=0cm 0.8cm 0cm 0cm,clip=true,scale=0.75]%
{./classi/client/presenter/user/SendData.png} \caption{Diagramma classe  \textit{SendData}}
\end{figure}

\begin{flushleft}
\begin{itemize}
\item \textbf{Descrizione:} Classe che ha il compito di gestire l'inserimento e l'invio di dati da parte degli utenti, per completare il passo corrente;
\item \textbf{Relazioni con altri componenti:}
\begin{sloppypar}
La classe comunica con l'interfaccia \texttt{\viewUser{}.I\fshyp{}Send\fshyp{}Da\fshyp{}ta} per creare il \textit{widget} che consente di inviare i dati, utilizza la classe \texttt{\collection{}.Pro\fshyp{}cess\fshyp{}Data\fshyp{}Col\fshyp{}lec\fshyp{}tion} per gestire e ottenere i dati dal \textit{server\ped{G}}, e infine invoca le seguenti classi che gestiscono l'invio di un tipo di dato specifico:
\begin{itemize}
	\item \texttt{\logicUser{}.SendText};
	\item \texttt{\logicUser{}.SendNumb};
	\item \texttt{\logicUser{}.SendImage};
	\item \texttt{\logicUser{}.SendPosition}.
\end{itemize}
\end{sloppypar}
\item \textbf{Attributi:}
\begin{sloppypar}
\begin{itemize}
\item \texttt{+ ProcessDataCollection processdata:}\\ campo dati di tipo \texttt{\collection{}.Pro\fshyp{}cess\fshyp{}Da\fshyp{}ta\fshyp{}Col\fshyp{}lec\fshyp{}tion} che consente di interagire con la lista dei dati inviati dall'utente relativa al processo in gestione presente nel \textit{server\ped{G}};
\item \texttt{+ Object template:}\\ oggetto ridefinito da \texttt{Backbone.View}, che contiene il \textit{template HTML\ped{G}} associato alla classe;
\item \texttt{+ Object el:}\\ oggetto ridefinito da \texttt{Backbone.View} che rappresenta l'elemento \textit{HTML\ped{G}} entro cui la classe ascolta eventi generati dagli utenti;
\item \texttt{+ String id:}\\ campo dati ridefinito da \texttt{Backbone.View} contente l'id della classe;
\end{itemize}
\end{sloppypar}
\item \textbf{Metodi:}
\begin{sloppypar}
\begin{itemize}
\item \texttt{+ void initialize():}\\ metodo ridefinito da \texttt{Backbone.View}, invocato alla costruzione di ciascun oggetto della classe, che consente di aggiungere una pagina \textit{HTML\ped{G}} associata al componente;
\item \texttt{+ void render():}\\ metodo ridefinito da \texttt{Backbone.View}, che consente di aggiungere alla pagina \textit{HTML\ped{G}} il \textit{template} campo dati della classe. Utilizza le classi \texttt{\logicUser{}.SendText} \texttt{\logicUser{}.SendNumb}, \texttt{\logicUser{}.SendImage} e \texttt{\logicUser{}.SendPosition} per renderizzare l'interfaccia relativa all'inserimento dei diversi tipi di dato;
\item \texttt{+ bool getData():}\\ controlla se i dati inseriti dall'utente sono corretti: se lo sono ritorna \texttt{true} e li aggiunge alla collezione \texttt{processData}, altrimenti ritorna \texttt{false};
\item \texttt{+ bool saveData():}\\ utilizza metodi del campo dati \texttt{processData}, per inviare i dati raccolti al \textit{server\ped{G}}.
\end{itemize}
\end{sloppypar}
\end{itemize}
\end{flushleft}



\subsubsection{Package \logicAdmin{}}
\paragraph{EventDispatcher}
\label{ev2}

\begin{figure}[H] \centering
\includegraphics[trim=0cm 0.8cm 0cm 0cm,clip=true,scale=0.75]%
{./classi/client/presenter/po/EventDispatcherB.png} \caption{Diagramma classe  \textit{EventDispatcher}}
\end{figure}

\begin{flushleft}
\begin{itemize}
\item \textbf{Descrizione:} Classe per la gestione della notifica di presenza di passi che richiedono approvazione.
\item \textbf{Relazioni con altri componenti:}
\begin{sloppypar}
La classe estende la classe \texttt{\logic{}.Base\fshyp{}Dis\fshyp{}pat\fshyp{}cher}.
\end{sloppypar}
\item \textbf{Attributi:}
\begin{sloppypar}
\begin{itemize}
	\item \texttt{- int INTERVALMS :} indica l'intervallo di pooling;
  	\item \texttt{+intervalId :} indica l'id dell'intervallo;
\end{itemize}
\end{sloppypar}
\item \textbf{Metodi:}
\begin{sloppypar}
\begin{itemize}  
	\item \texttt{- intervalFunction(collection : Backbone.Collection) : void:}\\ invoca il metodo notify se la collezione contiene dei nuovi passi che richiedono approvazione;
  	\item \texttt{+ startListen() : void :}\\ inizia a monitorare eventuali variazioni nella collezione di passi che richiedono approvazione;
  	\item \texttt{+ stopListen() : void :}\\ interrompe l'esecuzione della funzione periodica con id "intervalId";
  	\item \texttt{+ notify() : void:}\\ notifica gli observer.
\end{itemize}
\end{sloppypar}
\end{itemize}
\end{flushleft}

\paragraph{MainProcessOwner}
\label{mainProcessOwner}

\begin{figure}[H] \centering
\includegraphics[trim=0cm 0.8cm 0cm 0cm,clip=true,scale=0.75]%
{./classi/client/presenter/po/MainProcessOwner.png} \caption{Diagramma classe  \textit{MainProcessOwner}}
\end{figure}

\begin{flushleft}
\begin{itemize}
\item \textbf{Descrizione:} Classe che ha il compito della gestione generale della logica delle funzionalità \textit{Process Owner\ped{G}};
\item \textbf{Relazioni con altri componenti:}
\begin{sloppypar}
La classe comunica con il \textit{template} \texttt{\viewAdmin{}.I\fshyp{}Main\fshyp{}Pro\fshyp{}cess\fshyp{}Ow\fshyp{}ner} per la realizzazione dell'interfaccia grafica, con \texttt{\model{}.UserDataModel} per la gestione della sessione e con \texttt{\collectionp{}.ProcessDataCollection}
\end{sloppypar}
\item \textbf{Attributi:}
\begin{sloppypar}
\begin{itemize}
\item \texttt{+ UserDataModel session :} variabile necessaria per la gestione della sessione;
\item \texttt{+ int waitingDataNumber :} numero di dati attesi;
\item \texttt{+ ProcessDataCollection collection :} variabile necessaria per la gestione dei dati ricevuti dagli utenti riguardanti un processo
\item \texttt{+ Object template:}\\ oggetto ridefinito da \texttt{Backbone.View}, che contiene il \textit{template HTML\ped{G}} associato alla classe;
\item \texttt{+ Object el:}\\ oggetto ridefinito da \texttt{Backbone.View} che rappresenta l'elemento \textit{HTML\ped{G}} entro cui la classe ascolta eventi generati dagli utenti;
\item \texttt{+ String id:}\\ campo dati ridefinito da \texttt{Backbone.View} contente l'id della classe;
\end{itemize}
\end{sloppypar}
\item \textbf{Metodi:}
\begin{sloppypar}
\begin{itemize}
\item \texttt{+ initialize(options : Object) : void}\\ metodo ridefinito da \texttt{Backbone.View}, invocato alla costruzione di ciascun oggetto della classe, che consente di aggiungere una pagina \textit{HTML\ped{G}} associata al componente;
\item \texttt{+ render() : void}\\ metodo ridefinito da \texttt{Backbone.View}, che consente di aggiungere alla pagina \textit{HTML\ped{G}} il \textit{template} campo dati della classe;
\item \texttt{+ notifyWaitingData(collection : Backbone.Collection) : void:}\\ permette la gestione della notifica dell'evento: un passo richiede approvazione;
\item \texttt{+ update() : void:}\\ aggiorna il numero di passi che richiedono approvazione.
\end{itemize}
\end{sloppypar}
\end{itemize}
\end{flushleft}

\paragraph{OpenProcess}
\label{openProcess}

\begin{figure}[H] \centering
\includegraphics[trim=0cm 0.8cm 0cm 0cm,clip=true,scale=0.75]%
{./classi/client/presenter/po/OpenProcess.png} \caption{Diagramma classe  \textit{OpenProcess}}
\end{figure}

\begin{flushleft}
\begin{itemize}
\item \textbf{Descrizione:} Classe che ha il compito di gestire la ricerca e la selezione di un processo;
\item \textbf{Relazioni con altri componenti:}
\begin{sloppypar}
La classe comunica con il \textit{template} \texttt{\viewAdmin{}.I\fshyp{}O\fshyp{}pen\fshyp{}Pro\fshyp{}cess} per la realizzazione dell'interfaccia grafica, e con la classe \texttt{\collection{}Pro\fshyp{}cess\fshyp{}Col\fshyp{}lec\fshyp{}tion} per gestire e ottenere i dati dal \textit{server\ped{G}}.
\end{sloppypar}
\item \textbf{Attributi:}
\begin{sloppypar}
\begin{itemize}
\item \texttt{+ UserDataModel session:}\\ campo dati di tipo \texttt{\logic{}.User\fshyp{}Data\fshyp{}Model} utile per la gestione e recupero dati della sessione;
\item \texttt{+ ProcessCollection collection:}\\ campo dati di tipo \texttt{\collectionp{}.Pro\fshyp{}cess\fshyp{}Col\fshyp{}lec\fshyp{}tion} che contiene la lista dei processi non eliminati dal \textit{process owner\ped{G}};
\item \texttt{+ Object template:}\\ oggetto ridefinito da \texttt{Backbone.View}, che contiene il \textit{template HTML\ped{G}} associato alla classe;
\item \texttt{+ Object el:}\\ oggetto ridefinito da \texttt{Backbone.View} che rappresenta l'elemento \textit{HTML\ped{G}} entro cui la classe ascolta eventi generati dagli utenti;
\item \texttt{+ String id:}\\ campo dati ridefinito da \texttt{Backbone.View} contente l'id della classe;
\end{itemize}
\end{sloppypar}
\item \textbf{Metodi:}
\begin{sloppypar}
\begin{itemize}
\item \texttt{+ initialize(options : Object) : void:}\\ metodo ridefinito da \texttt{Backbone.View}, invocato alla costruzione di ciascun oggetto della classe, che consente di aggiungere una pagina \textit{HTML\ped{G}} associata al componente;
\item \texttt{+ render(options : int) : void:}\\ metodo ridefinito da \texttt{Backbone.View}, che consente di aggiungere alla pagina \textit{HTML\ped{G}} il \textit{template} campo dati della classe;
\item \texttt{+ void update():}\\ aggiorna il campo dati \texttt{collection} comunicando con il \textit{server\ped{G}}.
\end{itemize}
\end{sloppypar}
\end{itemize}
\end{flushleft}

\paragraph{NewProcess}
\label{newProcess}
\begin{figure}[H] \centering
\includegraphics[trim=0cm 0.8cm 0cm 0cm,clip=true,scale=0.75]%
{./classi/client/presenter/po/NewProcess.png} \caption{Diagramma classe  \textit{NewProcess}}
\end{figure}

\begin{flushleft}
\begin{itemize}
\item \textbf{Descrizione:} Classe che ha il compito di gestire la logica della definizione di un nuovo processo;
\item \textbf{Relazioni con altri componenti:}
\begin{sloppypar}
La classe comunica con il \textit{template} \texttt{\viewAdmin{}.I\fshyp{}New\fshyp{}pro\fshyp{}cess} per la realizzazione dell'interfaccia grafica e con la classe \texttt{\logicAdmin{}.Add\fshyp{}Step};
\end{sloppypar}
\item \textbf{Attributi:}
\begin{sloppypar}
\begin{itemize}
\item \texttt{+ ProcessModel process:}\\ campo dati di tipo \texttt{\model{}.Pro\fshyp{}cess\fshyp{}Mo\fshyp{}del} che contiene i dati del processo in definizione;
\item \texttt{+ Object template:}\\ oggetto ridefinito da \texttt{Backbone.View}, che contiene il \textit{template HTML\ped{G}} associato alla classe;
\item \texttt{+ Object el:}\\ oggetto ridefinito da \texttt{Backbone.View} che rappresenta l'elemento \textit{HTML\ped{G}} entro cui la classe ascolta eventi generati dagli utenti;
\item \texttt{+ String id:}\\ campo dati ridefinito da \texttt{Backbone.View} contente l'id della classe;
\item \texttt{+ AddStep addStepLogic:} campo istanza di \texttt{\logicAdmin{}.Add\fshyp{}Step}, necessario per l'aggiunta di passi in NewProcess;
\item \texttt{+Object blocks :} oggetto contenente i blocchi del processo in creazione. Un blocco è un entità per raggruppare logicamente un insieme di passi, e può essere sequenziale o non ordinato;    
\end{itemize}
\end{sloppypar}
\item \textbf{Metodi:}
\begin{sloppypar}
\begin{itemize}
\item \texttt{- getParam(param : String) : String :}\\ ritorna il paramentro get con nome "param" se presente nella url, altrimenti ritorna false;
\item \texttt{- printMessage(title : String, content : String) : void:}\\ apre un popup con titolo "title" e contenuto "content";
\item \texttt{- validateDate(dateInput : String, resultDate : Date) : Object:}\\ controlla la data e ritorna true o un eventuale stringa di descrizione dell'errore;
\item \texttt{- validateTime(timeInput : String, date : Date) : Object:}\\ controlla l'ora e ritorna true o un eventuale stringa di descrizione dell'errore;
\item \texttt{- validateDescription(description : String) : String :}\\ controlla il testo in input e ritorna true o un eventuale stringa di descrizione dell'errore;
\item \texttt{- validateImage(imageFile : Object) : String :}\\controlla l'immagine in input e ritorna true o un eventuale stringa di descrizione dell'errore;
\item \texttt{- saveOptions() : void :}\\ salva le opzioni sui blocchi impostate dall'utente;
\item \texttt{- printBlocksHelp(event : Object) : void :}\\ visualizza il pannello di help relativo all'aggiunta di blocchi;
\item \texttt{- showInput(event : Object) : void :}\\ mostra e rende obbligatori i campi dati selezionati;
\item \texttt{- changeTab(event : Object) : void :}\\ gestione dell'evento di cambio tab;
\item \texttt{- addStep(event : Object) : void :}\\ delega la gestione della creazione di un nuovo passo alla classe AddStep;
\item \texttt{- editStep(event : Object) : void :}\\ delega la gestione della modifica di un passo alla classe AddStep;
\item \texttt{- ascendBlock(event : Object) : void :}\\ gestisce lo spostamento di un blocco ad un livello superiore;
\item \texttt{- descendBlock(event : Object) : void:}\\ gestisce lo spostamento di un blocco ad un livello inferiore;
\item \texttt{- addUnorderedBlock(event : Object) : void :}\\ aggiunge un blocco non ordinato;
\item \texttt{- addSequentialBlock(event : Object) : void :}\\ aggiunge un blocco sequenziale;
\item \texttt{- deleteBlock(event : Object) : void :}\\ rimuove il blocco selezionato;
\item \texttt{- removeStep(event : Object) : void :}\\ rimuove il passo selezionato;
\item \texttt{- sortBlock(event : Object) : void :}\\ gestione del cambio dell'ordine dei passi di un blocco sequenziale;
\item \texttt{- saveDescription(event : Object) : void :}\\ salva la descrizione del processo in creazione;
\item \texttt{- cancelDescription(event : Object) : void:}\\ annulla le modifiche alla descrizione del processo in creazione;
\item \texttt{- parseBlock(event : Object) : void :}\\ rimuove i dati temporanei del blocco e imposta i valori di default;
\item \texttt{- saveProcess(event : Object) : void :}\\ salva il processo creato;
\item \texttt{- cancelProcess(event : Object) : void:}\\ cancella il processo creato;
\item \texttt{+ initialize(options : Object) : void:}\\ metodo ridefinito da \texttt{Backbone.View}, invocato alla costruzione di ciascun oggetto della classe, che consente di aggiungere una pagina \textit{HTML\ped{G}} associata al componente;
\item \texttt{+ render() : void :}\\ metodo ridefinito da \texttt{Backbone.View}, che consente di aggiungere alla pagina \textit{HTML\ped{G}} il \textit{template} campo dati della classe;
\end{itemize}
\end{sloppypar}
\end{itemize}
\end{flushleft}

\paragraph{AddStep}
\label{addStep}

\begin{figure}[H] \centering
\includegraphics[trim=0cm 0.8cm 0cm 0cm,clip=true,scale=0.75]%
{./classi/client/presenter/po/AddStep.png} \caption{Diagramma classe  \textit{AddStep}}
\end{figure}

\begin{flushleft}
\begin{itemize}
\item \textbf{Descrizione:} Classe che ha il compito di gestire la logica di definizione dei passi di un processo;
\item \textbf{Relazioni con altri componenti:}
\begin{sloppypar}
La classe comunica con il \textit{template} \texttt{\viewAdmin{}.I\fshyp{}Add\fshyp{}Step} per la realizzazione dell'interfaccia grafica e utilizza la classe \texttt{\modelUser{}.StepModel} per salvare i dati del passo in creazione.
\end{sloppypar}
\item \textbf{Attributi:}
\begin{sloppypar}
\begin{itemize}
\item \texttt{+ UserDataModel session :}\\ campo dati di tipo \texttt{\model{}.User\fshyp{}Mo\fshyp{}del} che contiene i dati della sessione;
\item \texttt{+ Object template:}\\ oggetto ridefinito da \texttt{Backbone.View}, che contiene il \textit{template HTML\ped{G}} associato alla classe;
\item \texttt{+ Object el:}\\ oggetto ridefinito da \texttt{Backbone.View} che rappresenta l'elemento \textit{HTML\ped{G}} entro cui la classe ascolta eventi generati dagli utenti;
\item \texttt{+ String id:}\\ campo dati ridefinito da \texttt{Backbone.View} contente l'id della classe;
\item \texttt{+ Object events:}\\ oggetto ridefinito da \texttt{Backbone.View} che associa ad ogni evento generato dagli utenti nella pagina \textit{HTML\ped{G}}, un metodo della classe;
\item \texttt{+ Object blocks:}\\ array contenente i blocchi del processo in creazione. Un blocco è un entità per raggruppare logicamente un insieme di passi, e può essere sequenziale o non ordinato;
\end{itemize}
\end{sloppypar}
\item \textbf{Metodi:}
\begin{sloppypar}
\begin{itemize}
\item \texttt{- printMessage(title : String, content : String): void :}\\ metodo invocato per l'apertura di un popup con titolo "title" e contenuto "content";
  \item \texttt{- showInput(event : Object):void :}\\ metodo per visualizare e rendere obbligatori i campi dati selezionati;
  \item \texttt{- changeCostraints(event : Object):void :}\\ cambia il vincolo di obbligatorietà dei dati geografici;
  \item \texttt{- validateDescription(description : String): String :}\\ controlla la validità della descrizione description e ritorna una stringa in caso di descrizione non valida;
  \item \texttt{- updateTextualData(event : Object, empty : boolean):void :}\\ aggiorna la lista dei dati testuali;
  \item \texttt{- updateImageData(event : Object, empty : boolean):void :}\\ aggiorna la lista dei dati di tipo immagine;
  \item \texttt{- updateNumericData(event : Object, empty : boolean):void :}\\ aggiorna la lista dei dati numerici;
  \item \texttt{- deleteData(event : Object):void :}\\ rimuove il dato selezionato;
  \item \texttt{- getData() : Object:}\\ restituisce i dati e i vincoli inseriti dall'utente relativi al passo in creazione;
  \item \texttt{- getGeographicData() : Object:}\\ restituisce vincoli geografici inseriti dall'utente;
  \item \texttt{- getTextualData(index : String):Object :}\\ restituisce la lista dei dati testuali inseriti dall'utente;
  \item \texttt{- getImageData(index : String):Object :}\\ restituisce la lista dei dati di tipo immagine inseriti dall'utente;
  \item \texttt{- getNumericData(index : String):Object :}\\ restituisce la lista dei dati numerici inseriti dall'utente;
  \item \texttt{- cancelStep(event : Object):void :}\\ annulla la modifica/creazione del passo;
  \item \texttt{- saveStep(event : Object):void :}\\ salva il passo se i dati inseriti dall'utente rispettano i vincoli;
\item \texttt{+ initialize(options : Object):void :}\\ metodo ridefinito da \texttt{Backbone.View}, invocato alla costruzione di ciascun oggetto della classe, che consente di aggiungere una pagina \textit{HTML\ped{G}} associata al componente;
\item \texttt{+ render(options : Object):void :}\\ metodo ridefinito da \texttt{Backbone.View}, che consente di aggiungere alla pagina \textit{HTML\ped{G}} il \textit{template} campo dati della classe;
\item \texttt{+ update(blockId : String, stepId : String):void :}\\ metodo che gestisce la richiesta di creazione e/o modifica di un passo.
\end{itemize}
\end{sloppypar}
\end{itemize}
\end{flushleft}

\paragraph{ManageProcess}
\label{manageProcess}
\begin{figure}[H] \centering
\includegraphics[trim=0cm 0.8cm 0cm 0cm,clip=true,scale=0.75]%
{./classi/client/presenter/po/ManageProcess.png} \caption{Diagramma classe  \textit{ManageProcess}}
\end{figure}
\begin{flushleft}
\begin{itemize}
\item \textbf{Descrizione:} Classe che ha il compito di gestire e accedere alle informazioni relative allo stato dei processi e ai dati inviati dagli utenti. Le operazioni di gestione dello stato comprendono la terminazione e l'eliminazione di un processo;
\item \textbf{Relazioni con altri componenti:}
\begin{sloppypar}
La classe comunica con il \textit{template} \texttt{\viewAdmin{}.I\fshyp{}Ma\fshyp{}na\fshyp{}ge\fshyp{}Pro\fshyp{}cess} per la realizzazione dell'interfaccia grafica, e con le classi \texttt{ProcessData} e \texttt{\modelAdmin{}.Pro\fshyp{}cess\fshyp{}Mo\fshyp{}del} per gestire e ottenere i dati dal \textit{server\ped{G}}.
\end{sloppypar}
\item \textbf{Attributi:}
\begin{sloppypar}
\begin{itemize}
\item \texttt{+ UserDataModel session:}\\ campo dati di tipo \texttt{\model{}.User\fshyp{}Data\fshyp{}Mo\fshyp{}del} che permette la gestione della sessione;
\item \texttt{+ ProcessModel process:}\\ campo dati di tipo \texttt{\modelAdmin{}.Pro\fshyp{}cess\fshyp{}Mo\fshyp{}del} che contiene i dati del processo in gestione;
\item \texttt{+ ProcessData processdata:}\\ campo dati di tipo ProcessData necessario per invocare l'omonimo \texttt{widget\ped{G}};
\item \texttt{+ Object template:}\\ oggetto ridefinito da \texttt{Backbone.View}, che contiene il \textit{template HTML\ped{G}} associato alla classe;
\item \texttt{+ Object el:}\\ oggetto ridefinito da \texttt{Backbone.View} che rappresenta l'elemento \textit{HTML\ped{G}} entro cui la classe ascolta eventi generati dagli utenti;
\item \texttt{+ String id:}\\ campo dati ridefinito da \texttt{Backbone.View} contente l'id della classe;
\item \texttt{+ Object events:}\\ oggetto che contiene tutti gli eventi input che vengono gestiti dalla suddetta classe;
\end{itemize}
\end{sloppypar}
\item \textbf{Metodi:}
\begin{sloppypar}
\begin{itemize}
\item \texttt{+ initialize(options : Object) : void:}\\ metodo ridefinito da \texttt{Backbone.View}, invocato alla costruzione di ciascun oggetto della classe, che consente di aggiungere una pagina \textit{HTML\ped{G}} associata al componente;
\item \texttt{+ render(option : Object, error : Object) : void:}\\ metodo ridefinito da \texttt{Backbone.View}, che consente di aggiungere alla pagina \textit{HTML\ped{G}} il \textit{template} campo dati della classe;
\item \texttt{+ void update():}\\ aggiorna i dati della pagina recuperandoli dal server;
\item \texttt{+ eliminateProcess() : void :}\\ permette la gestione della richiesta di eliminazione di un processo terminato dalla lista dei processi gestibili dal process owner;
\item \texttt{+ terminateProcess(event : Object) : void :}\\ permette gestione della richiesta di terminazione di un processo.
\item \texttt{- getParam(param : String) : String :}\\ ritorna il paramentro get con nome "param" se presente nella url, altrimenti ritorna false;
\item \texttt{- printMessage(title : String, content : String) : void}\\ apre un popup con titolo "title" e contenuto "content";
\item \texttt{- changeTab(event : Object) : void :}\\ gestione dell'evento di cambio tab;
\item \texttt{- activeLink(event : Object) : void:}\\ gestione della navigazione tra pagine tramite link contenuti all'interno di un tab;
\end{itemize}
\end{sloppypar}
\end{itemize}
\end{flushleft}

\paragraph{CheckStep}
\label{checkStep}

\begin{figure}[H] \centering
\includegraphics[trim=0cm 0.8cm 0cm 0cm,clip=true,scale=0.75]%
{./classi/client/presenter/po/CheckStep.png} \caption{Diagramma classe  \textit{CheckStep}}
\end{figure}

\begin{flushleft}
\begin{itemize}
\item \textbf{Descrizione:} Classe che ha il compito di definire la logica del controllo di un passo che richiede intervento umano per essere approvato;
\item \textbf{Relazioni con altri componenti:}
\begin{sloppypar}
La classe comunica con il \textit{template} \texttt{\viewAdmin{}.I\fshyp{}Check\fshyp{}Step} per la realizzazione dell'interfaccia grafica, e con le classi \texttt{\collectionp{}.Pro\fshyp{}cess\fshyp{}Da\fshyp{}ta\fshyp{}Col\fshyp{}lec\fshyp{}tion} e \texttt{\modelAdmin{}.Pro\fshyp{}cess\fshyp{}Mo\fshyp{}del} per gestire e ottenere i dati dal \textit{server\ped{G}}.
\end{sloppypar}
\item \textbf{Attributi:}
\begin{sloppypar}
\begin{itemize}
\item \texttt{+ UserDataModel session :} variabile di tipo \texttt{\model{}.User\fshyp{}Data\fshyp{}Model} necessaria per la gestione della sessione;
\item \texttt{+ ProcessCollection processes :} \texttt{\collectionp{}.Pro\fshyp{}cess\fshyp{}Col\fshyp{}lec\fshyp{}tion} necessaria per la gestione dei dati riguardanti la collezione di processi accessibili all'utente \texttt{process owner};
\item \texttt{+ StepCollection steps :}\texttt{\collectionp{}.\fshyp{}Step\fshyp{}Col\fshyp{}lec\fshyp{}tion} necessari per la gestione dei dati riguardanti la collezione dei passi di un processo
\item \texttt{+ Object approveDataTemplate :} template per l'approvazione dei dati;
\item \texttt{+ Object checkStepTemplate : } template per il controllo dei passi;
\item \texttt{+ ProcessDataCollection collection:}\\ campo dati di tipo \texttt{\collectionp{}.Pro\fshyp{}cess\fshyp{}Da\fshyp{}ta\fshyp{}Col\fshyp{}lec\fshyp{}tion} che contiene i dati inviati dagli utenti in attesa di approvazione;
\item \texttt{+ Object el:}\\ oggetto ridefinito da \texttt{Backbone.View} che rappresenta l'elemento \textit{HTML\ped{G}} entro cui la classe ascolta eventi generati dagli utenti;
\item \texttt{+ String id:}\\ campo dati ridefinito da \texttt{Backbone.View} contente l'id della classe;
\end{itemize}
\end{sloppypar}
\item \textbf{Metodi:}
\begin{sloppypar}
\begin{itemize}
\item \texttt{- getParam(param : String) :}\\ ritorna il valore del parametro \textit{param} se presente nella \textit{URL\ped{G}};
  \item \texttt{- printMessage(title : String, content : String) :}\\ apre un popup con titolo "title" e contenuto "content";
  \item \texttt{- updateStepData() :}\\ recupera le informazioni sui passi relativi ai dati che richidono intervento umano;
  \item \texttt{- updateProcessData() :}\\ recupera le informazioni sui processi relativi ai dati che richiedono intervento umano;
\item \texttt{+ initialize(options : Object) :}\\ metodo ridefinito da \texttt{Backbone.View}, invocato alla costruzione di ciascun oggetto della classe, che consente di aggiungere una pagina \textit{HTML\ped{G}} associata al componente;
\item \texttt{+ render(options : Object, error : Object) : void}\\ metodo ridefinito da \texttt{Backbone.View}, che consente di aggiungere alla pagina \textit{HTML\ped{G}} il \textit{template} campo dati della classe;
\item \texttt{update() :}\\ aggiorna la collezione dei dati che richiedono approvazione;
    \item \texttt{+ notifyWaitingData(collection : Backbone.Collection):}\\ permette la gestione dell'evento: "waitingDataNumber" passi richiedono approvazione;
    \item \texttt{+ approveData() :}\\ permette la gestione della richiesta di approvazione dei dati di un passo;
    \item \texttt{+ rejectData() :}\\ permette la gestione della richiesta di disapprovazione dei dati di un passo;
\end{itemize}
\end{sloppypar}
\end{itemize}
\end{flushleft}

\paragraph{ProcessData}
\label{pd}

\begin{figure}[H] \centering
\includegraphics[trim=0cm 0.8cm 0cm 0cm,clip=true,scale=0.75]%
{./classi/client/presenter/po/ProcessData.png} \caption{Diagramma classe  \textit{ProcessData}}
\end{figure}

\begin{flushleft}
\begin{itemize}
\item \textbf{Descrizione:} Classe che ha il compito di gestione della visualizzazione dei dati inviati dagli utenti riguardanti un processo creato
\item \textbf{Relazioni con altri componenti:}
\begin{sloppypar}
La classe comunica con il \textit{template} \texttt{\viewAdmin{}.I\fshyp{}Check\fshyp{}Step} per la realizzazione dell'interfaccia grafica, e con le classi \texttt{\collectionp{}.Pro\fshyp{}cess\fshyp{}Da\fshyp{}ta\fshyp{}Col\fshyp{}lec\fshyp{}tion} e \texttt{\modelAdmin{}.Pro\fshyp{}cess\fshyp{}Mo\fshyp{}del} per gestire e ottenere i dati dal \textit{server\ped{G}}.
\end{sloppypar}
\item \textbf{Attributi:}
\begin{sloppypar}
\begin{itemize}
\item \texttt{+ UserDataModel session :} variabile di tipo \texttt{\model{}.User\fshyp{}Data\fshyp{}Model} necessaria per la gestione della sessione;
\item \texttt{+ Object approveDataTemplate :} template per la renderizzazione del widget;
\item \texttt{+ ProcessDataCollection collection:}\\ campo dati di tipo \texttt{\collectionp{}.Pro\fshyp{}cess\fshyp{}Da\fshyp{}ta\fshyp{}Col\fshyp{}lec\fshyp{}tion} che contiene i dati inviati dagli utenti in attesa di approvazione;
\item \texttt{+ Object el:}\\ oggetto ridefinito da \texttt{Backbone.View} che rappresenta l'elemento \textit{HTML\ped{G}} entro cui la classe ascolta eventi generati dagli utenti;
\item \texttt{+ String id:}\\ campo dati ridefinito da \texttt{Backbone.View} contente l'id della classe;
\end{itemize}
\end{sloppypar}
\item \textbf{Metodi:}
\begin{sloppypar}
\begin{itemize}
\item \texttt{- updateStepData(stepId : String, options : Object) : void :}\\ aggiorna la pagina con i dati inviati dagli utenti riguardanti il passo "stepId";
  \item \texttt{- updateUserData(processId : String, username : String, options : Object) : void :}\\ aggiorna la pagina con i dati inviati dall'utente "usernamee" riguardanti il processo "processId";
\item \texttt{+ initialize(options : Object) : void) :}\\ metodo ridefinito da \texttt{Backbone.View}, invocato alla costruzione di ciascun oggetto della classe, che consente di aggiungere una pagina \textit{HTML\ped{G}} associata al componente;
\item \texttt{+ render(options : Object, error : Object) : void}\\ metodo ridefinito da \texttt{Backbone.View}, che consente di aggiungere alla pagina \textit{HTML\ped{G}} il \textit{template} campo dati della classe;
\item \texttt{+ update(param : String, options : Object) : void :}\\ aggiorna i dati della collezione "collection" recuperandoli dal server; di un passo;
\end{itemize}
\end{sloppypar}
\end{itemize}
\end{flushleft}