\subsection{Server}
Questa componente è incaricata di gestire la comunicazione con il client e di elaborarne le richieste restituendo i dati richiesti e quando necessario interroga la componente model per ottenere i dati dal database.
Tale componente è composta dalle classi:
\begin{itemize}
	\item \texttt{com.sirius.sequenziatore.server.presenter.common.SignUpController}
	\item \texttt{com.sirius.sequenziatore.server.presenter.common.LoginController}
	\item \texttt{com.sirius.sequenziatore.server.presenter.common.StepInfoController}
	\item \texttt{com.sirius.sequenziatore.server.presenter.common.ProcessInfoController}
	\item \texttt{com.sirius.sequenziatore.server.presenter.processowner.StepController}
	\item \texttt{com.sirius.sequenziatore.server.presenter.processowner.ProcessController}
	\item \texttt{com.sirius.sequenziatore.server.presenter.processowner.ApproveStepController}
	\item \texttt{com.sirius.sequenziatore.server.presenter.user.AccountController}
	\item \texttt{com.sirius.sequenziatore.server.presenter.user.UserStepController}
	\item \texttt{com.sirius.sequenziatore.server.presenter.user.UserProcessController}
	\item \texttt{com.sirius.sequenziatore.server.presenter.user.ReportController}
\end{itemize}
Nella prossime sessioni verranno trattate in dettaglio le seguenti classi dividendo l' esposizione per \textit{package}, si evidenzia come la voce mappatura base sia l' estensione della mappatura su cui si programma il sistema che sarà \textit{localhost:8080/sequenziatore/} , quindi tutte le mappature base saranno da considerarsi come aggiunte a seguito di /sequenziatore/ e successivamente le varie varianti dei metodi.
Tutte le classi \textit{controller} del \textit{presenter} dovranno essere marcate come \textit{@Controller} per essere riconosciute in modo corretto da \textit{Spring}.
\subsubsection{Package com.sirius.sequenziatore.server.presenter.common}
\begin{figure}[H] \centering
\
\includegraphics[trim=0cm 0.8cm 0cm 0cm,clip=true,scale=0.75]%
{./classi/server/presentercommon.png} \caption{Diagramma package - \texttt{com.sirius.sequenziatore.server.presenter.common}}
\end{figure}
All' interno di questa sezione verranno trattate tutte le classi contenute nel package \textit{common}.
\paragraph{SignUpController}%--------------------------------------------------------%
\
\begin{figure}[H] \centering
\includegraphics[trim=0cm 0.8cm 0cm 0cm,clip=true,scale=0.75]%
{./classi/server/signupcontroller.png} \caption{Diagramma classe - \texttt{SignUpController}}
\end{figure}
\begin{itemize}
	\item \textbf{Descrizione: } Questa classe dovrà gestire tutte le richieste di registrazione al sistema, sarà incaricata di inserire i dati nel database e di avvertire il client della riuscita della registrazione.
	\item \textbf{Mappatura base: }\textit{\textbackslash signup}
	\item \textbf{Relazioni con altri componenti: }
	La classe utilizzerà le seguenti classi:
	\begin{itemize}
		\item \texttt{com.sirius.sequenziatore.server.model.User;}
		\item \texttt{com.sirius.sequenziatore.server.model.UserDao;}
	\end{itemize}
	tramite le interfacce:
	\begin{itemize}
		\item \texttt{com.sirius.sequenziatore.server.model.ITransferObject;}
		\item \texttt{\sModel .IDataAccessObject;}
	\end{itemize}
	\item \textbf{Metodi: }\begin{itemize}
					\item \texttt{+registerUser(User toBeRegistered)}:\\
					 questo metodo gestirà un metodo \textbf{POST} e restituirà dovrà lanciare un' eccezione di tipo \texttt{HttpError} qual' ora ci siano stati problemi nella registrazione;
				\end{itemize}
\end{itemize}
\paragraph{LoginController}%--------------------------------------------------------------------%
\
\begin{figure}[H] \centering
\includegraphics[trim=0cm 0.8cm 0cm 0cm,clip=true,scale=0.75]%
{./classi/server/logincontroller.png} \caption{Diagramma classe - \texttt{LoginController}}
\end{figure}
\begin{itemize}
	\item \textbf{Descrizione: }Questa classe gestirà le richieste di \textit{log in}, dovrà controllare se l' utente esiste nel sistema e se le credenziali d' accesso sono corrette restituendo il tipo di utente altrimenti un errore se l' utente non esiste nel sistema;
	\item \textbf{Mappatura base: }\textit{\textbackslash login}
	\item \textbf{Relazioni con altri componenti: }
	La classe utilizzerà le seguenti classi:
	\begin{itemize}
		\item \texttt{com.sirius.sequenziatore.server.model.User;}
		\item \texttt{com.sirius.sequenziatore.server.model.UserDao;}
	\end{itemize}
	tramite le interfacce:
	\begin{itemize}
		\item \texttt{com.sirius.sequenziatore.server.model.ITransferObject;}
		\item \texttt{\sModel .IDataAccessObject;}
	\end{itemize}
	\item \textbf{Metodi: }\begin{itemize}
					\item \texttt{+checkLogin(User toBeLogged)}:\\
					 questo metodo gestirà un metodo di tipo \textbf{POST} , controllerà le credenziali di accesso e dovrà lanciare un' eccezione di tipo \texttt{HttpError} qualora ci siano stati problemi nella login;
				\end{itemize}
\end{itemize}
\paragraph{StepInfoController}%--------------------------------------------------------------------%
\begin{figure}[H] \centering
\
\includegraphics[trim=0cm 0.8cm 0cm 0cm,clip=true,scale=0.75]%
{./classi/server/stepinfocontroller.png} \caption{Diagramma classe - \texttt{StepInfoController}}
\end{figure}
\begin{itemize}
	\item \textbf{Descrizione: }Questa classe restituirà lo scheletro, quindi la composizione del passo richiesto;
	\item \textbf{Mappatura base: }\textit{\textbackslash step\textbackslash \{id\}}
	\item \textbf{Relazioni con altri componenti: }
	La classe utilizzerà le seguenti classi:
	\begin{itemize}
		\item \texttt{com.sirius.sequenziatore.server.model.Step;}
		\item \texttt{com.sirius.sequenziatore.server.model.StepDao;}
	\end{itemize}
	tramite le interfacce:
	\begin{itemize}
		\item \texttt{com.sirius.sequenziatore.server.model.ITransferObject;}
		\item \texttt{\sModel .IDataAccessObject;}
	\end{itemize}
	\item \textbf{Metodi: }\begin{itemize}
					\item \texttt{+Step getStepInformation()}:\\
					 il metodo gestisce una richiesta di tipo \textbf{GET} restituendo la struttura del passo con id uguale all' id fornito dopo averla recuperata dal \textit{database};
				\end{itemize}
\end{itemize}
\paragraph{ProcessInfoController}%----------------------------------------------------------------%
\begin{figure}[H] \centering
\includegraphics[trim=0cm 0.8cm 0cm 0cm,clip=true,scale=0.75]%
{./classi/server/processinfocontroller.png} \caption{Diagramma classe - \texttt{ProcessInfoController}}
\end{figure}
\begin{itemize}
	\item \textbf{Descrizione: } Questa classe dovrà restituire a chi lo richiede un processo dato l' \textit{id} con i suoi dati;
	\item \textbf{Mappatura base: }\textit{\textbackslash process\textbackslash \{id\}}
	\item \textbf{Relazioni con altri componenti: }
	La classe utilizzerà le seguenti classi:
	\begin{itemize}
		\item \texttt{com.sirius.sequenziatore.server.model.Process;}
		\item \texttt{com.sirius.sequenziatore.server.model.ProcessDao;}
	\end{itemize}
	tramite le interfacce:
	\begin{itemize}
		\item \texttt{com.sirius.sequenziatore.server.model.ITransferObject;}
		\item \texttt{\sModel .IDataAccessObject;}
	\end{itemize}
	\item \textbf{Metodi: }\begin{itemize}
					\item \texttt{+Process getProcessInformation()}:\\
					il metodo gestisce una richiesta di tipo \textbf{GET} e restituisce la struttura di un processo con id processo richiesto;
				\end{itemize}
\end{itemize}