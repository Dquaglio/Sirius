\subsection{Server}
Questa componente è incaricata di gestire la comunicazione con il client e di elaborarne le richieste restituendo i dati richiesti e quando necessario interroga la componente model per ottenere i dati dal database.
Tale componente è composta dalle classi:
\begin{itemize}
	\item \texttt{com.sirius.sequenziatore.server.presenter.common.SignUpController}
	\item \texttt{com.sirius.sequenziatore.server.presenter.common.LoginController}
	\item \texttt{com.sirius.sequenziatore.server.presenter.common.StepInfoController}
	\item \texttt{com.sirius.sequenziatore.server.presenter.common.ProcessInfoController}
	\item \texttt{com.sirius.sequenziatore.server.presenter.processowner.StepController}
	\item \texttt{com.sirius.sequenziatore.server.presenter.processowner.ProcessController}
	\item \texttt{com.sirius.sequenziatore.server.presenter.processowner.ApproveStepController}
	\item \texttt{com.sirius.sequenziatore.server.presenter.user.AccountController}
	\item \texttt{com.sirius.sequenziatore.server.presenter.user.UserStepController}
	\item \texttt{com.sirius.sequenziatore.server.presenter.user.UserProcessController}
	\item \texttt{com.sirius.sequenziatore.server.presenter.user.ReportController}
\end{itemize}
Nella prossime sessioni verranno trattate in dettaglio le seguenti classi dividendo l' esposizione per \textit{package}, si evidenzia come la voce mappatura base sia l' estensione della mappatura su cui si programma il sistema che sarà \textit{localhost:8080/sequenziatore/} , quindi tutte le mappature base saranno da considerarsi come aggiunte a seguito di /sequenziatore/ e successivamente le varie varianti dei metodi.
Tutte le classi \textit{controller} del \textit{presenter} dovranno essere marcate come \textit{@Controller} per essere riconosciute in modo corretto da Spring.
\subsubsection{Package com.sirius.sequenziatore.server.presenter.common}
\textbf{IMMAGINE DEL PACKAGE}
All' interno di questa sezione verranno trattate tutte le classi contenute nel package \textit{common}.
\paragraph{Classe SignUpController}%--------------------------------------------------------%
\begin{itemize}
	\item \textbf{Descrizione: } Questa classe dovrà gestire tutte le richieste di registrazione al sistema, sarà incaricata di inserire i dati nel database e di avvertire il client della riuscita della registrazione.
	\item \textbf{Mappatura base: }\textit{\textbackslash signup}
	\item \textbf{Attributi: }
	\item \textbf{Metodi: }\begin{itemize}
					\item +RegisterUser(Utente toBeRegistered) , questo metodo gestirà un metodo \textbf{POST} e restituirà un errore qual' ora ci siano stati problemi nella registrazione;
				\end{itemize}
\end{itemize}
\paragraph{LoginController}%--------------------------------------------------------------------%
\begin{itemize}
	\item \textbf{Descrizione: }Questa classe gestirà le richieste di \textit{log in}, dovrà controllare se l' utente esiste nel sistema e se le credenziali d' accesso siano corrette;
	\item \textbf{Mappatura base: }\textit{\textbackslash login}
	\item \textbf{Attributi: }
	\item \textbf{Metodi: }\begin{itemize}
					\item +CheckLogin(User toBeLogged) , questo metodo gestirà un metodo di tipo \textbf{POST} , controllerà le credenziali di accesso e dovrà restituire un errore qual' ora ci siano stati problemi nella login;
				\end{itemize}
\end{itemize}
\paragraph{StepInfoController}%--------------------------------------------------------------------%
\begin{itemize}
	\item \textbf{Descrizione: }Questa classe restituirà lo scheletro, quindi la composizione del passo richiesto;
	\item \textbf{Mappatura base: }\textit{\textbackslash step\textbackslash \{id\}}
	\item \textbf{Attributi: }
	\item \textbf{Metodi: }\begin{itemize}
					\item +Step GetStepInformation() il metodo gestisce una richiesta di tipo \textbf{GET} restituendo la struttura del passo con id uguale all' id fornito dopo averla recuperata dal \textit{database};
				\end{itemize}
\end{itemize}
\paragraph{ProcessInfoController}%----------------------------------------------------------------%
\begin{itemize}
	\item \textbf{Descrizione: } Questa classe dovrà restituire a chi lo richiede un processo dato l' \textit{id} con i suoi dati;
	\item \textbf{Mappatura base: }\textit{\textbackslash process\textbackslash \{id\}}
	\item \textbf{Attributi: }
	\item \textbf{Metodi: }\begin{itemize}
					\item 
				\end{itemize}
\end{itemize}