\subsection{Server}
Questa componente è incaricata di gestire la comunicazione con il client e di elaborarne le richieste restituendo i dati richiesti e quando necessario interroga la componente model per ottenere i dati dal database.
Tale componente è composta dalle classi:
\begin{itemize}
	\item com.sirius.sequenziatore.server.presenter.common.SignUpController
	\item com.sirius.sequenziatore.server.presenter.common.LoginController
	\item com.sirius.sequenziatore.server.presenter.common.StepInfoController
	\item com.sirius.sequenziatore.server.presenter.common.ProcessInfoController
	\item com.sirius.sequenziatore.server.presenter.processowner.StepController
	\item com.sirius.sequenziatore.server.presenter.processowner.ProcessController
	\item com.sirius.sequenziatore.server.presenter.processowner.ManageUserStepController
	\item com.sirius.sequenziatore.server.presenter.user.AccountController
	\item com.sirius.sequenziatore.server.presenter.user.UserStepController
	\item com.sirius.sequenziatore.server.presenter.user.UserProcessController
	\item com.sirius.sequenziatore.server.presenter.user.ReportController
\end{itemize}
Nella prossime sessioni verranno trattate in dettaglio le seguenti classi dividendo l' esposizione per \textit{package}.
\subsubsection{Package com.sirius.sequenziatore.server.presenter.common}
IMMAGINE DEL PACKAGE
All' interno di questa sezione verranno trattate tutte le classi contenute nel package \textit{common}.
\paragraph{Classe SignUpController}
\begin{description}
	\item[Descrizione]
	\item[Attributi]
	\item[Metodi]
\end{description}
\paragraph{LoginController}

\paragraph{StepInfoController}

\paragraph{ProcessInfoController}