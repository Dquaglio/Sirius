\subsubsection{Package com.sirius.sequenziatore.server.presenter.user}
\textbf{IMMAGINE PACKAGE}
\paragraph{UserStepController}%----------------------------------------------------------------%
\begin{itemize}
	\item \textbf{Descrizione: } Questa classe gestisce la ricezione di un passo di un utente tramite una richiesta di tipo \textit{POST}, tale passo dovrà essere inserito nel database, ponendo attenzione se è un passo che richiede approvazione o meno;
	\item \textbf{Mappatura base: } \textit{\textbackslash stepdata\textbackslash user}
	\item \textbf{Relazione con altri componenti: }
	\item \textbf{Attributi: } 
	\item \textbf{Metodi: }\begin{itemize}
					\item 
				\end{itemize}
\end{itemize}
\paragraph{UserProcessController}%----------------------------------------------------------------%
\begin{itemize}
	\item \textbf{Descrizione: } Questa classe permette all' utente varie operazioni, innanzitutto l' iscrizione ad un processo, poi restituisce il passo a cui è arrivato e il suo stato per tale processo e infine fornisce una lista di processi con tutti i processi a cui si può iscrivere e i processi per i quali può chiedere di fare il \textit{report};
	\item \textbf{Mappatura base: } \textit{\textbackslash user\textbackslash \{username\}}
	\item \textbf{Attributi: }
	\item \textbf{Metodi: }\begin{itemize}
					\item \texttt{+void ProcessSubscribe()} questo metodo mappa su \textbackslash subscribe\textbackslash \{processid\} e gestisce una richiesta di tipo \textit{POST} che permette ad un utente di iscriversi al processo voluto;
					\item \texttt{+Status GetProcessStatus()} questo metodo mappa su \textbackslash subscribe\textbackslash \{processid\} e gestisce una richiesta \textit{GET} che restituisce all' utente il proprio status per tale processo, restituendo il passo o i passi che può eseguire e quanti passi ha completato del processo;
					\item \texttt{+ProcessList GetListProcess()} questo processo mappa su \textbackslash processlist e gestisce una richiesta di tipo \textit{GET} andando e restituire una lista di processi che contiene tutti i processi  a cui è iscritto e quelli a cui si può iscrivere;
				\end{itemize}
\end{itemize}
\paragraph{AccountController}%----------------------------------------------------------------%
\begin{itemize}
	\item \textbf{Descrizione: } Classe che fornisce i dati di un utente e ne permette la modifica dei suddetti;
	\item \textbf{Mappatura base: } \textit{\textbackslash account\textbackslash \{username\}}
	\item \textbf{Attributi: } 
	\item \textbf{Metodi: }\begin{itemize}
					\item \texttt{+User GetUserData():}\\
					questo metodo gestisce una richiesta di tipo \textit{GET} e restituisce un oggetto di tipo User contenente tutti i dati di un utente;
				\end{itemize}
\end{itemize}
\paragraph{ReportController}%----------------------------------------------------------------%
\begin{itemize}
	\item \textbf{Descrizione: } 
	\item \textbf{Mappatura base: } \textit{\textbackslash process\textbackslash \{id\}}
	\item \textbf{Attributi: }
	\item \textbf{Metodi: }\begin{itemize}
					\item 
				\end{itemize}
\end{itemize}