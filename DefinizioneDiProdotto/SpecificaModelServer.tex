\subsection{Server}

Il \textit{model} lato \textit{server} gestisce la persistenza dei dati all'interno del \textit{database} consentendo interrogazione, inserimento, cancellazione e aggiornamento.

La componente è formata dalle seguenti \textit{classi}:
\begin{itemize}
	\item \hyperref[idataacessobject]{\smodel{}.IDataAcessObject};
	\item \hyperref[itransferobject]{\smodel{}.ITransferObject};
	\item \hyperref[userdao]{\smodel{}.UserDao};
	\item \hyperref[processdao]{\smodel{}.ProcessDao};
	\item \hyperref[processownerdao]{\smodel{}.ProcessOwnerDao};
	\item \hyperref[stepdao]{\smodel{}.StepDao};
\end{itemize}

\subsubsection{Package \model{}}

\paragraph{IDataAcessObject}
\label{idataacessobject}
\begin{flushleft}
\begin{itemize}
\item \textbf{Descrizione:} Interfaccia che permette di gestire la comunicazione e l'interrogazione con il \textit{database}.
\item \textbf{Metodi:}
\begin{sloppypar}
\begin{itemize}
\item \texttt{+ void setJdbcTemplate(JdbcTemplate jdbcTemplate):}\\ Imposta i parametri per l'accesso alla sorgente dei dati;
\item \texttt{+ ITransferObject getAll():}\\ Ritorna tutti i dati di competenza della classe che estende questa interfaccia.
\end{itemize}
\end{sloppypar}
\end{itemize}
\end{flushleft}

\paragraph{ITransferObject}
\label{itransferobject}
\begin{flushleft}
\begin{itemize}
\item \textbf{Descrizione:} Interfaccia realizzata dai tipi che modellano i dati del \textit{database}.
\end{itemize}
\end{flushleft}

\paragraph{UserDao}
\label{userdao}
\begin{flushleft}
\begin{itemize}
\item \textbf{Descrizione:} Classe che si occupa delle interrogazioni del \textit{database} relative agli utenti del sistema.
\item \textbf{Relazione con altre componenti:} la classe implementa la seguente interfaccia:
		\begin{itemize}
			\item \smodel{}.IDataAcessObject.
		\end{itemize}
		La classe invoca i metodi della classe:
		\begin{itemize}
			\item \smodel{}.User.
		\end{itemize}
\item \textbf{Attributi:}
\begin{sloppypar}
\begin{itemize}
\item \texttt{JdbcTemplate jdbcTemplate:}\\ Oggetto che fornisce l'accesso alla sorgente dei dati;
\end{itemize}
\end{sloppypar}
\item \textbf{Metodi:}
\begin{sloppypar}
\begin{itemize}
\item \texttt{+ User getUser(String userName):}\\ Ritorna l’utente con il nome utente specificato; 
\item \texttt{+ List<User> getAllUser():}\\ Ritorna tutti gli utenti;
\item \texttt{+ boolean insertUser(User user) :}\\ Aggiunge l'utente passato come parametro;
\item \texttt{+ public boolean updateUser(User user) :}\\ Aggiorna i dati dell'utente con il nome utente corrispondente a quello dell'utente passato, con i dati dell'utente passato.
\end{itemize}
\end{sloppypar}
\end{itemize}
\end{flushleft}

\paragraph{ProcessDao}
\label{processdao}
\begin{flushleft}
\begin{itemize}
\item \textbf{Descrizione:} Classe che si occupa delle interrogazioni del \textit{database} relative ai processi.
\item \textbf{Relazione con altre componenti:} la classe implementa la seguente interfaccia:
		\begin{itemize}
			\item \smodel{}.IDataAcessObject.
		\end{itemize}
		La classe invoca i metodi della classe:
		\begin{itemize}
			\item \smodel{}.Process.
		\end{itemize}
\item \textbf{Attributi:}
\begin{sloppypar}
\begin{itemize}
\item \texttt{JdbcTemplate jdbcTemplate:}\\ Oggetto che fornisce l'accesso alla sorgente dei dati;
\end{itemize}
\end{sloppypar}
\item \textbf{Metodi:}
\begin{sloppypar}
\begin{itemize}
\item \texttt{+ Process getProcess(int id)):}\\ Ritorna il processo con l'\texttt{id} specificato; 
\item \texttt{+ List<Process> getAllProcess():}\\ Ritorna tutti i processi;
\item \texttt{+ boolean insertProcess(Process process) :}\\ Aggiunge il processo passato come parametro;
\item \texttt{+ public boolean updateProcess(Process process) :}\\ Aggiorna i dati del processo con lo stesso \texttt{id} di quello del processo passato, con i dati del processo passato.
\end{itemize}
\end{sloppypar}
\end{itemize}
\end{flushleft}

\paragraph{ProcessOwnerDao}
\label{processownerdao}
\begin{flushleft}
\begin{itemize}
\item \textbf{Descrizione:} Classe che si occupa delle interrogazioni del \textit{database} relative all'autenticazione del \textit{ProcessOwner}.
\item \textbf{Relazione con altre componenti:} la classe implementa la seguente interfaccia:
		\begin{itemize}
			\item \smodel{}.IDataAcessObject.
		\end{itemize}
		La classe invoca i metodi della classe:
		\begin{itemize}
			\item \smodel{}.ProcessOwner.
		\end{itemize}
\item \textbf{Attributi:}
\begin{sloppypar}
\begin{itemize}
\item \texttt{JdbcTemplate jdbcTemplate:}\\ Oggetto che fornisce l'accesso alla sorgente dei dati;
\end{itemize}
\end{sloppypar}
\item \textbf{Metodi:}
\begin{sloppypar}
\begin{itemize}
\item \texttt{+ Process getProcessOwner(int id)):}\\ Ritorna l'oggetto rappresentante il \textit{ProcessOwner}. 
\end{itemize}
\end{sloppypar}
\end{itemize}
\end{flushleft}

\paragraph{StepDao}
\label{stepdao}
\begin{flushleft}
\begin{itemize}
\item \textbf{Descrizione:} Classe che si occupa delle interrogazioni del \textit{database} relative a tutte le operazioni sui passi dei processi.
\item \textbf{Relazione con altre componenti:} la classe implementa la seguente interfaccia:
		\begin{itemize}
			\item \smodel{}.IDataAcessObject.
		\end{itemize}
		La classe invoca i metodi della classe:
		\begin{itemize}
			\item \smodel{}.Step;
			\item \smodel{}.UserStep;
			\item \smodel{}.DataSent.
		\end{itemize}
\item \textbf{Attributi:}
\begin{sloppypar}
\begin{itemize}
\item \texttt{JdbcTemplate jdbcTemplate:}\\ Oggetto che fornisce l'accesso alla sorgente dei dati;
\end{itemize}
\end{sloppypar}
\item \textbf{Metodi:}
\begin{sloppypar}
\begin{itemize}
\item \texttt{+ Step getStep(int id):}\\ Ritorna il passo con l'\texttt{id} specificato; 
\item \texttt{+ List<Step> getAllStep():}\\ Ritorna tutti i passi;
\item \texttt{+ List<Step> getStepOf(int ProcessId):}\\ Ritorna tutti i passi appartenenti al processo di cui si è passato l'\texttt{id};
\item \texttt{+ boolean insertStep(Step step) :}\\ Aggiunge il passo passato come parametro;
\item \texttt{+ public boolean updateStep(Step step) :}\\ Aggiorna i dati del passo con l'\texttt{id} corrispondente a quello del passo passato, con i dati del passo passato;
\item \texttt{+ List<UserStep> userStep(String userName)}\\ Ritorna una lista di oggetti informativi sullo stato dei passi in corso da parte dell'utente di cui si è passato il nome utente;
\item \texttt{+ List<UserStep> userProcessStep(String userName, processId)}\\ Ritorna una lista di oggetti informativi sullo stato dei passi in corso appartenenti al processo di cui si è passato l'\texttt{id} da parte dell'utente di cui si è passato il nome utente;
\item \texttt{+ List<DataSent> getData(Step step)}\\ Ritorna tutti i dati da tutti gli utenti relativi al passo passato;
\item \texttt{+ List<DataSent> getData(String userName, Step step)}\\ Ritorna tutti i dati inviati dall'utente di cui si è passato il nome utente relativi al passo passato;
\item \texttt{+ boolean completeStep(String userName, Step step, List<DataSent> data, Step next)}\\Notifica e aggiorna nel \textit{database} lo stato dell'utente quando completa o tenta di completare un passo.
\end{itemize}
\end{sloppypar}
\end{itemize}
\end{flushleft}
