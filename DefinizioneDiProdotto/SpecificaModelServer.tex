\subsection{Server}

Il \textit{model} lato \textit{server} gestisce la persistenza dei dati all'interno del \textit{database} consentendo interrogazione, inserimento, cancellazione e aggiornamento.

La componente è formata dalle seguenti \textit{classi}:
\begin{itemize}
	\item \hyperref[idataacessobject]{\smodel{}.IDataAcessObject};
	\item \hyperref[itransferobject]{\smodel{}.ITransferObject};
	\item \hyperref[userdao]{\smodel{}.UserDao};
	\item \hyperref[processdao]{\smodel{}.ProcessDao};
	\item \hyperref[processownerdao]{\smodel{}.ProcessOwnerDao};
	\item \hyperref[stepdao]{\smodel{}.StepDao};
	\item \hyperref[botuser]{\smodel{}.User};
	\item \hyperref[botprocess]{\smodel{}.Process};
	\item \hyperref[botblock]{\smodel{}.Block};
	\item \hyperref[botstep]{\smodel{}.Step};
	\item \hyperref[botdatasent]{\smodel{}.DataSent};
	\item \hyperref[botidatavalue]{\smodel{}.IDataValue};
	\item \hyperref[bottextualvalue]{\smodel{}.TextualValue};
	\item \hyperref[botnumericvalue]{\smodel{}.NumericValue};
	\item \hyperref[botimagevalue]{\smodel{}.ImageValue};
	\item \hyperref[botgeographicvalue]{\smodel{}.GeographicValue};
	\item \hyperref[botuserstep]{\smodel{}.UserStep};
	\item \hyperref[botpowner]{\smodel{}.ProcessOwner};
	\item \hyperref[bottextualdata]{\smodel{}.TextualData};
	\item \hyperref[botnumericdata]{\smodel{}.NumericData};
\end{itemize}

\subsubsection{Package \model{}}

\paragraph{IDataAcessObject}\
\label{idataacessobject}
\begin{figure}[H] \centering
\includegraphics[trim=0cm 0.8cm 0cm 0cm,clip=true,scale=0.75]%
{./classi/server/model/IDataAccessObject.png} \caption{Diagramma interfaccia IDataAcessObject}
\end{figure}
\begin{flushleft}
\begin{itemize}
\item \textbf{Descrizione:} Interfaccia che permette di gestire la comunicazione e l'interrogazione con il \textit{database}.
\item \textbf{Metodi:}
\begin{sloppypar}
\begin{itemize}
\item \texttt{+ void setJdbcTemplate(JdbcTemplate jdbcTemplate):}\\ Imposta i parametri per l'accesso alla sorgente dei dati;
\item \texttt{+ ITransferObject getAll():}\\ Ritorna tutti i dati di competenza della classe che estende questa interfaccia.
\end{itemize}
\end{sloppypar}
\end{itemize}
\end{flushleft}

\paragraph{ITransferObject}
\label{itransferobject}
\begin{flushleft}
\begin{itemize}
\item \textbf{Descrizione:} Interfaccia realizzata dai tipi che modellano i dati del \textit{database}.
\end{itemize}
\end{flushleft}

\paragraph{UserDao}\
\label{userdao}
\begin{figure}[H] \centering
\includegraphics[trim=0cm 0.8cm 0cm 0cm,clip=true,scale=0.75]%
{./classi/server/model/UserDao.png} \caption{Diagramma classe UserDao}
\end{figure}
\begin{flushleft}
\begin{itemize}
\item \textbf{Descrizione:} Classe che si occupa delle interrogazioni del \textit{database} relative agli utenti del sistema.
\item \textbf{Relazione con altre componenti:} la classe implementa la seguente interfaccia:
		\begin{itemize}
			\item \smodel{}.IDataAcessObject.
		\end{itemize}
		La classe invoca i metodi della classe:
		\begin{itemize}
			\item \smodel{}.User.
		\end{itemize}\item \textbf{Attributi:}
\begin{sloppypar}
\begin{itemize}
\item \texttt{JdbcTemplate jdbcTemplate:}\\ Oggetto che fornisce l'accesso alla sorgente dei dati;
\end{itemize}
\end{sloppypar}
\item \textbf{Metodi:}
\begin{sloppypar}
\begin{itemize}
\item \texttt{+ setJdbcTemplate(JdbcTemplate jdbcTemplate):}\\ Imposta i parametri per l'accesso alla sorgente dei dati;
\item \texttt{+ List<ITransferObject> getAll():}\\ Ritorna tutti gli utenti come ITransferObject; 
\item \texttt{+ User getUser(String userName):}\\ Ritorna l’utente con il nome utente specificato; 
\item \texttt{+ List<User> getAllUser():}\\ Ritorna tutti gli utenti;
\item \texttt{+ boolean insertUser(User user) :}\\ Aggiunge l'utente passato come parametro;
\item \texttt{+ boolean updateUser(User user) :}\\ Aggiorna i dati dell'utente con il nome utente corrispondente a quello dell'utente passato, con i dati dell'utente passato.
\item \texttt{+ List<User> getUserByProcess(int idProcess):}\\ Ritorna la lista degli utenti iscritti al  processo con l'id specificato.
\end{itemize}
\end{sloppypar}
\end{itemize}
\end{flushleft}

\paragraph{ProcessDao}\
\label{processdao}
\begin{figure}[H] \centering
\includegraphics[trim=0cm 0.8cm 0cm 0cm,clip=true,scale=0.75]%
{./classi/server/model/ProcessDao.png} \caption{Diagramma classe ProcessDao}
\end{figure}
\begin{flushleft}
\begin{itemize}
\item \textbf{Descrizione:} Classe che si occupa delle interrogazioni del \textit{database} relative ai processi.
\item \textbf{Relazione con altre componenti:} la classe implementa la seguente interfaccia:
		\begin{itemize}
			\item \smodel{}.IDataAcessObject.
		\end{itemize}
		La classe invoca i metodi della classe:
		\begin{itemize}
			\item \smodel{}.Process;
			\item \smodel{}.Block;
			\item \smodel{}.Step;
			\item \smodel{}.TextualData;
			\item \smodel{}.NumericData;
			\item \smodel{}.GeographicData;
			\item \smodel{}.ImageData.
		\end{itemize}\item \textbf{Attributi:}
\begin{sloppypar}
\begin{itemize}
\item \texttt{- JdbcTemplate jdbcTemplate:}\\ Oggetto che fornisce l'accesso alla sorgente dei dati;
\end{itemize}
\end{sloppypar}
\item \textbf{Metodi:}
\begin{sloppypar}
\begin{itemize}
\item \texttt{+ setJdbcTemplate(JdbcTemplate jdbcTemplate):}\\ Imposta i parametri per l'accesso alla sorgente dei dati;
\item \texttt{+ List<ITransferObject> getAll():}\\ Ritorna tutti i processi come ITransferObject; 
\item \texttt{+ Process getProcess(int id)):}\\ Ritorna il processo con l'\texttt{id} specificato; 
\item \texttt{+ List<Process> getAllProcess():}\\ Ritorna tutti i processi;
\item \texttt{+ boolean insertProcess(Process process, List<Block> blocks):}\\ Aggiunge il processo passato come parametro inserendovi i blocchi di passi passati come parametri.
\item \texttt{+ public boolean updateProcess(Process process) :}\\ Aggiorna i dati del processo con lo stesso \texttt{id} di quello del processo passato, con i dati del processo passato;
\item \texttt{+ boolean deleteProcess(Process process):}\\ Pianifica l'eliminazione di un processo passato, il quale viene rimosso definitivamente qualora nessun utente via sia più iscritto;
\item \texttt{+ List<Process> getNotEliminated():}\\ Restituisce la lista dei processi per i quali non è ancora stata pianificata l'eliminazione;
\item \texttt{+ List<Process> getProcesses(String username): }\\ Restituisce la lista di processi a cui è iscritto l'utente con il nome specificato;
\item \texttt{+ List<Process> getSubscribableProcesses(String username):}\\ Restituisce la lista dei processi a cui può iscriversi l'utente con il nome specificato;
\item \texttt{+ boolean subscribe(String username, int processId):}\\ Iscrive l'utente con il nome specificato al processo con l'\texttt{id} specificato;
\item \texttt{+boolean unsubscribe(String username, int processId):}\\ Disiscrive l'utente con il nome specificato dal processo con l'\texttt{id} specificato;
\end{itemize}
\end{sloppypar}
\end{itemize}
\end{flushleft}

\paragraph{ProcessOwnerDao}\
\label{processownerdao}
\begin{figure}[H] \centering
\includegraphics[trim=0cm 0.8cm 0cm 0cm,clip=true,scale=0.75]%
{./classi/server/model/ProcessOwnerDao.png} \caption{Diagramma classe ProcessOwnerDao}
\end{figure}
\begin{flushleft}
\begin{itemize}
\item \textbf{Descrizione:} Classe che si occupa delle interrogazioni del \textit{database} relative all'autenticazione del \textit{ProcessOwner}.
\item \textbf{Relazione con altre componenti:} la classe implementa la seguente interfaccia:
		\begin{itemize}
			\item \smodel{}.IDataAcessObject.
		\end{itemize}
		La classe invoca i metodi della classe:
		\begin{itemize}
			\item \smodel{}.ProcessOwner.
		\end{itemize}
\item \textbf{Attributi:}
\begin{sloppypar}
\begin{itemize}
\item \texttt{- JdbcTemplate jdbcTemplate:}\\ Oggetto che fornisce l'accesso alla sorgente dei dati;
\end{itemize}
\end{sloppypar}
\item \textbf{Metodi:}
\begin{sloppypar}
\begin{itemize}
\item \texttt{+ setJdbcTemplate(JdbcTemplate jdbcTemplate):}\\ Imposta i parametri per l'accesso alla sorgente dei dati;
\item \texttt{+ List<ITransferObject> getAll():}\\ Ritorna tutto come ITransferObject; 
\item \texttt{+ Process getProcessOwner():}\\ Ritorna l'oggetto rappresentante il \textit{ProcessOwner}. 
\end{itemize}
\end{sloppypar}
\end{itemize}
\end{flushleft}

\paragraph{StepDao}\
\label{stepdao}
\begin{figure}[H] \centering
\includegraphics[trim=0cm 0.8cm 0cm 0cm,clip=true,scale=0.75]%
{./classi/server/model/StepDao.png} \caption{Diagramma classe StepDao}
\end{figure}
\begin{flushleft}
\begin{itemize}
\item \textbf{Descrizione:} Classe che si occupa delle interrogazioni del \textit{database} relative a tutte le operazioni sui passi dei processi.
\item \textbf{Relazione con altre componenti:} la classe implementa la seguente interfaccia:
		\begin{itemize}
			\item \smodel{}.IDataAcessObject.
		\end{itemize}
		La classe invoca i metodi della classe:
		\begin{itemize}
			\item \smodel{}.Step;
			\item \smodel{}.UserStep;
			\item \smodel{}.DataSent.
			\item \smodel{}.Block;
			\item \smodel{}.TextualValue;
			\item \smodel{}.NumericValue;
			\item \smodel{}.GeographicValue;
			\item \smodel{}.ImageValue.			
		\end{itemize}\item \textbf{Attributi:}
\begin{sloppypar}
\begin{itemize}
\item \texttt{- JdbcTemplate jdbcTemplate:}\\ Oggetto che fornisce l'accesso alla sorgente dei dati;
\end{itemize}
\end{sloppypar}
\item \textbf{Metodi:}
\begin{sloppypar}
\begin{itemize}
\item \texttt{+ setJdbcTemplate(JdbcTemplate jdbcTemplate):}\\ Imposta i parametri per l'accesso alla sorgente dei dati;
\item \texttt{+ List<ITransferObject> getAll():}\\ Ritorna tutti i processi come ITransferObject; 
\item \texttt{+ Step getStep(int id):}\\ Ritorna il passo con l'\texttt{id} specificato; 
\item \texttt{+ List<Step> getAllStep():}\\ Ritorna tutti i passi;
\item \texttt{+ List<Step> getStepOf(int ProcessId):}\\ Ritorna tutti i passi appartenenti al processo di cui si è passato l'\texttt{id};
\item \texttt{+ List<Step> getStespOf(int processId):} \\Ritorna la lista dei passi appartenenti al processo con l'\texttt{id} specificato;
\item \texttt{+ List<UserStep> userSteps(String userName)}\\ Ritorna una lista di oggetti informativi sullo stato dei passi in corso da parte dell'utente di cui si è passato il nome utente;
\item \texttt{+ List<UserStep> userProcessStep(String userName, processId)}\\ Ritorna una lista di oggetti informativi sullo stato dei passi in corso appartenenti al processo di cui si è passato l'\texttt{id} da parte dell'utente di cui si è passato il nome utente;
\item \texttt{+ List<UserStep> getApprovedOrRejected(String username)}\\ Ritorna una lista di oggetti informativi relativi ai passi richiedenti approvazioni che sono stati rifiutati o accettati dal \texttt{Process Owner} per l'utente specificato;
\item \texttt{+ boolean updateUserStep(UserStep userStep):}\\ Aggiornato lo stato del passo per l'utente in questione.
\item \texttt{+ List<DataSent> getData(int stepId):}\\ Ritorna tutti i dati da tutti gli utenti relativi al passo passato con l'\texttt{id} passato;
\item \texttt{+ DataSent getData(String userName, int stepId):}\\ Ritorna tutti i dati inviati dall'utente di cui si è passato il nome utente relativi al passo con l'\texttt{id} passato;
\item \texttt{+ List<DataSent> getProcessData(String username, int processId):} Ritorna tutti i dati inviati dall'utente con il nome specificato per il processo con l'\texttt{id} specificato;
\item \texttt{+ List<DataSent> getWaitingData():}\\ Ritorna tutti i dati di tutti i passi in attesa di approvazione;
\item \texttt{+ boolean delete(String username, int processId):}\\ Elimina tutte le informazioni su passi in corso dell'utente con il nome specificato appartenenti al processo con l'\texttt{id} specificato;
\item \texttt{+ boolean deleteUserStep(UserStep userStep):}\\ Elimina l'oggetto informativo sui passi in corso passato come parametro;
\item \texttt{+ boolean completeStep(String userName, Step step, DataSent data, Step next):}\\ Notifica e aggiorna nel \textit{database} lo stato dell'utente quando completa o tenta di completare un passo;
\item \texttt{- boolean nextBlock(String username, int actualBlockId):}\\ Metodo accessorio utilizzato dal  metodo precedente per l'avanzamento tra blocchi di passi.
\end{itemize}
\end{sloppypar}
\end{itemize}
\end{flushleft}

\paragraph{User}\
\label{botuser}
\begin{figure}[H] \centering
\includegraphics[trim=0cm 0.8cm 0cm 0cm,clip=true,scale=0.75]%
{./classi/server/model/User.png} \caption{Diagramma classe User}
\end{figure}
\begin{flushleft}
\begin{itemize}
\item \textbf{Descrizione:} Classe che modella gli utenti del sistema e che funge da interscambio dei dati di quest'ultimi con il \textit{database}.
\item \textbf{Relazione con altre componenti:} la classe implementa la seguente interfaccia:
		\begin{itemize}
			\item \smodel{}.ITransferObject.
		\end{itemize}
\item \textbf{Attributi:}
\begin{sloppypar}
\begin{itemize}
\item \texttt{- String username:}\\ Nome utente;
\item \texttt{- String password:}\\ Password dell'utente;
\item \texttt{- String name:}\\Nome anagrafico dell'utente;
\item \texttt{- String surName:}\\Cognome dell'utente;
\item \texttt{- Date dateOfBirth:}\\Data di nascita dell'utente;
\item \texttt{- String email:}\\Indirizzo di posta elettronica dell'utente.
\end{itemize}
\end{sloppypar}
\item \textbf{Metodi:}
\begin{sloppypar}
\begin{itemize}
\item \texttt{+ String getUsername():}\\ Ritorna il nome utente;
\item \texttt{+ void setUsername(String username):}\\ Imposta il nome utente;
\item \texttt{+ String getPassword():}\\ Ritorna la password dell'utente;
\item \texttt{+ void setPassword(String password):}\\ Imposta la password dell'utente;
\item \texttt{+ String getName():}\\ Ritorna il nome anagrafico dell'utente;
\item \texttt{+ void setName(String name):}\\ Imposta il nome anagrafico dell'utente;
\item \texttt{+ String getSurName():}\\ Ritorna il cognome dell'utente;
\item \texttt{+ void setSurName(String surName):}\\ Imposta il cognome dell'utente;
\item \texttt{+ Date getDateOfBirth():}\\ Ritorna la data di nascita dell'utente;
\item \texttt{+ void setDateOfBirth(Date dateOfBirth):}\\ Imposta la data di nascita dell'utente;
\item \texttt{+ String getEmail():}\\ Ritorna l'indirizzo di posta elettronica dell'utente;
\item \texttt{+ void setEmail(String email):}\\ Imposta l'indirizzo di posta elettronica dell'utente.
\end{itemize}
\end{sloppypar}
\end{itemize}
\end{flushleft}

\paragraph{Process}\
\label{botprocess}
\begin{figure}[H] \centering
\includegraphics[trim=0cm 0.8cm 0cm 0cm,clip=true,scale=0.75]%
{./classi/server/model/Process.png} \caption{Diagramma classe Process}
\end{figure}
\begin{flushleft}
\begin{itemize}
\item \textbf{Descrizione:} Classe che modella i processi del sistema e che funge da interscambio dei dati di quest'ultimi con il \textit{database}.
\item \textbf{Relazione con altre componenti:} la classe implementa la seguente interfaccia:
		\begin{itemize}
			\item \smodel{}.ITransferObject.
		\end{itemize}
\item \textbf{Attributi:}
\begin{sloppypar}
\begin{itemize}
\item \texttt{- String name:}\\ Nome del processo;
\item \texttt{- String description:}\\ Descrizione del processo;
\item \texttt{- int completionsMax:}\\ Numero massimo di completamenti del processo;
\item \texttt{- Date dateOfTermination:}\\ Data di terminazione del processo;
\item \texttt{- boolean terminated:}\\ Booleano vero quando il processo è terminato;
\item \texttt{- boolean eliminated:}\\ Booleano vero se pianificata l'eliminazione del processo;
\item \texttt{- String imageUrl:} \\Indirizzo dell'immagine associata al processo; 
\item \texttt{- int id:}\\ Codice identificativo \texttt{id} associato al processo.
\end{itemize}
\end{sloppypar}
\item \textbf{Metodi:}
\begin{sloppypar}
\begin{itemize}
\item \texttt{+ String getName():}\\ Ritorna il nome del processo;
\item \texttt{+ void setName(String name):}\\ Imposta il nome del processo;
\item \texttt{+ String getDescription():}\\ Ritorna la descrizione del processo;
\item \texttt{+ void setDescription(String description):}\\ Imposta la descrizione del processo;
\item \texttt{+ int getCompletionsMax():}\\ Restituisce il numero massimo di completamenti del processo;
\item \texttt{+ void setCompletionsMax(int completionsMax):}\\ Imposta il numero massimo di completamenti del processo;
\item \texttt{+ Date getDateOfTermination():}\\ Ritorna data di terminazione del processo;
\item \texttt{+ void setDateOfTermination(Date dateOfTermination):}\\ Imposta la data di terminazione del processo;
\item \texttt{+ boolean isTerminated():}\\ Ritorna vero se il processo è terminato;
\item \texttt{+ void setTerminated(boolean terminated):}\\ Imposta vero se il processo è terminato;
\item \texttt{+ boolean isEliminated():}\\ Ritorna vero se è pianificata l'eliminazione;
\item \texttt{+ void setEliminated():} \\Pianifica l'eliminazione del progetto;
\item \texttt{+ int getId():}\\ Ritorna codice identificativo \texttt{id} associato al processo;
\item \texttt{+ void setId(int id):}\\ Imposta codice identificativo \texttt{id} associato al processo.
\end{itemize}
\end{sloppypar}
\end{itemize}
\end{flushleft}

\paragraph{Block}\
\label{botblock}
\begin{figure}[H] \centering
\includegraphics[trim=0cm 0.8cm 0cm 0cm,clip=true,scale=0.75]%
{./classi/server/model/Block.png} \caption{Diagramma classe Block}
\end{figure}
\begin{flushleft}
\begin{itemize}
\item \textbf{Relazione con altre componenti:} la classe implementa la seguente interfaccia:
		\begin{itemize}
			\item \smodel{}.ITransferObject.
		\end{itemize}
		La classe contiene istanze di:
		\begin{itemize}
			\item \smodel{}.Step.
		\end{itemize}
\item \textbf{Attributi:}
\begin{sloppypar}
\begin{itemize}
\item \texttt{+ public enum BlockTypes{SEQUENTIAL, UNORDERED}:}\\ Enumerazione relativa al tipo di blocco;
\item \texttt{- int id:}\\ Codice identificativo \texttt{id} associato al blocco;
\item \texttt{- BlockTypes type:}\\ Tipo del blocco;
\item \texttt{- int requiredSteps:}\\ Numero minimo di passi richiesti se il blocco è non ordinato;
\item \texttt{- boolean first:}\\ Vero se il blocco è il primo del processo;
\item \texttt{- int nextBlockId:}\\ Codice identificativo \texttt{id} dell'eventuale blocco successivo;
\item \texttt{- List<Step> steps:}\\ Lista dei passi del blocco.
\end{itemize}
\end{sloppypar}
\item \textbf{Metodi:}
\begin{sloppypar}
\begin{itemize}
\item \texttt{+ int getId():}\\ Ritorna il codice identificativo \texttt{id} associato al blocco;
\item \texttt{+ void setId(int id):}\\ Imposta il codice identificativo \texttt{id} associato al blocco;
\item \texttt{+ BlockTypes getType():}\\ Ritorna il tipo del blocco;
\item \texttt{+ void setType(BlockTypes type):}\\ Imposta il tipo del blocco;
\item \texttt{+ int getRequiredSteps():}\\ Ritorna il numero minimo di passi richiesti;
\item \texttt{+ void setRequiredSteps(int requiredSteps):}\\ Imposta il numero minimo di passi richiesti;
\item \texttt{+ int getNextBlockId():}\\ Ritorna il codice identificativo \texttt{id} dell'eventuale blocco successivo;
\item \texttt{+ void setNextBlockId(int nextBlockId):}\\ Imposta il codice identificativo \texttt{id} dell'eventuale blocco successivo;
\item \texttt{+ List<Step> getSteps():}\\ Ritorna la lista dei passi del blocco;
\item \texttt{+ void setSteps(List<Step> steps):}\\ Imposta la lista dei passi del blocco;
\item \texttt{+ boolean isFirst():}\\ Ritorna vero se il blocco è il primo del processo;
\item \texttt{+ void setFirst(boolean first):}\\ Imposta vero se il blocco è il primo del processo;
\end{itemize}
\end{sloppypar}
\end{itemize}
\end{flushleft}

\paragraph{Step}\
\label{botstep}
\begin{figure}[H] \centering
\includegraphics[trim=0cm 0.8cm 0cm 0cm,clip=true,scale=0.75]%
{./classi/server/model/Step.png} \caption{Diagramma classe Step}
\end{figure}
\begin{flushleft}
\begin{itemize}
\item \textbf{Relazione con altre componenti:} la classe implementa la seguente interfaccia:
		\begin{itemize}
			\item \smodel{}.ITransferObject.
		\end{itemize}
		La classe contiene istanze di:
		\begin{itemize}
			\item \smodel{}.TextualData;
			\item \smodel{}.NumericData;
			\item \smodel{}.GeograpgicData;
			\item \smodel{}.ImageData.
		\end{itemize}
\item \textbf{Attributi:}
\begin{sloppypar}
\begin{itemize}
\item \texttt{- int id:}\\ Codice identificativo \texttt{id} associato al passo;
\item \texttt{- boolean first:}\\ Booleano vero se il passo è primo per il processo padre.
\item \texttt{- String description:}\\ Descrizione del passo;
\item \texttt{- int nextStepId:}\\ Codice identificativo \texttt{id} dell'eventuale passo successivo;
\item \texttt{- boolean requiresApproval:}\\ Booleano vero se il passo richiede l'approvazione del \texttt{Process Owner};
\item \texttt{- boolean optional:}\\ Booleano vero se il passo può essere saltato;
\item \texttt{- int processId:}\\ Codice identificativo \texttt{id} del processo a cui appartiene il passo;
\item \texttt{- List<NumericData> numericData:}\\ Lista di eventuali dati numerici richiesti per il superamento del passo;
\item \texttt{- List<TextualData> textualData:}\\ Lista di eventuali dati testuali richiesti per il superamento del passo;
\item \texttt{- List<ImageData> imageData:}\\ Lista di eventuali immagini richieste per il superamento del passo;
\item \texttt{- GeographicData geographicData:}\\ Eventuale dato geografico richiesto per il superamento del passo.
\end{itemize}
\end{sloppypar}
\item \textbf{Metodi:}
\begin{sloppypar}
\begin{itemize}
\item \texttt{+ int getId():}\\ Ritorna codice identificativo \texttt{id} associato al passo;
\item \texttt{+ void setId(int id):}\\ Imposta codice identificativo \texttt{id} associato al passo;
\item \texttt{+ boolean isFirst():}\\ Ritorna vero se il passo è primo per il processo padre;
\item \texttt{+ void setFirst():}\\ Imposta vero se il passo è primo per il processo padre;
\item \texttt{+ String getDescription():}\\ Ritorna descrizione del passo;
\item \texttt{+ void setDescription(String description):}\\ Imposta descrizione del passo;
\item \texttt{+ int getNextStepId():}\\ Ritorna il codice identificativo \texttt{id} dell'eventuale passo successivo;
\item \texttt{+ void setNextStepId(int nextStepId):}\\ Imposta il codice identificativo \texttt{id} dell'eventuale passo successivo;
\item \texttt{+ boolean requiresApproval():}\\ Ritorna vero se il passo richiede approvazione;
\item \texttt{+ void setRequiresApproval(boolean requiresApproval):}\\ Impostare a vero se il passo richiede approvazione;
\item \texttt{+ boolean isOptional():}\\ Ritorna vero se il passo è saltabile;
\item \texttt{+ void setOptional(boolean optional):}\\ Impostare a vero se il passo è saltabile;
\item \texttt{+ int getProcessId():}\\ Ritorna il codice identificativo \texttt{id} del processo di cui il passo fa parte;
\item \texttt{+ void setProcessId(int processId):}\\ Imposta il codice identificativo \texttt{id} del processo di cui il passo fa parte;
\item \texttt{+ List<NumericData> getNumericData():}\\ Ritorna la lista degli eventuali dati numerici richiesti per superare il passo;
\item \texttt{+ void setNumericData(List<NumericData> numericData):}\\ Imposta la lista degli eventuali dati numerici richiesti per superare il passo;
\item \texttt{+ List<TextualData> getTextualData():}\\ Ritorna la lista degli eventuali dati testuali richiesti per superare il passo;
\item \texttt{+ void setTextualData(List<TextualData> textualData):}\\ Imposta la lista degli eventuali dati testuali richiesti per superare il passo;
\item \texttt{+ List<ImageData> getImageData():}\\ Ritorna la lista delle eventuali immagini richieste per superare il passo;
\item \texttt{+ void setImageData(List<ImageData> imageData):}\\ Imposta la lista delle eventuali immagini richieste per superare il passo;
\item \texttt{+ GeographicData getRequiredPosition():}\\ Ritorna l'eventuale dato geografico richiesto per il superamento del passo;
\item \texttt{+ void setRequiredPosition(GeographicData requiredPosition):}\\ Imposta l'eventuale dato geografico richiesto per il superamento del passo;
\item \texttt{+ boolean isFirst():}\\ Ritorna vero se il passo è il primo di un blocco sequenziale;
\item \texttt{+ void setFirst(boolean first):}\\ Impostare a vero se il passo è il primo di un blocco sequenziale.
\end{itemize}
\end{sloppypar}
\end{itemize}
\end{flushleft}

\paragraph{DataSent}\
\label{botdatasent}
\begin{figure}[H] \centering
\includegraphics[trim=0cm 0.8cm 0cm 0cm,clip=true,scale=0.75]%
{./classi/server/model/DataSent.png} \caption{Diagramma classe DataSent}
\end{figure}
\begin{flushleft}
\begin{itemize}
\item \textbf{Descrizione:} Classe che modella i dati ricevuti dagli utenti che funge da interscambio  con il \textit{database}.
\item \textbf{Relazione con altre componenti:} la classe implementa la seguente interfaccia:
		\begin{itemize}
			\item \smodel{}.ITransferObject.
		\end{itemize}
				La classe contiene instanze della classe:
		\begin{itemize}
			\item \smodel{}.IDataValue.
		\end{itemize}
\item \textbf{Attributi:}
\begin{sloppypar}
\begin{itemize}
\item \texttt{- String username:}\\ Nome utente dell'utente che ha inviato il dato;
\item \texttt{- List<IDataValue> values:}\\ Oggetti con il valori dei dati;
\item \texttt{- Date sentTime:}\\ Data e ora di consegna dei dati;
\item \texttt{- int stepId:}\\ Codice \texttt{id} del passo richiedente il dato.
\end{itemize}
\end{sloppypar}
\item \textbf{Metodi:}
\begin{sloppypar}
\begin{itemize}
\item \texttt{+ String getUsername():}\\ Ritorna nome utente dell'utente che ha inviato il dato;
\item \texttt{+ void setUsername(String username):}\\ Imposta nome utente dell'utente che ha inviato il dato;
\item \texttt{+ List<IDataValue> getValues():}\\ Ritorna lista di oggetti con il valori dei dati;
\item \texttt{+ void setValues(List<IDataValue> values):}\\ Imposta lista di oggetti con il valori dei dati;
\item \texttt{+ Date getSentTime():}\\ Ritorna data e ora di consegna dei dati;
\item \texttt{+ void setSentTime(Date sentTime):}\\ Imposta data e ora di consegna dei dati;
\item \texttt{+ int getStepId():}\\ Ritorna codice \texttt{id} del passo richiedente il dato;
\item \texttt{+ void setStepId(int stepId):}\\ Imposta codice \texttt{id} del passo richiedente il dato.
\end{itemize}
\end{sloppypar}
\end{itemize}
\end{flushleft}

\paragraph{IDataValue}\
\label{botidatavalue}
\begin{figure}[H] \centering
\includegraphics[trim=0cm 0.8cm 0cm 0cm,clip=true,scale=0.75]%
{./classi/server/model/IDataValue.png} \caption{Diagramma interfaccia IDataValue}
\end{figure}
\begin{flushleft}
\begin{itemize}
\item \textbf{Descrizione:} Interfaccia che modella i valori dei dati ricevuti.
\item \textbf{Attributi:}
\begin{sloppypar}
\begin{itemize}
\item \texttt{+ enum DataTypes{NUMERIC, TEXTUAL, IMAGE, GEOGRAPHIC}:}\\ Enumerazione riguardante il tipo di dato;
\end{itemize}
\end{sloppypar}
\item \textbf{Metodi:}
\begin{sloppypar}
\begin{itemize}
\item \texttt{+ int getDataId():}\\ Ritorna codice \texttt{id} associato al valore;
\item \texttt{+ void setDataId(int id):}\\ Imposta codice \texttt{id} associato al valore.
\item \texttt{+ DataTypes getType():}\\ Ritorna il tipo del valore;
\item \texttt{+ void setType(DataTypes type):}\\ Imposta il tipo del valore.
\end{itemize}
\end{sloppypar}
\end{itemize}
\end{flushleft}

\paragraph{TextualValue}\
\label{bottextualvalue}
\begin{figure}[H] \centering
\includegraphics[trim=0cm 0.8cm 0cm 0cm,clip=true,scale=0.75]%
{./classi/server/model/TextualValue.png} \caption{Diagramma classe TextualValue}
\end{figure}
\begin{flushleft}
\begin{itemize}
\item \textbf{Descrizione:} Classe che modella i valori dei dati testuali.
\item \textbf{Relazione con altre componenti:} la classe implementa la seguente interfaccia:
		\begin{itemize}
			\item \smodel{}.IDataValue.
		\end{itemize}
\item \textbf{Attributi:}
\begin{sloppypar}
\begin{itemize}
\item \texttt{- int dataId:}\\ Codice \texttt{id} associato al valore;
\item \texttt{- DataTypes type;}\\ Tipo del valore;
\item \texttt{- String value:}\\ Valore testuale.
\end{itemize}
\end{sloppypar}
\item \textbf{Metodi:}
\begin{sloppypar}
\begin{itemize}
\item \texttt{+ String getValue():}\\ Ritorna valore testuale;
\item \texttt{+ void setValue(String value):}\\ Imposta valore testuale;
\item \texttt{+ int getDataId():}\\ Ritorna codice \texttt{id} associato al valore;
\item \texttt{+ void setDataId(int id):}\\ Imposta codice \texttt{id} associato al valore.
\item \texttt{+ DataTypes getType():}\\ Ritorna il tipo del valore;
\item \texttt{+ void setType(DataTypes type):}\\ Imposta il tipo del valore.
\end{itemize}
\end{sloppypar}
\end{itemize}
\end{flushleft}

\paragraph{NumericValue}\
\label{botnumericvalue}
\begin{figure}[H] \centering
\includegraphics[trim=0cm 0.8cm 0cm 0cm,clip=true,scale=0.75]%
{./classi/server/model/NumericValue.png} \caption{Diagramma classe NumericValue}
\end{figure}
\begin{flushleft}
\begin{itemize}
\item \textbf{Descrizione:} Classe che modella i valori dei dati numerici.
\item \textbf{Relazione con altre componenti:} la classe implementa la seguente interfaccia:
		\begin{itemize}
			\item \smodel{}.IDataValue.
		\end{itemize}
\item \textbf{Attributi:}
\begin{sloppypar}
\begin{itemize}
\item \texttt{- int dataId:}\\ Codice \texttt{id} associato al valore;
\item \texttt{- DataTypes type;}\\ Tipo del valore;
\item \texttt{- double value:}\\ Valore numerico.
\end{itemize}
\end{sloppypar}
\item \textbf{Metodi:}
\begin{sloppypar}
\begin{itemize}
\item \texttt{+ double getValue():}\\ Ritorna valore numerico;
\item \texttt{+ void setValue(double value):}\\ Imposta valore numerico.
\item \texttt{+ int getDataId():}\\ Ritorna codice \texttt{id} associato al valore;
\item \texttt{+ void setDataId(int id):}\\ Imposta codice \texttt{id} associato al valore.
\item \texttt{+ DataTypes getType():}\\ Ritorna il tipo del valore;
\item \texttt{+ void setType(DataTypes type):}\\ Imposta il tipo del valore.
\end{itemize}
\end{sloppypar}
\end{itemize}
\end{flushleft}

\paragraph{ImageValue}\
\label{botimagevalue}
\begin{figure}[H] \centering
\includegraphics[trim=0cm 0.8cm 0cm 0cm,clip=true,scale=0.75]%
{./classi/server/model/ImageValue.png} \caption{Diagramma classe ImageValue}
\end{figure}
\begin{flushleft}
\begin{itemize}
\item \textbf{Descrizione:} Classe che modella i valori dei dati immagine.
\item \textbf{Relazione con altre componenti:} la classe implementa la seguente interfaccia:
		\begin{itemize}
			\item \smodel{}.IDataValue.
		\end{itemize}
\item \textbf{Attributi:}
\begin{sloppypar}
\begin{itemize}
\item \texttt{- int dataId:}\\ Codice \texttt{id} associato al valore;
\item \texttt{- DataTypes type;}\\ Tipo del valore;
\item \texttt{- String imageUrl:}\\ Percorso \textit{URL} dell'immagine.
\end{itemize}
\end{sloppypar}
\item \textbf{Metodi:}
\begin{sloppypar}
\begin{itemize}
\item \texttt{+ String getImageUrl():}\\ Ritorna percorso \textit{URL} dell'immagine;
\item \texttt{+ void setImageUrl(String imageUrl):}\\ Imposta percorso \textit{URL} dell'immagine;
\item \texttt{+ int getDataId():}\\ Ritorna codice \texttt{id} associato al valore;
\item \texttt{+ void setDataId(int id):}\\ Imposta codice \texttt{id} associato al valore.
\item \texttt{+ DataTypes getType():}\\ Ritorna il tipo del valore;
\item \texttt{+ void setType(DataTypes type):}\\ Imposta il tipo del valore.
\end{itemize}
\end{sloppypar}
\end{itemize}
\end{flushleft}

\paragraph{GeographicValue}\
\label{botgeographicvalue}
\begin{figure}[H] \centering
\includegraphics[trim=0cm 0.8cm 0cm 0cm,clip=true,scale=0.75]%
{./classi/server/model/GeographicValue.png} \caption{Diagramma classe GeographicValue}
\end{figure}
\begin{flushleft}
\begin{itemize}
\item \textbf{Descrizione:} Classe che modella i valori dei dati geografici.
\item \textbf{Relazione con altre componenti:} la classe implementa la seguente interfaccia:
		\begin{itemize}
			\item \smodel{}.IDataValue.
		\end{itemize}
\item \textbf{Attributi:}
\begin{sloppypar}
\begin{itemize}
\item \texttt{- int dataId:}\\ Codice \texttt{id} associato al valore;
\item \texttt{- DataTypes type;}\\ Tipo del valore;
\item \texttt{- double latitude:}\\ Latitudine;
\item \texttt{- double longitude:}\\ Longitudine;
\item \texttt{- double altitude:}\\ Altitudine.
\end{itemize}
\end{sloppypar}
\item \textbf{Metodi:}
\begin{sloppypar}
\begin{itemize}
\item \texttt{+ double getLatitude():}\\ Ritorna latitudine;
\item \texttt{+ void setLatitude(double latitude):}\\ Imposta latitudine;
\item \texttt{+ double getLongitude():}\\ Ritorna longitudine;
\item \texttt{+ void setLongitude(double longitude):}\\ Imposta longitudine;
\item \texttt{+ double getAltitude():}\\ Ritorna altitudine;
\item \texttt{+ void setAltitude(double altitude):}\\ Imposta altitudine;
\item \texttt{+ int getDataId():}\\ Ritorna codice \texttt{id} associato al valore;
\item \texttt{+ void setDataId(int id):}\\ Imposta codice \texttt{id} associato al valore.
\item \texttt{+ DataTypes getType():}\\ Ritorna il tipo del valore;
\item \texttt{+ void setType(DataTypes type):}\\ Imposta il tipo del valore.
\end{itemize}
\end{sloppypar}
\end{itemize}
\end{flushleft}

\paragraph{UserStep}\
\label{botuserstep}
\begin{figure}[H] \centering
\includegraphics[trim=0cm 0.8cm 0cm 0cm,clip=true,scale=0.75]%
{./classi/server/model/UserStep.png} \caption{Diagramma classe UserStep}
\end{figure}
\begin{flushleft}
\begin{itemize}
\item \textbf{Descrizione:} Classe che modella i passi in corso e che funge da interscambio dei dati di quest'ultimi con il \textit{database}.
\item \textbf{Relazione con altre componenti:} la classe implementa la seguente interfaccia:
		\begin{itemize}
			\item \smodel{}.ITransferObject.
		\end{itemize}
\item \textbf{Attributi:}
\begin{sloppypar}
\begin{itemize}
\item \texttt{+ enum stepStates\{ONGOING, EXPECTANT, REJECTED, APPROVED\}:}\\ Enumerazione stato avanzamento;
\item \texttt{- int currentStepId:}\\ Codice \texttt{id} del passo attuale;
\item \texttt{- stepStates state:}\\ Stato avanzamento;
\item \texttt{- String username:}\\ Nome utente dell'utente del caso.
\end{itemize}
\end{sloppypar}
\item \textbf{Metodi:}
\begin{sloppypar}
\begin{itemize}
\item \texttt{+ int getCurrentStepId():}\\ Ritorna il codice \texttt{id} del passo attuale;
\item \texttt{+ void setCurrentStepId(int currentStepId):}\\ Imposta il codice \texttt{id} del passo attuale;
\item \texttt{+ stepStates getState():}\\ Ritorna stato avanzamento;
\item \texttt{+ void setStates(stepStates state):}\\ Imposta stato avanzamento;
\item \texttt{+ String getUsername():}\\ Restituisce nome utente dell'utente in caso;
\item \texttt{+ void setUsername(String username):}\\ Imposta nome utente dell'utente in caso.
\end{itemize}
\end{sloppypar}
\end{itemize}
\end{flushleft}

\paragraph{ProcessOwner}\
\label{botpowner}
\begin{figure}[H] \centering
\includegraphics[trim=0cm 0.8cm 0cm 0cm,clip=true,scale=0.75]%
{./classi/server/model/ProcessOwner.png} \caption{Diagramma classe ProcessOwner}
\end{figure}
\begin{flushleft}
\begin{itemize}
\item \textbf{Descrizione:} Classe che modella il ProcessOwner e che funge da interscambio dei dati di quest'ultimo con il \textit{database}.
\item \textbf{Relazione con altre componenti:} la classe implementa la seguente interfaccia:
		\begin{itemize}
			\item \smodel{}.ITransferObject.
		\end{itemize}
\item \textbf{Attributi:}
\begin{sloppypar}
\begin{itemize}
\item \texttt{- String username:}\\ Nome utente \textit{Process Owner};
\item \texttt{- String password:}\\ Password \textit{Process Owner}.
\end{itemize}
\end{sloppypar}
\item \textbf{Metodi:}
\begin{sloppypar}
\begin{itemize}
\item \texttt{+ String getUsername():}\\ Ritorna il nome utente del \textit{Process Owner};
\item \texttt{+ void setUsername(String username):}\\ Imposta il nome utente del \textit{Process Owner};
\item \texttt{+ String getPassword():}\\ Ritorna la password del \textit{Process Owner};
\item \texttt{+ void setPassword(String password):}\\ Imposta la password del \textit{Process Owner}. 
\end{itemize}
\end{sloppypar}
\end{itemize}
\end{flushleft}

\paragraph{TextualData}\
\label{bottextualdata}
\begin{figure}[H] \centering
\includegraphics[trim=0cm 0.8cm 0cm 0cm,clip=true,scale=0.75]%
{./classi/server/model/TextualData.png} \caption{Diagramma classe TextualData}
\end{figure}
\begin{flushleft}
\begin{itemize}
\item \textbf{Descrizione:} Classe che modella i valori dei dati testuali.
\item \textbf{Attributi:}
\begin{sloppypar}
\begin{itemize}
\item \texttt{- int dataId:}\\ Codice \texttt{id} associato al dato;
\item \texttt{- DataTypes type;}\\ Tipo del valore;
\item \texttt{- String description:}\\ Valore testuale.
\end{itemize}
\end{sloppypar}
\item \textbf{Metodi:}
\begin{sloppypar}
\begin{itemize}
\item \texttt{+ String getDescription():}\\ Ritorna descrizione;
\item \texttt{+ void setDescription(String value):}\\ Imposta descrizione;
\item \texttt{+ int getDataId():}\\ Ritorna codice \texttt{id} associato al dato;
\item \texttt{+ void setDataId(int id):}\\ Imposta codice \texttt{id} associato al dato.
\item \texttt{+ DataTypes getType():}\\ Ritorna il tipo del dato;
\item \texttt{+ void setType(DataTypes type):}\\ Imposta il tipo del dato.
\end{itemize}
\end{sloppypar}
\end{itemize}
\end{flushleft}

\paragraph{NumericData}\
\label{botnumericdata}
\begin{figure}[H] \centering
\includegraphics[trim=0cm 0.8cm 0cm 0cm,clip=true,scale=0.75]%
{./classi/server/model/NumericData.png} \caption{Diagramma classe NumericData}
\end{figure}
\begin{flushleft}
\begin{itemize}
\item \textbf{Descrizione:} Classe che modella i valori dei dati numerici.
\item \textbf{Attributi:}
\begin{sloppypar}
\begin{itemize}
\item \texttt{- int dataId:}\\ Codice \texttt{id} associato al datp;
\item \texttt{- DataTypes type;}\\ Tipo del dato;
\item \texttt{- double maxValue:}\\ Valore massimo;
\item \texttt{- double minValue:}\\ Valore minimo.
\end{itemize}
\end{sloppypar}
\item \textbf{Metodi:}
\end{itemize}
\end{flushleft}