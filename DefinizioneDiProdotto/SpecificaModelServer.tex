\subsection{Server}

Il \textit{model} lato \textit{server} gestisce la persistenza dei dati all'interno del \textit{database} consentendo interrogazione, inserimento, cancellazione e aggiornamento.

La componente è formata dalle seguenti \textit{classi}:
\begin{itemize}
	\item \hyperref[idataacessobject]{\smodel{}.IDataAcessObject};
	\item \hyperref[itransferobject]{\smodel{}.ITransferObject};
	\item \hyperref[userdao]{\smodel{}.UserDao};
	\item \hyperref[processdao]{\smodel{}.ProcessDao};
	\item \hyperref[processownerdao]{\smodel{}.ProcessOwnerDao};
	\item \hyperref[stepdao]{\smodel{}.StepDao};
	\item \hyperref[botuser]{\smodel{}.User};
	\item \hyperref[botprocess]{\smodel{}.Process};
	\item \hyperref[botstep]{\smodel{}.Step};
	\item \hyperref[botdata]{\smodel{}.Data};
	\item \hyperref[botdatatype]{\smodel{}.DataType};
	\item \hyperref[botcondition]{\smodel{}.Condition};
	\item \hyperref[botconstraint]{\smodel{}.Constraint};
	\item \hyperref[botnumericconstraint]{\smodel{}.NumericConstraint};
	\item \hyperref[bottemporalconstraint]{\smodel{}.TemporalConstraint};
	\item \hyperref[botgeographicconstraint]{\smodel{}.GeographicConstraint};
	\item \hyperref[botdatasent]{\smodel{}.DataSent};
	\item \hyperref[botidatavalue]{\smodel{}.IDataValue};
	\item \hyperref[bottextualvalue]{\smodel{}.TextualValue};
	\item \hyperref[botnumericvalue]{\smodel{}.NumericValue};
	\item \hyperref[botimagevalue]{\smodel{}.ImageValue};
	\item \hyperref[botgeographicvalue]{\smodel{}.GeographicValue};
	\item \hyperref[botuserstep]{\smodel{}.UserStep};
	\item \hyperref[botpowner]{\smodel{}.ProcessOwner};
\end{itemize}

\subsubsection{Package \model{}}

\paragraph{IDataAcessObject}
\label{idataacessobject}
\begin{flushleft}
\begin{itemize}
\item \textbf{Descrizione:} Interfaccia che permette di gestire la comunicazione e l'interrogazione con il \textit{database}.
\item \textbf{Metodi:}
\begin{sloppypar}
\begin{itemize}
\item \texttt{+ void setJdbcTemplate(JdbcTemplate jdbcTemplate):}\\ Imposta i parametri per l'accesso alla sorgente dei dati;
\item \texttt{+ ITransferObject getAll():}\\ Ritorna tutti i dati di competenza della classe che estende questa interfaccia.
\end{itemize}
\end{sloppypar}
\end{itemize}
\end{flushleft}

\paragraph{ITransferObject}
\label{itransferobject}
\begin{flushleft}
\begin{itemize}
\item \textbf{Descrizione:} Interfaccia realizzata dai tipi che modellano i dati del \textit{database}.
\end{itemize}
\end{flushleft}

\paragraph{UserDao}
\label{userdao}
\begin{flushleft}
\begin{itemize}
\item \textbf{Descrizione:} Classe che si occupa delle interrogazioni del \textit{database} relative agli utenti del sistema.
\item \textbf{Relazione con altre componenti:} la classe implementa la seguente interfaccia:
		\begin{itemize}
			\item \smodel{}.IDataAcessObject.
		\end{itemize}
		La classe invoca i metodi della classe:
		\begin{itemize}
			\item \smodel{}.User.
		\end{itemize}
\item \textbf{Attributi:}
\begin{sloppypar}
\begin{itemize}
\item \texttt{JdbcTemplate jdbcTemplate:}\\ Oggetto che fornisce l'accesso alla sorgente dei dati;
\end{itemize}
\end{sloppypar}
\item \textbf{Metodi:}
\begin{sloppypar}
\begin{itemize}
\item \texttt{+ User getUser(String userName):}\\ Ritorna l’utente con il nome utente specificato; 
\item \texttt{+ List<User> getAllUser():}\\ Ritorna tutti gli utenti;
\item \texttt{+ boolean insertUser(User user) :}\\ Aggiunge l'utente passato come parametro;
\item \texttt{+ public boolean updateUser(User user) :}\\ Aggiorna i dati dell'utente con il nome utente corrispondente a quello dell'utente passato, con i dati dell'utente passato.
\end{itemize}
\end{sloppypar}
\end{itemize}
\end{flushleft}

\paragraph{ProcessDao}
\label{processdao}
\begin{flushleft}
\begin{itemize}
\item \textbf{Descrizione:} Classe che si occupa delle interrogazioni del \textit{database} relative ai processi.
\item \textbf{Relazione con altre componenti:} la classe implementa la seguente interfaccia:
		\begin{itemize}
			\item \smodel{}.IDataAcessObject.
		\end{itemize}
		La classe invoca i metodi della classe:
		\begin{itemize}
			\item \smodel{}.Process.
		\end{itemize}
\item \textbf{Attributi:}
\begin{sloppypar}
\begin{itemize}
\item \texttt{- JdbcTemplate jdbcTemplate:}\\ Oggetto che fornisce l'accesso alla sorgente dei dati;
\end{itemize}
\end{sloppypar}
\item \textbf{Metodi:}
\begin{sloppypar}
\begin{itemize}
\item \texttt{+ Process getProcess(int id)):}\\ Ritorna il processo con l'\texttt{id} specificato; 
\item \texttt{+ List<Process> getAllProcess():}\\ Ritorna tutti i processi;
\item \texttt{+ boolean insertProcess(Process process) :}\\ Aggiunge il processo passato come parametro;
\item \texttt{+ public boolean updateProcess(Process process) :}\\ Aggiorna i dati del processo con lo stesso \texttt{id} di quello del processo passato, con i dati del processo passato.
\end{itemize}
\end{sloppypar}
\end{itemize}
\end{flushleft}

\paragraph{ProcessOwnerDao}
\label{processownerdao}
\begin{flushleft}
\begin{itemize}
\item \textbf{Descrizione:} Classe che si occupa delle interrogazioni del \textit{database} relative all'autenticazione del \textit{ProcessOwner}.
\item \textbf{Relazione con altre componenti:} la classe implementa la seguente interfaccia:
		\begin{itemize}
			\item \smodel{}.IDataAcessObject.
		\end{itemize}
		La classe invoca i metodi della classe:
		\begin{itemize}
			\item \smodel{}.ProcessOwner.
		\end{itemize}
\item \textbf{Attributi:}
\begin{sloppypar}
\begin{itemize}
\item \texttt{- JdbcTemplate jdbcTemplate:}\\ Oggetto che fornisce l'accesso alla sorgente dei dati;
\end{itemize}
\end{sloppypar}
\item \textbf{Metodi:}
\begin{sloppypar}
\begin{itemize}
\item \texttt{+ Process getProcessOwner(int id)):}\\ Ritorna l'oggetto rappresentante il \textit{ProcessOwner}. 
\end{itemize}
\end{sloppypar}
\end{itemize}
\end{flushleft}

\paragraph{StepDao}
\label{stepdao}
\begin{flushleft}
\begin{itemize}
\item \textbf{Descrizione:} Classe che si occupa delle interrogazioni del \textit{database} relative a tutte le operazioni sui passi dei processi.
\item \textbf{Relazione con altre componenti:} la classe implementa la seguente interfaccia:
		\begin{itemize}
			\item \smodel{}.IDataAcessObject.
		\end{itemize}
		La classe invoca i metodi della classe:
		\begin{itemize}
			\item \smodel{}.Step;
			\item \smodel{}.UserStep;
			\item \smodel{}.DataSent.
		\end{itemize}
\item \textbf{Attributi:}
\begin{sloppypar}
\begin{itemize}
\item \texttt{- JdbcTemplate jdbcTemplate:}\\ Oggetto che fornisce l'accesso alla sorgente dei dati;
\end{itemize}
\end{sloppypar}
\item \textbf{Metodi:}
\begin{sloppypar}
\begin{itemize}
\item \texttt{+ Step getStep(int id):}\\ Ritorna il passo con l'\texttt{id} specificato; 
\item \texttt{+ List<Step> getAllStep():}\\ Ritorna tutti i passi;
\item \texttt{+ List<Step> getStepOf(int ProcessId):}\\ Ritorna tutti i passi appartenenti al processo di cui si è passato l'\texttt{id};
\item \texttt{+ boolean insertStep(Step step) :}\\ Aggiunge il passo passato come parametro;
\item \texttt{+ public boolean updateStep(Step step) :}\\ Aggiorna i dati del passo con l'\texttt{id} corrispondente a quello del passo passato, con i dati del passo passato;
\item \texttt{+ List<UserStep> userStep(String userName)}\\ Ritorna una lista di oggetti informativi sullo stato dei passi in corso da parte dell'utente di cui si è passato il nome utente;
\item \texttt{+ List<UserStep> userProcessStep(String userName, processId)}\\ Ritorna una lista di oggetti informativi sullo stato dei passi in corso appartenenti al processo di cui si è passato l'\texttt{id} da parte dell'utente di cui si è passato il nome utente;
\item \texttt{+ List<DataSent> getData(Step step)}\\ Ritorna tutti i dati da tutti gli utenti relativi al passo passato;
\item \texttt{+ List<DataSent> getData(String userName, Step step)}\\ Ritorna tutti i dati inviati dall'utente di cui si è passato il nome utente relativi al passo passato;
\item \texttt{+ boolean completeStep(String userName, Step step, List<DataSent> data, Step next)}\\Notifica e aggiorna nel \textit{database} lo stato dell'utente quando completa o tenta di completare un passo.
\end{itemize}
\end{sloppypar}
\end{itemize}
\end{flushleft}

\paragraph{User}
\label{botuser}
\begin{flushleft}
\begin{itemize}
\item \textbf{Descrizione:} Classe che modella gli utenti del sistema e che funge da interscambio dei dati di quest'ultimi con il \textit{database}.
\item \textbf{Relazione con altre componenti:} la classe implementa la seguente interfaccia:
		\begin{itemize}
			\item \smodel{}.ITransferObject.
		\end{itemize}
\item \textbf{Attributi:}
\begin{sloppypar}
\begin{itemize}
\item \texttt{- String userName:}\\ Nome utente;
\item \texttt{- String password:}\\ Password dell'utente;
\item \texttt{- String name:}\\Nome anagrafico dell'utente;
\item \texttt{- String surName:}\\Cognome dell'utente;
\item \texttt{- Date dateOfBirth:}\\Data di nascita dell'utente;
\item \texttt{- String email:}\\Indirizzo di posta elettronica dell'utente;
\item \texttt{- int id:}\\Codice identificativo \texttt{id} associato all'utente.
\end{itemize}
\end{sloppypar}
\item \textbf{Metodi:}
\begin{sloppypar}
\begin{itemize}
\item \texttt{+ String getUserName():}\\ Ritorna il nome utente;
\item \texttt{+ void setUserName(String userName):}\\ Imposta il nome utente;
\item \texttt{+ String getPassword():}\\ Ritorna la password dell'utente;
\item \texttt{+ void setPassword(String password):}\\ Imposta la password dell'utente;
\item \texttt{+ String getName():}\\ Ritorna il nome anagrafico dell'utente;
\item \texttt{+ void setName(String name):}\\ Imposta il nome anagrafico dell'utente;
\item \texttt{+ String getSurName():}\\ Ritorna il cognome dell'utente;
\item \texttt{+ void setSurName(String surName):}\\ Imposta il cognome dell'utente;
\item \texttt{+ Date getDateOfBirth():}\\ Ritorna la data di nascita dell'utente;
\item \texttt{+ void setDateOfBirth(Date dateOfBirth):}\\ Imposta la data di nascita dell'utente;
\item \texttt{+ String getEmail():}\\ Ritorna l'indirizzo di posta elettronica dell'utente;
\item \texttt{+ void setEmail(String email):}\\ Imposta l'indirizzo di posta elettronica dell'utente;
\item \texttt{+ int getId():}\\ Ritorna il codice \texttt{id} associato all'utente;
\item \texttt{+ void setId(int id):}\\ Imposta il codice \texttt{id} associato all'utente.
\end{itemize}
\end{sloppypar}
\end{itemize}
\end{flushleft}

\paragraph{Process}
\label{botprocess}
\begin{flushleft}
\begin{itemize}
\item \textbf{Descrizione:} Classe che modella i processi del sistema e che funge da interscambio dei dati di quest'ultimi con il \textit{database}.
\item \textbf{Relazione con altre componenti:} la classe implementa la seguente interfaccia:
		\begin{itemize}
			\item \smodel{}.ITransferObject.
		\end{itemize}
\item \textbf{Attributi:}
\begin{sloppypar}
\begin{itemize}
\item \texttt{- String name:}\\ Nome del processo;
\item \texttt{- String description:}\\ Descrizione del processo;
\item \texttt{- int completionsMax:}\\ Numero massimo di completamenti del processo;
\item \texttt{- Date dateOfTermination:}\\ Data di terminazione del processo;
\item \texttt{- boolean terminated:}\\ Boolano vero quando il processo è terminato;
\item \texttt{- int maxTree:}\\ Massimo alberi del processo;
\item \texttt{- List<Integer> stepsId:}\\ Lista di codici \texttt{id} relativi ai passi del processo;
\item \texttt{- int id:}\\ Codice identificativo \texttt{id} associato al processo.
\end{itemize}
\end{sloppypar}
\item \textbf{Metodi:}
\begin{sloppypar}
\begin{itemize}
\item \texttt{+ String getName():}\\ Ritorna il nome del processo;
\item \texttt{+ void setName(String name):}\\ Imposta il nome del processo;
\item \texttt{+ String getDescription():}\\ Ritorna la descrizione del processo;
\item \texttt{+ void setDescription(String description):}\\ Imposta la descrizione del processo;
\item \texttt{+ int getCompletionsMax():}\\ Restituisce il numero massimo di completamenti del processo;
\item \texttt{+ void setCompletionsMax(int completionsMax):}\\ Imposta il numero massimo di completamenti del processo;
\item \texttt{+ Date getDateOfTermination():}\\ Ritorna data di terminazione del processo;
\item \texttt{+ void setDateOfTermination(Date dateOfTermination):}\\ Imposta la data di terminazione del processo;
\item \texttt{+ boolean isTerminated():}\\ Ritorna vero se il processo è terminato;
\item \texttt{+ void setTerminated(boolean terminated):}\\ Imposta vero se il processo è terminato;
\item \texttt{+ int getMaxTree():}\\ Ritorna il massimo alberi del processo;
\item \texttt{+ void setMaxtree(int maxTree):}\\ Imposta il massimo alberi del processo;
\item \texttt{+ List<Integer> getStepsId():}\\ Ritorna lista di codici \texttt{id} relativi ai passi del processo;
\item \texttt{+ void setStepsId(List<Integer> stepsId):}\\ Imposta lista di codi \texttt{id} relativi ai passi del processo;
\item \texttt{+int getId():}\\ Ritorna codice identificativo \texttt{id} associato al processo;
\item \texttt{+void setId(int id):}\\ Imposta codice identificativo \texttt{id} associato al processo.
\end{itemize}
\end{sloppypar}
\end{itemize}
\end{flushleft}

\paragraph{Step}
\label{botstep}
\begin{flushleft}
\begin{itemize}
\item \textbf{Descrizione:} Classe che modella i passi del sistema e che funge da interscambio dei dati di quest'ultimi con il \textit{database}.
\item \textbf{Relazione con altre componenti:} la classe implementa la seguente interfaccia:
		\begin{itemize}
			\item \smodel{}.ITransferObject.
		\end{itemize}
		La classe contiene istanze di:
		\begin{itemize}
			\item \smodel{}.Condition;
			\item \smodel{}.Data.
		\end{itemize}
\item \textbf{Attributi:}
\begin{sloppypar}
\begin{itemize}
\item \texttt{- int id:}\\ Codice identificativo \texttt{id} associato al passo;
\item \texttt{- String description:}\\ Descrizione del passo;
\item \texttt{- List<Data> data:}\\ Lista con i campi dato del passo;
\item \texttt{- List<Condition> conditions:}\\ Lista delle condizioni di avanzamento del passo;
\item \texttt{- int processId:}\\ Codice identificativo \texttt{id} associato al processo padre;
\item \texttt{- boolean first:}\\ Booleano vero se il passo è primo per il processo padre.
\end{itemize}
\end{sloppypar}
\item \textbf{Metodi:}
\begin{sloppypar}
\begin{itemize}
\item \texttt{+ int getId():}\\ Ritorna codice identificativo \texttt{id} associato al passo;
\item \texttt{+ void setId(int id):}\\ Imposta codice identificativo \texttt{id} associato al passo;
\item \texttt{+ String getDescription():}\\ Ritorna descrizione del passo;
\item \texttt{+ void setDescription(String description):}\\ Imposta descrizione del passo;
\item \texttt{+ List<Data> getData():}\\ Ritorna lista con i campi dato del passo;
\item \texttt{+ void setData(List<Data> data):}\\ Imposta lista con i campi dato del passo;
\item \texttt{+ List<Condition> getConditions():}\\ Ritorna lista delle condizioni di avanzamento del passo;
\item \texttt{+ void setConditions(List<Condition conditions):}\\ Imposta lista delle condizioni di avanzamento del passo;
\item \texttt{+ int getProcessId():}\\ Ritorna codice \texttt{id} associato al processo padre;
\item \texttt{+ void setProcessId(int processId):}\\ Imposta codice \texttt{id} associato al processo padre;
\item \texttt{+ boolean isFirst():}\\ Ritorna vero se il passo è primo per il processo padre;
\item \texttt{+ void setFirst():}\\ Imposta vero se il passo è primo per il processo padre. 
\end{itemize}
\end{sloppypar}
\end{itemize}
\end{flushleft}

\paragraph{Data}
\label{botdata}
\begin{flushleft}
\begin{itemize}
\item \textbf{Descrizione:} Classe che modella i campi dato richiesti.
\item \textbf{Attributi:}
\begin{sloppypar}
\begin{itemize}
\item \texttt{- String name:}\\ Nome del campo dati;
\item \texttt{- DataType type:}\\ Tipo di dato richiesto dal campo;
\item \texttt{- int id;}\\Codice identificativo \texttt{id} associato al campo dati.
\end{itemize}
\end{sloppypar}
\item \textbf{Metodi:}
\begin{sloppypar}
\begin{itemize}
\item \texttt{+ String getName():}\\ Ritorna il nome del campo dati;
\item \texttt{+ void setName(String name):}\\ Imposta il nome del campo dati;
\item \texttt{+ DataType getType():}\\ Ritorna il tipo di dato richiesto dal campo;
\item \texttt{+ void setType(DataType type):}\\ Imposta il tipo di dato richiesto dal campo;
\item \texttt{+ int getId():}\\ Ritorna il codice \texttt{id} associato al campo dati;
\item \texttt{+ void setId(int id):}\\ Imposta il codice \texttt{id} associato al campo dati.
\end{itemize}
\end{sloppypar}
\end{itemize}
\end{flushleft}

\paragraph{DataType}
\label{botdatatype}
\begin{flushleft}
\begin{itemize}
\item \textbf{Descrizione:} Enumerazione tipo di dato.
\item \textbf{Attributi:}
\begin{sloppypar}
\begin{itemize}
\item \texttt{+ enum DataType\{TEXTUAL, NUMERIC, IMAGE, GEOGRAPHIC}:\}\\ Enumerazione tipo di dato.
\end{itemize}
\end{sloppypar}
\end{itemize}
\end{flushleft}

\paragraph{Condition}
\label{botcondition}
\begin{flushleft}
\begin{itemize}
\item \textbf{Descrizione:} Classe che modella le condizioni di avanzamento di un passo.
\item \textbf{Relazione con altre componenti:}La classe contiene istanze di:
		\begin{itemize}
			\item \smodel{}.Constraint;
		\end{itemize}
\item \textbf{Attributi:}
\begin{sloppypar}
\begin{itemize}
\item \texttt{- int id:}\\ Codice identificativo \texttt{id} associato alla condizione di avanzamento;
\item \texttt{- boolean requiresApproval:}\\ Booleano vero se è richiesta l'approvazione del \textit{Process Owner};
\item \texttt{- List<Constraint> constraints:}\\ Lista di vincoli che soddisfano la condizione di avanzamento;
\item \texttt{- boolean optional:}\\ Booleano vero se la condizione è opzionale per l'avanzamento;
\item \texttt{- int nextStepId:}\\ Codice identificativo \texttt{id} associato al passo succesivo.
\end{itemize}
\end{sloppypar}
\item \textbf{Metodi:}
\begin{sloppypar}
\begin{itemize}
\item \texttt{+ int getId():}\\ Ritorna il codice \texttt{id} associato alla condizione di avanzamento;
\item \texttt{+ void setId(int id):}\\ Imposta il codice \texttt{id} associato alla condizione di avanzamento;
\item \texttt{+ boolean isRequiresApproval():}\\ Ritorna vero se è richiesta l'approvazione del \textit{Process Owner};
\item \texttt{+ void setRequiresApproval(boolean requiresApproval):}\\ Imposta vero se è richiesta l'approvazione del \textit{Process Owner};
\item \texttt{+ List<Constraint> getConstraints():}\\ Ritorna lista di vincoli che soddisfano la condizione di avanzamento;
\item \texttt{+ void setConstraints(List<Constraint> constraints):}\\ Imposta lista di vincoli che soddisfano la condizione di avanzamento;
\item \texttt{+ boolean isOptional():}\\ Ritorna vero se la condizione è opzionale per l'avanzamento;
\item \texttt{+ void setOptional(boolean optional):}\\ Imposta vero se la condizione è opzionale per l'avanzamento;
\item \texttt{+ int getNextStepId():}\\ Ritorna codice \texttt{id} del passo successivo;
\item \texttt{+ void setNextStepId(int nextStepId):}\\ Imposta codice \texttt{id} del passo succesivo.
\end{itemize}
\end{sloppypar}
\end{itemize}
\end{flushleft}

\paragraph{Constraint}
\label{botconstraint}
\begin{flushleft}
\begin{itemize}
\item \textbf{Descrizione:} Classe astratta che modella i vincoli.
\item \textbf{Attributi:}
\begin{sloppypar}
\begin{itemize}
\item \texttt{- Data associatedData:}\\ Campo dati su cui è posto il vincolo.
\end{itemize}
\end{sloppypar}
\item \textbf{Metodi:}
\begin{sloppypar}
\begin{itemize}
\item \texttt{+ Data getAssociatedData():}\\ Ritorna il campo dati su cui è posto il vincolo;
\item \texttt{+ void setAssociatedData(Data associatedData):}\\ Imposta il campo dati sui cui è posto il vincolo.
\end{itemize}
\end{sloppypar}
\end{itemize}
\end{flushleft}

\paragraph{NumericConstraint}
\label{botnumericconstraint}
\begin{flushleft}
\begin{itemize}
\item \textbf{Descrizione:} Classe che modella i vincoli numerici.
\item \textbf{Relazione con altre componenti:} la classe estende la seguente classe:
		\begin{itemize}
			\item \smodel{}.Constraint.
		\end{itemize}
\item \textbf{Attributi:}
\begin{sloppypar}
\begin{itemize}
\item \texttt{- int id:}\\ Codice identificativo \texttt{id} associato al vincolo;
\item \texttt{- int minDigits:}\\ Minimo numero di cifre;
\item \texttt{- int maxDigits:}\\ Massimo numero di cifre;
\item \texttt{- boolean decimal:}\\ Booleano vero se atteso un decimale;
\item \texttt{- double minValue:}\\ Valore minimo;
\item \texttt{- double maxValue:}\\ Valore massimo;
\end{itemize}
\end{sloppypar}
\item \textbf{Metodi:}
\begin{sloppypar}
\begin{itemize}
\item \texttt{+ int getId():}\\ Ritorna codice \texttt{id} del vincolo;
\item \texttt{+ void setId(int id):}\\ Imposta codice \texttt{id} del vincolo;
\item \texttt{+ int getMinDigits():}\\ Ritorna minimo numero di cifre;
\item \texttt{+ void setMinDigits(int minDigits):}\\ Imposta minimo numero di cifre;
\item \texttt{+ int getMaxDigits():}\\ Ritorna massimo numero di cifre;
\item \texttt{+ void setMaxDigits(int maxDigits):}\\ Imposta massimo numero di cifre;
\item \texttt{+ boolean isDecimal():}\\ Ritorna vero se atteso un decimale;
\item \texttt{+ void setDecimal(boolean decimal):}\\ Imposta vero se atteso un decimale;
\item \texttt{+ double getMinValue():}\\ Ritorna valore minimo;
\item \texttt{+ void setMinValue(double minValue):}\\ Imposta valore minimo;
\item \texttt{+ double getMaxValue():}\\ Ritorna valore massimo;
\item \texttt{+ void setMaxValue(double maxValue):}\\ Imposta valore massimo;
\end{itemize}
\end{sloppypar}
\end{itemize}
\end{flushleft}

\paragraph{TemporalConstraint}
\label{bottemporalconstraint}
\begin{flushleft}
\begin{itemize}
\item \textbf{Descrizione:} Classe che modella i vincoli temporali.
\item \textbf{Relazione con altre componenti:} la classe estende la seguente classe:
		\begin{itemize}
			\item \smodel{}.Constraint.
		\end{itemize}
\item \textbf{Attributi:}
\begin{sloppypar}
\begin{itemize}
\item \texttt{- int id:}\\ Codice identificativo \texttt{id} associato al vincolo;
\item \texttt{- Date begin:}\\ Inizio arco temporale valido;
\item \texttt{- Date end:}\\ Fine arco temporale valido.
\end{itemize}
\end{sloppypar}
\item \textbf{Metodi:}
\begin{sloppypar}
\begin{itemize}
\item \texttt{+ int getId():}\\ Ritorna codice \texttt{id} del vincolo;
\item \texttt{+ void setId(int id):}\\ Imposta codice \texttt{id} del vincolo;
\item \texttt{+ Date getBegin():}\\ Ritorna inizio arco temporale valido;
\item \texttt{+ void setBegin(Date begin):}\\ Imposta inizio arco temporale valido;
\item \texttt{+ Date getEnd():}\\ Ritorna fine arco temporale valido;
\item \texttt{+ void setEnd(Date end):}\ Imposta fine arco temporale valido.
\end{itemize}
\end{sloppypar}
\end{itemize}
\end{flushleft}

\paragraph{GeographicConstraint}
\label{botgeographicconstraint}
\begin{flushleft}
\begin{itemize}
\item \textbf{Descrizione:} Classe che modella i vincoli geografici.
\item \textbf{Relazione con altre componenti:} la classe estende la seguente classe:
		\begin{itemize}
			\item \smodel{}.Constraint.
		\end{itemize}
\item \textbf{Attributi:}
\begin{sloppypar}
\begin{itemize}
\item \texttt{- int id:}\\ Codice identificativo \texttt{id} associato al vincolo;
\item \texttt{- double latitude:}\\ Latitudine richiesta;
\item \texttt{- double longitude:}\\ Longitudine richiesta;
\item \texttt{- double altitude:}\\ Altitudine richiesta;
\item \texttt{- double radius:}\\ Raggio di tolleranza.
\end{itemize}
\end{sloppypar}
\item \textbf{Metodi:}
\begin{sloppypar}
\begin{itemize}
\item \texttt{+ int getId():}\\ Ritorna codice \texttt{id} del vincolo;
\item \texttt{+ void setId(int id):}\\ Imposta codice \texttt{id} del vincolo;
\item \texttt{+ double getLatitude():}\\ Ritorna latitudine richiesta;
\item \texttt{+ void setLatitude(double latitude):}\\ Imposta latitudine richiesta;
\item \texttt{+ double getLongitude():}\\ Ritorna longitudine richiesta;
\item \texttt{+ void setLongitude(double longitude):}\\ Imposta longitudine richiesta;
\item \texttt{+ double getAltitude():}\\ Ritorna altitudine richiesta;
\item \texttt{+ void setAltitude(double altitude):}\\ Imposta altitudine richiesta;
\item \texttt{+ double getRadius():}\\ Ritorna raggio di tolleranza;
\item \texttt{+ void setRadius(double radius):}\\ Imposta raggio di tolleranza.
\end{itemize}
\end{sloppypar}
\end{itemize}
\end{flushleft}

\paragraph{DataSent}
\label{botdatasent}
\begin{flushleft}
\begin{itemize}
\item \textbf{Descrizione:} Classe che modella i dati ricevuti dagli utenti che funge da interscambio  con il \textit{database}.
\item \textbf{Relazione con altre componenti:} la classe implementa la seguente interfaccia:
		\begin{itemize}
			\item \smodel{}.ITransferObject.
		\end{itemize}
				La classe invoca i metodi della classe:
		\begin{itemize}
			\item \smodel{}.IDataValue.
		\end{itemize}
\item \textbf{Attributi:}
\begin{sloppypar}
\begin{itemize}
\item \texttt{- String user:}\\ Nome utente dell'utente che ha inviato il dato;
\item \texttt{- DataType type:}\\ Tipo del dato inviato;
\item \texttt{- IDataValue value:}\\ Oggetto con il valore del dato;
\item \texttt{- int stepId:}\\Codice \texttt{id} del passo richiedente il dato.
\end{itemize}
\end{sloppypar}
\item \textbf{Metodi:}
\begin{sloppypar}
\begin{itemize}
\item \texttt{+ String getUser():}\\ Ritorna nome utente dell'utente che ha inviato il dato;
\item \texttt{+ void setUser(String user):}\\ Imposta nome utente dell'utente che ha inviato il dato;
\item \texttt{+ DataType getType():}\\ Ritorna il tipo del dato inviato;
\item \texttt{+ void setType(DataType type):}\\ Imposta il tipo del dato inviato;
\item \texttt{+ IDataValue getValue():}\\ Ritorna oggetto con il valore del dato;
\item \texttt{+ void setValue(IDataValue value):}\\ Imposta oggetto con il valore del dato;
\item \texttt{+ int getStepId():}\\ Ritorna codice \texttt{id} del passo richiedente il dato;
\item \texttt{+ void setStepId(int stepId):}\\ Imposta codice \texttt{id} del passo richiedente il dato.
\end{itemize}
\end{sloppypar}
\end{itemize}
\end{flushleft}

\paragraph{IDataValue}
\label{botidatavalue}
\begin{flushleft}
\begin{itemize}
\item \textbf{Descrizione:} Interfaccia che modella i valori dei dati ricevuti.
\item \textbf{Metodi:}
\begin{sloppypar}
\begin{itemize}
\item \texttt{+ int getId():}\\ Ritorna codice \texttt{id} associato al valore;
\item \texttt{+ void setId(int id):}\\ Imposta codice \texttt{id} associato al valore.
\end{itemize}
\end{sloppypar}
\end{itemize}
\end{flushleft}

\paragraph{TextualValue}
\label{bottextualvalue}
\begin{flushleft}
\begin{itemize}
\item \textbf{Descrizione:} Classe che modella i valori dei dati testuali.
\item \textbf{Relazione con altre componenti:} la classe implementa la seguente interfaccia:
		\begin{itemize}
			\item \smodel{}.IDataValue.
		\end{itemize}
\item \textbf{Attributi:}
\begin{sloppypar}
\begin{itemize}
\item \texttt{- int id:}\\ Codice \texttt{id} associato al valore;
\item \texttt{- String value:}\\ Valore testuale.
\end{itemize}
\end{sloppypar}
\item \textbf{Metodi:}
\begin{sloppypar}
\begin{itemize}
\item \texttt{+ String getValue():}\\ Ritorna valore testuale;
\item \texttt{+ void setValue(String value):}\\ Imposta valore testuale.
\end{itemize}
\end{sloppypar}
\end{itemize}
\end{flushleft}

\paragraph{NumericValue}
\label{botnumericvalue}
\begin{flushleft}
\begin{itemize}
\item \textbf{Descrizione:} Classe che modella i valori dei dati numerici.
\item \textbf{Relazione con altre componenti:} la classe implementa la seguente interfaccia:
		\begin{itemize}
			\item \smodel{}.IDataValue.
		\end{itemize}
\item \textbf{Attributi:}
\begin{sloppypar}
\begin{itemize}
\item \texttt{- int id:}\\ Codice \texttt{id} associato al valore;
\item \texttt{- double value:}\\ Valore numerico.
\end{itemize}
\end{sloppypar}
\item \textbf{Metodi:}
\begin{sloppypar}
\begin{itemize}
\item \texttt{+ double getValue():}\\ Ritorna valore numerico;
\item \texttt{+ void setValue(double value):}\\ Imposta valore numerico.
\end{itemize}
\end{sloppypar}
\end{itemize}
\end{flushleft}

\paragraph{ImageValue}
\label{botimagevalue}
\begin{flushleft}
\begin{itemize}
\item \textbf{Descrizione:} Classe che modella i valori dei dati immagine.
\item \textbf{Relazione con altre componenti:} la classe implementa la seguente interfaccia:
		\begin{itemize}
			\item \smodel{}.IDataValue.
		\end{itemize}
\item \textbf{Attributi:}
\begin{sloppypar}
\begin{itemize}
\item \texttt{- int id:}\\ Codice \texttt{id} associato al valore;
\item \texttt{- String imageUrl:}\\ Percorso \textit{URL} dell'immagine.
\end{itemize}
\end{sloppypar}
\item \textbf{Metodi:}
\begin{sloppypar}
\begin{itemize}
\item \texttt{+ String getImageUrl():}\\ Ritorna percorso \textit{URL} dell'immagine;
\item \texttt{+ void setImageUrl(String imageUrl):}\\ Imposta percorso \textit{URL} dell'immagine.
\end{itemize}
\end{sloppypar}
\end{itemize}
\end{flushleft}

\paragraph{GeographicValue}
\label{botgeographicvalue}
\begin{flushleft}
\begin{itemize}
\item \textbf{Descrizione:} Classe che modella i valori dei dati geografici.
\item \textbf{Relazione con altre componenti:} la classe implementa la seguente interfaccia:
		\begin{itemize}
			\item \smodel{}.IDataValue.
		\end{itemize}
\item \textbf{Attributi:}
\begin{sloppypar}
\begin{itemize}
\item \texttt{- int id:}\\ Codice \texttt{id} associato al valore;
\item \texttt{- double latitude:}\\ Latitudine;
\item \texttt{- double longitude:}\\ Longitudine;
\item \texttt{- double altitude:}\\ Altitudine.
\end{itemize}
\end{sloppypar}
\item \textbf{Metodi:}
\begin{sloppypar}
\begin{itemize}
\item \texttt{+ double getLatitude():}\\ Ritorna latitudine;
\item \texttt{+ void setLatitude(double latitude):}\\ Imposta latitudine;
\item \texttt{+ double getLongitude():}\\ Ritorna longitudine;
\item \texttt{+ void setLongitude(double longitude):}\\ Imposta longitudine;
\item \texttt{+ double getAltitude():}\\ Ritorna altitudine;
\item \texttt{+ void setAltitude(double altitude):}\\ Imposta altitudine.
\end{itemize}
\end{sloppypar}
\end{itemize}
\end{flushleft}

\paragraph{UserStep}
\label{botuserstep}
\begin{flushleft}
\begin{itemize}
\item \textbf{Descrizione:} Classe che modella i passi in corso e che funge da interscambio dei dati di quest'ultimi con il \textit{database}.
\item \textbf{Relazione con altre componenti:} la classe implementa la seguente interfaccia:
		\begin{itemize}
			\item \smodel{}.ITransferObject.
		\end{itemize}
\item \textbf{Attributi:}
\begin{sloppypar}
\begin{itemize}
\item \texttt{+ enum stepStates\{ONGOING, EXPECTANT, REJECTED, APPROVED\}:}\\ Enumerazione stato avanzamento;
\item \texttt{- int currentStepId:}\\ Codice \texttt{id} del passo attuale;
\item \texttt{- stepStates state:}\\ Stato avanzamento;
\item \texttt{- String user:}\\ Nome utente dell'utente del caso.
\end{itemize}
\end{sloppypar}
\item \textbf{Metodi:}
\begin{sloppypar}
\begin{itemize}
\item \texttt{+ int getCurrentStepId():}\\ Ritorna il codice \texttt{id} del passo attuale;
\item \texttt{+ void setCurrentStepId(int currentStepId):}\\ Imposta il codice \texttt{id} del passo attuale;
\item \texttt{+ stepStates getState():}\\ Ritorna stato avanzamento;
\item \texttt{+ void setStates(stepStates state):}\\ Imposta stato avanzamento;
\item \texttt{+ String getUser():}\\ Restituisce nome utente dell'utente in caso;
\item \texttt{+ void setUser(String user):}\\ Imposta nome utente dell'utente in caso.
\end{itemize}
\end{sloppypar}
\end{itemize}
\end{flushleft}

\paragraph{ProcessOwner}
\label{botpowner}
\begin{flushleft}
\begin{itemize}
\item \textbf{Descrizione:} Classe che modella il ProcessOwner e che funge da interscambio dei dati di quest'ultimo con il \textit{database}.
\item \textbf{Relazione con altre componenti:} la classe implementa la seguente interfaccia:
		\begin{itemize}
			\item \smodel{}.ITransferObject.
		\end{itemize}
\item \textbf{Attributi:}
\begin{sloppypar}
\begin{itemize}
\item \texttt{- String userName:}\\ Nome utente \textit{Process Owner};
\item \texttt{- String password:}\\ Password \textit{Process Owner}.
\end{itemize}
\end{sloppypar}
\item \textbf{Metodi:}
\begin{sloppypar}
\begin{itemize}
\item \texttt{+ String getUserName():}\\ Ritorna il nome utente del \textit{Process Owner};
\item \texttt{+ void setUserName(String userName):}\\ Imposta il nome utente del \textit{Process Owner};
\item \texttt{+ String getPassword():}\\ Ritorna la password del \textit{Process Owner};
\item \texttt{+ void setPassword(String password):}\\ Imposta la password del \textit{Process Owner}. 
\end{itemize}
\end{sloppypar}
\end{itemize}
\end{flushleft}