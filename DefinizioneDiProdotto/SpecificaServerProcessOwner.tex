\subsubsection{Package com.sirius.sequenziatore.server.presenter.processowner}
\textbf{IMMAGINE PACKAGE}
\paragraph{StepController}%----------------------------------------------------------------%
\begin{description}
	\item[Descrizione] Questa classe dovrà fornire al \textit{process owner} tutti i dati inseriti dagli utenti per un dato passo, quindi dovrà restituire una collezione di dati al process owner il quale potrà visionarli;
	\item[Mappatura base:] \textit{\textbackslash stepdata\textbackslash \{idstep\}\textbackslash processowner}
	\item[Attributi:]
	\item[Metodi:]\begin{itemize}
					\item 
				\end{itemize}
\end{description}
\paragraph{ProcessController}%----------------------------------------------------------------%
\begin{description}
	\item[Descrizione] 
	\item[Mappatura base:] \textit{\textbackslash process\textbackslash controller}
	\item[Attributi:]
	\item[Metodi:]\begin{itemize}
					\item 
				\end{itemize}
\end{description}
\paragraph{ApproveStepController}%----------------------------------------------------------------%
\begin{description}
	\item[Descrizione] Questa classe serve per fornire al \textit{process owner} i dati da approvare e per gestire quali passi siano stati approvati quali no, qualora un passo non venga approvato, verrà rimosso dal \textit{database};
	\item[Mappatura base:] \textit{\textbackslash approvedata}
	\item[Attributi:] 
	\item[Metodi:]\begin{itemize}
					\item +StepList GetStepToApprove(), il metodo gestisce una richiesta di tipo \textit{GET}, e restituirà un oggetto di tipo StepList contenente tutti i dati che richiedono approvazione;
					\item +void ApproveResponse(Step), il metodo gestisce una richiesta di tipo \textit{POST}, riceve un passo che ha subito la moderazione del \textit{process owner}, tale passo verrà eliminato dal database se il processowner lo ha rifiutato altrimenti verrà approvato definitivamente;  
				\end{itemize}
\end{description}