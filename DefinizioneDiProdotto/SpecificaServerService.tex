\subsubsection{Package com.sirius.sequenziatore.server.service}
\begin{figure}[H] \centering \includegraphics[width=%
\textwidth]
{./classi/server/serverservice.png} \caption{Diagramma package - \texttt{com.sirius.sequenziatore.server.service}}
\end{figure}
\paragraph{SignUpService}%--------------------------------------------------------%
\
\begin{figure}[H] \centering
\includegraphics[trim=0cm 0.8cm 0cm 0cm,clip=true,scale=0.75]%
{./classi/server/signupservice.png} \caption{Diagramma classe - \texttt{SignUpService}}
\end{figure}
\begin{itemize}
	\item \textbf{Descrizione: } Questa classe dovrà elaborare tutte le richieste di registrazione al sistema ricevute dal client, sarà incaricata di inserire i dati nel database.
	\item \textbf{Relazioni con altri componenti: }
	La classe utilizzerà le seguenti classi:
	\begin{itemize}
		\item \texttt{com.sirius.sequenziatore.server.model.UserDao;}
		\item \texttt{com.sirius.sequenziatore.server.model.User;}
	\end{itemize}
	\item \textbf{Attributi:}\begin{itemize}
					\item \texttt{-UserDao userDao}:\\
					oggetto usato per inserire il nuovo utente nel database;
	\end{itemize}
	\item \textbf{Metodi: }\begin{itemize}
					\item \texttt{+void registerUser(User toBeRegistered)}:\\
					 questo metodo gestirà una richiesta di tipo \textbf{POST} e dovrà lanciare un' eccezione di tipo \texttt{HttpError} qual' ora ci siano stati problemi nella registrazione;
				\end{itemize}
\end{itemize}