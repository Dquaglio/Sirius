\section{Processo di pianificazione}
\subsection{Software per la pianificazione}
Per il processo di pianificazione del progetto nonché gestione delle risorse è stato adottato il software GanttProject, software open source basato su piattaforma Java. Qui di seguito vengono elencate le principali caratteristiche che hanno portato alla scelta di questo strumento:
\begin{itemize}
\item Portabilità, essendo un software basato su Java;
\item Open-source;
\item Compatibile con MicrosoftProject;
\item Può generare grafici Work Breakdown Structure (WBS\ped{G});
\item Fornisce la possibilità di creare grafici di Gantt;
\item Può generare grafici Program Evalutation and Review Tecnique (PERT\ped{G});
\item In grado di gestire e generare grafici delle risorse assegnate.
\end{itemize}

\subsection{Procedura di pianificazione}
Il \textit{responsabile di progetto} per ogni attività indicata nel documento \textit{Piano di Progetto} dovrà creare un nuovo progetto seguendo la procedura qui descritta:

\begin{enumerate}
\item Inserire una milestone;
\item Inserire le attività da svolgere;
\item Inserire le rispettive sotto-attività;
\item Calcolare ed inserire i periodi di slack qualora fosse necessario;
\item Creare le risorse;
\item Assegnare le risorse create ad ogni attività;
\item Salvare la baseline\ped{G}.
\end{enumerate}

Sarà decisa a discrezione del \textit{Responsabile di Progetto} per ogni attività la possibilità di assegnare un surplus di ore, queste ore supplementari verranno scelte basandosi sulla criticità dell'attività considerata.

\begin{itemize}
\item Per le attività non critici non è previsto alcun surplus di ore;
\item Per le attività di media criticità il suplus di ore potrà essere del 15\%;
\item Per le attività di criticità massima il suplus di ore potrà essere del 30\%.
\end{itemize}
