\section{Processo di analisi}
\subsubsection{Classificazione dei requisiti}
Ogni requisito deve essere definito tramite  una descrizione testuale e un codice identificativo, classificante e univoco avente la seguente forma:

\begin{center}\{Tipologia\}\{Importanza\}\{Categoria\}\{Identificatore\}\end{center}


\begin{itemize}

\item Tipologia:
\begin{itemize}
\item \textbf{F:} requisito funzionale;
\item \textbf{Q:} requisito di qualità;
\item \textbf{V:} requisito di vincolo;
\end{itemize}

\item Importanza:
\begin{itemize}
\item \textbf{OB:} requisito obbligatorio;
\item \textbf{DE:} requisito desiderabile;
\item \textbf{OP:} requisito opzionale;
\end{itemize}


\item Categoria:
\begin{itemize}
\item \textbf{U:} requisito funzionale riguardante la parte utente;
\item \textbf{L:} requisito funzionale riguardante la parte utente autenticato;
\item \textbf{A:} requisito funzionale riguardante la parte amministratore;
\item vuoto se il requisito è di qualità oppure di vincolo;
\end{itemize}


\item Identificatore è un codice gerarchico composto da uno o più numeri separati da un punto, in cui l'ultimo numero è un identificatore incrementale intero.\\
La rimanente parte di codice viene utilizzata quando il requisito da definire è sotto-requisito di un altro, e identifica il requisito gerarchicamente superiore.

\end{itemize}

\subsubsection{Casi d'uso}
Per ciascun caso d'uso deve essere fornito un codice identificativo, una descrizione testuale e un diagramma UML.

\begin{flushleft}
Il codice identificativo deve rispettare la seguente forma:
\end{flushleft}

\begin{center}UC\{Tipologia\}\{Identificatore\}\end{center}
Tipologia può essere U, L o A, che stanno rispettivamente per utente non autenticato, utente autenticato e amministratore.\\
Identificatore è un codice gerarchico composto da uno o più numeri separati da un punto, in cui l'ultimo numero è un identificatore incrementale intero.
La rimanente parte di codice viene utilizzata quando il caso d'uso da definire è una specifica o estensione di un altro, e identifica il caso d'uso gerarchicamente superiore.

\begin{flushleft}
La descrizione deve contenere i seguenti dettagli:
\end{flushleft}

\begin{itemize}
\item Descrizione del caso d'uso;
\item Attori coinvolti;
\item Precondizione;
\item Scenario principale dello svolgersi degli eventi;
\item Senari alternativi;
\item Post-condizione.
\end{itemize}

\begin{flushleft}
Il diagramma deve rispettare le regole della notazione UML 2.x\ped{G}.
\end{flushleft}
\subsection{Strumento per il tracciamento}
Per il tracciamento è stato sviluppato un semplice programma denominato Sirius RTg. Questo programma è stato sviluppato utilizzando PHP\ped{G}, CSS\ped{G} ed HTML\ped{G}. Sirius RTg è attualmente un programma il cui sviluppo non è terminato, principalmente per permettere una aggiunta di funzionalità in caso di necessità. Sirius RTg, alla versione 1.4.0, fornisce le seguenti funzionalità:
\begin{itemize}
\item Inserimento requisito e relativo tracciamento;
\item Visualizzazione dello script per la tabella dei requisiti e relativo tracciamento.
\end{itemize}
\subsubsection{Inserimento requisito e relativo tracciamento}
Questa funzionalità è fornita all'esterno attraverso un'interfaccia scritta in HTML e CSS.\\
L'interfaccia è costituita da uno semplice \textit{form}, in cui è possibile inserire:
\begin{itemize}
\item Codice requisito;
\item Descrizione del requisito;
\item Categoria;
\item Importanza;
\item Tipo;
\item Relativi casi d'uso;
\item Relative fonti.
\end{itemize}
Ogni requisito deve avere obbligatoriamente definito il Tipo, l'Importanza, ed il Codice requisito. Il codice deve necessariamente identificare univocamente il requisito, altrimenti un uso ridondante di codici verrà notificato all'utente. Obbligatoria la categoria, in caso il requisiti sia della parte utente.
\subsubsection{Visualizza script}
Questa funzionalità serve per la stampa a video dei vari script in latex\ped{G} per i vari requisiti e il relativo tracciamento.\\ Visualizza script è composto dalle seguenti funzionalità:
	\begin{itemize}
		\item Visualizzazione script per Requisiti di tipo utente amministratore;
		\item Visualizzazione script per Requisiti di tipo utente-utente autenticato;
		\item Visualizzazione script per Requisiti di vincolo;
		\item Visualizzazione script per Requisiti di qualità;
		\item Visualizzazione script tracciamento Requisiti-uc;
		\item Visualizzazione script tracciamento Uc-requisiti.
	\end{itemize}
Ogni script stampato su video, segue le regole definite nelle \textit{Norme di Progetto}.\\
Anche se Visualizzazione script utente e Visualizzazione script utente autenticato sono due sotto-funzionalità distinte di Visualizza script, devono essere utilizzate assieme per produrre la tabella dei requisiti di tipo utente; infatti Visualizzazione script utente stampa l'intestazione della tabella e la parte dei requisiti utente, mentre Visualizzazione script utente autenticato stampa la parte dei requisiti utente autenticato e i comandi necessari per chiudere la tabella.\\
Tutti gli altri Script, invece possono essere usati singolarmente e a video, oltre al contenuto comparirà la relativa intestazione e chiusura della tabella.