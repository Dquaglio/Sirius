\section{Processo di analisi}
\subsubsection{Classificazione dei requisiti}
Ogni requisito deve essere definito tramite  una descrizione testuale e un codice identificativo, classificante e univoco avente la seguente forma:

\begin{center}\{Tipologia\}\{Importanza\}\{Identificatore\}\end{center}


\begin{itemize}

\item Tipologia:
\begin{itemize}
\item \textbf{F:} requisito funzionale;
\item \textbf{Q:} requisito di qualità;
\item \textbf{V:} requisito di vincolo;
\end{itemize}

\item Importanza:
\begin{itemize}
\item \textbf{OB:} requisito obbligatorio;
\item \textbf{DE:} requisito desiderabile;
\item \textbf{OP:} requisito opzionale;
\end{itemize}

\item Identificatore è un codice gerarchico composto da uno o più numeri separati da un punto, in cui l'ultimo numero è un identificatore incrementale intero.\\
La rimanente parte di codice viene utilizzata quando il requisito da definire è sotto-requisito di un altro, e identifica il requisito gerarchicamente superiore.

\end{itemize}

\subsubsection{Casi d'uso}
Per ciascun caso d'uso deve essere fornito un codice identificativo, una descrizione testuale e un diagramma UML.

\begin{flushleft}
Il codice identificativo deve rispettare la seguente forma:
\end{flushleft}

\begin{center}UC\{Tipologia\}\{Identificatore\}\end{center}
Tipologia può essere U o A, che stanno rispettivamente per Utente o Amministratore
dove Identificatore è un codice gerarchico composto da uno o più numeri separati da un punto, in cui l'ultimo numero è un identificatore incrementale intero.\\
La rimanente parte di codice viene utilizzata quando il caso d'uso da definire è una specifica o estensione di un altro, e identifica il caso d'uso gerarchicamente superiore.

\begin{flushleft}
La descrizione deve contenere i seguenti dettagli:
\end{flushleft}

\begin{itemize}
\item Descrizione del caso d'uso;
\item Attori coinvolti;
\item Precondizione;
\item Scenario principale dello svolgersi degli eventi;
\item Senari alternativi;
\item Postcondizione.
\end{itemize}

\begin{flushleft}
Il diagramma deve rispettare le regole della notazione \ped{G}UML 2.x.
\end{flushleft}
\subsection{Strumento per il tracciamento}