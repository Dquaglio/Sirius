\section{Documentazione}
\subsection{LaTeX}
\subsection{Template}
\subsection{Classificazione Documenti}
\subsection{Versionamento}
\subsection{Codifica}
\subsubsection{Codifica e convenzioni}
\begin{itemize}

\item Tutti i file contenenti documentazione dovranno essere conformi alla codifica UTF-8_G;

\item I file contenenti codice java_G  dovranno essere conformi alla codifica UTF-16_G;

\item Gli sviluppatori dovranno attenersi alle \textit{"Code Conventions for the Java Programming Language"} reperibili nel sito ufficiale \textit{oracle} al link:\\ \\
\centerline{ \url{http://www.oracle.com/technetwork/java/codeconv-138413.html}};

\item A discrezione del \textit{responsabile di progetto} sarà possibile effettuare delle modifiche alle convenzioni riportate.
\end{itemize}

\subsection{Struttura Documentazione}
\subsection{Struttura Documentazione del codice}
\subsection{Norme tipografiche}
\subsubsection{Generali}
\begin{itemize}
\item Ogni documento deve essere in lingua italiana, altre lingue possono essere utilizzate per riferirsi a termini tecnici informatici o in situazioni che lo richiedono strettamente;

\item Ogni documento deve essere grammaticalmente, sintatticamente e semanticamente corretto, cercando di essere meno verboso possibile;

\item Utilizzare il più possibile elenchi puntati invece di lunghe frasi.

\end{itemize}
\subsubsection{Punteggiatura}
\begin{itemize}
\item Non si usa mai un punto alla fine di un titolo: di capitolo, di paragrafo, di sottoparagrafo;

\item Ogni elemento di un elenco puntato termina con un punto e virgola, se è l'ultimo elemento con un punto;

\item Prima di ogni segno di punteggiatura non va mai messo uno spazio bianco, dopo invece lo spazio bianco va messo sempre;

\item Il testo racchiuso tra parentesi non deve aprirsi o chiudersi con un carattere di spaziatura ne terminare con un carattere di punteggiatura.

\end{itemize}
\subsubsection{Ortografia}

\begin{itemize}

\item Le lettere maiuscole vanno poste solo all'inizio di ogni elemento di un elenco puntato e dove lo prevede l'ortografia italiana (all'inizio di un periodo o dopo un segno di punteggiatura forte, cioè dopo il punto fermo, i puntini di sospensione, il punto esclamativo ed il punto interrogativo). È inoltre utilizzata l'iniziale maiuscola nel nome del team, del progetto, dei documenti, dei ruoli di progetto, delle fasi di lavoro.


\end{itemize}

\subsubsection{Stile}
\begin{itemize}
\item Se si devono elencare delle di istruzioni in serie o una divisione in paragrafi e sotto-paragrafi è necessario utilizzare un elenco numerato, altrimenti è preferibile un elenco puntato;

\item Il primo livello di profondità degli elenchi puntati è contrassegnato da un pallino nero pieno, il secondo da un trattino, il terzo da un asterisco;

\item Le date dovranno essere espresse nella forma \textbf{aaaa-mm-gg} secondo lo standard  \ped{G}\textbf{ISO G 8601:2004};

\item Gli orari dovranno essere espressi nella forma \textbf{hh:mm} secondo lo standard \textbf{ISO G 8601:2004};

\item \textbf{URL} ed indirizzi mail dovrano essere preceduto dal comando \LaTeX{} \verb!\url!;

\item Ogni prima occorrenza di una parola presente sul \textit{Glossario} sarà preceduta da pedice \ped{G}, se un termine contenuto nel glossario è composto da un insieme di parole sarà scritto in grassetto, in modo da marcarlo interamente;

\item Stile di testo:

\begin{itemize}

\item Il corsivo deve essere utilizzato nelle citazioni, nelle abbreviazioni, per il nome delle figure di rilievo (es. \textit{committente}) e per il nome dei documenti (es. \textit{Analisi dei requisiti});

\item il grassetto deve essere utilizzato per evidenziare (se si reputa necessario) le parole chiave ed i passaggi particolarmente rilevanti.

\end{itemize}

\end{itemize}

\subsection{Glossario}
\subsubsection{Parole notevoli}
\subsubsection{Acronimi}
\subsection{Codifica files}