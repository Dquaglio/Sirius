\section{Introduzione}

\subsection{Scopo del documento}
Questo documento è stato redatto al fine di definire e standardizzare tutte le norme che ogni componente del team \gruppo{} dovrà adottare durante il periodo di svolgimento del capitolato \progetto, commissionato dalla \textit{Zucchetti} S.P.A\ped{G}. In particolare si andranno di seguito ad elencare convenzioni e norme per:
\begin{itemize}
\item Le comunicazioni e le relazioni inter-personali;
\item Il processo di sviluppo dell'applicazione;
\item Il processo di pianificazione del progetto;
\item Il processo di \textit{Analisi};
\item La redazione e verifica della correttezza dei documenti;
\item L'utilizzo degli strumenti dell'ambiente di lavoro.
\end{itemize}

Ogni membro del gruppo \gruppo{} ha l'obbligo di visionare questo documento.
Tutti i componenti del gruppo sono tenuti a sottostare alle norme qui descritte di modo da garantire un corretto e coerente insieme di file prodotti, assicurando conseguentemente un continuo miglioramento dell'efficienza nello sviluppo.
Infine, qualora lo si ritenesse necessario, ogni membro del team potrà contattare l'\textit{Amministratore} per discutere norme esistenti o per valutarne l'adozione di nuove.

\subsection{Ref. Glossario}
Al fine di rendere più leggibile e comprensibile i documenti, i termini tecnici, di dominio, gli acronimi e le parole che necessitano di essere chiarite, sono riportate nel documento \Glossario{}.\\
Tutte le prime occorrenze di vocaboli presenti nel \textit{Glossario} devono essere seguite da una ``G'' maiuscola in pedice.