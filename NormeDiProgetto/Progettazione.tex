\section{Progettazione}
Questa sezione descrive le norme cui i progettisti sono tenuti ad attenersi durante la progettazione dell'applicazione.

\subsection{Tecnica di progettazione}


\subsection{Diagrammi}
Per quanto concerne i diagrammi sar� adottato il linguaggio di modellazione \textbf{UML} 2.0,
tramite questo linguaggio si andranno a definire:
\begin{itemize}
\item \textbf{Diagrammi dei package:} ossia elementi di raggruppamento di classi. Tali elementi dovranno figurare durante la progettazione generale e dovranno essere identificati univocamente al fine di stabilire come i suddetti interagiscono tra di loro.

\item \textbf{Diagrammi delle classi:} che consentono di descrivere tipi di entit�, le loro caratteristiche e come queste entit� interagiscono tra di loro.

\item \textbf{Diagrammi di sequenza:} #da vedere

\item \textbf{Diagrammi di attivit�:} atti a mostrare i flussi di attivit� che i vari tipi di utenti potranno compiere all'interno dell'applicazione.

\end{itemize}

\subsection{Design pattern}
Per ogni design pattern utilizzato sar� necessario definire:
\begin{itemize}
\item Una \textbf{descrizione generale} che riporta la struttura del design pattern scelto;
\item Una \textbf{motivazione} che descriva i vantaggi che ne comporta il suo uso;
\item Il \textbf{contesto applicativo} che associa al design pattern il contesto ove � stato adottato.
\end{itemize}

\subsection{Architettura generale}