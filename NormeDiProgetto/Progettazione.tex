\section{Progettazione}
Questa sezione descrive le norme cui i progettisti dovranno attenersi durante la stesura del documento:  \textit{Specifica tecnica} ed il documento \textbf{definizione di prodotto}. Queste norme sono dettate al fine di poter redigere un documento il più possibile formale e senza ambiguità.

\subsection{Specifica tecnica}
\subsubsection{Diagrammi}
Per quanto concerne i diagrammi sarà adottato il linguaggio \textit{Unified Modelling Language}(UML) 2.0,
tramite questo linguaggio si andranno a definire:
\begin{itemize}
\item \textbf{Diagrammi dei package}: ossia elementi di raggruppamento di classi. Tali elementi dovranno figurare durante la progettazione generale e dovranno essere identificati univocamente al fine di stabilire come i suddetti interagiscono tra di loro.

\item \textbf{Diagrammi di sequenza}: I quali andranno a descrivere come un gruppo di oggetti andranno a collaborare per implementare collettivamente un comportamento;

\item \textbf{Diagrammi di attività}: Atti a mostrare i flussi di attivit che gli utenti potranno percorrere durante l'uso dell'applicazione;

\item \textbf{Diagrammi delle classi}: I quali consentono di descrivere tipi di entità con le loro caratteristiche.
\end{itemize}

\subsubsection{Design pattern}
Per ogni design pattern utilizzato sarà necessario andare a definire i seguenti punti:
\begin{itemize}
\item Una \textbf{descrizione generale} che riporta la struttura del design pattern scelto;
\item Una \textbf{motivazione} che descriva i vantaggi che ne comporta il suo uso;
\item Il \textbf{contesto applicativo} che associa ai design pattern utilizzati il contesto dove sono stati adottati.
\end{itemize}

\subsubsection{Nomenclatura delle classi}
Ogni classe descritta nel documento specifica tecnica seguirà il seguente schema:
\begin{itemize}
\item \textbf{Nome}: enuncia il nome della classe che andrà descritta, devono essere obbligatoriamente in inglese e devono comparire con l'iniziale in maiuscolo.
\item \textbf{Package}:enuncia il pacchetto, ed i relativi sotto-pacchetti, all'interno dei quali è contenuta la classe di interesse.
Ogni package deve seguire la seguente notazione: Pack1::Pack2::..::Pack-n;
ove con:
\begin{itemize}
\item Pack1, si intende il package principale;
\item Pack(x)::Pack(y) indica che indica che y è sotto package di x;
\end{itemize}

\item \textbf{Descrizione}: deve contenere una breve ma significativa descrizione testuale riguardante l'utilizzo della classe;
\item \textbf{Relazione con altri componenti}: viene specificato se la classe di interesse ha relazioni con le classi di altri componenti.
\end{itemize}

\subsubsection{Tracciamento}

\subsection{Test} I progettisti avranno il compito di definire delle classi e dei test fittizi con lo scopo di valutare il lavoro svolto.
I test saranno suddivisi in:
\begin{itemize}
\item Test di integrazione;
\item Test di unità;
\end{itemize}

\subsubsection{Tracciamento dei test}

\paragraph{Tracciamento requisiti-componenti}
\begin{itemize}
\item \textbf{Requisito}:
\item \textbf{Descrizione req.}:
\item \textbf{Nome componente}:
\end{itemize}

\paragraph{Tracciamento componenti-requisiti}
La struttura della tabella sarà la seguente:
\begin{itemize}
\item \textit{Requisito}: contenente il codice univoco e classificante del requisito;
\item \textit{Componente}: contente il nome del package ed il nome della classe.
\end{itemize}

\paragraph{Test di sistema ed integrazione}
Il tracciamento dei test sarà inserito in forma tabellare nel documento \PianoDiQualifica, e dovrà rispettare il seguente stampo:
\begin{itemize}
\item \textbf{Codice test}: il quale dovrà essere obbligatoriamente univoco ed atto ad identificare il test;
\item \textbf{Descrizione}: la quale specifica lo scopo del test;
\item \textbf{Requisito annesso}: che specifica il requisito cui il test fa riferimento;
\item \textbf{Stato}: che riporta se il test è stato effettuato.
\end{itemize}

\subsection{Definizione di prodotto}
Ci poniamo di implementare questa sezione nel momento in cui andremo a descrivere l'architettura di basso livello del sistema.