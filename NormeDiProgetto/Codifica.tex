\section{Codifica files}
\subsection{Convenzioni di codifica}
Al fine di produrre codice ordinato e leggibile, in modo da semplificare il più possibile l'attività di manutenzione, per quanto concerne la programmazione \textit{Java\ped{G}}, si adotteranno le norme imposte dalla \textit{Java code conventions}, reperibili all'indirizzo:
\begin{center}
\href{http://www.oracle.com/technetwork/java/codeconvtoc-136057.html}{http://www.oracle.com/technetwork/java/codeconvtoc-136057.html}
\end{center}
Variazioni e modifiche a queste convenzioni possono essere richieste all'\textit{Amministratore}, allegandone la motivazione.
Se l'\textit{Amministratore} riterrà opportune le variazioni presentate, sarà tenuto a notificarlo al team seguendo le convenzioni imposte dalle \textit{Norme di Progetto}.

\subsection{Nomi}
Sarà adottata la notazione \textit{CamelCase} al fine di identificare facilmente il nome di variabili, classi, metodi, funzioni, interfacce. Più specificatamente bisognerà rispettare le seguenti regole:
\begin{itemize}
\item \textbf{variabili, metodi, funzioni, interfacce}:dovranno avere la prima lettera minuscola;
\item \textbf{classi}: dovranno avere la prima lettera maiuscola.
\end{itemize}
Inoltre sarà obligatorio utitilizzare la lingua \textbf{inglese} per assegnare i suddetti nomi. 
Inoltre è opportuno ma non obbligatorio che i nomi (di variabili, metodi, etc..) siano esplicativi rispetto al ruolo che assumono nel contesto di applicazione.

\subsection{Commenti ed intestazioni}

Ogni file contenente codice dovrà essere provvisto di un intestazione che rispetta la seguente forma:
\\
\\
\textcolor{green}{
\textbf{file}: che riporta il nome del file; \\
\textbf{author}: che riporta il nome dell'autore;\\
\textbf{date}: che riporta la data di creazione; \\
\textbf{lastModified}: che riporta la data di ultima modifica;\\
\textbf{brief}: che descrive brevemente lo scopo e contenuto del file.\\
}

Le classi dovranno obbligatoriamente essere provviste di commenti che contengono:
\\
\\
\textcolor{green}{
\textbf{class}: nome della classe;\\
\textbf{brief}: breve descrizione della classe.\\
}


Per quanto concerne i metodi, essi dovranno essere provvisti di commenti che si adeguano alla seguente forma:
\\
\\
\textcolor{green}{
\textbf{brief}: breve descrizione del compito del metodo;\\
\textbf{param}: che comparirà tante volte quanti sono i parametri in input al metodo, e ne riporterà il nome ed il tipo;\\
\textbf{return}: nome e tipo del valore ritornato dalla funzione.\\}

Infine per le interfacce la forma sarà la seguente:
\\
\\
\textcolor{green}{
\textbf{brief}: breve descrizione dello scopo dell'interfaccia.\\}

Qualora fosse impossibile utilizzare la tecnica del \textit{refractoring}\ped{G} per ristrutturare il codice, nel qualcaso si trattasse di codice particolarmente difficile da comprendere, è possibile dedicare un commento aggiuntivo:
\\
\\
\textcolor{green}{spiegazione approfondita: ...\\}\\
al fine di facilitarne la comprensione.
