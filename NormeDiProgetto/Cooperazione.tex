\section{Cooperazione}


\subsection{Comunicazioni}
\subsubsection{Comunicazioni interne}
Le comunicazioni interne generiche tra i membri del gruppo Github si svolgono attraverso la mailing list privata (gruppo Google) e facebook.

Le comunicazioni e il coordinamento formali (issue tracking, formazione, ecc.) invece si appoggiano alla piattaforma \url{https://swesirius.teamworkpm.net}
\subsubsection{Comunicazioni esterne}
Le comunicazioni formali esterne avvengono attraverso l’indirizzo email: \url{swesirius@gmail.com}

\subsection{Riunioni}
Con riunioni si intende qualsiasi incontro fra Proponente / Committente e un gruppo di rappresentanza (composto da almeno la maggioranza assoluta) del gruppo di progetto.

\subsubsection{Richiesta}

La richiesta di indire una riunione esterna può essere avanzata da qualsiasi componente del gruppo; e compito del Responsabile contattare ed organizzare l'evento con l'interessato. 
Una volta pianificato con il resto del gruppo secondo i passi indicati alla sezione 3.1 con tag [Riunione Esterna].

\subsubsection{Esito}

Ad ogni incontro il Responsabile ha il compito di stilare un verbale (cap. 5.8) che evidenzia i chiarimenti emersi durante l'incontro.

\subsection{Verbali incontri}

Per Verbali degli incontri si intendono quei documenti redatti dal Responsabile di Progetto in occasione di incontri esterni ed interni.
Per tali documenti e prevista una sola stesura in quanto promemoria dell'incontro avvenuto.
Non e perciò previsto il versionamento.
I Verbali degli incontri devono essere denominati secondo il seguente criterio:

\begin{center}
Verbale\{tipo incontro\}\{data incontro\}
\end{center}

\begin{itemize}
\item tipo incontro: indica il tipo di incontro e I (interno) o E (esterno);
\item data incontro: indica la data in cui  e stato tenuto l'incontro seguendo il formato.
\end{itemize}
  
  
La prima pagina di ogni verbale deve obbligatoriamente contenere i seguenti campi, nell'ordine indicato:

\begin{itemize}
\item data;
\item luogo: espresso nel formato;
\item ora di ritrovo: espressa nel formato;
\item Ora dell'incontro: fhhg:fmmg;
\item Durata dell'incontro: fxg min;
\item partecipanti interni: lista degli appartenenti Sirius presenti all'incontro;
\item partecipanti esterni: rappresentanti della ditta/e per ogni partecipante indicare il campo ruolo che rappresenta il ruolo assunto all'interno dell'azienda a cui fanno capo; nel caso il partecipante sia il Committente, il campo viene compilato con Committente;
\item contenuto: la decisione del formato  e lasciata al Responsabile di Progetto, il quale adotta lo stile più consono in base al tipo di incontro svolto;
\item firme: devono essere comprese quelle di tutti i partecipanti del gruppo Sirius a conferma della presa visione del documento.
\end{itemize}

