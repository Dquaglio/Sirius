\section{Cooperazione}
\subsection{Comunicazioni interne}
L'apparato di comunicazioni interne ufficiali sarà gestito tramite una \textit{mailing list\ped{G}} basata su \textit{Google Groups}.\\
Il nome della \textit{mailing list} è Sirius esattamente come il nome del gruppo(\gruppo).\\ Vige l'obbligo di utilizzare la \textit{mailing list} solamente per le comunicazioni interne ufficiali, così da evitare intasamenti superflui che graverebbero sul lavoro di verbalizzazione delle comunicazioni di rilevo, in quanto renderebbero più complessa ed inutilmente lunga l'estrazione delle informazioni utili. Tuttavia è preferibile che la comunicazione tra i vari componenti del team avvenga principalmente durante gli incontri che si terranno in un luogo fisico comune (sezione 2.2, Riunioni interne).\\
Al fine di facilitare anche le comunicazioni informali, sono stati adottati due strumenti di \textit{instant messaging\ped{G}} e videoconferenza quali Skype e Google Plus ed un strumento di \textit{web storage} per lo scambio di dati ufficiosi (sezione 8.1, Gestione condivisione file).

\subsection{Riunioni interne}
Le riunioni del gruppo \gruppo{} avranno una frequenza almeno settimanale. Il giorno della settimana, il luogo e l'ora in cui riunirsi ufficialmente sarà deciso dal \textit{Responsabile di Progetto} su consultazione degli altri membri del team e sarà comunicato tramite il gruppo \textit{Google Groups} Sirius. Chiunque non abbia la possibilità di essere presente fisicamente nel luogo della riunione, dovrà possibilmente restare in contatto con il nucleo dei membri mediante videoconferenza.\\
Qualunque membro del team può richiedere al \textit{Responsabile di Progetto} di indire una riunione allegandone l'argomento di discussione ed una sua breve descrizione, a seguito della comunicazione il \textit{Responsabile di Progetto} deciderà se indire la suddetta riunione generale, cioè con obbligatoria presenza di tutti i membri del gruppo. Qualsiasi riunione surplus a quella settimanale deve essere indetta con almeno 2 giorni di anticipo, in modo da verificare la disponibilità del gruppo.\\
Se dovessero essere necessarie riunioni che non richiedono la presenza del gruppo nella sua totalità, ogni membro potrà presentare la richiesta di ritrovo tramite l'apposita \textit{mailing list} (sezione 2.1, Comunicazioni interne), richiedendo la disponibilità degli specifici membri del team che riterrà necessari, questo poiché è auspicabile che alcune figure come ad esempio \textit{Progettista} ed \textit{Analista} collaborino tra di loro frequentemente.\\
Le riunioni che non coinvolgono interamente il team non necessitano dell'approvazione del \textit{Responsabile di Progetto} in modo da ridurre il suo carico di lavoro, nonostante questo le decisioni effettuate durante queste discussioni inter-membri dovranno comunque essere verbalizzate.

\subsection{Comunicazioni Esterne}
Per le comunicazioni esterne di ogni tipo è stato creato in indirizzo \textit{e-mail} del team:
\begin{center}
\href{swesirius@gmail.com}{swesirius@gmail.com}
\end{center}

Il \textit{Responsabile di Progetto} rappresenta il team stesso, sarà quindi incaricato di mantenere i contatti con proponente, committente, ed eventuali altre figure non facenti parte del nucleo del \textit{team}, tramite questo indirizzo \textit{e-mail}, inoltre sarà sempre parte del suo compito aggiornare i membri del gruppo stesso riguardo le corrispondenze pervenute attenendosi alle istruzioni della sezione 2.1, Comunicazioni interne.

\subsection{Incontri esterni}
Il \textit{Responsabile di Progetto} ha inoltre l'onere di organizzare eventuali incontri esterni (per chiarificazioni o quant'altro) con \textit{Proponente} o \textit{Committente/i}. Ogni membro del gruppo può richiedere un incontro esterno al \textit{Responsabile di Progetto}, presentando una motivazione valida.
Infine, il \textit{Responsabile di Progetto} dopo aver valutato personalmente la proposta, dovrà presentarla al gruppo (con allegata la motivazione ed il nome di chi l'ha richiesta) quindi, per l'approvazione definitiva di quest'ultima, almeno due membri escluso l'artefice dovranno dare ulteriore conferma e disponibilità, in caso contrario la proposta sarà bocciata.