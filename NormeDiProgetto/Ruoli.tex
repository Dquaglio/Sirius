\section{Ruoli di progetto}
\subsection{Responsabile di progetto}
Il \textit{responsabile di progetto} è colui che ha potere decisionale sul progetto, l'entità quindi che ha l'incarico di approvare le scelte effettuate ed il lavoro svolto. Questo potere decisionale si ripercuote sulla responsabilità che detiene nel presentare il risultato del prodotto creato al proponente.
Il responsabile di progetto deve inoltre occuparsi della gestione (assegnazione, modifica) dei \textit{ticket}, ed assicurarsi che i verificatori seguano sistematicamente le regole imposte dalle \textit{Norme di progetto}.
Riassumendo, tale figura ha responsabilità sotto il profilo di:
\begin{itemize}
\item Approvazione dell'offerta economica;
\item Approvazione dei documenti;
\item Analisi e gestione delle risorse;
\item Analisi e gestione dei rischi;
\item Pianificazione del piano di lavoro;
\item Coordinazione del team;
\item Controllo del regolare svolgimento delle attività;
\item Evitare conflitti di interesse tra redattori e verificatori.
\end{itemize}
Per ciò che concerne la documentazione, il responsabile di progetto è colui che deve redigere il \textit{Piano di progetto} e collaborare nella stesura del \textit{Piano di qualifica}.
\subsection{Amministratore}
La figura dell'\textit{Amministratore} è responsabile per tutto ciò che è pertinente all'efficienza ed operatività dell'ambiente di lavoro. 
I suoi compiti di primaria importanza sono qui di seguito riportati:
\begin{itemize}
\item Gestione del versionamento del prodotto;
\item Ricerca degli strumenti più adatti allo svolgimento del progetto;
\item Automatizzazione delle attività che possono essere risolte tramite strumenti e procedure automatiche, in modo da snellire il carico di lavoro ad personam;
\item Creare o ricercare strumenti che possano controllare/monitorare la qualità del prodotto;
\item Normare le procedure standard di pianificazione del progetto, coordinazione del team, redazione dei documenti, produzione del codice.

L'amministratore deve redigere interamente le \textit{Norme di progetto} nonché collaborare nel \textit{Piano di qualifica}.
\end{itemize}
\subsection{Analista}
L'\textit{Analista} è competente nell'attività di analisi ed astrazione dei requisiti di progetto.
Di seguito vengono riportate le sue mansioni principali:
\begin{itemize}
\item Astrarre i requisiti dal problema in modo da creare una specifica di progetto comprensibile dal progettista, dal proponente e dal committente;
\item Comprendere i requisiti meno espliciti del problema da affrontare.
\end{itemize}

I documenti: \textit{Analisi dei Requisiti} e \textit{Studio di Fattibilità} devono essere stilati dall'\textit{Analista}.
Nel \textit{Piano di qualifica} dovra' illustrare il livello di qualita' richiesta e le procedure da attuare per raggiungerla.

\subsection{Verificatore}
Il verificatore e' colui che deve effettuare l'attivita' di verifica. Questa figura deve attenersi alle regole imposte dalle \textit{Norme di progetto}, e tramite l'ausilio degli strumenti illustrati nel \textit{Piano di qualifica} avra' il compito di assicurare che:
\begin{itemize}
\item Le attivita' svolte siano coerenti agli standard adottati.
\item La documentazione sia conforme alle \textit{Norme di progetto};
\end{itemize}

\subsection{Progettista}
Il progettista è colui che ha la responsabilità sull'attività di progettazione. I suoi compiti possono essere riassunti come segue:
\begin{itemize}
\item Agire in modo che il progetto sia sviluppato tramite tecnologie al più possibile stabili e note;
\item Agire in modo che il progetto sia sviluppato seguendo soluzioni (come ad esempio soluzioni progettuali) ottimizzate e note;
\item Creare una soluzione progettuale adeguata, ossia comprensibile ed attuabile;
\item Agira sulle scelte progettuali in modo da sviluppare un prodotto facilmente manutenibile.
\end{itemize}
Tale ruolo avrà il compito di redigere la \textit{Specifica Tecnica}, la \textit{Definizione di Prodotto} e la parte inerente alla metrica di verifica nel \textit{Piano di Qualifica}.
\subsection{Programmatore}
Come deducibile è la figura che si occuperà dell'attività di codifica. Le principali mansioni di questo ruolo si riassumono nei seguenti punti:
\begin{itemize}
\item Implementare le soluzioni progettuali specificate dal \textit{Progettista}, senza discuterle o senza prendere alcuna iniziativa personale.
\item Scrivere codice manutenibile rispettando gli standard imposti per la codifica;
\item Documentare tutto il codice generato, in modo chiaro e coinciso;
\item Implementare i test per la verifica e validazione del codice.
\end{itemize}
Il \textit{Programmatore} si occuperà infine di scrivere il \textit{Manuale Utente}.