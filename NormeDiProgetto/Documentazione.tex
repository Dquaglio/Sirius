\section{Processo di documentazione}
\subsection{Template}
Ogni documento dovrà essere generato includendo il \textit{template} \LaTeX presente nella cartella "Modello".
Questo modello è stato creato prima dell'inizio della redazione di ogni altro documento del team \gruppo, la sua modifica può avvenire solo presentando all'\textit{Amministratore} una richiesta formale, allegandone la motivazione ed il tipo di modifica richiesta. Se l'\textit{Amministratore} riterrà opportuno effettuare il cambiamento, prima di apportare la modifica dovrà avvertire l'intero team al fine di evitare disguidi.

\subsection{Classificazione documenti}
\subsubsection{Documenti formali}
Sono catalogati come formali tutti i documenti approvati dal \textit{Responsabile di Progetto}, ovvero i documenti ritenuti pronti per essere visionati dal committente. Tali documenti, prima di raggiungere l'approvazione dovranno aver superato con successo la procedura di verifica e validazione riportata nel \textit{Piano di Qualifica}.
\subsubsection{Documenti informali}
Tutti i documenti che non sono stati approvati dal \textit{Responsabile di Progetto} sono da ritenersi informali, e di utilizzo esclusivamente interno. Tutti i documenti non versionati sono da ritenersi non ufficiali.

\subsection{Versionamento documenti}
Il versionamento di tutta la documentazione del gruppo \gruppo è stato organizzato secondo le seguenti convenzioni:
\begin{itemize}
\item Il numero di versionamento deve essere nella forma:

\begin{center}
\textbf{X}, \textbf{Y}, \textbf{Z}
\end{center}

con \textbf{X}, \textbf{Y}, \textbf{Z} numeri interi non negativi;
\item Tutti gli elementi devono salire di una sola unità alla volta.
\end{itemize}

Di seguito vengono inoltre riportati i significati che possono assumere le variazioni della versione del documento:
\begin{itemize}
\item La \textbf{X} rappresenta il numero di uscite formali del documento, ogni qual volta un documento verrà pubblicato il valore della cifra \textbf{Y} e della cifra \textbf{Z} verrà azzerato. Riportando quanto detto più precisamente:
\begin{enumerate}
\item X assumerà il valore: 1, alla \textbf{revisione dei requisiti};
\item X assumerà il valore: 2, alla \textbf{revisione di progettazione};
\item X assumerà il valore: 3, alla \textbf{revisione di qualifica};
\item X assumerà il valore: 4, alla \textbf{revisione di accettazione}.
\end{enumerate}
\item La \textbf{Y} rappresenta il numero di \textit{push} effettuati sul branch master in GitHub (sezione 8.2.1, GitHub), ossia il numero di volte in cui sono state compiute importanti modifiche al documento.
\item La \textbf{Z} rappresenta il numero di modifiche minori apportate al documento durante la fase di sviluppo.
Aumenta al termine di ogni sessione di lavoro sul documento.
\end{itemize}

Ogni documento formale riporterà un diario delle modifiche contenente le trasformazioni più rilevanti che ha attraversato sotto forma tabellare.
Una nuova riga di questa tabella sarà aggiunta solo nel momento in cui saranno la \textbf{X} o la \textbf{Y} ad aumentare.

\subsection{Struttura Documentazione}
\subsubsection{Header}
Ogni pagina esclusa la prima presenta un header raffigurante il logo del gruppo sulla sinistra, mentre sulla destra il nome del team ed il nome del progetto.
\subsubsection{Footer}
Ogni pagina esclusa la prima presenta un footer riportante il nome del documento corredato della versione sulla sinistra, mentre sulla destra il numero della pagina.
Per il numero di pagina delle prime quattro facciate saranno utilizzati i numeri romani, a seguire invece verranno utilizzati i numeri occidentali.
\subsubsection{Prima pagina}
La prima pagina di ogni documento conterrà:
\begin{itemize}
\item Il logo del \textit{team}, riportante la scritta \gruppo;
\item Il titolo del progetto;
\item Il nome del documento e la sua versione;
\item Il nome del corso;
\item L'anno di sviluppo del progetto;
\end{itemize}

\subsubsection{Seconda pagina}
La seconda pagina di ogni documento conterrà:
\begin{itemize}
\item  Informazioni sul documento come segue:
\begin{itemize}
\item Titolo del documento;
\item Data di creazione;
\item Versione attuale;
\item Utilizzo, che specifica se il documento è per utilizzo interno o esterno;
\item Nome file;
\item Redazione;
\item Revisione;
\item Approvazione;
\item Distribuito da, a cui seguirà il nome del gruppo.
\end{itemize}
\item Un sommario riportante una breve descrizione;
\end{itemize}

\subsubsection{Terza pagina}
La terza pagina di ogni documento conterrà:
\begin{itemize}
\item Un diario delle modifiche apportate al documento, dall'inizio fino alla versione corrente.
\end{itemize}

\subsubsection{Quarta pagina}
La quarta pagina di ogni documento ne riporterà l'indice, è possibile che l'indice si estenda per più di una singola pagina. 

\subsection{Norme tipografiche}
\subsubsection{Generali}
\begin{itemize}
\item Ogni documento deve essere in lingua italiana, altre lingue possono essere utilizzate per riferirsi a termini tecnici informatici o in situazioni che lo richiedono strettamente;

\item Ogni documento deve essere grammaticalmente, sintatticamente e semanticamente corretto, cercando di essere meno verboso possibile;

\item Utilizzare il più possibile elenchi puntati invece di lunghe frasi.

\end{itemize}
\subsubsection{Punteggiatura}
\begin{itemize}
\item Non si usa mai un punto alla fine di un titolo: di capitolo, di paragrafo, di sotto-paragrafo;

\item Ogni elemento di un elenco puntato termina con un punto e virgola, se è l'ultimo elemento con un punto;

\item Prima di ogni segno di punteggiatura non va mai messo uno spazio bianco, dopo invece lo spazio bianco va messo sempre;


\item Il testo racchiuso tra parentesi non deve aprirsi o chiudersi con un carattere di spaziatura ne terminare con un carattere di punteggiatura.

\end{itemize}
\subsubsection{Ortografia}

\begin{itemize}

\item Le lettere maiuscole vanno poste solo all'inizio di ogni elemento di un elenco puntato e dove lo prevede l'ortografia italiana (all'inizio di un periodo o dopo un segno di punteggiatura forte, cioè dopo il punto fermo, i puntini di sospensione, il punto esclamativo ed il punto interrogativo). È inoltre utilizzata l'iniziale maiuscola nel nome del team, del progetto, dei documenti, dei ruoli di progetto.


\end{itemize}

\subsubsection{Stile}
\begin{itemize}
\item Se si devono elencare delle di istruzioni in serie o una divisione in paragrafi e sotto-paragrafi è necessario utilizzare un elenco numerato, altrimenti è preferibile un elenco puntato;

\item Il primo livello di profondità degli elenchi puntati è contrassegnato da un pallino nero pieno, il secondo da un trattino, il terzo da un asterisco;

\item Le date dovranno essere espresse nella forma \textbf{aaaa-mm-gg} secondo lo standard  \ped{G}\textbf{ISO G 8601:2004};

\item Gli orari dovranno essere espressi nella forma \textbf{hh:mm} secondo lo standard \textbf{ISO G 8601:2004};

\item \textbf{URL} ed indirizzi mail dovrano essere preceduto dal comando \LaTeX \verb+ \+url;

\item Ogni prima (e possibilmente anche successiva) occorrenza di una parola presente sul \textit{Glossario} sarà seguita da pedice \ped{G}.

\item Stile di testo:

\begin{itemize}

\item Il corsivo deve essere utilizzato nelle citazioni, nelle abbreviazioni, per il nome delle figure di rilievo (es. \textit{committente}, per termini stranieri, \textit{Responsabile di Progetto}) e per il nome dei documenti (es. \textit{Analisi dei requisiti});

\item il grassetto deve essere utilizzato per evidenziare (se si reputa necessario) le parole chiave ed i passaggi particolarmente rilevanti.

\end{itemize}

\end{itemize}
\subsection{Calcolo indice di Gulpease}
In ogni documento redatto il verificatore dovrà calcolare l'indice di Gulpease, ossia il valore di leggibilità del documento.
Per raggiungere il seguente scopo è disponibile uno script online, reperibile al sito:\\
\\
\href{http://xoomer.virgilio.it/roberto-ricci/variabilialeatorie/esperimenti/leggibilita.htm}{http://www.xoomer.virgilio.it/roberto-ricci/variabilialeatorie/esperimenti/leggibilita.htm}
\\ \\
Questo script già esistente è stato verificato prima di essere adottato, in modo da scongiurare il rischio di incompatibilità tra i documenti redatti e la forma che doveva avere l'input per lo script.
Se l'indice risultante di un documento si troverà in un range compreso tra lo 0 ed il 40, sarà necessario ricercare nel testo frasi troppo lunghe e complesse per reimpostarle.


\subsection{Glossario}
Durante la stesura di un documento, ogni qual volta il redattore riterrà necessario chiarire il significato di un termine utilizzato sarà tenuto ad aggiungerlo nel \textit{glossario}. Vige la regola di inserire i nuovi termini sul glossario parallelamente alla stesura del documento su cui si sta lavorando.\\
Il \textit{glossario} sarà strutturato seguendo questo schema:
\begin{itemize}
\item Nel file \LaTeX ogni parola sarà contenuta nel \textbf{tag}: "elemento";
\item A seguire, andando a capo-riga, sarà riportata la descrizione del termine.
\end{itemize}
Il \textit{glossario} sarà inoltre suddiviso in due sezioni:
\begin{itemize}
\item Termini;
\item Acronimi.
\end{itemize}
Termini ed acronimi dovranno essere necessariamente elencati in ordine alfabetico, la definizione dovrà essere breve ed esplicativa, inoltre sempre la definizione non potrà iniziare con una \textbf{E accentata}.