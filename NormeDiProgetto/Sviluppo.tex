\section{Norme di sviluppo}
\subsection{Analisi dei requisiti}
I requisiti del prodotto software devono essere specificati nel documento \emph{Analisi dei Requisiti}, la cui stesura deve rispettare le fasi descritte in questa sezione.

\subsubsection{Classificazione dei requisiti}
Ogni requisito deve essere definito tramite  una descrizione testuale e un codice identificativo, classificante e univoco avente la seguente forma:

\begin{center}\{Tipologia\}\{Importanza\}\{Categoria\}\{Identificatore\}\end{center}


\begin{itemize}

\item Tipologia:
\begin{itemize}
\item \textbf{F:} requisito funzionale;
\item \textbf{Q:} requisito di qualità;
\item \textbf{V:} requisito di vincolo;
\end{itemize}

\item Importanza:
\begin{itemize}
\item \textbf{OB:} requisito obbligatorio;
\item \textbf{DE:} requisito desiderabile;
\item \textbf{OP:} requisito opzionale;
\end{itemize}


\item Categoria:
\begin{itemize}
\item \textbf{U:} requisito per la parte utente;
\item \textbf{L:} requisito per la parte utente autenticato;
\item \textbf{A:} requisito per la parte utente amministratore;
\end{itemize}

\item Identificatore è un codice gerarchico composto da uno o più numeri separati da un punto, in cui l'ultimo numero è un identificatore incrementale intero.\\
La rimanente parte di codice viene utilizzata quando il requisito da definire è sotto-requisito di un altro, e identifica il requisito gerarchicamente superiore.

\end{itemize}

\subsubsection{Casi d'uso}
Per ciascun caso d'uso deve essere fornito un codice identificativo, una descrizione testuale e un diagramma UML.

\begin{flushleft}
Il codice identificativo deve rispettare la seguente forma:
\end{flushleft}

\begin{center}UC\{Categoria\}\{Identificatore\}\end{center}
Categoria, come per i requisiti,può essere U,L o A, che stanno rispettivamente per Utente, Utente Autenticato, Utente amministratore; l'uso di categoria è dettato dalla necessità di fornire più ordine e facilitare la comprensione degli use case, nella fattispecie, per facilitare l'individuazione a cui è rivolta la funzionalità.
dove Identificatore è un codice gerarchico composto da uno o più numeri separati da un punto, in cui l'ultimo numero è un identificatore incrementale intero.\\
La rimanente parte di codice viene utilizzata quando il caso d'uso da definire è una specifica o estensione di un altro, e identifica il caso d'uso gerarchicamente superiore.

\begin{flushleft}
La descrizione deve contenere i seguenti dettagli:
\end{flushleft}

\begin{itemize}
\item Descrizione del caso d'uso;
\item Attori coinvolti;
\item Precondizione;
\item Scenario principale dello svolgersi degli eventi;
\item Senari alternativi;
\item Postcondizione.
\end{itemize}

\begin{flushleft}
Il diagramma deve rispettare le regole della notazione \ped{G}UML 2.x.
\end{flushleft}

\subsubsection{Tracciamento}
Ogni requisito deve essere associato ad una fonte di provenienza tra le seguenti:

\begin{itemize}
\item Capitolato;
\item Verbale;
\item Interna.
\end{itemize}
Inoltre, per ciascun requisito, devono essere specificati i casi d'uso di deduzione.\\
Ad ogni caso d'uso deve corrispondere almeno un requisito.

\subsection{Codifica}
\subsubsection{Codifica e convenzioni}
\begin{itemize}

\item Tutti i file contenenti documentazione dovranno essere conformi alla codifica \ped{G}UTF-8;

\item I file contenenti codice \ped{G}java  dovranno essere conformi alla codifica \ped{G}UTF-16;

\item Gli sviluppatori dovranno attenersi alle \textit{"Code Conventions for the Java Programming Language"} reperibili nel sito ufficiale \textit{oracle} al link:\\ \\
\centerline{ \url{http://www.oracle.com/technetwork/java/codeconv-138413.html}};

\item A discrezione del \textit{responsabile di progetto} sarà possibile effettuare delle modifiche alle convenzioni riportate.
\end{itemize}
