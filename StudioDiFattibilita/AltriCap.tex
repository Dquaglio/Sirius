
\section{Altri capitolati}
\subsection{Premessa}
Di seguito riportiamo un breve elenco dei requisiti rilevati dalla lettura dei capitolati e una breve analisi che evidenzia le motivazioni che hanno portato all'esclusione degli stessi.\\
\subsection{C1 MaaP Analisi}
Elenco generico dei requisiti richiesti dal capitolato:
\begin{itemize} 
\item Utilizzo di strumenti open source\ped{G};
\item Gestione di database\ped{G}, e conoscenza di database non relazionali;
\item Conoscenza di base del \textbf{mondo \textit{business}}, per creare report adatti alla categoria dell'utilizzatore finale;
\item Linguaggio di programmazione per la gestione dei dati Javascript\ped{G};
\item Il capitolato prevede la realizzazione lato server, lato client\ped{G}, creazione dinamica di pagine web, modulo per l'interfacciamento al database e recupero dei dati.
\end{itemize}
Dall'analisi del capitolato esso risulta essere un progetto interessante per quanto riguarda la strumentazione da utilizzare, ma allo stesso tempo le risorse del team \textit{Sirius} non sarebbero adatte alla mole di lavoro richiesta. Dallo studio si può prevedere che le ore di ricerca di strumenti open source\ped{G}, creazione di conoscenze riguardanti database non relazionali, unite alla realizzazione di tutto il pacchetto richiesto dal cliente, avrebbero portato ad un sovraccarico di lavoro.
\\
\subsection{C2 Ring Analisi}
Elenco generico dei requisiti richiesti dal capitolato:
\begin{itemize} 
\item Conoscenze di biologia;
\item Implementare algoritmi che richiamano la teoria dei grafi;
\item Linguaggio di programmazione C++\ped{G} o Java\ped{G};
\item Interfacciamento con software specifici inerenti al settore di utilizzo.
\end{itemize}
Il capitolato presentato riguarda la realizzazione di un'interessante applicativo per la gestione delle strutture proteiche. La scelta di declinare questo capitolato è stata dettata principalmente dalle nulle conoscenze di chimica che risultano fondamentali per lo sviluppo del progetto.\\
\subsection{C3 Romeo Analisi}
Elenco generico dei requisiti richiesti dal capitolato:
\begin{itemize} 
\item Elaborazione immagini sia bidimensionali che tridimensionali;
\item Algoritmi per l'estrazione di dati da immagini;
\item Studio del protocollo di elaborazione dati.
\end{itemize}
L'utilizzo di strumenti informatici come ausilio delle strutture mediche è sicuramente un aspetto interessante da sviluppare. Nel capitolato sopracitato gli strumenti informatici avrebbero coadiuvato gli esami clinici restituendo un risultato sotto forma di immagine, questa sarebbe stata il punto di partenza per la prima analisi automatizzata atta ad evidenziare determinate specifiche. Le elaborazioni anche se elementari di immagini richiedono tempo ed algoritmi ottimizzati in quanto la mole di dati da elaborare potrebbe essere molto elevata. Lo studio e la ricerca di questi algoritmi di elaborazione avrebbe richiesto troppo tempo in relazione alla tipologia di gruppo. Anche se le conoscenze mediche sarebbero passate in secondo piano in quanto la richiesta prevedeva lo sviluppo di un solo algoritmo di elaborazione, il team \textit{Sirius} ha comunque deciso di declinare questo capitolato.   \\
\subsection{C5 Sgad Analisi}
Elenco generico dei requisiti richiesti dal capitolato:
\begin{itemize} 
\item Architettura distribuita;
\item Conoscenze di grafica per la creazione di gioco;
\item Vincoli precisi riguardo i giocatori;
\item Utilizzo database.
\end{itemize}
Lo studio di architetture distribuite avrebbe portato ogni componente del gruppo ad acquisire conoscenze importanti circa gli aspetti analizzati e specifiche competenze che solo dopo un approfondito studio si potrebbero acquisire. Nonostante questo, la realizzazione di un'architettura di questo tipo sarebbe stata molto dispendiosa in merito a tempo, il quale sarebbe stato sottratto alla realizzazione della parte grafica su web. Quest'ultima ragione ha portato all'esclusione del suddetto capitolato.\\