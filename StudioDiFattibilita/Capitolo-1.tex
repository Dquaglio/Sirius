
\section{Introduzione}
\subsection{Scopo del documento}
Questo Documento ha lo scopo di fornire spiegazioni circa la scelta del capitolato C04 da parte dell'azienda Sirius. Ricercando aspetti negativi e positivi per ogni proposta che hanno poi portato alla scelta o al respingimento.\\

\section{Capitolato scelto C4 Seq}
\subsection{Analisi}
Il capitolato proposto dai committenti prof. Vardanega e prof. Cardin per conto della Zucchetti spa prevede la realizzazione di sequenziatore da integrare nei dispositivi mobili.
Di seguito gli aspetti che hanno portato alla scelta del capitolato:
\begin{itemize}
\item Gli strumenti da utilizzare (HTML5, javascript, database) per lo svolgimento del progetto sono di forte interesse da parte di tutti i componenti del gruppo;
\item Lo studio di "workflow" e della organizzazione di passi step-step dipendenti;
\item carico di lavoro non particolarmente elevato per i requisiti minimi;
\item Conoscenza di base di mobile tecnology;
\end{itemize}
Di seguito i fattori di rischio che si presenteranno con la scelta del capitolato:
\begin{itemize}
\item La descrizione del capitolato non è del tutto chiara ed esaustiva, ma a seguito di un 'incontro con l'azienda Zucchetti spa contiamo di risolvere gran parte di questi problemi;
\item Le tecnologie mobile sono in continua evoluzione e questo potrà essere un fattore di rischio in quanto potrebbe succedere di utilizzare standard ancora in definizione; per esempio HTML5 che è richiesto dal capitolato;
\end{itemize}
La documentazione presenta qualche lacuna che provvederemo a colmare con un incontro con l'azienda committente quanto prima.\\
\subsection{Conclusioni}
Il mercato mobile è in continua evoluzione e sia al momento che nei prossimi periodi ci sarà sempre maggiore richiesta di conoscenze inerenti il mobile e il suo interfacciamento con i servizi di rete. Per questo motivo creare una conoscenza di base ed una piccola esperienza sull'utilizzo di questi strumenti potrebbe essere in futuro un'opportunità nonchè un punto di partenza per la nostra carriera lavorativo. Come la programmazione java lato server che abbiamo avuto modo di conoscere tramite il normale corso di studi, e che sicuramente approfondiremo durante lo sviluppo del progetto, è ormai diventata una base indispensabile per la programmazione server-side. Inoltre, il tema proposto non richiede conoscenze particolari e lo stesso potrebbe essere valido come punto di partenza per successivi adattamenti specifici. \\

\section{Altri capitolati}
\subsection{Premessa}
Di seguito riportiamo l'elenco dei requisiti rilevati dalla lettura del capitolato e una breve analisi che evidenzia le motivazioni di tale scelta.\\
\subsection{C1 MaaP Analisi}
Elenco generico dei requisiti richiesti dal capitolato:
\begin{itemize} 
\item Utilizzo di strumenti open source
\item Gestione di database, e conoscenza di database non relazionali
\item Conoscenza di base del "mondo business"(cit.), per creare report adatti alla categoria dell'utilizzatore finale.
\item Linguaggio di programmazione per la gestione dei dati Javascript
\item Il capitolato prevede la realizzazione di: lato server, lato client, creazione dinamica di pagine web, modulo per l'interfacciamento al database e recupero dei dati.
\end{itemize}
Da questa analisi del capitolato proposto, il presente risulta essere un lavoro interessante per quanto riguarda la strumentazione da utilizzare, ma allo stesso tempo le risorse di Sirius non sarebbero adatte alla mole di lavoro richiesta. Dal nostro studio possiamo prevedere che le ore di ricerca di strumenti open source, creazione di conoscenze riguardanti database non relazionali, unite alla realizzazione di tutto il pacchetto richiesto dal cliente, avrebbero portato ad un sovraccarico.
Inoltre, riteniamo che le conoscenze di database costruite durante il normale corso di studi siano più che sufficienti al momento e che un'approfondimento relativo alla gestione di basi di dati non relazionali potrà raramente portare risultati in ambito lavorativo.
\\
\subsection{C2 Ring Analisi}
Elenco generico dei requisiti richiesti dal capitolato:
\begin{itemize} 
\item Conoscenze di biologia;
\item Implementare algoritmi che richiamano la teoria dei grafi;
\item linguaggio di programmazione C++ o Java;
\item Interfacciamento con software specifici inerente il settore di utilizzo;
\end{itemize}
Il capitolato presentato riguarda la realizzazione di un'interessante applicativo per la gestione delle strutture proteiche. La scelta di declinare questo capitolato è stata dettata principalmente dalle nulle conoscenze di biologia che risultano fondamentali per lo sviluppo del progetto.\\
\subsection{C3 Romeo Analisi}
Elenco generico dei requisiti richiesti dal capitolato:
\begin{itemize} 
\item Elaborazione immagini sia 2D che 3D;
\item Algoritmi di estrazione dati da immagini;
\item Studio del protocollo di elaborazione dati;
\end{itemize}
L'utilizzo di strumenti informatici come ausilio delle strutture mediche presenti è sicuramente un aspetto interessante da sviluppare. Nel capitolato sopracitato gli strumenti informatici avrebbero coadiuvato gli esami clinici che restituiscono un risultato sotto forma di immagine, e questa sarebbe stata il punto di partenza per la prima analisi automatizzata per evidenziare i primi risultati. Le elaborazioni anche se elementari di immagini richiedono tempo ed algoritmi ottimizzati in quanto la mole di dati da elaborare potrebbe essere molto elevata. Lo studio e la ricerca di questi algoritmi di elaborazione avrebbe richiesto troppo tempo per la nostra tipologia di gruppo. Anche se le conoscenze mediche sarebbero passate in secondo piano in quanto la richiesta prevedeva lo sviluppo di un solo algoritmo di elaborazione,abbiamo deciso di declinare questo capitolato.   \\
\subsection{C5 Sgad Analisi}
Elenco generico dei requisiti richiesti dal capitolato:
\begin{itemize} 
\item Architettura distribuita;
\item Conoscenze di grafica per la creazione di gioco;
\item Vincoli precisi riguardo i giocatori;
\item Utilizzo database;
\end{itemize}
Lo studio di architetture distribuite su richieste del client avrebbe portato ad ogni componente del gruppo conoscenze importanti circa gli aspetti analizzati e specifiche competenze che solo dopo approfondito studio si potrebbero acquisire. Nonostante questo, la realizzazione di un'architettura di questo tipo sarebbe stata molto dispendiosa, e la realizzazione della parte di grafica su web non avrebbe ricevuto il giusto tempo che richiede una parte così interessante per i componenti di Sirius. Per questo motivo ogniuno di noi preferisce approfondire la parte di grafica e studio delle architetture distribuite in un secondo momento quando se ne presenterà l'occasione.\\
