\section{Introduzione}
\subsection{Glossario}
Al fine di facilitare la compresione del seguente documento, ed in generale di ogni documento che verrà forinto da parte del team Sirius, è stato creato appositamente un glossario (\textit{Glossario v1.2.0.pdf}) contenente la definizione dei termini più complessi o di quelli che necessitano un approfondimento.
$ Questi vocaboli sono contrassegnati in ogni documento dal pedice G (_G).
$ \subsection{Scopo del documento}
Tale documento si prefigge lo scopo di riassumere la discussione e le scelte che hanno portato il team Sirius alla scelta del capitolato \textit{C04 Sequenziatore}.
\subsection{Fonti}
\begin{itemize}
\item Capitolato d'appalto C1:\textit{ MaaP: MongoDB as an admin Platform} \\
\url{http://www.math.unipd.it/ ~ tullio/IS-1/2013/Progetto/C1.pdf;}
\item  Capitolato d'appalto C2: \textit{RING: Residue Interaction Network Generator}\\
\url{http://www.math.unipd.it/ ~ tullio/IS-1/2013/Progetto/C2.pdf;}
\item  Capitolato d'appalto C3: \textit{Romeo: Medical Imaging Cluster Analysis Tool}\\
\url{http://www.math.unipd.it/ ~ tullio/IS-1/2013/Progetto/C3.pdf;}
\item Capitolato d'appalto C4: \textit{Seq: Gestore di processi sequenziali con esecuzione da smartphone}\\
\url{http://www.math.unipd.it/ ~ tullio/IS-1/2013/Progetto/C4.pdf;}
\item Capitolato d'appalto C5:  \textit{SGAD: Social Game con Architettura Distribuita}
\url{http://www.math.unipd.it/ ~ tullio/IS-1/2013/Progetto/C5.pdf.}
\end{itemize}
\section{Capitolato C4 Seq}
\subsection{Analisi}
\subsection{Introduzione}
Il capitolato proposto dai committenti Prof. Vardanega e Prof. Cardin per conto della Zucchetti S.P.A prevede la realizzazione di un Sequenziatore, inteso come strumento per la verifica del compimento di passi astratti.
\subsubsection{Studio dei fattori favorevoli}
\begin{itemize}
\item Gli strumenti da utilizzare (HTML5, javascript, database) per lo svolgimento del progetto sono di forte interesse da parte di tutti i componenti del gruppo;
\item Lo studio del concetto di \textbf{workflow}, della creazione, validazione, verifica di passi \textbf{step by step} dipendenti, considerando la sua relazione con la gestione del lavoro in team;
\item Le conoscenze di base del team Sirius sulla tecnologia mobile sono in parte già acquisite da 3 membri su 6;
\item La libertà concessa dal capitolato permette di spaziare maggiormente con le idee, sia riguardo la sua implementazione che riguardo possibili features aggiuntive.
\end{itemize}
\subsubsection{Studio delle criticità}
\begin{itemize}
\item Il capitolato lascia ampia scelta riguardo le tecnologie da utilizzare, questa libertà potrà causare futuri conflitti nelle scelte progettuali;
\item Essendo HTML5 uno standard non completamente supportato, si corre il rischio di scrivere codice vincolato solamente all'utilizzo dei broswer in grado di supportarlo;
\item Si intravede una difficoltà nel concretizzare l'idea astratta di sequenziatore, questo a causa del fatto che i requisiti e le funzionalità del programma da implementare sono parse parzialmente vaghe nel capitolato.
\end{itemize}

\subsection{Analisi del mercato}
Dal capitolato si evince che il prodotto andrà ad inserirsi all'interno del mercato del mondo mobile, mercato tutt'ora in espansione anche se in parte già saturo. Sviluppare quindi un applicazione multipiattaforma grazie ad HTML5 garantirebbe ad essa una maggior probabilità di essere conosciuta.
Essendo inoltre molto ampia la tipologia di utilizzi che uno strumento come il sequenziatore può fornire, si intuisce che gli utenti potenzialmente interessati sono innumerevoli.

\subsection{Conclusioni}
 Alla luce di tutto ciò, creare una conoscenza di base ed una piccola esperienza sull'utilizzo degli strumenti sopracitati potrebbe rivelarsi in futuro un'opportunità, nonchè un punto di partenza, per la nostra carriera lavorativa. Un esempio è la programmazione java lato server, base indispensabile, che abbiamo già avuto modo di conoscere tramite il corso di studi e che sicuramente approfondiremo durante lo sviluppo del progetto. Inoltre, il tema proposto non richiede conoscenze particolari che divergono dall'ambiente informatico, portandoci quindi a concludere che sviluppare il capitolato: \textit{Seq: Gestore di processi sequenziali con esecuzione da smartphone}, era una scelta condivisa da tutti i membri del gruppo. \\

\section{Altri capitolati}
\subsection{Premessa}
Di seguito riportiamo l'elenco dei requisiti rilevati dalla lettura dei capitolati e una breve analisi che evidenzia le motivazioni che ci hanno portato ad escluderli.\\
\subsection{C1 MaaP Analisi}
Elenco generico dei requisiti richiesti dal capitolato:
\begin{itemize} 
\item Utilizzo di strumenti open source;
\item Gestione di database, e conoscenza di database non relazionali;
\item Conoscenza di base del \textbf{mondo business}, per creare report adatti alla categoria dell'utilizzatore finale;
\item Linguaggio di programmazione per la gestione dei dati Javascript;
\item Il capitolato prevede la realizzazione lato server, lato client, creazione dinamica di pagine web, modulo per l'interfacciamento al database e recupero dei dati.
\end{itemize}
Dall' analisi del capitolato esso risulta essere un progetto interessante per quanto riguarda la strumentazione da utilizzare, ma allo stesso tempo le risorse del team Sirius non sarebbero adatte alla mole di lavoro richiesta. Dal nostro studio possiamo prevedere che le ore di ricerca di strumenti open source, creazione di conoscenze riguardanti database non relazionali, unite alla realizzazione di tutto il pacchetto richiesto dal cliente, avrebbero portato ad un sovraccarico di lavoro.
\\
\subsection{C2 Ring Analisi}
Elenco generico dei requisiti richiesti dal capitolato:
\begin{itemize} 
\item Conoscenze di biologia;
\item Implementare algoritmi che richiamano la teoria dei grafi;
\item Linguaggio di programmazione C++ o Java;
\item Interfacciamento con software specifici inerente il settore di utilizzo.
\end{itemize}
Il capitolato presentato riguarda la realizzazione di un'interessante applicativo per la gestione delle strutture proteiche. La scelta di declinare questo capitolato è stata dettata principalmente dalle nulle conoscenze di biologia che risultano fondamentali per lo sviluppo del progetto.\\
\subsection{C3 Romeo Analisi}
Elenco generico dei requisiti richiesti dal capitolato:
\begin{itemize} 
\item Elaborazione immagini sia 2D che 3D;
\item Algoritmi di estrazione dati da immagini;
\item Studio del protocollo di elaborazione dati.
\end{itemize}
L'utilizzo di strumenti informatici come ausilio delle strutture mediche presenti è sicuramente un aspetto interessante da sviluppare. Nel capitolato sopracitato gli strumenti informatici avrebbero coadiuvato gli esami clinici restituendo un risultato sotto forma di immagine, questa sarebbe stata il punto di partenza per la prima analisi automatizzata atta ad evidenziare determinate specifiche. Le elaborazioni anche se elementari di immagini richiedono tempo ed algoritmi ottimizzati in quanto la mole di dati da elaborare potrebbe essere molto elevata. Lo studio e la ricerca di questi algoritmi di elaborazione avrebbe richiesto troppo tempo per la nostra tipologia di gruppo. Anche se le conoscenze mediche sarebbero passate in secondo piano in quanto la richiesta prevedeva lo sviluppo di un solo algoritmo di elaborazione, abbiamo deciso di declinare questo capitolato.   \\
\subsection{C5 Sgad Analisi}
Elenco generico dei requisiti richiesti dal capitolato:
\begin{itemize} 
\item Architettura distribuita;
\item Conoscenze di grafica per la creazione di gioco;
\item Vincoli precisi riguardo i giocatori;
\item Utilizzo database.
\end{itemize}
Lo studio di architetture distribuite su richieste del client avrebbe portato ogni componente del gruppo ad acquisire conoscenze importanti circa gli aspetti analizzati e specifiche competenze che solo dopo un approfondito studio si potrebbero acquisire. Nonostante questo, la realizzazione di un'architettura di questo tipo sarebbe stata molto dispendiosa in merito a tempo che sarebbe stato sottratto alla realizzazione della parte grafica su web. Quest'ultima ragione ha portato ad escludere il suddetto capitolato.\\