\section{Introduzione}
\subsection{Glossario}
Al fine di facilitare la comprensione del seguente documento, ed in generale di ogni documento che verrà fornito da parte del team \gruppo , è stato creato appositamente un glossario (\textit{\Glossario}) contenente la definizione dei termini più complessi o di quelli che necessitano un approfondimento. Questi vocaboli sono contrassegnati in ogni documento dal pedice G (\ped{G}).
\subsection{Fonti}
\begin{itemize}
\item Capitolato d'appalto C1:\textit{ MaaP: MongoDB as an admin Platform} \\
\url{http://www.math.unipd.it/ ~ tullio/IS-1/2013/Progetto/C1.pdf;}
\item  Capitolato d'appalto C2: \textit{RING: Residue Interaction Network Generator}\\
\url{http://www.math.unipd.it/ ~ tullio/IS-1/2013/Progetto/C2.pdf;}
\item  Capitolato d'appalto C3: \textit{Romeo: Medical Imaging Cluster Analysis Tool}\\
\url{http://www.math.unipd.it/ ~ tullio/IS-1/2013/Progetto/C3.pdf;}
\item Capitolato d'appalto C4: \textit{Seq: Gestore di processi sequenziali con esecuzione da smartphone}\\
\url{http://www.math.unipd.it/ ~ tullio/IS-1/2013/Progetto/C4.pdf;}
\item Capitolato d'appalto C5:  \textit{SGAD: Social Game con Architettura Distribuita}\\
\url{http://www.math.unipd.it/ ~ tullio/IS-1/2013/Progetto/C5.pdf.}
\end{itemize}