\section{Capitolato C4 Seq}
\subsection{Introduzione}
Il capitolato proposto dai committenti Prof. Vardanega e Prof. Cardin per conto della Zucchetti \textit{S.P.A}\ped{G} prevede la realizzazione di un \progetto , inteso come strumento per la verifica del compimento di passi astratti.
\subsection{Studio dei fattori favorevoli}
\begin{itemize}
\item Gli strumenti da utilizzare (HTML5, Javascript, database) per lo svolgimento del progetto sono di forte interesse da parte di tutti i componenti del gruppo;
\item Lo studio del concetto di \textbf{workflow}, della creazione, validazione, verifica di passi \textbf{step by step} dipendenti, considerando la sua relazione con la gestione del lavoro in team;
\item Le conoscenze di base del team \textit{Sirius} sulla tecnologia mobile sono in parte già acquisite da 3 membri su 6;
\item La libertà concessa dal capitolato permette di spaziare maggiormente con le idee, sia riguardo la sua implementazione che riguardo possibili \textbf{features}\ped{G} aggiuntive.
\end{itemize}
\subsection{Studio delle criticità}
\begin{itemize}
\item Il capitolato lascia ampia scelta riguardo le tecnologie da utilizzare, questa libertà potrà causare futuri conflitti nelle scelte progettuali;
\item Essendo HTML5 uno standard non completamente supportato, si corre il rischio di scrivere codice vincolato solamente all'utilizzo dei browser in grado di supportarlo;
\item Si intravede una difficoltà nel concretizzare l'idea astratta di \progetto , questo a causa del fatto che i requisiti e le funzionalità del programma da implementare sono parse parzialmente vaghe nel capitolato.
\end{itemize}

\subsection{Analisi di mercato}
Dal capitolato si evince che il prodotto andrà ad inserirsi all'interno del mercato del mondo mobile, mercato tuttora in espansione anche se in parte già saturo. Sviluppare quindi un applicazione multi-piattaforma grazie ad HTML5 garantirebbe ad essa una maggior probabilità di essere conosciuta.
Essendo inoltre molto ampia la tipologia di utilizzi che uno strumento come \textit{Sequenziatore} può fornire, si intuisce che gli utenti potenzialmente interessati sono innumerevoli.

\subsection{Conclusioni}
Alla luce di tutto ciò, creare una conoscenza di base ed una piccola esperienza sull'utilizzo degli strumenti sopracitati potrebbe rivelarsi in futuro un'opportunità, nonché un punto di partenza, per le carriere lavorative dei componenti del team. Un esempio è la programmazione Java lato server, base indispensabile, che i componenti di \textit{Sirius} hanno avuto modo di conoscere tramite il corso di studi e che sicuramente approfondiranno durante lo sviluppo del progetto. Inoltre, il tema proposto non richiede conoscenze particolari che divergono dall'ambiente informatico, portandoci quindi a concludere che sviluppare il capitolato: \textit{Seq: Gestore di processi sequenziali con esecuzione da smartphone} è una scelta condivisa da tutti i membri del gruppo. \\