\section{Definizioni}
Sono qui di seguito elencate le definizioni di tutti i termini contrassegnati con il pedice "G" negli altri documenti.
Si ricorda che tutti i termini saranno riportati in ordine alfabetico:

\subsection*{"A"}
\begin{itemize}
\itemsep2em
\item \textbf{Admin}:\\ abbreviazione di Administrator, rappresenta l'amministratore di sistema;
\item \textbf{Acceptable}:\\ termine che indica il grado minimo di accettazione di un parametro;
\item \textbf{Altervista}:\\ si tratta di una piattaforma web italiana che offre la possibilità di aprire immediatamente e gratuitamente un sito web, un forum o un blog;
\item \textbf{Android}:\\ sistema operativo open source\ped{G} per dispositivi mobili sviluppato da Google Inc.
\end{itemize}
\subsection*{"B"}
\begin{itemize}
\itemsep2em
\item \textbf{Bounded}: letteralmente: legato a;
\item \textbf{Browser}:\\ programma utilizzato per la navigazione dei contenuti presenti nel World Wide Web;
tecnicamente, un browser e un’applicazione client\ped{G} che utilizza il protocollo HTTP\ped{g} per inoltrare le richieste dell’utente ad un web server\ped{G};
\item \textbf{Branch}:\\ nel contesto di GitHub, opzione che permette lo sviluppo parallelo da parte di più utenti di uno stesso file;
\item \textbf{Back end}:\\ in una struttura client/server, il back endrappresenta la parte server, ossia la parte inaccessibile all'utente ma solamente agli amministratori.
\end{itemize}
\subsection*{"C"}
\begin{itemize}
\itemsep2em
\item \textbf{C++}:\\ il C++ è un linguaggio di programmazione orientato agli oggetti;
\item \textbf{Capability}:\\ indica il grado di capacità di compiere un determinato compito. Da non confondere con capability, termine informatico;
\item \textbf{Chrome}:\\ browser\ped{G} proprietario sviluppato da Google, basato su un progetto opensource\ped{G};
\item \textbf{Chrome for iOS\ped{G}}:\\ versione per iOS\ped{G} di Chrome\ped{G};
\item \textbf{Chromium}:\\ browser\ped{G} opensource, è il core di Chrome\ped{G};
\item \textbf{Controller(Spring)}: rappresenta una componente del programma che riceve delle HttpServletRequest (richieste http) e delle HttpServletResponse, ed è in grado di partecipare al workflow MVC;
\item \textbf{CSS}:\\ linguaggio usato per definire la formattazione di documenti HTML, ad esempio in siti web e relative pagine web;
\item \textbf{Client}:\\ computer o programma che inoltra le richieste dell’utente ad un programma server\ped{G}.
\end{itemize}

\subsection*{"D"}
\begin{itemize}
\itemsep2em
\item \textbf{Database}:\\ il termine database, base di dati o banca dati (a volte abbreviato con la sigla DB), indica un archivio dati, o un insieme di archivi ben strutturati, in cui le informazioni in esso contenute sono strutturate e collegate tra loro secondo un particolare modello logico;
\item \textbf{Driver}: file accessorio al ssitema operativo e che consente la comunicazione fra il computer ed una periferica.
\end{itemize}

\subsection*{"E"}
\begin{itemize}
\itemsep2em
\item \textbf{Efficienza}:\\ capacità di azione o di produzione con il minimo di scarto, di spesa, di risorse e di tempo impiegati; 
\item \textbf{Efficacia}:\\ si intende la capacità di raggiungere un determinato obiettivo.
\end{itemize}

\subsection*{"F"}
\begin{itemize}
\itemsep2em
\item \textbf{Feature}:\\ feature può essere tradotto in italiano come: caratteristica;
\item \textbf{Form}:\\ un form è un modulo da compilare, esso generalmente è composto da una serie di campi (testuali e non) che l'utente dei riempire al fine di specificare determinate informazioni.
\item \textbf{Framework}:\\ struttura logica di supporto su cui un \textit{software} può essere progettato e realizzato.
\item \textbf{Front-end}:\\ in una struttura client/server, front end rappresenta il client, è un termine largamente utilizzato per caratterizzare le interfacce che hanno come destinatario l'utente.
\end{itemize}

\subsection*{"H"}
\begin{itemize}
\itemsep2em
\item \textbf{Hardware}:\\ termine generico per indicare la parte elettronica di un sistema;
\item \textbf{HTML}:\\ linguaggio di markup\ped{G} per la strutturazione delle pagine web.
\end{itemize}

\subsection*{"I"}
\begin{itemize}
\itemsep2em
\item \textbf{iOS}:\\ sistema operativo\ped{G} sviluppato da Apple Inc, disponibile solamente su dispositivi mobili Apple;
\item \textbf{Internet Explorer}:\\ browser\ped{G} proprietario sviluppato da Microsoft.
\end{itemize}

\subsection*{"J"}
\begin{itemize}
\itemsep2em
\item \textbf{Java}:\\ linguaggio di programmazione orientato agli oggetti, specificatamente progettato per essere il più possibile indipendente dalla piattaforma di esecuzione;
\item \textbf{Javascript}:\\ linguaggio di scripting\ped{G} orientato agli oggetti comunemente usato nella creazione di siti web.
\end{itemize}

\subsection*{"M"}
\begin{itemize}
\itemsep2em
\item \textbf{Markup}:\\ in generale un linguaggio di markup è un insieme di regole che descrivono i meccanismi di rappresentazione (strutturali, semantici o presentazionali) di un testo;
\item \textbf{Maturity}:\\ misura di quanto è governato il sistema di processi dell'azienda;
\item \textbf{Milestone}:\\ letteralmente significa pietra miliare, indica importanti traguardi intermedi nello svolgimento del progetto;
\item \textbf{Mozilla Firefox}:\\ browser\ped{G} opensource\ped{G} sviluppato da Mozilla Foundation;
\item \textbf{MySQL}:\\ sistema per la gestione di basi di dati relazionali.
\end{itemize}

\subsection*{"O"}
\begin{itemize}
\itemsep2em
\item \textbf{Open source}:\\ in informatica, indica un software i cui autori (più precisamente i detentori dei diritti) ne permettono e favoriscono il libero studio e l'apporto di modifiche da parte di altri programmatori indipendenti;
\item \textbf{Opera}:\\ browser\ped{G} free sviluppato da Opera Software.
\end{itemize}

\subsection*{"P"}
\begin{itemize}
\itemsep2em
\item \textbf{Package}:si intende una collezione di classi e di interfacce correlate;
\item \textbf{Passo}:\\ nel contesto del progetto \progetto{}, un passo è inteso come una unità di un workflow\ped{G}, e può essere visto come un insieme di dati che devono essere raccolti al fine di completarlo;
\item \textbf{Processo}:\\ rete di cambiamenti, attività o azioni collegate tra loro;
\item \textbf{Process owner}:\\ per process owner nel contesto del progetto si intende un utente il quale ha privilegi speciali circa l'accesso delle informazioni (dati) inviati dagli utenti, questi privilegi sono utili a determinare la correttezza di dei data che necessitano una verifica umana.
\end{itemize}
\subsection*{"R"}
\itemsep2em
\begin{itemize}
\item \textbf{Root}:\\ tradotto letteralmente in: radice, specifica la cartella principale da cui si sviluppano tutte le sottocartelle.
\item \textbf{Routing}:\\ funzione di un commutatore che decide su quale porta o interfaccia inviare un elemento di comunicazione ricevuto.
\end{itemize}
\subsection*{"S"}
\begin{itemize}
\itemsep2em
\item \textbf{Safari}:\\ browser\ped{G} sviluppato da Apple Inc.;
\item \textbf{Safari for iOs}:\\ versione per iOs\ped{G} di Safari;
\item \textbf{Server}:\\ componente che fornisce un qualsiasi servizio ad altre componenti denominate client\ped{G} attraverso una rete di computer o direttamente in locale;
\item \textbf{Sistema operativo}: software\ped che ha il compito di gestire e controllare tutto il traffico di dati all’interno del computer e fra questo e tutte le periferiche, operando anche come intermediario fra hardware\ped{G} e software\ped{G} di sistema ed i diversi programmi in esecuzione;
\item \textbf{Slack}:\\ si tratta di un lasco di tempo che funge da cuscinetto tra attività;
\item \textbf{Scripting}:\\ in informatica è un linguaggio di programmazione interpretato, destinato in genere a compiti di automazione del sistema operativo o delle applicazioni, o ad essere usato all'interno delle pagine web;
\item \textbf{Software}:\\ termine generico che definisce programmi e procedure utilizzati per far eseguire al computer un determinato compito;
\item \textbf{Step by step}:\\ letteralmente tradotto in "passo a passo", indica la dipendenza di un passo\ped{G} con il suo passo successivo;
\item \textbf{Stub}: inteso come articolo ancora in formato di bozza, quindi sufficiente per dare un'idea del contenuto di un certo argomento, ma lontano dal fornire una spiegazione esaustiva.
\end{itemize} 

\subsection*{"T"}
\begin{itemize}
\itemsep2em
\item \textbf{Ticket}:\\ strumento tramite il quale vengono assegnati dei compiti da svolgere nel contesto di un team che lavora sullo stesso progetto.
\end{itemize}

\subsection*{"U"}
\begin{itemize}
\itemsep2em
\item \textbf{URL}:\\ sequenza di caratteri che identifica univocamente l'indirizzo di una risorsa \textit{web}.
\end{itemize}

\subsection*{"W"}
\begin{itemize}
\itemsep2em

\item \textbf{WBS}:\\ detta anche Struttura di Scomposizione del Lavoro (traduzione letterale) o Struttura Analitica di Progetto è l'elenco di tutte le attività di un progetto;
\item \textbf{Websocket}:\\una tecnologia che permette maggiore interazione tra un browser e un server, facilitando la realizzazione di applicazioni che forniscono contenuti e giochi in tempo reale;
\item \textbf{Workflow}:\\ tradotto letteralmente in: flusso di lavoro, indica un insieme di passi\ped{G} da svolgere sequenzialmente;
\item \textbf{W3C}: organizzazione internazionale non governativa che ha come scopo sviluppare le potenzialità del web, la principale attività del w3c è quella di stabilire standard tecnici per il world wide web, sia per quanto concerne i linguaggi di markup sia per quanto concerne i protocolli di comunicazione.
\end{itemize}

\section*{Acronimi}
Qui di seguito vengono riportai tutti gli acronimi utilizzati nella redazione dei documenti:
\begin{itemize}
\itemsep2em
\item \textbf{PDF}:\\ formato di un file basato su un linguaggio di descrizione di pagina, utile per rappresentare documenti in modo indipendente dall'hardware e dal software utilizzati per generarli o per visualizzarli;
\item \textbf{PERT}:\\ Il diagramma reticolare di PERT (Program Evaluation and Review Technique) descrive la sequenza cronologica delle azioni pianificate per il completamento di un progetto nel suo complesso. Esso rappresenta graficamente il piano d’azione;
\item \textbf{PHP}:\\ Hypertext Preprocessor, linguaggio di programmazione originariamente concepito per la programmazione di pagine web dinamiche;
\item \textbf{RR}:\\ Revisione dei Requisiti;
\item \textbf{RP}:\\ Revisione di Progettazione;
\item \textbf{RA}:\\ Revisione di Qualifica;
\item \textbf{RQ}:\\ Revisione di Accettazione;
\item \textbf{S.P.A.}:\\ Società per azioni;
\item \textbf{UML}:\\ in ingegneria del software, UML (Unified Modeling Language, "linguaggio di modellazione unificato") è un linguaggio di modellazione e specifica basato sul paradigma object-oriented.
\end{itemize}