\section{Descrizione messaggi di errore}
\label{errori}

\subsection{E1: javascript disabilitato}
\label{e1}
Questo errore viene visualizzato quando nel \textit{Browser\ped{G}} utilizzato non è attivato l'utilizzo di \textit{Javascript\ped{G}}.
Per attivare \textit{Javascript} dalle opzioni del \textit{Browser\ped{G}}, è possibile seguire le istruzioni presenti all'indirizzo \url{http://aboutjavascript.com/it.html}.

\subsection{E2: credenziali non corrette}
\label{e2}
Questo errore si verifica se, durante la procedura di autenticazione, l'\textit{username} o la \textit{password} inserite non sono corrette.
La guida al corretto utilizzo della funzionalità di autenticazione è presente nel capitolo \hyperref[autenticazione]{Autenticazione}.

\subsection{E3: errore di connessione}
\label{e3}
Questo errore si verifica se, quando viene effettuata una richiesta al \textit{server\ped{G}}, il dispositivo utilizzato non ha accesso ad alcuna connessione.
Per proseguire con l'utilizzo dell'applicazione è necessario accedere alla rete.

\subsection{E4: nome processo già utilizzato}
\label{e4}
Questo errore si verifica quando si richiede il salvataggio di un nuovo processo, il cui nome scelto è già utilizzato da un altro processo.
Per risolvere il problema è sufficiente reinserire il nome, e ritentare il salvataggio del processo.

\subsection{E5: dato mancante}
\label{e5}
Questo errore si verifica quando si richiede il salvataggio di un nuovo processo, del quale non sono stati inseriti tutti i dati obbligatori.
Per risolvere il problema è sufficiente inserire il dato segnalato dal messaggio d'errore.
La guida al corretto utilizzo della funzionalità di creazione processo è presente nei seguenti capitoli:
\begin{itemize}
\item \hyperref[creazione]{Creazione di un processo}:
\begin{itemize}
\item \hyperref[definizione]{Definizione di un processo};
\item \hyperref[addstep]{Aggiunta di un passo};
\item \hyperref[vincoli]{Aggiunta di un criterio di superamento}.
\end{itemize}
\end{itemize}