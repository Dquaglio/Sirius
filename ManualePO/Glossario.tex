\section{Glossario}
\label{glossario}

Sono qui di seguito elencate in ordine alfabetico, le definizioni di tutti i termini contrassegnati con il pedice "G" nel documento.

\subsection*{"A"}
\addcontentsline{toc}{subsection}{\protect\numberline{}A}
\begin{itemize}
\item \textbf{Android}:\\ sistema operativo open source\ped{G} per dispositivi mobili sviluppato da Google Inc.
\end{itemize}
\subsection*{"B"}
\begin{itemize}
\item \textbf{Browser}:\\ programma utilizzato per la navigazione dei contenuti presenti nel World Wide Web;
tecnicamente, un browser e un’applicazione client\ped{G} che utilizza il protocollo HTTP\ped{g} per inoltrare le richieste dell’utente ad un web server\ped{G}.
\end{itemize}

\subsection*{"C"}
\addcontentsline{toc}{subsection}{\protect\numberline{}C}
\begin{itemize}
\item \textbf{Chrome}:\\ browser\ped{G} proprietario sviluppato da Google, basato su un progetto opensource\ped{G};
\item \textbf{Chrome for iOS\ped{G}}:\\ versione per iOS\ped{G} di Chrome\ped{G};
\item \textbf{Chromium}:\\ browser\ped{G} opensource, è il core di Chrome\ped{G};
\item \textbf{Client}:\\ computer o programma che inoltra le richieste dell’utente ad un programma server\ped{G}.
\end{itemize}

\subsection*{"D"}
\addcontentsline{toc}{subsection}{\protect\numberline{}D}
\begin{itemize}
\item \textbf{Dati di sessione}:\\ Dati salvati nel dispositivo dell'utente, che associano l'utente ad una applicazione. Nel progetto \progetto{}, tali dati consentono al programma di conoscere se l'utente è autenticato, o se deve effettuare il \textit{login}.
\end{itemize}

\subsection*{"H"}
\addcontentsline{toc}{subsection}{\protect\numberline{}H}
\begin{itemize}
\item \textbf{HTML}:\\ linguaggio di markup\ped{G} per la strutturazione delle pagine web.
\end{itemize}

\subsection*{"I"}
\addcontentsline{toc}{subsection}{\protect\numberline{}I}
\begin{itemize}
\item \textbf{iOS}:\\ sistema operativo\ped{G} sviluppato da Apple Inc, disponibile solamente su dispositivi mobili Apple;
\item \textbf{Internet Explorer}:\\ browser\ped{G} proprietario sviluppato da Microsoft.
\end{itemize}

\subsection*{"M"}
\addcontentsline{toc}{subsection}{\protect\numberline{}M}
\begin{itemize}
\item \textbf{Markup}:\\ in generale un linguaggio di markup è un insieme di regole che descrivono i meccanismi di rappresentazione (strutturali, semantici o presentazionali) di un testo;
\item \textbf{Mozilla Firefox}:\\ browser\ped{G} opensource\ped{G} sviluppato da Mozilla Foundation.
\end{itemize}

\subsection*{"O"}
\addcontentsline{toc}{subsection}{\protect\numberline{}O}
\begin{itemize}
\item \textbf{Open source}:\\ in informatica, indica un software i cui autori (più precisamente i detentori dei diritti) ne permettono e favoriscono il libero studio e l'apporto di modifiche da parte di altri programmatori indipendenti;
\item \textbf{Opera}:\\ browser\ped{G} free sviluppato da Opera Software.
\end{itemize}

\subsection*{"P"}
\addcontentsline{toc}{subsection}{\protect\numberline{}P}
\begin{itemize}
\item \textbf{Passo}:\\ nel contesto del progetto \progetto{}, un passo è inteso come una unità di un workflow\ped{G}, e può essere visto come un insieme di dati che devono essere raccolti al fine di completarlo;
\item \textbf{PDF}:\\ formato di un file basato su un linguaggio di descrizione di pagina, utile per rappresentare documenti in modo indipendente dall'hardware e dal software utilizzati per generarli o per visualizzarli;
\item \textbf{Process owner}:\\ per process owner nel contesto del progetto si intende un utente che può creare processi a cui gli utenti base possono partecipare, e il quale ha privilegi speciali riguardo l'accesso delle informazioni inviate dagli utenti; questi privilegi sono utili a determinare la correttezza di dei data che necessitano una verifica umana.
\end{itemize}

\subsection*{"S"}
\addcontentsline{toc}{subsection}{\protect\numberline{}S}
\begin{itemize}
\item \textbf{Safari}:\\ browser\ped{G} sviluppato da Apple Inc.;
\item \textbf{Safari for iOs}:\\ versione per iOs\ped{G} di Safari;
\item \textbf{Server}:\\ componente che fornisce un qualsiasi servizio ad altre componenti denominate client\ped{G} attraverso una rete di computer o direttamente in locale;
\item \textbf{Sistema operativo}: software\ped che ha il compito di gestire e controllare tutto il traffico di dati all’interno del computer e fra questo e tutte le periferiche, operando anche come intermediario fra hardware\ped{G} e software\ped{G} di sistema ed i diversi programmi in esecuzione;
\item \textbf{Software}:\\ termine generico che definisce programmi e procedure utilizzati per far eseguire al computer un determinato compito.
\end{itemize} 

\subsection*{"U"}
\addcontentsline{toc}{subsection}{\protect\numberline{}U}
\begin{itemize}
\item \textbf{URL}:\\ sequenza di caratteri che identifica univocamente l'indirizzo di una risorsa \textit{web}.
\end{itemize}