\subsection{Tecniche di analisi statica}
L'analisi statica è una tecnica di analisi applicabile sia alla documentazione che al codice e permette di effettuare la verifica di quanto prodotto individuando errori ed anomalie. Essa può essere svolta in due modi diversi ma complementari tra di loro in quanto per utilizzare \textit{inspection} bisogna prima aver effettuato \textit{walkthrough}
\subsubsection{Inspection}
Questa tecnica, di analisi statica, consiste nella verifica di sezioni ben definite di un documento o del codice. Questo tipo di controlli per i documenti sono usualmente definiti tramite una lista di controllo (checklist) redatta anticipatamente rispetto all'attività di verifica da intraprendere. Per la verifica dei documenti, la lista di controllo è stata elaborata a seguito di analisi eseguite tramite \textit{walkthrough}, ed evidenziando gli errori più ricorrenti riscontrati. \textit{Inspection} è una strategia rapida in quanto permette l'analisi di alcuni parti ritenute critiche nella checklist senza bisogno di una lettura integrale di documento o di tutto il codice in oggetto.
\subsubsection{Walkthrough}
\textit{Walkthrough} è una tecnica di analisi statica che consiste nella lettura critica a largo raggio di tutto il documento. In questa tipologia di analisi il \textit{Verificatore} utilizza molto tempo per la lettura e correzione del documento o codice. Questa tecnica viene di solito utilizzata nella prima parte dello sviluppo di progetti in quanto, la poca esperienza del \textit{Verificatore} non permette un'altro tipo di verifica. Al termine di questo primo set di analisi \textit{walkthrough} viene usualmente definita una lista di controllo che permetta di ricercare in primo luogo gli errori più ricorrenti, e maggiormente riscontrati. \textit{Walkthrough} è un'attività onerosa e collaborativa che richiede l'intervento di più persone per essere efficiente ed efficace
\subsection{Tecniche di analisi dinamica}
L'analisi dinamica si applica solamente al prodotto software e consiste nell'esecuzione del codice mediante l'uso di test predisposti per verificarne il funzionamento o rilevare possibili difetti di implementazione eseguendo tutto o solo una parte del codice.
La \textbf{ripetibilità} del test è una caratteristica fondamentale per questo tipo di test, in quanto dichiara che il codice con un certo \textit{input} produce sempre lo stesso \textit{output} su uno specifico ambiente. In questo modo si è in grado di riscontrare problemi e verificare la correttezza del prodotto.
Per questo \gruppo ~ha deciso di definire a priori le seguenti caratteristiche:
\begin{itemize}
\item \textbf{Ambiente}: sistema \textit{hardware} e quello \textit{software} sui quali è stato pianificato l'utilizzo del prodotto, di essi si deve definire uno stato iniziale dal quale poter iniziare ad eseguire i test;
\item \textbf{Specifica di \textit{input}}: definire quali sono gli \textit{input} e quali devono essere gli \textit{output} attesi;
\item \textbf{Procedure}: definire quali devono essere i test ed in che ordine devono essere analizzati i risultati ottenuti.
\end{itemize}
Di seguito sono definiti cinque diversi tipi di test.
\subsubsection{Test di unità} 
Per test di unità si intende la verifica di ogni singola unità di prodotto software tramite l'utilizzo di stub\ped{G}, driver\ped{G} e logger\ped{G}. Per unità si intende la più piccola porzione di codice che è utile verificare singolarmente e che viene prodotta da un unico programmatore. Tramite questo tipo di test si vogliono testare i vari le unità per rilevare errori di implementazione da parte dei programmatori.
\subsubsection{Test di integrazione}
I test di integrazione prevedono la verifica dei componenti del sistema che vengono aggiunti incrementando il prodotto di origine e si prefigge quindi di analizzare la combinazione di due o più unità software che hanno quindi superato i test di unità. Questa tecnica di verifica serve ad individuare errori residui nella programmazione dei singoli moduli: come modifiche delle interfacce e comportamenti inaspettati di componenti software di parti terze e che pregiudicherebbero la validità del prodotto. Per effettuare tali test può essere necessario l'aggiunta di componenti software fittizie e non ancora implementate al fine di non pregiudicare negativamente l'esito dell'analisi.
\subsubsection{Test di sistema}
Consiste nella validazione del sistema attraverso la verifica della copertura di tutti i requisiti obbligatori individuati in \infoAR, e tracciati  grazie allo strumento messo a punto da \gruppo;
\subsubsection{Test di regressione}
I test di regressione vengono eseguiti quando si apportano delle modifiche a parte del software e questi consistono nella riesecuzione dei test riguardanti le i componenti che hanno subito modifiche e che precedentemente non erano soggetti ad errori.
Tale operazione viene aiutata dal tracciamento, che permette di individuare e ripetere facilmente i test di unità, integrazione ed eventualmente di sistema che sono stati potenzialmente influenzati dalle modifiche.
\subsubsection{Test di accettazione}
Si tratta del collaudo del prodotto software sotto il controllo del proponente. Se il collaudo viene superato in modo positivo, il sistema viene rilasciato e la commessa si conclude.