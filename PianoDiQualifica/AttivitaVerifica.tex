\section{Resoconto attivit\'a di verifica}
\subsection{Riassunto dell'attivit\'a di verifica su RR}
\gruppo, ha deciso di valorizzare soprattuto i requisiti desiderabili per fare in modo di tenere alto l'indice di efficienza\ped{G}. Valutando, durante lo stato di avanzamento, quali di questi requisiti saranno successivamente sviluppati nella realizzazione del prodotto software. Questo tipo di scelte risultano infatti difficoltose allo stato attuale in quanto una pianificazione approfondita necessiterebbe di \textit{lead time} \ped{G} precisi che non sono al momento tra le conoscenze dei componenti del gruppo.
Nel periodo di tempo che ha portato \gruppo ~alla consegna di questa revisione sono stati verificati i \textbf{documenti} ed i \textbf{processi}.\\
I \textbf{documenti} sono stati verificati anche durante le operazioni di redazione per portare a conoscenza dei contenuti tutti i componenti del gruppo di lavoro.
L'analisi statica, in primo luogo utilizzando la tecnica del \textit{walkthrough}, ha portato alla redazione  di una lista di controllo che verr\'a poi incrementata ed utilizzata nell'analisi finale del documento prima di procedere alla consegna. Una volta rilevati gli errori questi sono stati notificati al redattore che ha proceduto alla correzione, evidenziando gli errori frequenti che sono stati utilizzati per migliorare il processo di verifica. \gruppo ~adotta il ciclo PDCA per rendere pi\'u efficente\ped{G} ed efficace\ped{G} nel tempo il processo di verifica.\\
L'attivit\'a di verifica, inoltre, utilizzando la tecnica \textit{inspection} \'e stata utilizzata principalmente per la verifica dei grafici dei casi d'uso. Per verificare la correttezza dei requisiti richiesti e la successiva completezza ci si \'e affidati ad un particolare strumento di tracciamento definito in \infoNDP.
L'avanzamento dei processi, dettato dal \infoPDP, \'e stato mantenuto in controllo tramite una costante verifica delle metriche definite in questo documento e di cui troviamo una rappresentazione grafica in seguito.
\subsection{Dettaglio dell'attivit\'a di verifica su RR}
\subsubsection{Documenti}
Indice di \textit{gulpease} per i documenti redatti:\\
\begin{tabular}{| >{\centering\arraybackslash}m{1in} | >{\centering\arraybackslash}m{1in} | >{\centering\arraybackslash}m{1in}|}
\hline
\textbf{Documento} & \textbf{Valore di accettazione} & \textbf{Esito} \\
\hline
\infoPDP & $>$40 & \textit{Positivo}\\
\hline
\infoNDP & $>$40 & \textit{Positivo}\\
\hline
\infoAR & $>$40 & \textit{Positivo}\\
\hline
\infoPDQ & $>$40 & \textit{Positivo}\\
\hline
\infoSDF & $>$40 & \textit{Positivo}\\
\hline
\end{tabular}\\
\begin{center}
Tabella 1: esito del calcolo indice di gulpease per ogni documento.
\end{center}
\subsubsection{Processi}
\gruppo ~ha condotto l'attivit\'a di verifica per i processi. In questo primo periodo il processo di documentazione \'e predominante nella pianificazione delle risorse. Di seguito viene riportato l'indice \textbf{SV (\textit{schedule variance})} per le attivit\'a eseguite e i risultati sono i seguenti:
\begin{center}
\begin{tabular}{| >{\centering\arraybackslash}m{1in} | >{\centering\arraybackslash}m{1in} | >{\centering\arraybackslash}m{1in} | >{\centering\arraybackslash}m{1in} | >{\centering\arraybackslash}m{1in} |}
\hline
\textbf{Attivit\'a} & \textbf{Ore pianificate} & \textbf{Ore rilevate} & \textbf{SV rilevato} & \textbf{SV accettazione} \\
\hline
Norme di Progetto & 17 H & 17 H & 0 H & $>$ -1 H\\
\hline
Studio di Fattibilit\'a & 8 H & 14 H & \textbf{-6 H} & \textbf{$>$ -1 H}\\
\hline
Analisi dei Requisiti & 70 H & 68 H & 2 H & $>$ -4 H\\
\hline
Piano di Progetto & 37 H & 35 H & 2 H & $>$ -2 H\\
\hline
Piano di Qualifica & 26 H & 22 H & 4 H & $>$ -2 H\\
\hline
\end{tabular}
\end{center}
\begin{center}
Tabella 2 : Indice SV per le attivit\'a.
\end{center}
Da una prima analisi, si denota che \gruppo ~ha pianificato in modo preciso le attivit\'a.
L'attivita' di Studio di Fattibilit\'a, essendo stato uno dei primi documenti che \gruppo ~ha redatto, la pianificazione non \'e stata precisa questo ha portato ad un SV dell'attivit\'a fuori dal range di accettazione. Le cause di questo problema sono da ricercare anche nella poca confidenza con gli strumenti di \textit{editor} testi e con gli strumenti di condivisione. La singola occorrenza del problema, non \'e quindi indice di allarme per gli altri processi che saranno pianificati nell'avanzamento del prodotto.
\begin{itemize}
\item SV-totale = 2 H;
\end{itemize}
SV-totale maggiore di zero denota che \gruppo ~st\'a producendo pi\'u velocemente rispetto a quanto pianificato. Questo pu\'o essere una diretta conseguenza dell'aggiunta di uno \textit{slack} temporale nella pianificazione delle attivit\'a.
Il team ha valutato la possibilit\'a di ridurre il tempo di \textit{slack}, per fare in modo che la pianificazione corrisponda alla realt\'a; ma data la variabilit\'a delle attivit\'a che \gruppo ~intende svolgere nel proseguo del progetto e la poca esperienza, \'e stato deciso di non modificare tale valore.\\
Di seguito viene riportato l'indice \textbf{BV (\textit{budget variance})} per le attivit\'a eseguite e i risultati sono i seguenti:
\begin{center}
\begin{tabular}
{| >{\centering\arraybackslash}m{1in} | >{\centering\arraybackslash}m{1in} | >{\centering\arraybackslash}m{1in} | >{\centering\arraybackslash}m{1in} | >{\centering\arraybackslash}m{1in} |}
\hline
\textbf{Attivit\'a} & \textbf{Costo pianificato} & \textbf{Costo consuntivo} & \textbf{BV rilevato} & \textbf{BV limite} \\
\hline
Norme di Progetto & 325.00 € & 325.00 € & 0.00 € & $>$ -32.50 €\\
\hline
Studio di Fattibilit\'a & 180.00 € & 310.00 € & \textbf{-130.00 €} & \textbf{$>$ -18.00 €}\\
\hline
Analisi dei Requisiti & 1630.00 € & 1600.00 € & 30.00 € & $>$ -16.30 €\\
\hline
Piano di Progetto & 1005.00 € & 945.00 € & 60.00 € & $>$ -10.05 €\\
\hline
Piano di Qualifica & 490.00 € & 420.00 € & 70.00 € & $>$ -49.00 €\\
\hline
\end{tabular}
\end{center}
\begin{center}
Tabella 3: Indice BV per le attivit\'a.
\end{center}
Come descritto sopra per SV, anche BV denota che il preventivo di costo previsto per le attivit\'a svolte \'e stato corretto.
In particolare nell'attivit\'a di Studio di Fattibilit\'a il costo a consuntivo \'e stato maggiore rispetto a quello preventivato. Questo \'e da collegare al costo orario dell'amministratore e non meno alle cause elencate sopra per l'indice SV che ne \'e strettamente collegato.
Complessivamente \gruppo ~ha ottenuto:
\begin{itemize}
\item BV-totale = 30.00 €.
\end{itemize}
il risultato ottenuto \'e una diretta conseguenza di un preventivo appropriato, e quindi ad un piccolo margine di guadagno nel budget di spesa dell'intero progetto.