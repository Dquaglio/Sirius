\section{Resoconto attività di verifica}
\subsection{Riassunto dell'attività di verifica su RR}
\gruppo, ha deciso di valorizzare sopratutto i requisiti desiderabili per fare in modo di tenere alto l'indice di efficienza\ped{G}. Valutando, durante lo stato di avanzamento, quali di questi requisiti saranno successivamente sviluppati nella realizzazione del prodotto software. Questo tipo di scelte risultano infatti difficoltose allo stato attuale in quanto una pianificazione approfondita necessiterebbe di \textit{lead time} \ped{G} precisi che non sono al momento tra le conoscenze dei componenti del gruppo.
Nel periodo di tempo che ha portato \gruppo ~alla consegna di questa revisione sono stati verificati i \textbf{documenti} ed i \textbf{processi}.\\
I \textbf{documenti} sono stati verificati anche durante le operazioni di redazione per portare a conoscenza dei contenuti tutti i componenti del gruppo di lavoro.
L'analisi statica, in primo luogo utilizzando la tecnica del \textit{walkthrough}, ha portato alla redazione  di una lista di controllo che verrà poi incrementata ed utilizzata nell'analisi finale del documento prima di procedere alla consegna. Una volta rilevati gli errori questi sono stati notificati al redattore che ha proceduto alla correzione, evidenziando gli errori frequenti che sono stati utilizzati per migliorare il processo di verifica. \gruppo ~adotta il ciclo PDCA per rendere più efficiente\ped{G} ed efficace\ped{G} nel tempo il processo di verifica.\\
L'attività di verifica, inoltre, utilizzando la tecnica \textit{inspection} è stata utilizzata principalmente per la verifica dei grafici dei casi d'uso. Per verificare la correttezza dei requisiti richiesti e la successiva completezza ci si è affidati ad un particolare strumento di tracciamento definito in \infoNDP.
L'avanzamento dei processi, dettato dal \infoPDP, è stato mantenuto in controllo tramite una costante verifica delle metriche definite in questo documento e di cui troviamo una rappresentazione grafica in seguito.
\subsection{Dettaglio dell'attività di verifica su RR}
\subsubsection{Documenti}
Indice di \textit{Gulpease} per i documenti redatti:\\
\begin{longtable}{lllXr}
\toprule
\textbf{Documento} & \textbf{Valore di accettazione} & \textbf{Esito} \\
\toprule
\infoPDP & $>$40 & \textit{Positivo}\\
\midrule
\infoNDP & $>$40 & \textit{Positivo}\\
\midrule
\infoAR & $>$40 & \textit{Positivo}\\
\midrule
\infoPDQ & $>$40 & \textit{Positivo}\\
\midrule
\infoSDF & $>$40 & \textit{Positivo}\\
\bottomrule
\caption{Esito del calcolo indice di Gulpease per ogni documento per RR}
\end{longtable}
\subsubsection{Processi}
\gruppo ~ha condotto l'attività di verifica per i processi. In questo primo periodo il processo di documentazione è predominante nella pianificazione delle risorse. Di seguito viene riportato l'indice \textbf{SV (\textit{schedule variance})} per le attività eseguite e i risultati sono i seguenti:
\begin{longtable}{lllllXr}
\toprule
\textbf{Attività} & \textbf{Ore} & \textbf{Ore} & \textbf{SV} & \textbf{SV} \\
& \textbf{pianificate} & \textbf{rilevate} & \textbf{rilevato} & \textbf{accettazione}\\
\toprule
Norme di Progetto & 17 H & 17 H & 0 H & $>$ -1 H\\
\midrule
Studio di Fattibilità & 8 H & 14 H & \textbf{-6 H} & \textbf{$>$ -1 H}\\
\midrule
Analisi dei Requisiti & 70 H & 68 H & 2 H & $>$ -4 H\\
\midrule
Piano di Progetto & 37 H & 35 H & 2 H & $>$ -2 H\\
\midrule
Piano di Qualifica & 26 H & 22 H & 4 H & $>$ -2 H\\
\bottomrule
\caption{Indice SV per le attività per RR}
\end{longtable}
Da una prima analisi, si denota che \gruppo ~ha pianificato in modo preciso le attività.
L'attività di Studio di Fattibilità, essendo stato uno dei primi documenti che \gruppo ~ha redatto, la pianificazione non è stata precisa questo ha portato ad un SV dell'attività fuori dal range di accettazione. Le cause di questo problema sono da ricercare anche nella poca confidenza con gli strumenti di \textit{editor} testi e con gli strumenti di condivisione. La singola occorrenza del problema, non è quindi indice di allarme per gli altri processi che saranno pianificati nell'avanzamento del prodotto.
\begin{itemize}
\item SV-totale = 2 H;
\end{itemize}
SV-totale maggiore di zero denota che \gruppo ~stà producendo più velocemente rispetto a quanto pianificato. Questo può essere una diretta conseguenza dell'aggiunta di uno \textit{slack} temporale nella pianificazione delle attività.
Il team ha valutato la possibilità di ridurre il tempo di \textit{slack}, per fare in modo che la pianificazione corrisponda alla realtà; ma data la variabilità delle attività che \gruppo ~intende svolgere nel \gruppo ha effettuato l'analisi dei documenti e ha rilevato la conformità controllando l'indice di Gulpease di tutti i documenti prodotti o modificati in questo periodo di tempo.proseguo del progetto e la poca esperienza, è stato deciso di non modificare tale valore.\\
Di seguito viene riportato l'indice \textbf{BV (\textit{budget variance})} per le attività eseguite e i risultati sono i seguenti:
\begin{longtable}{lllllXr}
\toprule
\textbf{Attività} & \textbf{Costo} & \textbf{Costo} & \textbf{BV} & \textbf{BV} \\
& \textbf{pianificato} & \textbf{consuntivo} & \textbf{rilevato} & \textbf{limite}\\
\toprule
Norme di Progetto & 325.00 € & 325.00 € & 0.00 € & $>$ -32.50 €\\
\midrule
Studio di Fattibilità & 180.00 € & 310.00 € & \textbf{-130.00 €} & \textbf{$>$ -18.00 €}\\
\midrule
Analisi dei Requisiti & 1630.00 € & 1600.00 € & 30.00 € & $>$ -16.30 €\\
\midrule
Piano di Progetto & 1005.00 € & 945.00 € & 60.00 € & $>$ -10.05 €\\
\midrule
Piano di Qualifica & 490.00 € & 420.00 € & 70.00 € & $>$ -49.00 €\\
\bottomrule
\caption{Indice BV per le attività per RR}
\end{longtable}
Come descritto sopra per SV, anche BV denota che il preventivo di costo previsto per le attività svolte è stato corretto.
In particolare nell'attività di Studio di Fattibilità il costo a consuntivo è stato maggiore rispetto a quello preventivato. Questo è da collegare al costo orario dell'amministratore e non meno alle cause elencate sopra per l'indice SV che ne è strettamente collegato.
Complessivamente \gruppo ~ha ottenuto:
\begin{itemize}
\item BV-totale = 30.00 €.
\end{itemize}
il risultato ottenuto è una diretta conseguenza di un preventivo appropriato, e quindi ad un piccolo margine di guadagno nel budget di spesa dell'intero progetto.
%Analisi REVISIONE DI PROGETTAZIONE
\subsection{Riassunto dell'attività di verifica su RP}
Il processo di progettazione è per \gruppo ~un importante strumento per giungere ai periodi successivi con maggiore sicurezza del lavoro da fare e con una stima precisa delle risorse e degli obiettivi che si è deciso di perseguire. Nonostante le tempistiche ridotte, l'organizzazione del lavoro ha voluto che prima di procedere alla progettazione vera e propria si correggessero le opportunità di miglioramento riscontrate in RR, con l'obiettivo di consolidare il grado di maturità dei processi esistenti prima di aggiungerne di nuovi. Nella progettazione, descritta nella \infoST, si è provveduto a delineare la struttura del software provando a pianificare la tipologia di test che \gruppo~ andrà ad eseguire nei periodi a seguire. 
\subsection{Dettaglio dell'attività di verifica su RP}
\subsubsection{Documenti}
Indice di \textit{Gulpease} per i documenti redatti:\\
\begin{longtable}{lllXr}
\toprule
\textbf{Documento} & \textbf{Valore di accettazione} & \textbf{Esito} \\
\toprule
\infoPDP & $>$40 & \textit{Positivo}\\
\midrule
\infoNDP & $>$40 & \textit{Positivo}\\
\midrule
\infoAR & $>$40 & \textit{Positivo}\\
\midrule
\infoPDQ & $>$40 & \textit{Positivo}\\
\midrule
\infoSDF & $>$40 & \textit{Positivo}\\
\midrule
\infoST & $>$40 & \textit{Positivo}\\
\bottomrule
\caption{Esito del calcolo indice di Gulpease per ogni documento per RP}
\end{longtable}
\subsubsection{Processi}
\gruppo ~ha condotto l'attività di verifica per i processi. In questo primo periodo il processo di documentazione è predominante nella pianificazione delle risorse. Di seguito viene riportato l'indice \textbf{SV (\textit{schedule variance})} per le attività eseguite e i risultati sono i seguenti:
\begin{longtable}{lllllXr}
\toprule
\textbf{Attività} & \textbf{Ore} & \textbf{Ore} & \textbf{SV} & \textbf{SV} \\
& \textbf{pianificate} & \textbf{rilevate} & \textbf{rilevato} & \textbf{accettazione}\\
\toprule
Norme di Progetto & 5 H & 9 H & \textbf{-4 H} & $>$ -1 H\\
\midrule
Analisi dei Requisiti & 53 H & 25 H & 27 H & $>$ -3 H\\
\midrule
Piano di Progetto & 42 H & 25 H & 17 H & $>$ -2 H\\
\midrule
Piano di Qualifica & 7 H & 23 H & \textbf{-16 H} & $>$ -1 H\\
\midrule
Specifica Tecnica & 74 H & 90 H & \textbf{-32 H} & $>$ -4 H\\
\bottomrule
\caption{Indice SV per le attività per RP}
\end{longtable}

Da una prima analisi, si denota che \gruppo ~ha pianificato in modo preciso le attività.
Questo ha fatto in modo che la progettazione architetturale sia stata svolta più velocemente rispetto a quanto pianificato. La causa principale può essere ricercata nella velocità di accordo tra i progettisti di \gruppo, che hanno definito in modo rapido la progettazione del prodotto. Questo ha portato al risultato positivo.
\begin{itemize}
\item SV-totale = 2 H;
\end{itemize}
SV-totale maggiore di zero denota che \gruppo ~stà producendo più velocemente rispetto a quanto pianificato. Questo può essere una diretta conseguenza dell'aggiunta di uno \textit{slack} temporale nella pianificazione delle attività.
Il team ha valutato la possibilità di ridurre il tempo di \textit{slack}, per fare in modo che la pianificazione corrisponda alla realtà; ma data la variabilità delle attività che \gruppo ~intende svolgere nel proseguo del progetto e la poca esperienza, è stato deciso di non modificare tale valore.\\
Di seguito viene riportato l'indice \textbf{BV (\textit{budget variance})} per le attività eseguite e i risultati sono i seguenti:
\begin{longtable}{lllllXr}
\toprule
\textbf{Attività} & \textbf{Costo} & \textbf{Costo} & \textbf{BV} & \textbf{BV} \\
& \textbf{pianificato} & \textbf{consuntivo} & \textbf{rilevato} & \textbf{limite}\\
\toprule
Norme di Progetto & 61.00 € & 105.00 € & \textbf{-44.00} € & $>$ -6.10 €\\
\midrule
Analisi dei Requisiti & 1253.00 € & 578.00 € & 675.00 € & $>$ -125.30 €\\
\midrule
Piano di Progetto & 1182.00 € & 705.00 € & 477.00 € & $>$ -118.20 €\\
\midrule
Piano di Qualifica & 71.00 € & 582.00 € & \textbf{-511.00 €} & $>$ -7.10 €\\
\midrule
Specifica Tecnica & 1584.00 € & 2132.00 € & \textbf{-548.00 €} & $>$ -158.40 €\\
\bottomrule
\caption{Indice BV per le attività per RP}
\end{longtable}
Come descritto sopra per SV, anche BV denota che il preventivo di costo previsto per le attività svolte è stato corretto.
In particolare la stesura della specifica tecnica ha richiesto più budget di quello preventivato, le altre attività non hanno però consumato del tutto il proprio budget. Questo nonostante le correzioni da effettuare sulla documentazione prodotta in RR.
Complessivamente \gruppo ~ha ottenuto:
\begin{itemize}
\item BV-totale = 49.00 €.
\end{itemize}
il risultato ottenuto è una diretta conseguenza di un preventivo appropriato, e quindi ad un piccolo margine di guadagno nel budget di spesa dell'intero progetto.
%Attività di verifica su RQ
\subsection{Riassunto dell'attività di verifica su RQ} %da aggiornare
\gruppo, ha deciso di valorizzare sopratutto i requisiti desiderabili per fare in modo di tenere alto l'indice di efficienza\ped{G}. Valutando, durante lo stato di avanzamento, quali di questi requisiti saranno successivamente sviluppati nella realizzazione del prodotto software. Questo tipo di scelte risultano infatti difficoltose allo stato attuale in quanto una pianificazione approfondita necessiterebbe di \textit{lead time} \ped{G} precisi che non sono al momento tra le conoscenze dei componenti del gruppo.
Nel periodo di tempo che ha portato \gruppo ~alla consegna di questa revisione sono stati verificati i \textbf{documenti} ed i \textbf{processi}.\\
I \textbf{documenti} sono stati verificati anche durante le operazioni di redazione per portare a conoscenza dei contenuti tutti i componenti del gruppo di lavoro.
L'analisi statica, in primo luogo utilizzando la tecnica del \textit{walkthrough}, ha portato alla redazione  di una lista di controllo che verrà poi incrementata ed utilizzata nell'analisi finale del documento prima di procedere alla consegna. Una volta rilevati gli errori questi sono stati notificati al redattore che ha proceduto alla correzione, evidenziando gli errori frequenti che sono stati utilizzati per migliorare il processo di verifica. \gruppo ~adotta il ciclo PDCA per rendere più efficiente\ped{G} ed efficace\ped{G} nel tempo il processo di verifica.\\
L'attività di verifica, inoltre, utilizzando la tecnica \textit{inspection} è stata utilizzata principalmente per la verifica dei grafici dei casi d'uso. Per verificare la correttezza dei requisiti richiesti e la successiva completezza ci si è affidati ad un particolare strumento di tracciamento definito in \infoNDP.
L'avanzamento dei processi, dettato dal \infoPDP, è stato mantenuto in controllo tramite una costante verifica delle metriche definite in questo documento e di cui troviamo una rappresentazione grafica in seguito.
\subsection{Dettaglio dell'attività di verifica su RQ}
\subsubsection{Documenti}
Indice di \textit{Gulpease} per i documenti redatti:\\
\begin{longtable}{lllXr}
\toprule
\textbf{Documento} & \textbf{Valore di accettazione} & \textbf{Esito} \\
\toprule
\infoPDP & $>$40 & \textit{Positivo}\\
\midrule
\infoNDP & $>$40 & \textit{Positivo}\\
\midrule
\infoAR & $>$40 & \textit{Positivo}\\
\midrule
\infoPDQ & $>$40 & \textit{Positivo}\\
\midrule
\infoSDF & $>$40 & \textit{Positivo}\\
\midrule
\infoDP & $>$40 & \textit{Positivo}\\
\midrule
\infoMU & $>$40 & \textit{Positivo}\\
\midrule
\infoMPO & $>$40 & \textit{Positivo}\\
\midrule
\infoST & $>$40 & \textit{Positivo}\\
\bottomrule
\caption{Esito del calcolo indice di Gulpease per ogni documento per RQ}
\end{longtable}
\subsubsection{Processi}
\gruppo ~ha condotto l'attività di verifica per i processi. Di seguito è riportato l'indice \textbf{SV (\textit{schedule variance})} per le attività eseguite.
\begin{longtable}{lllllXr}
\toprule
\textbf{Attività} & \textbf{Ore} & \textbf{Ore} & \textbf{SV} & \textbf{SV} \\
& \textbf{pianificate} & \textbf{rilevate} & \textbf{rilevato} & \textbf{accettazione}\\
\toprule
Norme di Progetto & 8 H & 6 H & 2 H & $>$ -1 H\\
\midrule
Piano di Progetto & 13 H & 6 H & 7 H & $>$ -1 H\\
\midrule
Piano di Qualifica & 19 H & 27 H & \textbf{-8 H} & $>$ \textbf{-1 H}\\
\midrule
Specifica Tecnica & 7 H & 56 H & \textbf{-49 H} & \textbf{$>$ -1 H}\\
\midrule
Definizione di Prodotto & 78 H & 84 H & \textbf{-6 H} & \textbf{$>$ -4 H}\\
\midrule
Codifica & 51 H & 48 H & 3 H & $>$ -3 H\\
\midrule
Manuale Utente & 17 H & 15 H & 2 H & $>$ -1 H\\
\midrule
Attività di Verifica & 93 H & 40 H & 53 H & $>$ -5 H\\
\bottomrule
\caption{Indice SV per le attività per RQ.}
\end{longtable}
Dall'analisi della tabella sopra si denota come la pianificazione del periodo sia stata poco efficace. Questo è dovuto ad una attività di progettazione svolta in modo non preciso, che ha portato ad una correzione e nuova stesura di alcuni documenti. Inoltre, si vede come la pianificazione dei test e la progettazione di dettaglio siano state sottovalutate nella pianificazione del periodo.
Del resto l'attività di verifica e di codifica, sulla quale si era pianificato un notevole monte ore, sono risultate essere attività sopravalutate.
\begin{itemize}
\item SV-totale = 4 H;
\end{itemize}

Di seguito viene riportato l'indice \textbf{BV (\textit{budget variance})} per le attività eseguite e i risultati sono i seguenti:
\begin{longtable}{lllllXr}
\toprule
\textbf{Attività} & \textbf{Costo} & \textbf{Costo} & \textbf{BV} & \textbf{BV} \\
& \textbf{pianificato} & \textbf{consuntivo} & \textbf{rilevato} & \textbf{limite}\\
\toprule
Norme di Progetto & 160.00 € & 120.00 € & 40.00 € & $>$ -16.00 €\\
\midrule
Piano di Progetto & 390.00 € & 180.00 € & 210.00 € & $>$ -39.00 €\\
\midrule
Piano di Qualifica & 475.00 € & 675.00 € & \textbf{-200.00 €} & $>$ -47.00 €\\
\midrule
Specifica Tecnica & 154.00 € & 1232.00 € & \textbf{-1078.00 €} & \textbf{$>$ -15.00 €}\\
\midrule
Definizione di Prodotto & 1759.00 € & 1812.00 € & \textbf{-53.00 €} & $>$ -176.00 €\\
\midrule
Codifica & 765.00 € & 720.00 € & 45.00 € & $>$ -77.00 €\\
\midrule
Manuale Utente & 374.00 € & 330.00 € & 44.00 € & $>$ -37.00 €\\
\midrule
Verifica & 1395.00 € & 600.00 € & 795.00 € & $>$ -140.00 €\\
\bottomrule
\caption{Indice BV per le attività per RQ}
\end{longtable}
Complessivamente \gruppo ~ha ottenuto:
\begin{itemize}
\item BV-totale = -197.00 €.
\end{itemize}
\subsubsection{Risultati delle misurazioni sul codice}
Di seguito i risultati dei test di analisi statica effettuati sul codice fin'ora prodotto. Si è deciso di non effettuare i test sulla parte HTML in quanto statica e priva di metodi da analizzare. Per ogni metrica viene indicata la media e il valore massimo, rilevati dall'analisi sul codice prodotto, sia parte server che parte client.
Il livello di accettabilità è valutato sul valore medio della metrica.\\
I valori fuori range saranno commentati di seguito.\\
\begin{longtable}{llllXr}
\toprule
\textbf{Metrica} & \textbf{Media} & \textbf{Massimo} & \textbf{Accettazione}\\
\toprule
Complessità ciclomatica & 4 & 13 & Acc.\\%1-15
\midrule
Livelli di annidamento & 1,46 & 3 & Acc.\\%1-6
\midrule
Attributi per classe & 2,3 & 8 & Acc.\\%0-16
\midrule
Parametri per metodo & 3 & 7 & Acc.\\%0-8
\midrule
Linee di codice per linee commento & 32\% & 44\% & Acc.\\%>0,25
\midrule
Copertura & 53\% &  & \textbf{N. Acc.}\\%80 percento
\bottomrule
\caption{Risultati delle misurazioni sul codice}
\end{longtable}
La copertura non è risultata accettabile in quanto si dovrebbe stabilizzare sopra al 80\%, questo sarà l'obiettivo dei antecedente i test di validazione aumentare la copertura dei test per fare in modo di limitare il codice non testato.
%da aggiornare
%
%
%
%
\subsection{Riassunto dell'attività di verifica su RA} 
Al momento tutti per gli opzionali ci mancano gestione del profilo e del PDF per i tecnici il caricamento su sourceforge
\gruppo, ha deciso di valorizzare sopratutto i requisiti desiderabili per fare in modo di tenere alto l'indice di efficienza\ped{G}.
Nel periodo di tempo che ha portato \gruppo ~alla consegna di questa revisione sono stati verificati i \textbf{documenti} ed i \textbf{processi}.\\
I \textbf{documenti} sono stati verificati anche durante le operazioni di redazione per portare a conoscenza dei contenuti tutti i componenti del gruppo di lavoro e per essere pronti per la consegna al committente.
L'analisi statica, in primo luogo utilizzando la tecnica del \textit{walkthrough}, ha portato alla redazione  di una lista di controllo che verrà poi incrementata ed utilizzata nell'analisi finale del documento prima di procedere alla consegna. Una volta rilevati gli errori questi sono stati notificati al redattore che ha proceduto alla correzione, evidenziando gli errori frequenti che sono stati utilizzati per migliorare il processo di verifica. \gruppo ~adotta il ciclo PDCA per rendere più efficiente\ped{G} ed efficace\ped{G} nel tempo il processo di verifica.\\
L'avanzamento dei processi, dettato dal \infoPDP, è stato mantenuto in controllo tramite una costante verifica delle metriche definite in questo documento e di cui troviamo una rappresentazione grafica in seguito.
Inoltre, in questo periodo l'attività di testing ha occupato gran parte delle ore disponibili. Le unità sono state testate tramite i test di unità descritti sopra e attraverso l'utilizzo di strumenti definiti in\infoNDP. 
\subsection{Dettaglio dell'attività di verifica su RA}
\subsubsection{Documenti}
Indice di \textit{Gulpease} per i documenti redatti:\\
\begin{longtable}{lllXr}
\toprule
\textbf{Documento} & \textbf{Valore di accettazione} & \textbf{Esito} \\
\toprule
\infoPDP & $>$40 & \textit{Positivo}\\
\midrule
\infoNDP & $>$40 & \textit{Positivo}\\
\midrule
\infoAR & $>$40 & \textit{Positivo}\\
\midrule
\infoPDQ & $>$40 & \textit{Positivo}\\
\midrule
\infoSDF & $>$40 & \textit{Positivo}\\
\midrule
\infoDP & $>$40 & \textit{Positivo}\\
\midrule
\infoMU & $>$40 & \textit{Positivo}\\
\midrule
\infoMPO & $>$40 & \textit{Positivo}\\
\midrule
\infoST & $>$40 & \textit{Positivo}\\
\bottomrule
\caption{Esito del calcolo indice di Gulpease per ogni documento per RA}
\end{longtable}
\subsubsection{Processi}
\gruppo ~ha condotto l'attività di verifica per i processi. Di seguito è riportato l'indice \textbf{SV (\textit{schedule variance})} per le attività eseguite.
\begin{longtable}{lllllXr}
\toprule
\textbf{Attività} & \textbf{Ore} & \textbf{Ore} & \textbf{SV} & \textbf{SV} \\
& \textbf{pianificate} & \textbf{rilevate} & \textbf{rilevato} & \textbf{accettazione}\\
\toprule
Norme di Progetto & 8 H & 6 H & 2 H & $>$ -1 H\\
\midrule
Piano di Progetto & 13 H & 6 H & 7 H & $>$ -1 H\\
\midrule
Piano di Qualifica & 19 H & 27 H & \textbf{-8 H} & $>$ \textbf{-1 H}\\
\midrule
Specifica Tecnica & 7 H & 56 H & \textbf{-49 H} & \textbf{$>$ -1 H}\\
\midrule
Definizione di Prodotto & 78 H & 84 H & \textbf{-6 H} & \textbf{$>$ -4 H}\\
\midrule
Codifica & 51 H & 48 H & 3 H & $>$ -3 H\\
\midrule
Manuale Utente & 17 H & 15 H & 2 H & $>$ -1 H\\
\midrule
Attività di Verifica & 93 H & 40 H & 53 H & $>$ -5 H\\
\bottomrule
\caption{Indice SV per le attività per RQ.}
\end{longtable}
Dall'analisi della tabella sopra si denota come la pianificazione del periodo sia stata poco efficace. Questo è dovuto ad una attività di progettazione svolta in modo non preciso, che ha portato ad una correzione e nuova stesura di alcuni documenti. Inoltre, si vede come la pianificazione dei test e la progettazione di dettaglio siano state sottovalutate nella pianificazione del periodo.
Del resto l'attività di verifica e di codifica, sulla quale si era pianificato un notevole monte ore, sono risultate essere attività sopravalutate.
\begin{itemize}
\item SV-totale = 4 H;
\end{itemize}

Di seguito viene riportato l'indice \textbf{BV (\textit{budget variance})} per le attività eseguite e i risultati sono i seguenti:
\begin{longtable}{lllllXr}
\toprule
\textbf{Attività} & \textbf{Costo} & \textbf{Costo} & \textbf{BV} & \textbf{BV} \\
& \textbf{pianificato} & \textbf{consuntivo} & \textbf{rilevato} & \textbf{limite}\\
\toprule
Norme di Progetto & 160.00 € & 120.00 € & 40.00 € & $>$ -16.00 €\\
\midrule
Piano di Progetto & 390.00 € & 180.00 € & 210.00 € & $>$ -39.00 €\\
\midrule
Piano di Qualifica & 475.00 € & 675.00 € & \textbf{-200.00 €} & $>$ -47.00 €\\
\midrule
Specifica Tecnica & 154.00 € & 1232.00 € & \textbf{-1078.00 €} & \textbf{$>$ -15.00 €}\\
\midrule
Definizione di Prodotto & 1759.00 € & 1812.00 € & \textbf{-53.00 €} & $>$ -176.00 €\\
\midrule
Codifica & 765.00 € & 720.00 € & 45.00 € & $>$ -77.00 €\\
\midrule
Manuale Utente & 374.00 € & 330.00 € & 44.00 € & $>$ -37.00 €\\
\midrule
Verifica & 1395.00 € & 600.00 € & 795.00 € & $>$ -140.00 €\\
\bottomrule
\caption{Indice BV per le attività per RQ}
\end{longtable}
Complessivamente \gruppo ~ha ottenuto:
\begin{itemize}
\item BV-totale = -197.00 €.
\end{itemize}
\subsubsection{Risultati delle misurazioni sul codice}
Di seguito i risultati dei test di analisi statica effettuati sul codice fin'ora prodotto. Si è deciso di non effettuare i test sulla parte HTML in quanto statica e priva di metodi da analizzare. Per ogni metrica viene indicata la media e il valore massimo, rilevati dall'analisi sul codice prodotto, sia parte server che parte client.
Il livello di accettabilità è valutato sul valore medio della metrica.\\
I valori fuori range saranno commentati di seguito.\\
\begin{longtable}{llllXr}
\toprule
\textbf{Metrica} & \textbf{Media} & \textbf{Massimo} & \textbf{Accettazione}\\
\toprule
Complessità ciclomatica & 4 & 13 & Acc.\\%1-15
\midrule
Livelli di annidamento & 1,46 & 3 & Acc.\\%1-6
\midrule
Attributi per classe & 2,3 & 8 & Acc.\\%0-16
\midrule
Parametri per metodo & 3 & 7 & Acc.\\%0-8
\midrule
Linee di codice per linee commento & 32\% & 44\% & Acc.\\%>0,25
\midrule
Copertura & 53\% &  & \textbf{N. Acc.}\\%80 percento
\bottomrule
\caption{Risultati delle misurazioni sul codice}
\end{longtable}
La copertura non è risultata accettabile in quanto si dovrebbe stabilizzare sopra al 80\%, questo sarà l'obiettivo dei antecedente i test di validazione aumentare la copertura dei test per fare in modo di limitare il codice non testato.
