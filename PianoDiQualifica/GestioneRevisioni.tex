\section{Gestione amministrativa della revisione}
\subsection{Comunicazione e risoluzione anomalie}
Un'anomalia consiste in una deviazione del prodotto dalle aspettative prefissate. Per la gestione e risoluzione di anomalie ci si affida allo strumento di
\textit{ticketing} adottato da \gruppo, e normato in \infoNDP. 
Il \textit{Verificatore}, per ogni anomalia riscontrata, dovrà aprire un nuovo ticket indirizzato al \textit{Responsabile di Progetto}, il quale, dopo aver valutato l'impatto costi/benefici lo approverà e aprirà in ticket per il \textit{Programmatore} che ha sviluppato quella parte di software o redatto il documento.
Per la procedura di creazione del ticket si rimanda a \infoNDP.
\subsection{Trattamento delle discrepanze}
Una discrepanza è un errore di coerenza tra il prodotto realizzato e quello atteso. \gruppo ~interpreta la discrepanza come una forma di anomalia non grave, e per questo verrà trattata come tale.
\subsection{Procedure di controllo di qualità di processo}
Le procedure di controllo qualità di processo si basano sul ciclo di Deming, che lo arricchisce. Fissare gli obiettivi, è la prima attività del ciclo di Deming di processo; tutte le successive attività devono mutare nel tempo per fare in modo che possano essere migliorate in modo continuativo. I processi saranno pianificati dettagliatamente e ogni pianificazione prevederà dei valori attesi dallo stesso, questi saranno confrontati con i risultati ottenuti alla terminazione del processo e analizzati. Se l'analisi di tali misure evidenzia valori che si discostano, in modo negativo, dal valore atteso, questo sarà indice di un'opportunità di miglioramento. Per ognuno di questi valori si ricercheranno le cause e si definiranno specifiche soluzioni intervenendo sul processo stesso ed eventualmente anche sul valore definito nella pianificazione iniziale.
\gruppo ~ha definito le principali misurazioni di processo:
\begin{itemize}
\item \textit{Lead time} preventivato e \textit{lead time} a consuntivo;
\item Risorse utilizzate durante il processo;
\item Cicli di processo;
\item Attinenza alla pianificazione iniziale;
\item Soddisfacimento dei requisiti richiesti.
\end{itemize}
Se non vengono rilevati problemi relativi ad un processo, è possibile aumentare l'efficenza del processo studiando tecniche migliorative che permettano di abbassare il \textit{lead time} o il numero di risorse impiegate, garantendo sempre che il prodotto finale abbia un elevato grado di soddisfacimento dei requisiti richiesti. Ad ogni modo, ogni singola misurazione può essere utilizzata per una più specifica pianificazione nelle successive esecuzioni di processo.