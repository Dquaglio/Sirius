\section{Gestione amministrativa della revisione}
\subsection{Comunicazione e risoluzione anomalie}
Un'anomalia consiste in una deviazione del prodotto dalle aspettative prefissate. Per la gestione e risoluzione di anomalie ci si affida allo strumento di
\textit{ticketing} adottato da \gruppo, e normato in \infoNDP. 
Il \textit{Verificatore}, per ogni anomalia riscontrata, dovrà aprire un nuovo ticket indirizzato al \textit{Responsabile di Progetto}, il quale, dopo aver valutato l'impatto costi/benefici lo approverà e aprirà in ticket per il \textit{Programmatore} che ha sviluppato quella parte di software o redatto il documento.
Per la procedura di creazione del ticket si rimanda a \infoNDP.
\subsection{Trattamento delle discrepanze}
Una discrepanza è un errore di coerenza tra il prodotto realizzato e quello atteso. \gruppo ~interpreta la discrepanza come una forma di anomalia non grave, e per questo verrà trattata come tale.