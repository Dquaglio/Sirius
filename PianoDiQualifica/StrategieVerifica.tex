\section{Visione generale della strategia di verifica}
\subsection{Introduzione}
Il team \gruppo ~ha deciso di porre al centro di ogni fase l'attività di verifica in quanto essa certifica la qualità del prodotto. L'attività di verifica sarà continua in tutte le fasi del progetto.\\
\subsection{Organizzazione}
Il processo di qualifica accompagnerà tutte le fasi di ciclo di vita del software. Ogni procedura di verifica sarà schedulata attraverso appositi strumenti e i risultati saranno analizzati in questo documento. Tramite il diario delle modifiche è possibile tenere traccia dell'attività di verifica effettuata ed operare delle verifiche circoscritte ai soli cambiamenti.
\gruppo ~ha deciso di adottare una politica per lo sviluppo del progetto a ciclo di vita incrementale, come definito meglio in \infoPDP; tale scelta ha portato ad organizzare le attività di verifica nei periodi temporali antecedenti le revisioni.
In particolare le operazioni di controllo verranno istanziate quando il prodotto da analizzare avrà raggiunto uno stato in cui presenti differenze sostanziali rispetto allo stato precedente.
Lo schema che rappresenta l'organizzazione e la pianificazione delle attività di verifica è il \textbf{modello a V}(V-model). Questo prevede che invece di discendere lungo una linea retta, dopo la fase di programmazione risale con una tipica forma a V. Il modello dimostra la relazione tra ogni fase del ciclo di vita dello sviluppo del software e la sua fase di \textit{testing}.
\begin{figure} [H]
\centering
     \includegraphics[scale=0.8]{../modello/img/V}\\
     \caption{Modello a V}\label{fig:1}
\end{figure}
\subsection{Pianificazione strategica temporale}
Al fine di rispettare in modo ristretto le scadenze citate di seguito e spiegate in modo approfondito in \infoPDP, \gruppo ~ha deciso di pianificare in modo approfondito e sistematico l'attività di verifica. Facendo in modo di rilevare e risolvere nel più breve tempo possibile gli errori che vengono rilevati per evitare che questi possano creare maggiori problematiche nell' avanzamento del prodotto software. Le scadenze sono di due tipi:
\begin{itemize}
\item Scadenze formali
\begin{itemize}
\item Revisione dei requisiti (RR): \textit{2014-03-05};
\item Revisione di accettazione (RA): \textit{2014-07-18}.
\end{itemize}
\item Scadenze di avanzamento
\begin{itemize}
\item Revisione di progettazione (RP): \textit{2014-04-02};
\item Revisione di qualifica (RQ): \textit{2014-07-02}.
\end{itemize}
\end{itemize}
\subsection{Descrizione della procedura di pianificazione adottata}
Per gestire al meglio la pianificazione, \gruppo ~prevede l'attività di verifica in tutti e quattro i periodi di avanzamento del prodotto, che sono paralleli alle scadenze definite in \infoPDP.
\begin{itemize}
\item \textbf{Analisi}: in questa prima fase il compito del \textit{Verificatore} è innanzitutto relativo alla documentazione e alla correttezza del tracciamento dei requisiti. Ogni documento che servirà per la consegna della RR, una volta ultimata la fase di redazione, verrà verificato in modo definitivo seguendo la procedura cos\'i definita:
\begin{enumerate}
\item Verrà controllata la correttezza dei contenuti rispetto alle aspettative del documento tramite una rilettura accurata;
\item Verrà controllata la correttezza grammaticale;
\item Verrà controllato che il documento rispetti le norme definite in \infoNDP ~tramite la lista di controllo presente in tale documento.
\item Verrà verificato che ogni requisito funzionale rilevato abbia una corrispondenza in almeno un caso d'uso e che questo sia tracciato tramite il software di tracciamento che \gruppo ~ha deciso di utilizzare;
\item Verrà verificato che ogni requisito di vincolo e di qualità sia tracciato tramite il software di tracciamento che \gruppo ~ha deciso di utilizzare.
\end{enumerate}
\item \textbf{Progettazione}: il \textit{Verificatore} ha l'importante compito di controllare il soddisfacimento dei requisiti indicati in fase di analisi. Inoltre, si devono verificare che i processi che portano all'incremento dei documenti redatti nel precedente periodo siano conformi alle procedure e regole descritte in \infoNDP;
\item \textbf{Programmazione}: il \textit{Verificatore} provvederà a controlli periodici e pianificati di porzioni di codice, inizialmente di tipo statico per poi passare a dei controlli di tipo dinamico per valutare la correttezza del software;
\item \textbf{Collaudo}: in questa fase le verifiche saranno esclusivamente di tipo dinamico per garantire che il prodotto risponda a tutti i requisiti indicati e a tutte le richieste del committente: sia implicite che esplicite.
Per le indicazioni precise circa la procedura di verifica adottata dal gruppo si fa riferimento a \infoNDP.
\end{itemize}
Per i diversi periodi l'attività di verifica sarà diversa in base alla tipologia di lavoro svolta. Essa verterà sulla verifica della documentazione all'analisi del prodotto software garantendo che il risultato sia efficace\ped{G} rispetto alla procedura analizzata, non perdendo di efficienza\ped{G} contrattuale. 
\subsection{Slack}
\gruppo, conscio della poca esperienza nella pianificazione e gestione di progetti di questo tipo, ha deciso di inserire degli \textit{slack}\ped{G} temporale durante la pianificazione delle attività. Tale scelta è approfondita in \infoNDP ~che ne definisce la quantità e in \infoPDP ~che ne analizza le motivazioni.
L'aggiunta di slack temporali, oltre a portare al progetto una pianificazione più precisa, comporta un aumento dei costi che è però commisurato all'aumento della qualità finale.
\subsection{Obiettivi}
\subsubsection{Qualità di processo}
Al fine di garantire la qualità di prodotto è necessario ricercare la qualità dei processi che lo definiscono. Per questo \gruppo ~ha deciso di adottare lo standard ISO/IEC 15504 denominato SPICE il quale fornisce le indicazioni necessarie a valutare l'idoneità dei processi attualmente in uso.
Per applicare correttamente questo modello, ed adattarlo alla gestione attuata dal gruppo di lavoro, si è deciso di utilizzare il ciclo di Deming o PDCA: il quale definisce una metodologia di controllo dei processi durante il loro ciclo di vita permettendo, inoltre, di migliorare in modo continuo la qualità.
Tramite al principio PDCA, descritto nella sezione A.1, è garantito un miglioramento continuo dei processi e quindi della qualità degli stessi: come diretta conseguenza si otterrà il miglioramento qualitativo del prodotto risultante.
Per attenere questo è necessario che il processo sia in controllo e quindi:
\begin{itemize}
\item Effettuare una dettagliata pianificazione dei processi;
\item Pianificare anche il numero di risorse da utilizzare;
\end{itemize}
Altre due metriche utilizzate per la qualità di processo e per mantenere il processo in controllo sono BV che monitora il il consumo del budget nel tempo e SV che valuta lo stato di avanzamento temporale attuale rispetto a quello pianificato. Le specifiche di queste due metriche sono indicate in \infoNDP. Valori ottimali di BV ed SV indicano un elevato grado di conoscenza ed integrazione del processo come indicato nel CMMI nella sezione A.3.
La qualità dei processi viene monitorata anche tramite la qualità del prodotto: infatti un prodotto di bassa qualità indica un processo da migliorare. Se ad un processo non vengono rilevati problemi, è possibile apportare dei miglioramenti: riducendo il numero di cicli iterattivi, il tempo o le risorse ma garantendo che l'esecuzione del processo sia fedele al piano e soddisfi i requisiti. In questo modo si aumenta l'efficenza del processo e se ne determina un'adattamento positivo alla realtà del team di lavoro valutabile in efficacia.
\subsubsection{Qualità di prodotto}
Al fine di aumentare il valore del prodotto e di garantire il corretto funzionamento dello stesso è necessario fissare degli obiettivi e garantire che questi vengano effettivamente rispettati.
Lo standard ISO/IEC 9126 è stato redatto con lo scopo di descrivere questi obiettivi e di delineare delle metriche capaci di misurare il raggiungimento degli stesso.
Ognuno di questi obiettivi verrà valutato al termine del processo di verifica per avere dei risultati da analizzare. La valutazione dei risultati ottenuti è molto importante, soprattutto nei primi momenti di creazione del gruppo, in quanto questo deve essere un punto di partenza per modificare, creare o ridefinire le procedure in modo da evitare il ripercuotersi sistematico degli errori.
Di seguito elenchiamo le caratteristiche che \gruppo ~si impegna a garantire per il prodotto che andrà a realizzare: Oltre alla descrizione della caratteristica qui vengono definite le metriche, i parametri di accettazione. 
\begin{itemize}
\item \textbf{Funzionalità}
L’applicazione prodotta deve soddisfare tutti i requisiti obbligatori individuati in \infoAR ~nel modo più completo ed economico possibile, garantendo la sicurezza del prodotto e dei suoi componenti, e adeguandosi alle norme e alle prescrizioni imposte. Inoltre, data la natura del prodotto \progetto, \gruppo ~ha deciso di prestare particolare attenzione al'interoperabilità del codice: intesa come la capacità di agire con altri sistemi.
\begin{itemize}
\item \textbf{misura:} percentuale di requisiti soddisfatti;
\item \textbf{metrica:} la soglia di sufficienza sia il soddisfacimento di tutti i requisiti obbligatori;
\end{itemize}

\item \textbf{Affidabilità}
L’applicazione deve dimostrarsi robusta, di facile ripristino e recupero in caso di errori, e aderire alle norme e alle prescrizioni stabilite.
\begin{itemize}
\item \textbf{misura:} numero di esecuzioni dell'applicazione andate a buon fine;
\item \textbf{metrica:} le esecuzioni dovranno spaziare su tutta la gamma delle possibili casistiche. Il numero di esecuzioni andate a buon fine dovrà essere rapportato al numero totale delle casistiche considerate;
\end{itemize}

\item \textbf{Usabilità}
L’applicazione deve risultare comprensibile, facilmente apprendibile e soprattutto aderire a norme e prescrizioni per garantire facilità d’uso e soddisfacimento delle necessità dell’utente.
\begin{itemize}
\item \textbf{misura:} data l'aleatorietà della qualità richiesta, non si riesce a definire un'unità di misura obiettiva;
\item \textbf{metrica:} non è stata definita una metrica di usabilità; ma \gruppo ~cercherà di offrire la miglior esperienza di utilizzo per tutti coloro che usano il prodotto;
\end{itemize}

\item \textbf{Efficienza}
L’applicazione deve fornire tutte le funzionalità nel minor tempo possibile e con il minimo utilizzo di risorse.
\begin{itemize}
\item \textbf{misura:} il tempo di latenza per ottenere una risposta dal programma;
il tempo di latenza per ottenere una risposta simulando un sovraccarico della rete;
\item \textbf{metrica:} i tempi di latenza dovranno essere in linea con le tempistiche rilevate con l'utilizzo di architetture dello stesso tipo di quelle definite nel prodotto;
\end{itemize}

\item \textbf{Manutenibilità}
L’applicazione deve essere analizzabile, facilmente modificabile e verificabile; inoltre dovrà ridurre il rischio di comportamenti inaspettati al seguito dell'effettuazione di modifiche. Inoltre la documentazione prodotta deve essere chiara e comprensibile.
\begin{itemize}
\item \textbf{misura:} le misurazioni per garantire questa caratteristica sono diverse e non esclusive e sono descritte in seguito;
\item \textbf{metrica:} le metriche da utilizzare sono descritte in seguito;
\end{itemize}

\item \textbf{Portabilità}
L’applicazione deve essere adattabile e compatibile con ambienti d’uso diversi, e con i quali dovrà coesistere condividendo risorse e anche per questo sarà validata da strumenti forniti dal W3C.
\begin{itemize}
\item \textbf{misura:} L'applicazione dovrà essere eseguibile con i browser indicati in \infoAR;
\item \textbf{metrica:} soddisfacimento dei requisiti di compatibilità e validazione tramite strumenti W3C;
\end{itemize}
\end{itemize}
Inoltre, \gruppo ~ ha definito delle altre caratteristi che andranno ricercate per il prodotto:
\begin{itemize}

\item \textbf{Semplicità}: realizzazione del prodotto nella maniera più semplice possibile, ma non semplicistica;
\item \textbf{Incapsulamento}: il codice deve avere visibilità minima e permettere un utilizzo dall’esterno solamente mediante interfacce; ciò aumenta la manutenibilità e la possibilità di riuso del codice;
\item \textbf{Coesione}: le funzionalità che concorrono allo stesso fine devono risiedere nello stesso componente; favorisce semplicità, manutenibilità, riusabilità e riduce l’indice di dipendenza.
\end{itemize}
Al fine di controllare la qualità di prodotto raggiunta \gruppo ha adottato le seguenti procedure:
\begin{itemize}
\item \textbf{\textit{Quality assurance}}: insieme di attività atte a garantire il raggiungimento degli obiettivi di qualità;
\item \textbf{Verifica}: determina se l'output di una fase è \textbf{consistente}, \textbf{completo} e \textbf{corretto}. Il processo di verifica è eseguito in modo uniforme e durante tutto il periodo di realizzazione del prodotto. I risultati della verifica sono analizzati di volta in volta e riportati in seguito al documento;
\item \textbf{Validazione}: e si intende la conferma in modo oggettivo che il sistema risponde ai requisiti.
\end{itemize}
La definizione delle metriche e gli strumenti utilizzati per la rilevazione delle stesse sono specificate in \infoNDP. Di seguito per ogni metrica indichiamo le misure desiderabili: divise in range ottimale e range di accettazione. D.S indica che sarà definito in seguito.\\
\begin{tabular}{|c|c|c|c|}
\hline
\textbf{Metrica} & \textbf{Accettazione} & \textbf{Ottimale}\\
\hline
SV\footnote{}&$>$-(ore prev x5\%)&$>$0\\
\hline
BV\footnote{}&$>$-(costo prev x10\%)&$>$0\\
\hline
Gulpease\footnote{}&40-100&50-100\\
\hline
Complessità ciclomatica\footnote{}&1-15&1-10\\
\hline
Livelli di annidamento\footnote{}&1-6&1-3\\
\hline
Attributi per classe\footnote{}&0-16&3-8\\
\hline
Parametri per metodo\footnote{}&0-8&0-4\\
\hline
Linee di codice per &&\\linee di commento\footnote{}&$>$0,25&$>$0,30\\
\hline
Accoppiamento&D.S.&D.S.\\
\hline
Copertura&80\%-100\%&85\%-100\%\\
\hline
\end{tabular}\\
\begin{center}
Tabella 1: parametri delle metriche adottate.
\end{center}
I parametri ottimali e di accettazione sono riferiti a:
\begin{enumerate}
\item Parametro calcolato da \gruppo ~valutando l'inesperienza del gruppo;
\item Parametro calcolato da \gruppo ~valutando l'inesperienza del gruppo;
\item Parametro calcolato in base all'utenza della documentazione;
\item Il valore 10 come massimo fu raccomandato da T.J.McCabe, l'inventore di tale metrica;
\item Valore ideale valutando la chiarezza di codifica;
\item Valore ideale valutando la chiarezza di codifica;
\item Valore ideale valutando la chiarezza di codifica;
\item Il valore 0,30 è ricavato dal rapporto 22/78. Valori ricavati dalle medie dichiarate da Ohlo(Open source network).
\end{enumerate}
\subsection{Risorse umane e responsabilità}
Il responsabile delle relazioni con il committente è il \textit{Responsabile di Progetto} il quale è responsabile anche dell'operato dei \textit{Verificatori}. Mentre i \textit{Verificatori} sono a loro volta responsabili delle attività di verifica e delle procedure di controllo qualità . Il \textit{Verificatore}, inoltre, avrà un ruolo attivo in tutto il ciclo di vita del prodotto software, ma esso dovrà evitare di influire in modo troppo pesante sui processi software per evitare un rapporto costi/benefici troppo elevato. Compito dell' \textit{Amministratore} è quello di supporto a tutte le attività fornendo una solida infrastruttura software anche per il processo di verifica in ogni fase lavorativa.
Per una descrizione più approfondita di ruoli e responsabilità si rimanda a \infoNDP.\\
\subsection{Risorse software}
Al fine di effettuare la fase di verifica e validazione nel modo più sistematico possibile sono stati messi a disposizione di tutti i \textit{Verificatori} dall' \textit{Amministratore} un pacchetto di prodotti software il più specifico possibile rispetto alle esigenze del team. Inoltre, è sempre compito dell' \textit{Amministratore} formare ogni verificatore all'utilizzo dei prodotti che permettano la verifica, evidenziando, se richiesto le funzionalità non utilizzate per ogni prodotto. \gruppo{} ha deciso di adottare questo tipo di formazione per valorizzare il lavoro di chi effettivamente utilizza i prodotti di verifica, dando la possibilità che proprio da queste figure nascano idee e proposte di miglioramento che saranno poi valutate dal \textit{Responsabile di progetto} congiuntamente all'\textit{Amministratore}.
Gli strumenti necessari al raggiungimento degli obiettivi definiti è indicato in \infoNDP, che ne definiscono anche le specifiche tecniche e l'utilizzo.
\subsection{Tecniche di analisi}
\gruppo ~ha deciso di adottare diverse tecniche di analisi, elencate di seguito e descritte in \infoNDP.
Per l'analisi statica che è una tecnica di analisi applicabile sia alla documentazione che al codice e permette di effettuare la verifica di quanto prodotto individuando errori ed anomalie. Le tecniche di analisi statica utilizzate sono di tipo \textbf{inspection} per la documentazione, solo a seguito di una prima analisi di tipo \textbf{walkthrough}. Per il codice verrà adottata principalmente una tipologia di analisi \textbf{inspection} per gli errori che sono più ricorrenti o che sono stati rilevati nelle verifiche precedenti.
\gruppo ~ha deciso di verificare il prodotto software tramite analisi dinamica ciò consiste nell'esecuzione del codice mediante l'uso di test predisposti per verificarne il funzionamento o rilevare possibili difetti di implementazione eseguendo tutto o solo una parte del codice.\\
I test verteranno su test di unità per le porzioni di codice prodotte, test di integrazione per le componenti aggiunte in modo incrementale; test di sistema per verificare la corretta esecuzione del sistema; eventuali test di regressione per le modifiche apportate a componenti già testati; e infine test di accettazione per validare il prodotto finale.
Nella progettazione di dettaglio verranno definiti in modo specifico le caratteristiche per i test.