\section{Introduzione}
\subsection{Glossario}
Al fine di facilitare la comprensione del seguente documento, ed in generale di ogni documento che verrà fornito da parte del team \gruppo , è stato creato appositamente un glossario (\textit{\Glossario}) contenente la definizione dei termini più complessi o di quelli che necessitano un approfondimento. Questi vocaboli sono contrassegnati in ogni documento dal pedice G (\ped{G}).
\subsection{Riferimenti normativi}
\begin{itemize}
\item ISO/IEC Standard 12207:1995;
\item IEEE Std 610.12-1990 - Standard Glossary of Software Engineering Terminology;
\item IEEE Std 730TM-2002 (revision of IEEE Std 730-1998) - Standard for Software Quality Assurance Plans;
\item \infoNDP .
\end{itemize}
\subsection{Riferimenti informativi}
\begin{itemize}
\item Informazioni sul sito del docente \\
\url{http://www.math.unipd.it/~tullio/IS-1/2013/Progetto/PD01c.html#PQ}
\end{itemize}
\subsection{Scopo del documento}
Il documento si prefigge di illustrare la strategia complessiva di verifica e validazione proposta dal fornitore per pervenire al collaudo del sistema con la massima efficienza ed efficacia; garantendo in particolar modo:\\
\begin{itemize}
\item la correttezza del prodotto;
\item il soddisfacimento del cliente.
\end{itemize}
\subsection{Scopo del prodotto}
asd\\
