\section{Introduzione}
\subsection{Scopo del Prodotto}
Lo scopo del progetto \progetto{}, è di fornire un servizio di gestione di processi definiti da una serie di passi da eseguirsi in sequenza o senza un ordine predefinito, utilizzabile da dispositivi mobili di tipo smaptphone o tablet.
\subsection{Glossario}
Al fine di rendere più leggibile e comprensibile i documenti, i termini tecnici, di dominio, gli acronimi e le parole che necessitano di essere chiarite, sono riportate nel documento \Glossario{}.\\
Ogni occorrenza di vocaboli presenti nel \textit{Glossario} deve essere seguita da una ``G'' maiuscola in pedice.
\subsection{Riferimenti}
\subsubsection{Normativi}
\begin{itemize}
\item ISO/IEC Standard 12207:1995;
\item ISO/IEC 9126;
\item IEEE Std 730TM-2002 (revision of IEEE Std 730-1998) - Standard for Software Quality Assurance Plans;
\item \textbf{Norme di progetto:} \infoNDP ;
\item Capitolato d'appalto C4: \textit{sequenziatore}.
\end{itemize}
\subsubsection{Informativi}
\begin{itemize}
\item Informazioni sul sito del docente;
\item "Software Engineering (9th edition)" Ian Sommerville Pearson Education Addison-Wesley;
\item Ciclo di Deming - Estratto da Software Engineering (9th edition)
Ian Sommerville Pearson Education Addison-Wesley;
\item Capability Maturity Model Integration (CMMI) - Estratto da Software Engineering (9th edition)
Ian Sommerville Pearson Education Addison-Wesley;
\item SWEBOK cap.11 - \textit{Software Quality};
\item \textbf{Piano di progetto:} \infoPDP ;
\item Indice di Gulpease.
\end{itemize}
\subsection{Scopo del documento}
Il documento si prefigge di illustrare la strategia complessiva di verifica e validazione proposta dal team \gruppo ~per pervenire al collaudo del sistema con la massima efficacia\ped{G}.In questo documento, inoltre, definiamo gli obiettivi di qualità intesa come il rispetto dei requisiti e prestazioni enunciati esplicitamente, la conformità agli standard di sviluppo esplicitamente documentati e le caratteristiche implicite che si aspetta da un prodotto software. Garantendo in particolar modo ed in modo macroscopico:
\begin{itemize}
\item La correttezza del prodotto;
\item La verifica continua sulle attività svolte;
\item Il soddisfacimento del cliente.
\end{itemize}