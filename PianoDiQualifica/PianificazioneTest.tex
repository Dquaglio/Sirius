\section{Pianificazione dei test}
Di seguito elenchiamo tutti i test di validazione, sistema ed integrazione previsti, prevedendo però ad un successivo aggiornamento per i test di unità. Per quanto riguarda le tempistiche di esecuzione dei test si faccia riferimento a \infoPDP.
Il valore \textbf{N.A} è da intendersi come non applicato, in quanto tali test saranno eseguiti successivamente nello svolgimento del progetto.
\subsection{Test di sistema}
In questa sezione vengono descritti i test di sistema che consentiranno a \gruppo ~di verificare il comportamento dinamico del sistema rispetto ai requisiti descritti in \infoAR. I test sotto riportati sono relativi ai requisiti software individuati e meritevoli di test.
\subsubsection{Descrizione dei test di sistema}
\subsubsection{Ambito utente}
\LTXtable{\textwidth}{RequisitiUtente.tex}
\subsubsection{Ambito process owner}
\LTXtable{\textwidth}{RequisitiAmministratore.tex}
\subsection{Requisiti di vincolo}
\LTXtable{\textwidth}{RequisitiVincolo.tex}
\subsection{Test di integrazione}
In questa sezione verranno descritti i test di integrazione, da utilizzare per i vari componenti descritti nella progettazione ad alto livello, che permettono di verificare la corretta integrazione ed il corretto flusso dei dati all’interno del sistema.
Si è deciso di utilizzare una strategia di integrazione incrementale bottom-up che permette di sviluppare e verificare le componenti in parallelo.\\
Assemblando le componenti in modo incrementale i difetti rilevati da un test sono da
attribuirsi, con maggior probabilità, all’ultima parte aggiunta e si rende ogni passo di
integrazione reversibile consentendo di retrocedere verso uno stato noto e sicuro.
In questo modo i componenti base vengono testati più volte riducendo la possibile presenza di errori.\\
Nel diagramma 
\subsubsection{Descrizione dei test di integrazione}
Di seguito sono elencati i test di integrazione relativo ai componenti indicati in \infoST. I test di integrazione valutano l'integrazione dei vari componenti mano a mano che vengono aggregati al progetto, facendo in modo di non perdere le funzionalità acquisite fino a quel momento.
\begin{longtable}{lllllXr}
\toprule
\textbf{Test} & \textbf{Descrizione} & \textbf{Componente} & \textbf{Test} & \textbf{Stato}\\
&&&\textbf{di integrazione}&\\
\toprule
TU1&login utente e login process owner &Login & & Pianificato\\
\midrule
TU2&creazione processo&NuovoProcesso&&N.A.\\
\midrule
TU3&visualizzazione dati processo&VisualizzaProcesso&&N.A.\\
\midrule
TU4&Approvazione passo&ApprovazionePasso&&N.A.\\
\bottomrule
\caption{Tabella dei test di integrazione process owner}
\end{longtable}

\begin{longtable}{lllllXr}
\toprule
\textbf{Test} & \textbf{Descrizione} & \textbf{Componente} & \textbf{Test} & \textbf{Stato}\\
&&&\textbf{di integrazione}&\\
\toprule
TU5&gestione dati&GestioneUser&&N.A.\\
\midrule
TU6&iscrizione/disiscrizione&Registrazione&&N.A.\\
\midrule
TU7&esecuzione passo&EsecuzionePasso&&N.A.\\
\midrule
TU8&Visualizzazione report finale&Stampa&&N.A.\\
\bottomrule
\caption{Tabella dei test di integrazione user}
\end{longtable}
\subsection{Test di unità}
I test di unità sono effettuati creando una nuova classe basata sulla precedente denominata \textit{"testnomeclasse"} ed invocando i metodi di test con i rispettivi parametri. Per la definizione delle classi si faccia riferimento a \infoDP.
\subsubsection{Test di unità per view}
% VIEW

\newcommand{\viewAdmin}{se\fshyp{}quen\fshyp{}zia\fshyp{}to\fshyp{}re.cli\fshyp{}ent::view::pro\fshyp{}cess\fshyp{}ow\fshyp{}ner}

\newcommand{\iViewAdmin}{se\fshyp{}quen\fshyp{}zia\fshyp{}to\fshyp{}re::cli\fshyp{}ent::view::i\fshyp{}pro\fshyp{}cess\fshyp{}ow\fshyp{}ner}

\newcommand{\viewUser}{se\fshyp{}quen\fshyp{}zia\fshyp{}to\fshyp{}re::cli\fshyp{}ent::view::u\fshyp{}ser}

\newcommand{\iViewUser}{se\fshyp{}quen\fshyp{}zia\fshyp{}to\fshyp{}re::cli\fshyp{}ent::view::i\fshyp{}U\fshyp{}ser}

% PRESENTER

\newcommand{\iLogicAdmin}{se\fshyp{}quen\fshyp{}zia\fshyp{}to\fshyp{}re::cli\fshyp{}ent::pre\fshyp{}sen\fshyp{}ter::i\fshyp{}pro\fshyp{}cess\fshyp{}ow\fshyp{}ner::i\fshyp{}lo\fshyp{}gic}

\newcommand{\logicAdmin}{se\fshyp{}quen\fshyp{}zia\fshyp{}to\fshyp{}re::cli\fshyp{}ent::pre\fshyp{}sen\fshyp{}ter::pro\fshyp{}cess\fshyp{}ow\fshyp{}ner::lo\fshyp{}gic}

\newcommand{\iLogicUser}{se\fshyp{}quen\fshyp{}zia\fshyp{}to\fshyp{}re::cli\fshyp{}ent::pre\fshyp{}sen\fshyp{}ter::i\fshyp{}u\fshyp{}ser::i\fshyp{}lo\fshyp{}gic}

\newcommand{\logicUser}{se\fshyp{}quen\fshyp{}zia\fshyp{}to\fshyp{}re::cli\fshyp{}ent::pre\fshyp{}sen\fshyp{}ter::u\fshyp{}ser::lo\fshyp{}gic}

\newcommand{\serverCommunication}{se\fshyp{}quen\fshyp{}zia\fshyp{}to\fshyp{}re::cli\fshyp{}ent::pre\fshyp{}sen\fshyp{}ter::ser\fshyp{}ver\fshyp{}com\fshyp{}mu\fshyp{}ni\fshyp{}ca\fshyp{}tion}

\newcommand{\iServerCommunication}{se\fshyp{}quen\fshyp{}zia\fshyp{}to\fshyp{}re::cli\fshyp{}ent::pre\fshyp{}sen\fshyp{}ter::i\fshyp{}ser\fshyp{}ver\fshyp{}com\fshyp{}mu\fshyp{}ni\fshyp{}ca\fshyp{}tion}


% MODEL

\newcommand{\modelAdmin}{se\fshyp{}quen\fshyp{}zia\fshyp{}to\fshyp{}re::cli\fshyp{}ent::mo\fshyp{}del::lo\fshyp{}cal\fshyp{}da\fshyp{}ta\_pro\fshyp{}cess\fshyp{}\_\fshyp{}ow\fshyp{}ner}

\newcommand{\modelUser}{se\fshyp{}quen\fshyp{}zia\fshyp{}to\fshyp{}re::cli\fshyp{}ent::mo\fshyp{}del::lo\fshyp{}cal\fshyp{}da\fshyp{}ta\_u\fshyp{}ser}

\newcommand{\iModelAdmin}{se\fshyp{}quen\fshyp{}zia\fshyp{}to\fshyp{}re::cli\fshyp{}ent::mo\fshyp{}del::i\fshyp{}lo\fshyp{}cal\fshyp{}da\fshyp{}ta\_pro\fshyp{}cess\fshyp{}\_\fshyp{}ow\fshyp{}ner}

\newcommand{\iModelUser}{se\fshyp{}quen\fshyp{}zia\fshyp{}to\fshyp{}re::cli\fshyp{}ent::mo\fshyp{}del::i\fshyp{}lo\fshyp{}cal\fshyp{}da\fshyp{}ta\_u\fshyp{}ser}

\newcommand{\model}{se\fshyp{}quen\fshyp{}zia\fshyp{}to\fshyp{}re::cli\fshyp{}ent::mo\fshyp{}del}
Per la \textit{View}, essendo le classi prive di metodi, viene creato un metodo TestView() che riceve come parametro il tipo di dato richiesto e se ne verifica il risultato ottenuto confrontandolo con quello atteso.
\paragraph{Login}
\begin{itemize}
\item \textbf{Descrizione:} \textit{Template HTML} che permette di gestire l'interfaccia grafica relativa alle richieste di autenticazione al sistema.
\end{itemize}

\paragraph{MainUser}
\begin{itemize}
\item \textbf{Descrizione:} Classe che permette la gestione delle principali componenti dell'interfaccia grafica dell'utente.
\end{itemize}

\paragraph{Register}
\begin{itemize}
\item \textbf{Descrizione:} \textit{Template HTML} che permette di gestire dell'interfaccia grafica relativa alle richieste di registrazione da parte dell'utente.
\end{itemize}

\paragraph{UserData}
\begin{itemize}
\item \textbf{Descrizione:} \textit{Template HTML} che permette la realizzazione dei \textit{widget} che consentono visualizzazione e modifica dei dati dell'utente.
\end{itemize}

\paragraph{OpenProcess}
\begin{itemize}
\item \textbf{Descrizione:} \textit{Template HTML} che permette di realizzare i \textit{widget} per consentire l'apertura di un processo tramite ricerca o selezionandolo da una lista.
\end{itemize}

\paragraph{ManagementProcess}
\begin{itemize}
\item \textbf{Descrizione:} \textit{Template HTML} che permette di realizzare i \textit{widget} per consentire la visualizzazione dello stato del processo selezionato e i vincoli per concludere il passo in corso.
\end{itemize}

\paragraph{SendData}
\begin{itemize}
\item \textbf{Descrizione:} \textit{Template HTML} che permette di realizzare i \textit{widget} per consentire l'invio dei dati richiesti per la conclusione del passo in esecuzione.
\end{itemize}

\paragraph{SendText}
\begin{itemize}
\item \textbf{Descrizione:} \textit{Template HTML} che permette di realizzare i \textit{widget} che consentono di inserire il testo da inviare per concludere il passo in esecuzione.
\end{itemize}

\paragraph{SendNumb}
\begin{itemize}
\item \textbf{Descrizione:} \textit{Template HTML} che permette agli oggetti che la implementano di realizzare i \textit{widget} che consentono di inserire i dati numerici da inviare per concludere il passo in esecuzione.
\end{itemize}

\paragraph{SendPosition}
\begin{itemize}
\item \textbf{Descrizione:} \textit{Template HTML} che permette  di realizzare i \textit{widget} che consentono di inviare la posizione geografica richiesta per la conclusione del passo in esecuzione.
\end{itemize}

\paragraph{SendImage}
\begin{itemize}
\item \textbf{Descrizione:} \textit{Template HTML} che permette di realizzare i \textit{widget} che consentono di inserire le immagini richieste per concludere i passo in esecuzione.
\end{itemize}

\paragraph{PrintProcess}
\begin{itemize}
\item \textbf{Descrizione:} \textit{Template HTML} che permette di realizzare i \textit{widget} che consentono il salvataggio dei \textit{report} sull'esecuzione del processo.
\end{itemize}

\paragraph{MainProcessOwner}
\begin{itemize}
\item \textbf{Descrizione:} Componente che permette la gestione delle principali componenti dell'interfaccia grafica dell'utente \textit{process owner\ped{G}}.
\end{itemize}

\paragraph{NewProcess}
\begin{itemize}
\item \textbf{Descrizione:} \textit{Template HTML} che permette di gestire l'interfaccia grafica che consente di creare nuovi processi.
\end{itemize}

\paragraph{AddStep}
\begin{itemize}
\item \textbf{Descrizione:} \textit{Template HTML} che permette di gestire l'interfaccia grafica che consente di definire un nuovo passo del processo in creazione.
\end{itemize}

\paragraph{OpenProcess}
\begin{itemize}
\item \textbf{Descrizione:} \textit{Template HTML} che permette di realizzare i\textit{widget} che consentono di aprire un processo tramite ricerca o selezionandolo da una lista.
\end{itemize}

\paragraph{ManageProcess}
\begin{itemize}
\item \textbf{Descrizione:} \textit{Template HTML} che permette di realizzare i\textit{widget} che consentono di gestire l'accesso ai dati inviati al\textit{server\ped{G}} dagli utenti.
\end{itemize}

\paragraph{CheckStep}
\begin{itemize}
\item \textbf{Descrizione:} \textit{Template HTML} che permette di realizzare i\textit{widget} che consentono di gestire l'approvazione dei passi che richiedono intervento umano.
\end{itemize}

\subsubsection{Test di unità per presenter}
Per le classi \textit{Presenter} vengono verificati i metodi della classe, per controllare che ritornino il valore atteso.
\paragraph{Router}
\begin{flushleft}
\begin{itemize}
\item \textbf{Descrizione:} Classe che permette di coordinare l'inizializzazione e la renderizzazione delle pagine, gestendo gli eventi e le azioni di cambio pagina;
\item \textbf{Attributi:}
\begin{sloppypar}
\begin{itemize}
\item \texttt{+ UserData session:}\\ oggetto di tipo \texttt{\model{}.U\fshyp{}ser\fshyp{}Da\fshyp{}ta}, che consente di gestire la sessione dell'utente;
\item \texttt{+ Backbone.View[] views:}\\ array che contiene le classi del presenter in esecuzione;
\item \texttt{+ Object routes:}\\ oggetto ridefinito da \texttt{Backbone.Router} che associa ad ogni evento di \textit{routing\ped{G}}, un metodo della classe;
\end{itemize}
\end{sloppypar}
\item \textbf{Metodi:}
\begin{sloppypar}
\begin{itemize}
\item \texttt{+ void home():}\\ gestisce l'evento di \textit{routing\ped{G} home};
\item \texttt{+ void processes():}\\ gestisce l'evento di \textit{routing\ped{G} processes};
\item \texttt{+ void newProcess():}\\ gestisce l'evento di \textit{routing\ped{G} newProcess};
\item \texttt{+ void checkStep():}\\ gestisce l'evento di \textit{routing\ped{G} checkStep};
\item \texttt{+ void process():}\\ gestisce l'evento di \textit{routing\ped{G} process};
\item \texttt{+ void register():}\\ gestisce l'evento di \textit{routing\ped{G} register};
\item \texttt{+ void user():}\\ gestisce l'evento di \textit{routing\ped{G} user};
\item \texttt{+ bool checkSession(String pageId):}\\ ritorna \texttt{true} solo se l'utente è autenticato; in caso contrario crea e renderizza la pagina di \textit{login};
\item \texttt{+ void load(String resource, String pageId):}\\ crea e aggiunge una vista di tipo \textit{resource} al campo dati \texttt{this.views}, all'indice \textit{pageId};
\item \texttt{+ void changePage(String pageId):}\\ imposta la pagina con id \textit{pageId} come attiva, ed esegue la transizione di cambio pagina.
\end{itemize}
\end{sloppypar}
\end{itemize}
\end{flushleft}

\paragraph{Login}
\begin{flushleft}
\begin{itemize}
\item \textbf{Descrizione:} Classe che ha il compito di gestire le richieste di autenticazione al sistema;
\item \textbf{Attributi:}
\begin{sloppypar}
\begin{itemize}
\item \texttt{+ UserDataModel model:}\\ campo dati di tipo \texttt{\model{}.U\fshyp{}ser\fshyp{}Mo\fshyp{}del} che contiene i dati di sessione dell'utente;
\item \texttt{+ Object template:}\\ oggetto ridefinito da \texttt{Backbone.View}, che contiene il \textit{template HTML\ped{G}} associato alla classe;
\item \texttt{+ Object el:}\\ oggetto ridefinito da \texttt{Backbone.View} che rappresenta l'elemento \textit{HTML\ped{G}} entro cui la classe ascolta eventi generati dagli utenti;
\item \texttt{+ Object events:}\\ oggetto ridefinito da \texttt{Backbone.View} che associa ad ogni evento generato dagli utenti nella pagina \textit{HTML\ped{G}}, un metodo della classe;
\end{itemize}
\end{sloppypar}
\item \textbf{Metodi:}
\begin{sloppypar}
\begin{itemize}
\item \texttt{+ void initialize():}\\ metodo ridefinito da \texttt{Backbone.View}, invocato alla costruzione di ciascun oggetto della classe, che consente di aggiungere una pagina \textit{HTML\ped{G}} associata al componente;
\item \texttt{+ void render():}\\ metodo ridefinito da \texttt{Backbone.View}, che consente di aggiungere alla pagina \textit{HTML\ped{G}} il \textit{template} campo dati della classe;
\item \texttt{+ void login(Event event):}\\ effettua una richiesta di \textit{login}, utilizzando il campo dati \model{} per comunicare con il \textit{server\ped{G}}.
\end{itemize}
\end{sloppypar}
\end{itemize}
\end{flushleft}

\paragraph{MainUser}
\begin{flushleft}
\begin{itemize}
\item \textbf{Descrizione:} Classe che ha il compito della gestione generale della logica delle funzionalità utente;
\item \textbf{Attributi:}
\begin{sloppypar}
\begin{itemize}
\item \texttt{+ UserDataModel model:}\\ campo dati di tipo \texttt{\model{}.U\fshyp{}ser\fshyp{}Mo\fshyp{}del} che contiene i dati di sessione dell'utente;
\item \texttt{+ Object template:}\\ oggetto ridefinito da \texttt{Backbone.View}, che contiene il \textit{template HTML\ped{G}} associato alla classe;
\item \texttt{+ Object el:}\\ oggetto ridefinito da \texttt{Backbone.View} che rappresenta l'elemento \textit{HTML\ped{G}} entro cui la classe ascolta eventi generati dagli utenti;
\item \texttt{+ Object events:}\\ oggetto ridefinito da \texttt{Backbone.View} che associa ad ogni evento generato dagli utenti nella pagina \textit{HTML\ped{G}}, un metodo della classe;
\end{itemize}
\end{sloppypar}
\item \textbf{Metodi:}
\begin{sloppypar}
\begin{itemize}
\item \texttt{+ void initialize():}\\ metodo ridefinito da \texttt{Backbone.View}, invocato alla costruzione di ciascun oggetto della classe, che consente di aggiungere una pagina \textit{HTML\ped{G}} associata al componente;
\item \texttt{+ void render():}\\ metodo ridefinito da \texttt{Backbone.View}, che consente di aggiungere alla pagina \textit{HTML\ped{G}} il \textit{template} campo dati della classe.
\end{itemize}
\end{sloppypar}
\end{itemize}
\end{flushleft}

\paragraph{Register}
\begin{flushleft}
\begin{itemize}
\item \textbf{Descrizione:} Classe che ha il compito di gestire le richieste di registrazione da parte dell'utente;
\item \textbf{Attributi:}
\begin{sloppypar}
\begin{itemize}
\item \texttt{+ UserDataModel model:}\\ campo dati di tipo \texttt{\model{}.U\fshyp{}ser\fshyp{}Mo\fshyp{}del} che contiene i dati utente e di sessione;
\item \texttt{+ Object template:}\\ oggetto ridefinito da \texttt{Backbone.View}, che contiene il \textit{template HTML\ped{G}} associato alla classe;
\item \texttt{+ Object el:}\\ oggetto ridefinito da \texttt{Backbone.View} che rappresenta l'elemento \textit{HTML\ped{G}} entro cui la classe ascolta eventi generati dagli utenti;
\item \texttt{+ Object events:}\\ oggetto ridefinito da \texttt{Backbone.View} che associa ad ogni evento generato dagli utenti nella pagina \textit{HTML\ped{G}}, un metodo della classe;
\end{itemize}
\end{sloppypar}
\item \textbf{Metodi:}
\begin{sloppypar}
\begin{itemize}
\item \texttt{+ void initialize():}\\ metodo ridefinito da \texttt{Backbone.View}, invocato alla costruzione di ciascun oggetto della classe, che consente di aggiungere una pagina \textit{HTML\ped{G}} associata al componente;
\item \texttt{+ void render():}\\ metodo ridefinito da \texttt{Backbone.View}, che consente di aggiungere alla pagina \textit{HTML\ped{G}} il \textit{template} campo dati della classe;
\item \texttt{+ void register(Event event):}\\ effettua una richiesta di registrazione, utilizzando il campo dati \model{} per comunicare con il \textit{server\ped{G}}.
\end{itemize}
\end{sloppypar}
\end{itemize}
\end{flushleft}

\paragraph{UserData}
\begin{flushleft}
\begin{itemize}
\item \textbf{Descrizione:} Classe che ha il compito di gestire la visualizzazione e la modifica dei dati dell'utente;
\item \textbf{Relazioni con altri componenti:}
\item \textbf{Attributi:}
\begin{sloppypar}
\begin{itemize}
\item \texttt{+ UserDataModel model:}\\ campo dati di tipo \texttt{\model{}.U\fshyp{}ser\fshyp{}Mo\fshyp{}del} che contiene i dati utente e di sessione;
\item \texttt{+ Object template:}\\ oggetto ridefinito da \texttt{Backbone.View}, che contiene il \textit{template HTML\ped{G}} associato alla classe;
\item \texttt{+ Object el:}\\ oggetto ridefinito da \texttt{Backbone.View} che rappresenta l'elemento \textit{HTML\ped{G}} entro cui la classe ascolta eventi generati dagli utenti;
\item \texttt{+ Object events:}\\ oggetto ridefinito da \texttt{Backbone.View} che associa ad ogni evento generato dagli utenti nella pagina \textit{HTML\ped{G}}, un metodo della classe;
\end{itemize}
\end{sloppypar}
\item \textbf{Metodi:}
\begin{sloppypar}
\begin{itemize}
\item \texttt{+ void initialize():}\\ metodo ridefinito da \texttt{Backbone.View}, invocato alla costruzione di ciascun oggetto della classe, che consente di aggiungere una pagina \textit{HTML\ped{G}} associata al componente;
\item \texttt{+ void render():}\\ metodo ridefinito da \texttt{Backbone.View}, che consente di aggiungere alla pagina \textit{HTML\ped{G}} il \textit{template} campo dati della classe;
\item \texttt{+ void editData():}\\ utilizza il campo dati \texttt{model} per salvare i dati modificati dall'utente nel \textit{server\ped{G}}.
\end{itemize}
\end{sloppypar}
\end{itemize}
\end{flushleft}

\paragraph{OpenProcess}
\begin{flushleft}
\begin{itemize}
\item \textbf{Descrizione:} Classe che ha il compito di selezionare, ricercare e aprire un processo fra quelli eseguibili;
\item \textbf{Attributi:}
\begin{sloppypar}
\begin{itemize}
\item \texttt{+ ProcessCollection collection:}\\ campo dati di tipo \texttt{\collection{}.Pro\fshyp{}cess\fshyp{}Col\fshyp{}lec\fshyp{}tion} che contiene la lista dei processi non terminati o non ancora eliminati dall'utente;
\item \texttt{+ Object template:}\\ oggetto ridefinito da \texttt{Backbone.View}, che contiene il \textit{template HTML\ped{G}} associato alla classe;
\item \texttt{+ Object el:}\\ oggetto ridefinito da \texttt{Backbone.View} che rappresenta l'elemento \textit{HTML\ped{G}} entro cui la classe ascolta eventi generati dagli utenti;
\item \texttt{+ String id:}\\ campo dati ridefinito da \texttt{Backbone.View} contente l'id della classe;
\end{itemize}
\end{sloppypar}
\item \textbf{Metodi:}
\begin{sloppypar}
\begin{itemize}
\item \texttt{+ void initialize():}\\ metodo ridefinito da \texttt{Backbone.View}, invocato alla costruzione di ciascun oggetto della classe, che consente di aggiungere una pagina \textit{HTML\ped{G}} associata al componente;
\item \texttt{+ void render():}\\ metodo ridefinito da \texttt{Backbone.View}, che consente di aggiungere alla pagina \textit{HTML\ped{G}} il \textit{template} campo dati della classe;
\item \texttt{+ void update():}\\ aggiorna il campo dati \texttt{collection} comunicando con il \textit{server\ped{G}}.
\end{itemize}
\end{sloppypar}
\end{itemize}
\end{flushleft}

\paragraph{ManagementProcess}
\begin{itemize}
\item \textbf{Descrizione:} Classe che ha il compito di gestire e accedere alle informazioni relative allo stato del processo selezionato.
\item \textbf{Attributi:}
\begin{sloppypar}
\begin{itemize}
\item \texttt{+ ProcessModel process:}\\ campo dati di tipo \texttt{\model{}.Pro\fshyp{}cess\fshyp{}Mo\fshyp{}del} che contiene i dati del processo in gestione;
\item \texttt{+ Object template:}\\ oggetto ridefinito da \texttt{Backbone.View}, che contiene il \textit{template HTML\ped{G}} associato alla classe;
\item \texttt{+ Object el:}\\ oggetto ridefinito da \texttt{Backbone.View} che rappresenta l'elemento \textit{HTML\ped{G}} entro cui la classe ascolta eventi generati dagli utenti;
\item \texttt{+ String id:}\\ campo dati ridefinito da \texttt{Backbone.View} contente l'id della classe;
\end{itemize}
\end{sloppypar}
\item \textbf{Metodi:}
\begin{sloppypar}
\begin{itemize}
\item \texttt{+ void initialize():}\\ metodo ridefinito da \texttt{Backbone.View}, invocato alla costruzione di ciascun oggetto della classe, che consente di aggiungere una pagina \textit{HTML\ped{G}} associata al componente;
\item \texttt{+ void render():}\\ metodo ridefinito da \texttt{Backbone.View}, che consente di aggiungere alla pagina \textit{HTML\ped{G}} il \textit{template} campo dati della classe;
\item \texttt{+ void update():}\\ aggiorna i campi dati \texttt{process} e \texttt{processData} comunicando con il \textit{server\ped{G}};
\item \texttt{+ String getParam(String param):}\\ ritorna il valore del parametro \textit{param} se presente nella \textit{URL\ped{G}}.
\end{itemize}
\end{sloppypar}
\end{itemize}
\end{flushleft}

\paragraph{PrintReport}
\begin{flushleft}
\begin{itemize}
\item \textbf{Descrizione:} Classe che ha il compito di gestire la creazione del report di fine processo;
\item \textbf{Attributi:}
\begin{sloppypar}
\begin{itemize}
\item \texttt{+ ProcessDataCollection processdata:}\\ campo dati di tipo \texttt{\collection{}.Pro\fshyp{}cess\fshyp{}Da\fshyp{}ta\fshyp{}Col\fshyp{}lec\fshyp{}tion} che contiene i dati inviati dall'utente relativi al processo in gestione;
\item \texttt{+ Object template:}\\ oggetto ridefinito da \texttt{Backbone.View}, che contiene il \textit{template HTML\ped{G}} associato alla classe;
\item \texttt{+ Object el:}\\ oggetto ridefinito da \texttt{Backbone.View} che rappresenta l'elemento \textit{HTML\ped{G}} entro cui la classe ascolta eventi generati dagli utenti;
\item \texttt{+ String id:}\\ campo dati ridefinito da \texttt{Backbone.View} contente l'id della classe;
\end{itemize}
\end{sloppypar}
\item \textbf{Metodi:}
\begin{sloppypar}
\begin{itemize}
\item \texttt{+ void initialize():}\\ metodo ridefinito da \texttt{Backbone.View}, invocato alla costruzione di ciascun oggetto della classe, che consente di aggiungere una pagina \textit{HTML\ped{G}} associata al componente;
\item \texttt{+ void render():}\\ metodo ridefinito da \texttt{Backbone.View}, che consente di aggiungere alla pagina \textit{HTML\ped{G}} il \textit{template} campo dati della classe.
\end{itemize}
\end{sloppypar}
\end{itemize}
\end{flushleft}

\paragraph{SendData}
\begin{flushleft}
\begin{itemize}
\item \textbf{Descrizione:} Classe che ha il compito di gestire l'inserimento e l'invio di dati da parte degli utenti, per completare il passo corrente;
\item \textbf{Attributi:}
\begin{sloppypar}
\begin{itemize}
\item \texttt{+ ProcessDataCollection processdata:}\\ campo dati di tipo \texttt{\collection{}.Pro\fshyp{}cess\fshyp{}Da\fshyp{}ta\fshyp{}Col\fshyp{}lec\fshyp{}tion} che consente di interagire con la lista dei dati inviati dall'utente relativa al processo in gestione presente nel \textit{server\ped{G}};
\item \texttt{+ Object template:}\\ oggetto ridefinito da \texttt{Backbone.View}, che contiene il \textit{template HTML\ped{G}} associato alla classe;
\item \texttt{+ Object el:}\\ oggetto ridefinito da \texttt{Backbone.View} che rappresenta l'elemento \textit{HTML\ped{G}} entro cui la classe ascolta eventi generati dagli utenti;
\item \texttt{+ String id:}\\ campo dati ridefinito da \texttt{Backbone.View} contente l'id della classe;
\end{itemize}
\end{sloppypar}
\item \textbf{Metodi:}
\begin{sloppypar}
\begin{itemize}
\item \texttt{+ void initialize():}\\ metodo ridefinito da \texttt{Backbone.View}, invocato alla costruzione di ciascun oggetto della classe, che consente di aggiungere una pagina \textit{HTML\ped{G}} associata al componente;
\item \texttt{+ void render():}\\ metodo ridefinito da \texttt{Backbone.View}, che consente di aggiungere alla pagina \textit{HTML\ped{G}} il \textit{template} campo dati della classe. Utilizza le classi \texttt{\logicUser{}.SendText} \texttt{\logicUser{}.SendNumb}, \texttt{\logicUser{}.SendImage} e \texttt{\logicUser{}.SendPosition} per renderizzare l'interfaccia relativa all'inserimento dei diversi tipi di dato;
\item \texttt{+ bool getData():}\\ controlla se i dati inseriti dall'utente sono corretti: se lo sono ritorna \texttt{true} e li aggiunge alla collezione \texttt{processData}, altrimenti ritorna \texttt{false};
\item \texttt{+ bool saveData():}\\ utilizza metodi del campo dati \texttt{processData}, per inviare i dati raccolti al \textit{server\ped{G}}.
\end{itemize}
\end{sloppypar}
\end{itemize}
\end{flushleft}

\paragraph{SendText}
\begin{flushleft}
\begin{itemize}
\item \textbf{Descrizione:} Classe che permette l'inserimento e il controllo di dati testuali inseriti dagli utenti;
\item \textbf{Attributi:}
\begin{sloppypar}
\begin{itemize}
\item \texttt{+ Object template:}\\ oggetto ridefinito da \texttt{Backbone.View}, che contiene il \textit{template HTML\ped{G}} associato alla classe;
\item \texttt{+ Object el:}\\ oggetto ridefinito da \texttt{Backbone.View} che rappresenta l'elemento \textit{HTML\ped{G}} entro cui la classe ascolta eventi generati dagli utenti;
\item \texttt{+ String id:}\\ campo dati ridefinito da \texttt{Backbone.View} contente l'id della classe;
\end{itemize}
\end{sloppypar}
\item \textbf{Metodi:}
\begin{sloppypar}
\begin{itemize}
\item \texttt{+ void initialize():}\\ metodo ridefinito da \texttt{Backbone.View}, invocato alla costruzione di ciascun oggetto della classe, che consente di aggiungere una pagina \textit{HTML\ped{G}} associata al componente;
\item \texttt{+ void render():}\\ metodo ridefinito da \texttt{Backbone.View}, che consente di aggiungere alla pagina \textit{HTML\ped{G}} il \textit{template} campo dati della classe;
\item \texttt{+ bool getData(ProcessDataModel data):}\\ controlla se i dati inseriti dall'utente sono corretti: se lo sono ritorna \texttt{true} e li aggiunge al riferimento \texttt{data}, altrimenti ritorna \texttt{false}.
\end{itemize}
\end{sloppypar}
\end{itemize}
\end{flushleft}

\paragraph{SendNumb}
\begin{flushleft}
\begin{itemize}
\item \textbf{Descrizione:} Classe che ha il compito di permettere l'inserimento e il controllo di dati numerici inseriti dagli utenti;
\item \textbf{Attributi:}
\begin{sloppypar}
\begin{itemize}
\item \texttt{+ Object template:}\\ oggetto ridefinito da \texttt{Backbone.View}, che contiene il \textit{template HTML\ped{G}} associato alla classe;
\item \texttt{+ Object el:}\\ oggetto ridefinito da \texttt{Backbone.View} che rappresenta l'elemento \textit{HTML\ped{G}} entro cui la classe ascolta eventi generati dagli utenti;
\item \texttt{+ String id:}\\ campo dati ridefinito da \texttt{Backbone.View} contente l'id della classe;
\end{itemize}
\end{sloppypar}
\item \textbf{Metodi:}
\begin{sloppypar}
\begin{itemize}
\item \texttt{+ void initialize():}\\ metodo ridefinito da \texttt{Backbone.View}, invocato alla costruzione di ciascun oggetto della classe, che consente di aggiungere una pagina \textit{HTML\ped{G}} associata al componente;
\item \texttt{+ void render():}\\ metodo ridefinito da \texttt{Backbone.View}, che consente di aggiungere alla pagina \textit{HTML\ped{G}} il \textit{template} campo dati della classe;
\item \texttt{+ bool getData(ProcessDataModel data):}\\ controlla se i dati inseriti dall'utente sono corretti: se lo sono ritorna \texttt{true} e li aggiunge al riferimento \texttt{data}, altrimenti ritorna \texttt{false}.
\end{itemize}
\end{sloppypar}
\end{itemize}
\end{flushleft}

\paragraph{SendImage}
\begin{flushleft}
\begin{itemize}
\item \textbf{Descrizione:} Classe che gestisce l'inserimento e il controllo di immagini inserite dagli degli utenti;
\item \textbf{Attributi:}
\begin{sloppypar}
\begin{itemize}
\item \texttt{+ Object template:}\\ oggetto ridefinito da \texttt{Backbone.View}, che contiene il \textit{template HTML\ped{G}} associato alla classe;
\item \texttt{+ Object el:}\\ oggetto ridefinito da \texttt{Backbone.View} che rappresenta l'elemento \textit{HTML\ped{G}} entro cui la classe ascolta eventi generati dagli utenti;
\item \texttt{+ String id:}\\ campo dati ridefinito da \texttt{Backbone.View} contente l'id della classe;
\end{itemize}
\end{sloppypar}
\item \textbf{Metodi:}
\begin{sloppypar}
\begin{itemize}
\item \texttt{+ void initialize():}\\ metodo ridefinito da \texttt{Backbone.View}, invocato alla costruzione di ciascun oggetto della classe, che consente di aggiungere una pagina \textit{HTML\ped{G}} associata al componente;
\item \texttt{+ void render():}\\ metodo ridefinito da \texttt{Backbone.View}, che consente di aggiungere alla pagina \textit{HTML\ped{G}} il \textit{template} campo dati della classe;
\item \texttt{+ bool getData(ProcessDataModel data):}\\ controlla se i dati inseriti dall'utente sono corretti: se lo sono ritorna \texttt{true} e li aggiunge al riferimento \texttt{data}, altrimenti ritorna \texttt{false}.
\end{itemize}
\end{sloppypar}
\end{itemize}
\end{flushleft}

\paragraph{SendPosition}
\begin{flushleft}
\begin{itemize}
\item \textbf{Descrizione:} Classe che ha il compito di gestire il calcolo e il controllo della posizione geografica dell'utente;
\item \textbf{Attributi:}
\begin{sloppypar}
\begin{itemize}
\item \texttt{+ Object template:}\\ oggetto ridefinito da \texttt{Backbone.View}, che contiene il \textit{template HTML\ped{G}} associato alla classe;
\item \texttt{+ Object el:}\\ oggetto ridefinito da \texttt{Backbone.View} che rappresenta l'elemento \textit{HTML\ped{G}} entro cui la classe ascolta eventi generati dagli utenti;
\item \texttt{+ String id:}\\ campo dati ridefinito da \texttt{Backbone.View} contente l'id della classe;
\end{itemize}
\end{sloppypar}
\item \textbf{Metodi:}
\begin{sloppypar}
\begin{itemize}
\item \texttt{+ void initialize():}\\ metodo ridefinito da \texttt{Backbone.View}, invocato alla costruzione di ciascun oggetto della classe, che consente di aggiungere una pagina \textit{HTML\ped{G}} associata al componente;
\item \texttt{+ void render():}\\ metodo ridefinito da \texttt{Backbone.View}, che consente di aggiungere alla pagina \textit{HTML\ped{G}} il \textit{template} campo dati della classe;
\item \texttt{+ bool getData(ProcessDataModel data):}\\ controlla se i dati inseriti dall'utente sono corretti: se lo sono ritorna \texttt{true} e li aggiunge al riferimento \texttt{data}, altrimenti ritorna \texttt{false}.
\end{itemize}
\end{sloppypar}
\end{itemize}
\end{flushleft}


\paragraph{MainProcessOwner}
\begin{flushleft}
\begin{itemize}
\item \textbf{Descrizione:} Classe che ha il compito della gestione generale della logica delle funzionalità \textit{Process Owner\ped{G}};
\item \textbf{Attributi:}
\begin{sloppypar}
\begin{itemize}
\item \texttt{+ Object template:}\\ oggetto ridefinito da \texttt{Backbone.View}, che contiene il \textit{template HTML\ped{G}} associato alla classe;
\item \texttt{+ Object el:}\\ oggetto ridefinito da \texttt{Backbone.View} che rappresenta l'elemento \textit{HTML\ped{G}} entro cui la classe ascolta eventi generati dagli utenti;
\item \texttt{+ String id:}\\ campo dati ridefinito da \texttt{Backbone.View} contente l'id della classe;
\end{itemize}
\end{sloppypar}
\item \textbf{Metodi:}
\begin{sloppypar}
\begin{itemize}
\item \texttt{+ void initialize():}\\ metodo ridefinito da \texttt{Backbone.View}, invocato alla costruzione di ciascun oggetto della classe, che consente di aggiungere una pagina \textit{HTML\ped{G}} associata al componente;
\item \texttt{+ void render():}\\ metodo ridefinito da \texttt{Backbone.View}, che consente di aggiungere alla pagina \textit{HTML\ped{G}} il \textit{template} campo dati della classe.
\end{itemize}
\end{sloppypar}
\end{itemize}
\end{flushleft}

\paragraph{OpenProcess}
\begin{flushleft}
\begin{itemize}
\item \textbf{Descrizione:} Classe che ha il compito di gestire la ricerca e la selezione di un processo;
\item \textbf{Attributi:}
\begin{sloppypar}
\begin{itemize}
\item \texttt{+ ProcessCollection collection:}\\ campo dati di tipo \texttt{\collection{}.Pro\fshyp{}cess\fshyp{}Col\fshyp{}lec\fshyp{}tion} che contiene la lista dei processi non eliminati dal \textit{process owner\ped{G}};
\item \texttt{+ Object template:}\\ oggetto ridefinito da \texttt{Backbone.View}, che contiene il \textit{template HTML\ped{G}} associato alla classe;
\item \texttt{+ Object el:}\\ oggetto ridefinito da \texttt{Backbone.View} che rappresenta l'elemento \textit{HTML\ped{G}} entro cui la classe ascolta eventi generati dagli utenti;
\item \texttt{+ String id:}\\ campo dati ridefinito da \texttt{Backbone.View} contente l'id della classe;
\end{itemize}
\end{sloppypar}
\item \textbf{Metodi:}
\begin{sloppypar}
\begin{itemize}
\item \texttt{+ void initialize():}\\ metodo ridefinito da \texttt{Backbone.View}, invocato alla costruzione di ciascun oggetto della classe, che consente di aggiungere una pagina \textit{HTML\ped{G}} associata al componente;
\item \texttt{+ void render():}\\ metodo ridefinito da \texttt{Backbone.View}, che consente di aggiungere alla pagina \textit{HTML\ped{G}} il \textit{template} campo dati della classe;
\item \texttt{+ void update():}\\ aggiorna il campo dati \texttt{collection} comunicando con il \textit{server\ped{G}}.
\end{itemize}
\end{sloppypar}
\end{itemize}
\end{flushleft}

\paragraph{NewProcess}
\begin{flushleft}
\begin{itemize}
\item \textbf{Descrizione:} Classe che ha il compito di gestire la logica della definizione di un nuovo processo;
\item \textbf{Attributi:}
\begin{sloppypar}
\begin{itemize}
\item \texttt{+ ProcessCollection collection:}\\ campo dati di tipo \texttt{\collection{}.Pro\fshyp{}cess\fshyp{}Col\fshyp{}lec\fshyp{}tion} che consente di interagire con la lista dei processi non eliminati dal \textit{process owner\ped{G}}, presente nel \textit{server\ped{G}};
\item \texttt{+ ProcessModel model:}\\ campo dati di tipo \texttt{\model{}.Pro\fshyp{}cess\fshyp{}Mo\fshyp{}del} che contiene i dati del processo in definizione;
\item \texttt{+ Object template:}\\ oggetto ridefinito da \texttt{Backbone.View}, che contiene il \textit{template HTML\ped{G}} associato alla classe;
\item \texttt{+ Object el:}\\ oggetto ridefinito da \texttt{Backbone.View} che rappresenta l'elemento \textit{HTML\ped{G}} entro cui la classe ascolta eventi generati dagli utenti;
\item \texttt{+ String id:}\\ campo dati ridefinito da \texttt{Backbone.View} contente l'id della classe;
\end{itemize}
\end{sloppypar}
\item \textbf{Metodi:}
\begin{sloppypar}
\begin{itemize}
\item \texttt{+ void initialize():}\\ metodo ridefinito da \texttt{Backbone.View}, invocato alla costruzione di ciascun oggetto della classe, che consente di aggiungere una pagina \textit{HTML\ped{G}} associata al componente;
\item \texttt{+ void render(String[] errors):}\\ metodo ridefinito da \texttt{Backbone.View}, che consente di aggiungere alla pagina \textit{HTML\ped{G}} il \textit{template} campo dati della classe, compilato con gli eventuali errori \texttt{errors};
\item \texttt{+ void newStep():}\\ utilizza la classe \texttt{\logicAdmin{}.Add\fshyp{}Step} per definire e aggiungere un nuovo passo al processo \texttt{model};
\item \texttt{+ bool getData():}\\ controlla se i dati inseriti dal \textit{process owner\ped{G}} sono corretti: se lo sono ritorna \texttt{true} e li aggiunge al processo \texttt{model}, altrimenti ritorna \texttt{false};
\item \texttt{+ bool saveProcess():}\\ utilizza metodi del campo dati \texttt{collection}, per inviare il processo \texttt{model} al \textit{server\ped{G}}.
\end{itemize}
\end{sloppypar}
\end{itemize}
\end{flushleft}

\paragraph{AddStep}
\begin{flushleft}
\begin{itemize}
\item \textbf{Descrizione:} Classe che ha il compito di gestire la logica di definizione dei passi di un processo;
\item \textbf{Attributi:}
\begin{sloppypar}
\begin{itemize}
\item \texttt{+ StepModel model:}\\ campo dati di tipo \texttt{\model{}.Step\fshyp{}Mo\fshyp{}del} che contiene i dati del passo in definizione;
\item \texttt{+ Object template:}\\ oggetto ridefinito da \texttt{Backbone.View}, che contiene il \textit{template HTML\ped{G}} associato alla classe;
\item \texttt{+ Object el:}\\ oggetto ridefinito da \texttt{Backbone.View} che rappresenta l'elemento \textit{HTML\ped{G}} entro cui la classe ascolta eventi generati dagli utenti;
\item \texttt{+ String id:}\\ campo dati ridefinito da \texttt{Backbone.View} contente l'id della classe;
\end{itemize}
\end{sloppypar}
\item \textbf{Metodi:}
\begin{sloppypar}
\begin{itemize}
\item \texttt{+ void initialize():}\\ metodo ridefinito da \texttt{Backbone.View}, invocato alla costruzione di ciascun oggetto della classe, che consente di aggiungere una pagina \textit{HTML\ped{G}} associata al componente;
\item \texttt{+ void render(String[] errors):}\\ metodo ridefinito da \texttt{Backbone.View}, che consente di aggiungere alla pagina \textit{HTML\ped{G}} il \textit{template} campo dati della classe, compilato con gli eventuali errori \texttt{errors};
\item \texttt{+ bool getData():}\\ controlla se i dati inseriti dal \textit{process owner\ped{G}} sono corretti: se lo sono ritorna \texttt{true} e li aggiunge al passo \texttt{model}, altrimenti ritorna \texttt{false}.
\end{itemize}
\end{sloppypar}
\end{itemize}
\end{flushleft}

\paragraph{ManageProcess}
\begin{flushleft}
\begin{itemize}
\item \textbf{Descrizione:} Classe che ha il compito di gestire e accedere alle informazioni relative allo stato dei processi e ai dati inviati dagli utenti. Le operazioni di gestione dello stato comprendono la terminazione e l'eliminazione di un processo;
\item \textbf{Attributi:}
\begin{sloppypar}
\begin{itemize}
\item \texttt{+ ProcessModel process:}\\ campo dati di tipo \texttt{\model{}.Pro\fshyp{}cess\fshyp{}Mo\fshyp{}del} che contiene i dati del processo in gestione;
\item \texttt{+ ProcessDataCollection processdata:}\\ campo dati di tipo \texttt{\collection{}.Pro\fshyp{}cess\fshyp{}Da\fshyp{}ta\fshyp{}Col\fshyp{}lec\fshyp{}tion} che contiene i dati inviati dagli utenti relativi al processo in gestione;
\item \texttt{+ Object template:}\\ oggetto ridefinito da \texttt{Backbone.View}, che contiene il \textit{template HTML\ped{G}} associato alla classe;
\item \texttt{+ Object el:}\\ oggetto ridefinito da \texttt{Backbone.View} che rappresenta l'elemento \textit{HTML\ped{G}} entro cui la classe ascolta eventi generati dagli utenti;
\item \texttt{+ String id:}\\ campo dati ridefinito da \texttt{Backbone.View} contente l'id della classe;
\end{itemize}
\end{sloppypar}
\item \textbf{Metodi:}
\begin{sloppypar}
\begin{itemize}
\item \texttt{+ void initialize():}\\ metodo ridefinito da \texttt{Backbone.View}, invocato alla costruzione di ciascun oggetto della classe, che consente di aggiungere una pagina \textit{HTML\ped{G}} associata al componente;
\item \texttt{+ void render():}\\ metodo ridefinito da \texttt{Backbone.View}, che consente di aggiungere alla pagina \textit{HTML\ped{G}} il \textit{template} campo dati della classe;
\item \texttt{+ void update():}\\ aggiorna i campi dati \texttt{process} e \texttt{processData} comunicando con il \textit{server\ped{G}};
\item \texttt{+ String getParam(String param):}\\ ritorna il valore del parametro \textit{param} se presente nella \textit{URL\ped{G}};
\end{itemize}
\end{sloppypar}
\end{itemize}
\end{flushleft}

\paragraph{CheckStep}
\begin{flushleft}
\begin{itemize}
\item \textbf{Descrizione:} Classe che ha il compito di definire la logica del controllo di un passo che richiede intervento umano per essere approvato;
\item \textbf{Attributi:}
\begin{sloppypar}
\begin{itemize}
\item \texttt{+ ProcessDataCollection processdata:}\\ campo dati di tipo \texttt{\collection{}.Pro\fshyp{}cess\fshyp{}Da\fshyp{}ta\fshyp{}Col\fshyp{}lec\fshyp{}tion} che contiene i dati inviati dagli utenti in attesa di approvazione;
\item \texttt{+ Object template:}\\ oggetto ridefinito da \texttt{Backbone.View}, che contiene il \textit{template HTML\ped{G}} associato alla classe;
\item \texttt{+ Object el:}\\ oggetto ridefinito da \texttt{Backbone.View} che rappresenta l'elemento \textit{HTML\ped{G}} entro cui la classe ascolta eventi generati dagli utenti;
\item \texttt{+ String id:}\\ campo dati ridefinito da \texttt{Backbone.View} contente l'id della classe;
\end{itemize}
\end{sloppypar}
\item \textbf{Metodi:}
\begin{sloppypar}
\begin{itemize}
\item \texttt{+ void initialize():}\\ metodo ridefinito da \texttt{Backbone.View}, invocato alla costruzione di ciascun oggetto della classe, che consente di aggiungere una pagina \textit{HTML\ped{G}} associata al componente;
\item \texttt{+ void render():}\\ metodo ridefinito da \texttt{Backbone.View}, che consente di aggiungere alla pagina \textit{HTML\ped{G}} il \textit{template} campo dati della classe;
\item \texttt{+ void update():}\\ aggiorna il campo dati \texttt{processData} comunicando con il \textit{server\ped{G}};
\item \texttt{+ String getParam(String param):}\\ ritorna il valore del parametro \textit{param} se presente nella \textit{URL\ped{G}};
\item \texttt{+ void approveData():}\\ salva nel \textit{server} lo stato "approvato" ai dati della collezione \textit{processData} dei quali il \textit{process owner\ped{G}} ha richiesto l'approvazione;
\item \texttt{+ void rejectData():}\\ salva nel \textit{server} lo stato "approvato" ai dati della collezione \textit{processData} che il \textit{process owner\ped{G}} ha respinto;
\end{itemize}
\end{sloppypar}
\end{itemize}
\subsubsection{Test di unità per model}
Per le classi \textit{Model} vengono verificati i metodi della classe, per controllare che ritornino il valore atteso.
\paragraph{UserModel}
\begin{flushleft}
\begin{itemize}
\item \textbf{Classe utilizzata per i test:} testUserModel;
\item \textbf{Descrizione:} Classe che permette di gestire i dati di una sessione di un utente autenticato o di un \textit{Process Owner\ped{G}};
\item \textbf{Verifica dei metodi:}
\begin{sloppypar}
\begin{itemize}
\item \texttt{+ void login(String username, String password):}\\ delega al server il controllo delle credenzili e, al completamento della richiesta, salva i dati di sessione in caso di successo;
\item \texttt{+ void logout():}\\ cancella di dati di sessione dell'utente;
\item \texttt{+ void signup():}\\ effettua una richiesta di registrazione al \textit{server\ped{G}} inviando i dati della classe.
\end{itemize}
\end{sloppypar}
\item \textbf{Esito:} Superato;
\end{itemize}
\end{flushleft}

\paragraph{ProcessModel}
\begin{flushleft}
\begin{itemize}
\item \textbf{Classe utilizzata per i test:} testProcessModel;
\item \textbf{Descrizione:} Classe che permette di gestire i dati di un processo, e di salvarli o recuperarli dal \textit{server\ped{G}};
\item \textbf{Verifica dei metodi:}
\begin{sloppypar}
\begin{itemize}
\item \texttt{+ void fetchProcess():}\\ recupera dal \textit{server\ped{G}} i dati del processo, e i dati dei passi che assegna alla collezione \texttt{steps}, sincronizzando le operazioni.
\end{itemize}
\end{sloppypar}
\item \textbf{Esito:} Superato;
\end{itemize}
\end{flushleft}

\paragraph{ProcessDataModel}
\begin{flushleft}
\begin{itemize}
\item \textbf{Classe utilizzata per i test:} testProcessDataModel;
\item \textbf{Descrizione:} verifica del corretto invio di dati da un utente relativi ad un processo, e e il relativo salvataggio e recupero dal \textit{server\ped{G}};
\item \textbf{Verifica dei metodi:}
\begin{sloppypar}
\begin{itemize}
\item \texttt{+ void subscribe(bool subscription):}\\ effettua una richiesta di iscrizione o disiscrizione al \textit{server\ped{G}} a seconda del valore del parametro \textit{subscription}, riguardante il processo con "id" \texttt{idProcesso};
\item \texttt{+ void sendData(int nextStep):}\\ invia al \textit{server\ped{G}} i dati della classe e l'id del prossimo passo da eseguire, che identifica una condizione del processo con "id" \texttt{idProcesso}.
\end{itemize}
\end{sloppypar}
\item \textbf{Esito:} Superato;
\end{itemize}
\end{flushleft}


\paragraph{ProcessCollection}
\begin{flushleft}
\begin{itemize}
\item \textbf{Classe utilizzata per i test:} testProcessCollection;
\item \textbf{Descrizione:} Classe che permette di gestire un insieme di dati inviati da un utente relativi ad un processo;
\item \textbf{Verifica dei metodi:}
\begin{sloppypar}
\begin{itemize}
\item \texttt{+ void fetchProcesses():}\\ verifica del corretto ritorno dal server della lista dei processi a cui l'utente identificato dai dati di sessione può accedere;
\item \texttt{+ void saveProcess(ProcessModel process):}\\ aggiunge il processo \texttt{process} alla collezione dei processi nel \textit{serve\ped{G}}.
\end{itemize}
\end{sloppypar}
\item \textbf{Esito:} Superato;
\end{itemize}
\end{flushleft}

\paragraph{ProcessDataCollection}
\begin{flushleft}
\begin{itemize}
\item \textbf{Classe utilizzata per i test:} testProcessDataCollection;
\item \textbf{Descrizione:} Classe che permette di gestire un insieme di dati inviati dagli utenti;
\item \textbf{Verifica dei metodi:}
\begin{sloppypar}
\begin{itemize}
\item \texttt{+ void fetchProcessData(int stepId):}\\ richiede al \textit{server\ped{G}} la lista dei dati inviati riguardanti il passo con "id" \texttt{stepId}, ai quali l'utente identificato dai dati di sessione può accedere;
\item \texttt{+ void fetchStepData(int processId):}\\ richiede al \textit{server\ped{G}} la lista dei dati inviati riguardanti il processo con "id" \texttt{processId}, ai quali l'utente identificato dai dati di sessione può accedere;
\item \texttt{+ void fetchWaitingData():}\\ richiede al \textit{server\ped{G}} la lista dei dati inviati che richiedono controllo umano;
\item \texttt{+ void approveData(int stepId, String username):}\\ invia al \textit{server\ped{G}} la richiesta di approvazione dei dati riguardanti il passo con "id" \texttt{stepId} e l'utente con username \texttt{username}.
\item \texttt{+ void rejectData(int stepId, String username):}\\ invia al \textit{server\ped{G}} l'esito negativo del controllo dei dati riguardanti il passo con "id" \texttt{stepId} e l'utente con username \texttt{username}.
\end{itemize}
\end{sloppypar}
\item \textbf{Esito:} Superato;
\end{itemize}
\end{flushleft}

\paragraph{IDataAcessObject}
\begin{flushleft}
\begin{itemize}
\item \textbf{Classe utilizzata per i test:} testIDataAcessObject;
\item \textbf{Descrizione:} Interfaccia che permette di gestire la comunicazione e l'interrogazione con il \textit{database}.
\item \textbf{Verifica dei metodi:}
\begin{sloppypar}
\begin{itemize}
\item \texttt{+ void setJdbcTemplate(JdbcTemplate jdbcTemplate):}\\ imposta i parametri per l'accesso alla sorgente dei dati;
\item \texttt{+ ITransferObject getAll():}\\ ritorna tutti i dati di competenza della classe che estende questa interfaccia.
\end{itemize}
\end{sloppypar}
\item \textbf{Esito:} Superato;
\end{itemize}
\end{flushleft}


\paragraph{UserDao}
\label{userdao}
\begin{flushleft}
\begin{itemize}
\item \textbf{Classe utilizzata per i test:} testUserDao;
\item \textbf{Descrizione:} classe che si occupa delle interrogazioni del \textit{database} relative agli utenti del sistema.
\item \textbf{Verifica dei metodi:}
\begin{sloppypar}
\begin{itemize}
\item \texttt{+ User getUser(String userName):}\\ ritorna l’utente con il nome utente specificato; 
\item \texttt{+ List<User> getAllUser():}\\ ritorna tutti gli utenti;
\item \texttt{+ boolean insertUser(User user) :}\\ aggiunge l'utente passato come parametro;
\item \texttt{+ public boolean updateUser(User user) :}\\ verifica del corretto aggiornamento dei dati dell'utente con il nome utente corrispondente a quello dell'utente passato, con i dati dell'utente passato.
\end{itemize}
\end{sloppypar}
\item \textbf{Esito:} Superato;
\end{itemize}
\end{flushleft}

\paragraph{ProcessDao}
\begin{flushleft}
\begin{itemize}
\item \textbf{Classe utilizzata per i test:} testProcessDao;
\item \textbf{Descrizione:} classe che si occupa delle interrogazioni del \textit{database} relative ai processi.
\item \textbf{Verifica dei metodi:}
\begin{sloppypar}
\begin{itemize}
\item \texttt{+ Process getProcess(int id)):}\\ ritorna il processo con l'\texttt{id} specificato; 
\item \texttt{+ List<Process> getAllProcess():}\\ ritorna tutti i processi;
\item \texttt{+ boolean insertProcess(Process process) :}\\ aggiunge il processo passato come parametro;
\item \texttt{+ public boolean updateProcess(Process process) :}\\ aggiorna i dati del processo con lo stesso \texttt{id} di quello del processo passato, con i dati del processo passato.
\end{itemize}
\end{sloppypar}
\item \textbf{Esito:} Superato;
\end{itemize}
\end{flushleft}

\paragraph{ProcessOwnerDao}
\begin{flushleft}
\begin{itemize}
\item \textbf{Classe utilizzata per i test:} testProcessOwnerDao;
\item \textbf{Descrizione:} classe che si occupa delle interrogazioni del \textit{database} relative all'autenticazione del \textit{ProcessOwner}.
\item \textbf{Verifica dei metodi:}
\begin{sloppypar}
\begin{itemize}
\item \texttt{+ Process getProcessOwner(int id)):}\\ ritorna l'oggetto rappresentante il \textit{ProcessOwner}. 
\end{itemize}
\end{sloppypar}
\item \textbf{Esito:} Superato;
\end{itemize}
\end{flushleft}

\paragraph{StepDao}
\begin{flushleft}
\begin{itemize}
\item \textbf{Classe utilizzata per i test:} testStepDao;
\item \textbf{Descrizione:} classe che si occupa delle interrogazioni del \textit{database} relative a tutte le operazioni sui passi dei processi.
\item \textbf{Verifica dei metodi:}
\begin{sloppypar}
\begin{itemize}
\item \texttt{+ Step getStep(int id):}\\ Ritorna il passo con l'\texttt{id} specificato; 
\item \texttt{+ List<Step> getAllStep():}\\ Ritorna tutti i passi;
\item \texttt{+ List<Step> getStepOf(int ProcessId):}\\ Ritorna tutti i passi appartenenti al processo di cui si è passato l'\texttt{id};
\item \texttt{+ boolean insertStep(Step step) :}\\ Aggiunge il passo passato come parametro;
\item \texttt{+ public boolean updateStep(Step step) :}\\ Aggiorna i dati del passo con l'\texttt{id} corrispondente a quello del passo passato, con i dati del passo passato;
\item \texttt{+ List<UserStep> userStep(String userName)}\\ Ritorna una lista di oggetti informativi sullo stato dei passi in corso da parte dell'utente di cui si è passato il nome utente;
\item \texttt{+ List<UserStep> userProcessStep(String userName, processId)}\\ Ritorna una lista di oggetti informativi sullo stato dei passi in corso appartenenti al processo di cui si è passato l'\texttt{id} da parte dell'utente di cui si è passato il nome utente;
\item \texttt{+ List<DataSent> getData(Step step)}\\ Ritorna tutti i dati da tutti gli utenti relativi al passo passato;
\item \texttt{+ List<DataSent> getData(String userName, Step step)}\\ Ritorna tutti i dati inviati dall'utente di cui si è passato il nome utente relativi al passo passato;
\item \texttt{+ boolean completeStep(String userName, Step step, List<DataSent> data, Step next)}\\Notifica e aggiorna nel \textit{database} lo stato dell'utente quando completa o tenta di completare un passo.
\end{itemize}
\end{sloppypar}
\item \textbf{Esito:} Superato;
\end{itemize}
\end{flushleft}

\paragraph{User}
\begin{flushleft}
\begin{itemize}
\item \textbf{Classe utilizzata per i test:} testUser;
\item \textbf{Descrizione:} verifica della corretta gestione gli utenti del sistema e corretta interazione con il database.
\item \textbf{Verifica dei metodi:}
\begin{sloppypar}
\begin{itemize}
\item \texttt{+ String getUserName():}\\ Ritorna il nome utente;
\item \texttt{+ void setUserName(String userName):}\\ Imposta il nome utente;
\item \texttt{+ String getPassword():}\\ Ritorna la password dell'utente;
\item \texttt{+ void setPassword(String password):}\\ Imposta la password dell'utente;
\item \texttt{+ String getName():}\\ Ritorna il nome anagrafico dell'utente;
\item \texttt{+ void setName(String name):}\\ Imposta il nome anagrafico dell'utente;
\item \texttt{+ String getSurName():}\\ Ritorna il cognome dell'utente;
\item \texttt{+ void setSurName(String surName):}\\ Imposta il cognome dell'utente;
\item \texttt{+ Date getDateOfBirth():}\\ Ritorna la data di nascita dell'utente;
\item \texttt{+ void setDateOfBirth(Date dateOfBirth):}\\ Imposta la data di nascita dell'utente;
\item \texttt{+ String getEmail():}\\ Ritorna l'indirizzo di posta elettronica dell'utente;
\item \texttt{+ void setEmail(String email):}\\ Imposta l'indirizzo di posta elettronica dell'utente;
\item \texttt{+ int getId():}\\ Ritorna il codice \texttt{id} associato all'utente;
\item \texttt{+ void setId(int id):}\\ Imposta il codice \texttt{id} associato all'utente.
\end{itemize}
\end{sloppypar}
\item \textbf{Esito:} Superato;
\end{itemize}
\end{flushleft}

\paragraph{Process}
\begin{flushleft}
\begin{itemize}
\item \textbf{Classe utilizzata per i test:} testProcess;
\item \textbf{Descrizione:} verifica della corretta gestione dei processi.
\item \textbf{Verifica dei metodi:}
\begin{sloppypar}
\begin{itemize}
\item \texttt{+ String getName():}\\ Ritorna il nome del processo;
\item \texttt{+ void setName(String name):}\\ Imposta il nome del processo;
\item \texttt{+ String getDescription():}\\ Ritorna la descrizione del processo;
\item \texttt{+ void setDescription(String description):}\\ Imposta la descrizione del processo;
\item \texttt{+ int getCompletionsMax():}\\ Restituisce il numero massimo di completamenti del processo;
\item \texttt{+ void setCompletionsMax(int completionsMax):}\\ Imposta il numero massimo di completamenti del processo;
\item \texttt{+ Date getDateOfTermination():}\\ Ritorna data di terminazione del processo;
\item \texttt{+ void setDateOfTermination(Date dateOfTermination):}\\ Imposta la data di terminazione del processo;
\item \texttt{+ boolean isTerminated():}\\ Ritorna vero se il processo è terminato;
\item \texttt{+ void setTerminated(boolean terminated):}\\ Imposta vero se il processo è terminato;
\item \texttt{+ int getMaxTree():}\\ Ritorna il massimo alberi del processo;
\item \texttt{+ void setMaxtree(int maxTree):}\\ Imposta il massimo alberi del processo;
\item \texttt{+ List<Integer> getStepsId():}\\ Ritorna lista di codici \texttt{id} relativi ai passi del processo;
\item \texttt{+ void setStepsId(List<Integer> stepsId):}\\ Imposta lista di codi \texttt{id} relativi ai passi del processo;
\item \texttt{+int getId():}\\ Ritorna codice identificativo \texttt{id} associato al processo;
\item \texttt{+void setId(int id):}\\ Imposta codice identificativo \texttt{id} associato al processo.
\end{itemize}
\end{sloppypar}
\item \textbf{Esito:} Superato;
\end{itemize}
\end{flushleft}

\paragraph{Step}
\begin{flushleft}
\begin{itemize}
\item \textbf{Classe utilizzata per i test:} testStep;
\item \textbf{Descrizione:} Verifica della corretta gestione dei passi.
\item \textbf{Verifica dei metodi:}
\begin{sloppypar}
\begin{itemize}
\item \texttt{+ int getId():}\\ Ritorna codice identificativo \texttt{id} associato al passo;
\item \texttt{+ void setId(int id):}\\ Imposta codice identificativo \texttt{id} associato al passo;
\item \texttt{+ String getDescription():}\\ Ritorna descrizione del passo;
\item \texttt{+ void setDescription(String description):}\\ Imposta descrizione del passo;
\item \texttt{+ List<Data> getData():}\\ Ritorna lista con i campi dato del passo;
\item \texttt{+ void setData(List<Data> data):}\\ Imposta lista con i campi dato del passo;
\item \texttt{+ List<Condition> getConditions():}\\ Ritorna lista delle condizioni di avanzamento del passo;
\item \texttt{+ void setConditions(List<Condition conditions):}\\ Imposta lista delle condizioni di avanzamento del passo;
\item \texttt{+ int getProcessId():}\\ Ritorna codice \texttt{id} associato al processo padre;
\item \texttt{+ void setProcessId(int processId):}\\ Imposta codice \texttt{id} associato al processo padre;
\item \texttt{+ boolean isFirst():}\\ Ritorna vero se il passo è primo per il processo padre;
\item \texttt{+ void setFirst():}\\ Imposta vero se il passo è primo per il processo padre. 
\end{itemize}
\end{sloppypar}
\item \textbf{Esito:} Superato;
\end{itemize}
\end{flushleft}

\paragraph{Data}
\begin{flushleft}
\begin{itemize}
\item \textbf{Classe utilizzata per i test:} testData;
\item \textbf{Descrizione:} verifica della corretta gestione dei dati;
\item \textbf{Verifica dei metodi:}
\begin{sloppypar}
\begin{itemize}
\item \texttt{+ String getName():}\\ Ritorna il nome del campo dati;
\item \texttt{+ void setName(String name):}\\ Imposta il nome del campo dati;
\item \texttt{+ DataType getType():}\\ Ritorna il tipo di dato richiesto dal campo;
\item \texttt{+ void setType(DataType type):}\\ Imposta il tipo di dato richiesto dal campo;
\item \texttt{+ int getId():}\\ Ritorna il codice \texttt{id} associato al campo dati;
\item \texttt{+ void setId(int id):}\\ Imposta il codice \texttt{id} associato al campo dati.
\end{itemize}
\end{sloppypar}
\item \textbf{Esito:} Superato;
\end{itemize}
\end{flushleft}

\paragraph{Condition}
\begin{flushleft}
\begin{itemize}
\item \textbf{Classe utilizzata per i test:} testCondition;
\item \textbf{Descrizione:} verifica della corretta gestione delle condizioni di terminazione passo;
\item \textbf{Verifica dei metodi:}
\begin{sloppypar}
\begin{itemize}
\item \texttt{+ int getId():}\\ Ritorna il codice \texttt{id} associato alla condizione di avanzamento;
\item \texttt{+ void setId(int id):}\\ Imposta il codice \texttt{id} associato alla condizione di avanzamento;
\item \texttt{+ boolean isRequiresApproval():}\\ Ritorna vero se è richiesta l'approvazione del \textit{Process Owner};
\item \texttt{+ void setRequiresApproval(boolean requiresApproval):}\\ Imposta vero se è richiesta l'approvazione del \textit{Process Owner};
\item \texttt{+ List<Constraint> getConstraints():}\\ ritorna lista di vincoli che soddisfano la condizione di avanzamento;
\item \texttt{+ void setConstraints(List<Constraint> constraints):}\\ Imposta lista di vincoli che soddisfano la condizione di avanzamento;
\item \texttt{+ boolean isOptional():}\\ Ritorna vero se la condizione è opzionale per l'avanzamento;
\item \texttt{+ void setOptional(boolean optional):}\\ Imposta vero se la condizione è opzionale per l'avanzamento;
\item \texttt{+ int getNextStepId():}\\ Ritorna codice \texttt{id} del passo successivo;
\item \texttt{+ void setNextStepId(int nextStepId):}\\ Imposta codice \texttt{id} del passo succesivo.
\end{itemize}
\end{sloppypar}
\item \textbf{Esito:} Superato;
\end{itemize}
\end{flushleft}

\paragraph{Constraint}
\begin{flushleft}
\begin{itemize}
\item \textbf{Classe utilizzata per i test:} testConstraint;
\item \textbf{Descrizione:} verifica della corretta gestione dei vincoli di passo;
\item \textbf{Verifica dei metodi:}
\begin{sloppypar}
\begin{itemize}
\item \texttt{+ Data getAssociatedData():}\\ Ritorna il campo dati su cui è posto il vincolo;
\item \texttt{+ void setAssociatedData(Data associatedData):}\\ Imposta il campo dati sui cui è posto il vincolo.
\end{itemize}
\end{sloppypar}
\item \textbf{Esito:} Superato;
\end{itemize}
\end{flushleft}

\paragraph{NumericConstraint}
\begin{flushleft}
\begin{itemize}
\item \textbf{Classe utilizzata per i test:} testNumeriConstraint;
\item \textbf{Descrizione:} corretta gestione dei vincoli numerici.
\item \textbf{Verifica dei metodi:}
\begin{sloppypar}
\begin{itemize}
\item \texttt{+ int getId():}\\ Ritorna codice \texttt{id} del vincolo;
\item \texttt{+ void setId(int id):}\\ Imposta codice \texttt{id} del vincolo;
\item \texttt{+ int getMinDigits():}\\ Ritorna minimo numero di cifre;
\item \texttt{+ void setMinDigits(int minDigits):}\\ Imposta minimo numero di cifre;
\item \texttt{+ int getMaxDigits():}\\ Ritorna massimo numero di cifre;
\item \texttt{+ void setMaxDigits(int maxDigits):}\\ Imposta massimo numero di cifre;
\item \texttt{+ boolean isDecimal():}\\ Ritorna vero se atteso un decimale;
\item \texttt{+ void setDecimal(boolean decimal):}\\ Imposta vero se atteso un decimale;
\item \texttt{+ double getMinValue():}\\ Ritorna valore minimo;
\item \texttt{+ void setMinValue(double minValue):}\\ Imposta valore minimo;
\item \texttt{+ double getMaxValue():}\\ Ritorna valore massimo;
\item \texttt{+ void setMaxValue(double maxValue):}\\ Imposta valore massimo;
\end{itemize}
\end{sloppypar}
\item \textbf{Esito:} Superato;
\end{itemize}
\end{flushleft}

\paragraph{TemporalConstraint}
\begin{flushleft}
\begin{itemize}
\item \textbf{Classe utilizzata per i test:} testTemporalContraint;
\item \textbf{Descrizione:} verifica corretta gestione dei vincoli temporali;
\item \textbf{Verifica dei metodi:}
\begin{sloppypar}
\begin{itemize}
\item \texttt{+ int getId():}\\ Ritorna codice \texttt{id} del vincolo;
\item \texttt{+ void setId(int id):}\\ Imposta codice \texttt{id} del vincolo;
\item \texttt{+ Date getBegin():}\\ Ritorna inizio arco temporale valido;
\item \texttt{+ void setBegin(Date begin):}\\ Imposta inizio arco temporale valido;
\item \texttt{+ Date getEnd():}\\ Ritorna fine arco temporale valido;
\item \texttt{+ void setEnd(Date end):}\ Imposta fine arco temporale valido.
\end{itemize}
\end{sloppypar}
\item \textbf{Esito:} Superato;
\end{itemize}
\end{flushleft}

\paragraph{GeographicConstraint}
\begin{flushleft}
\begin{itemize}
\item \textbf{Classe utilizzata per i test:} testGeographicContraint;
\item \textbf{Descrizione:} verifica della corretta gestione dei vincoli di posizione.
\item \textbf{Verifica dei metodi:}
\begin{sloppypar}
\begin{itemize}
\item \texttt{+ int getId():}\\ Ritorna codice \texttt{id} del vincolo;
\item \texttt{+ void setId(int id):}\\ Imposta codice \texttt{id} del vincolo;
\item \texttt{+ double getLatitude():}\\ Ritorna latitudine richiesta;
\item \texttt{+ void setLatitude(double latitude):}\\ Imposta latitudine richiesta;
\item \texttt{+ double getLongitude():}\\ Ritorna longitudine richiesta;
\item \texttt{+ void setLongitude(double longitude):}\\ Imposta longitudine richiesta;
\item \texttt{+ double getAltitude():}\\ Ritorna altitudine richiesta;
\item \texttt{+ void setAltitude(double altitude):}\\ Imposta altitudine richiesta;
\item \texttt{+ double getRadius():}\\ Ritorna raggio di tolleranza;
\item \texttt{+ void setRadius(double radius):}\\ Imposta raggio di tolleranza.
\end{itemize}
\end{sloppypar}
\item \textbf{Esito:} Superato;
\end{itemize}
\end{flushleft}

\paragraph{DataSent}
\begin{flushleft}
\begin{itemize}
\item \textbf{Classe utilizzata per i test:} testDataSent;
\item \textbf{Descrizione:} Verifica della corretta gestione dei dati inseriti dagli utenti;
\item \textbf{Verifica dei metodi:}
\begin{sloppypar}
\begin{itemize}
\item \texttt{+ String getUser():}\\ Ritorna nome utente dell'utente che ha inviato il dato;
\item \texttt{+ void setUser(String user):}\\ Imposta nome utente dell'utente che ha inviato il dato;
\item \texttt{+ DataType getType():}\\ Ritorna il tipo del dato inviato;
\item \texttt{+ void setType(DataType type):}\\ Imposta il tipo del dato inviato;
\item \texttt{+ IDataValue getValue():}\\ Ritorna oggetto con il valore del dato;
\item \texttt{+ void setValue(IDataValue value):}\\ Imposta oggetto con il valore del dato;
\item \texttt{+ int getStepId():}\\ Ritorna codice \texttt{id} del passo richiedente il dato;
\item \texttt{+ void setStepId(int stepId):}\\ Imposta codice \texttt{id} del passo richiedente il dato.
\end{itemize}
\end{sloppypar}
\item \textbf{Esito:} Superato;
\end{itemize}
\end{flushleft}

\paragraph{IDataValue}
\begin{flushleft}
\begin{itemize}
\item \textbf{Classe utilizzata per i test:} testIDataValue;
\item \textbf{Descrizione:} verifica del corretto ritorno per i dati ricevuti dagli utenti;
\item \textbf{Verifica dei metodi:}
\begin{sloppypar}
\begin{itemize}
\item \texttt{+ int getId():}\\ Ritorna codice \texttt{id} associato al valore;
\item \texttt{+ void setId(int id):}\\ Imposta codice \texttt{id} associato al valore.
\end{itemize}
\end{sloppypar}
\item \textbf{Esito:} Superato;
\end{itemize}
\end{flushleft}

\paragraph{TextualValue}
\begin{flushleft}
\begin{itemize}
\item \textbf{Classe utilizzata per i test:} testTextualValue;
\item \textbf{Descrizione:} verifica del corretto ritorno e importazione dei dati di tipo testuale;
\item \textbf{Verifica dei metodi:}
\begin{sloppypar}
\begin{itemize}
\item \texttt{+ String getValue():}\\ Ritorna valore testuale;
\item \texttt{+ void setValue(String value):}\\ Imposta valore testuale.
\end{itemize}
\end{sloppypar}
\item \textbf{Esito:} Superato;
\end{itemize}
\end{flushleft}

\paragraph{NumericValue}
\begin{flushleft}
\begin{itemize}
\item \textbf{Classe utilizzata per i test:} testNumericValue;
\item \textbf{Descrizione:} verifica del corretto ritorno e importazione dei dati di tipo numerico;
\item \textbf{Verifica dei metodi:}
\begin{sloppypar}
\begin{itemize}
\item \texttt{+ double getValue():}\\ Ritorna valore numerico;
\item \texttt{+ void setValue(double value):}\\ Imposta valore numerico.
\end{itemize}
\end{sloppypar}
\item \textbf{Esito:} Superato;
\end{itemize}
\end{flushleft}

\paragraph{ImageValue}
\begin{flushleft}
\begin{itemize}
\item \textbf{Classe utilizzata per i test:} testImageValue;
\item \textbf{Descrizione:} verifica del corretto ritorno e importazione dei dati di tipo immagine;
\item \textbf{Verifica dei metodi:}
\begin{sloppypar}
\begin{itemize}
\item \texttt{+ String getImageUrl():}\\ Ritorna percorso \textit{URL} dell'immagine;
\item \texttt{+ void setImageUrl(String imageUrl):}\\ Imposta percorso \textit{URL} dell'immagine.
\end{itemize}
\end{sloppypar}
\item \textbf{Esito:} Superato;
\end{itemize}
\end{flushleft}

\paragraph{GeographicValue}
\begin{flushleft}
\begin{itemize}
\item \textbf{Classe utilizzata per i test:} testGeographicValue;
\item \textbf{Descrizione:} verifica del corretto ritorno e importazione dei dati di tipo geografico;
\item \textbf{Verifica dei metodi:}
\begin{sloppypar}
\begin{itemize}
\item \texttt{+ double getLatitude():}\\ Ritorna latitudine;
\item \texttt{+ void setLatitude(double latitude):}\\ Imposta latitudine;
\item \texttt{+ double getLongitude():}\\ Ritorna longitudine;
\item \texttt{+ void setLongitude(double longitude):}\\ Imposta longitudine;
\item \texttt{+ double getAltitude():}\\ Ritorna altitudine;
\item \texttt{+ void setAltitude(double altitude):}\\ Imposta altitudine.
\end{itemize}
\end{sloppypar}
\end{itemize}
\end{flushleft}

\paragraph{UserStep}
\begin{flushleft}
\begin{itemize}
\item \textbf{Classe utilizzata per i test:} testUserStep;
\item \textbf{Descrizione:} verifica della corretta gestione dei passi che sono in corso;
\item \textbf{Verifica dei metodi:}
\begin{sloppypar}
\begin{itemize}
\item \texttt{+ int getCurrentStepId():}\\ Ritorna il codice \texttt{id} del passo attuale;
\item \texttt{+ void setCurrentStepId(int currentStepId):}\\ Imposta il codice \texttt{id} del passo attuale;
\item \texttt{+ stepStates getState():}\\ Ritorna stato avanzamento;
\item \texttt{+ void setStates(stepStates state):}\\ Imposta stato avanzamento;
\item \texttt{+ String getUser():}\\ Restituisce nome utente dell'utente in caso;
\item \texttt{+ void setUser(String user):}\\ Imposta nome utente dell'utente in caso.
\end{itemize}
\end{sloppypar}
\item \textbf{Esito:} Superato;
\end{itemize}
\end{flushleft}

\paragraph{ProcessOwner}
\begin{flushleft}
\begin{itemize}
\item \textbf{Classe utilizzata per i test:} testProcessOwner;
\item \textbf{Descrizione:} verifica della corretta gestione dell'ambito \textit{process owner};
\item \textbf{Verifica dei metodi:}
\begin{sloppypar}
\begin{itemize}
\item \texttt{+ String getUserName():}\\ Ritorna il nome utente del \textit{Process Owner};
\item \texttt{+ void setUserName(String userName):}\\ Imposta il nome utente del \textit{Process Owner};
\item \texttt{+ String getPassword():}\\ Ritorna la password del \textit{Process Owner};
\item \texttt{+ void setPassword(String password):}\\ Imposta la password del \textit{Process Owner}. 
\end{itemize}
\end{sloppypar}
\item \textbf{Esito:} Superato;
\end{itemize}
\end{flushleft}
\subsubsection{Descrizione dei test di validazione}
I test di validazione saranno definiti in seguito.



