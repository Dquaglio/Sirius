\section{Pianificazione dei test}
Di seguito elenchiamo tutti i test di validazione, sistema ed integrazione previsti, prevedendo però ad un successivo aggiornamento per i test di unità. Per quanto riguarda le tempistiche di esecuzione dei test si faccia riferimento a \infoPDP.
Il valore \textbf{N.A} è da intendersi come non applicato, in quanto tali test saranno eseguiti successivamente nello svolgimento del progetto.
\subsection{Test di sistema}
In questa sezione vengono descritti i test di sistema che consentiranno a \gruppo ~di verificare il comportamento dinamico del sistema rispetto ai requisiti descritti in \infoAR. I test sotto riportati sono relativi ai requisiti software individuati e meritevoli di test.
\subsubsection{Descrizione dei test di sistema}
\subsubsection{Ambito utente}
\LTXtable{\textwidth}{RequisitiUtente.tex}
\subsubsection{Ambito process owner}
\LTXtable{\textwidth}{RequisitiAmministratore.tex}
\subsection{Requisiti di vincolo}
\LTXtable{\textwidth}{RequisitiVincolo.tex}
\subsection{Test di integrazione}
In questa sezione verranno descritti i test di integrazione, da utilizzare per i vari componenti descritti nella progettazione ad alto livello, che permettono di verificare la corretta integrazione ed il corretto flusso dei dati all’interno del sistema.
Si è deciso di utilizzare una strategia di integrazione incrementale bottom-up che permette di sviluppare e verificare le componenti in parallelo.\\
Assemblando le componenti in modo incrementale i difetti rilevati da un test sono da
attribuirsi, con maggior probabilità, all’ultima parte aggiunta e si rende ogni passo di
integrazione reversibile consentendo di retrocedere verso uno stato noto e sicuro.
In questo modo i componenti base vengono testati più volte riducendo la possibile presenza di errori.\\
Nel diagramma 
\subsubsection{Descrizione dei test di integrazione}
Di seguito sono elencati i test di integrazione relativo ai componenti indicati in \infoST.
\LTXtable{\textwidth}{Dettagliotestintegrazione.tex}
%da fare
\subsection{Test di unità}
I test di unità sulle view sono effettuati creando una nuova classe basata sulla precedente e definita "testnome" ed invocando il metodo test con i rispettivi parametri.
\paragraph{Login}
\begin{itemize}
\item \textbf{Descrizione:} \textit{Template HTML} che permette di gestire l'interfaccia grafica relativa alle richieste di autenticazione al sistema.
\end{itemize}

\paragraph{MainUser}
\begin{itemize}
\item \textbf{Descrizione:} Classe che permette la gestione delle principali componenti dell'interfaccia grafica dell'utente.
\end{itemize}

\paragraph{Register}
\begin{itemize}
\item \textbf{Descrizione:} \textit{Template HTML} che permette di gestire dell'interfaccia grafica relativa alle richieste di registrazione da parte dell'utente.
\end{itemize}

\paragraph{UserData}
\begin{itemize}
\item \textbf{Descrizione:} \textit{Template HTML} che permette la realizzazione dei \textit{widget} che consentono visualizzazione e modifica dei dati dell'utente.
\end{itemize}

\paragraph{OpenProcess}
\begin{itemize}
\item \textbf{Descrizione:} \textit{Template HTML} che permette di realizzare i \textit{widget} per consentire l'apertura di un processo tramite ricerca o selezionandolo da una lista.
\end{itemize}

\paragraph{ManagementProcess}
\begin{itemize}
\item \textbf{Descrizione:} \textit{Template HTML} che permette di realizzare i \textit{widget} per consentire la visualizzazione dello stato del processo selezionato e i vincoli per concludere il passo in corso.
\end{itemize}

\paragraph{SendData}
\begin{itemize}
\item \textbf{Descrizione:} \textit{Template HTML} che permette di realizzare i \textit{widget} per consentire l'invio dei dati richiesti per la conclusione del passo in esecuzione.
\end{itemize}

\paragraph{SendText}
\begin{itemize}
\item \textbf{Descrizione:} \textit{Template HTML} che permette di realizzare i \textit{widget} che consentono di inserire il testo da inviare per concludere il passo in esecuzione.
\end{itemize}

\paragraph{SendNumb}
\begin{itemize}
\item \textbf{Descrizione:} \textit{Template HTML} che permette agli oggetti che la implementano di realizzare i \textit{widget} che consentono di inserire i dati numerici da inviare per concludere il passo in esecuzione.
\end{itemize}

\paragraph{SendPosition}
\begin{itemize}
\item \textbf{Descrizione:} \textit{Template HTML} che permette  di realizzare i \textit{widget} che consentono di inviare la posizione geografica richiesta per la conclusione del passo in esecuzione.
\end{itemize}

\paragraph{SendImage}
\begin{itemize}
\item \textbf{Descrizione:} \textit{Template HTML} che permette di realizzare i \textit{widget} che consentono di inserire le immagini richieste per concludere i passo in esecuzione.
\end{itemize}

\paragraph{PrintProcess}
\begin{itemize}
\item \textbf{Descrizione:} \textit{Template HTML} che permette di realizzare i \textit{widget} che consentono il salvataggio dei \textit{report} sull'esecuzione del processo.
\end{itemize}

\paragraph{MainProcessOwner}
\begin{itemize}
\item \textbf{Descrizione:} Componente che permette la gestione delle principali componenti dell'interfaccia grafica dell'utente \textit{process owner\ped{G}}.
\end{itemize}

\paragraph{NewProcess}
\begin{itemize}
\item \textbf{Descrizione:} \textit{Template HTML} che permette di gestire l'interfaccia grafica che consente di creare nuovi processi.
\end{itemize}

\paragraph{AddStep}
\begin{itemize}
\item \textbf{Descrizione:} \textit{Template HTML} che permette di gestire l'interfaccia grafica che consente di definire un nuovo passo del processo in creazione.
\end{itemize}

\paragraph{OpenProcess}
\begin{itemize}
\item \textbf{Descrizione:} \textit{Template HTML} che permette di realizzare i\textit{widget} che consentono di aprire un processo tramite ricerca o selezionandolo da una lista.
\end{itemize}

\paragraph{ManageProcess}
\begin{itemize}
\item \textbf{Descrizione:} \textit{Template HTML} che permette di realizzare i\textit{widget} che consentono di gestire l'accesso ai dati inviati al\textit{server\ped{G}} dagli utenti.
\end{itemize}

\paragraph{CheckStep}
\begin{itemize}
\item \textbf{Descrizione:} \textit{Template HTML} che permette di realizzare i\textit{widget} che consentono di gestire l'approvazione dei passi che richiedono intervento umano.
\end{itemize}

%\paragraph{Router}
\begin{flushleft}
\begin{itemize}
\item \textbf{Classe utilizzata per i test:} testRouter;
\item \textbf{Descrizione:} Verificare che la classe permetta l'inizializzazione e la gestione delle pagine partendo dagli eventi;
\item \textbf{Verifica dei metodi:}
\begin{sloppypar}
\begin{itemize}
\item \texttt{+ void home():}\\ gestisce l'evento di \textit{routing\ped{G} home};
\item \texttt{+ void processes():}\\ gestisce l'evento di \textit{routing\ped{G} processes};
\item \texttt{+ void newProcess():}\\ gestisce l'evento di \textit{routing\ped{G} newProcess};
\item \texttt{+ void checkStep():}\\ gestisce l'evento di \textit{routing\ped{G} checkStep};
\item \texttt{+ void process():}\\ gestisce l'evento di \textit{routing\ped{G} process};
\item \texttt{+ void register():}\\ gestisce l'evento di \textit{routing\ped{G} register};
\item \texttt{+ void user():}\\ gestisce l'evento di \textit{routing\ped{G} user};
\item \texttt{+ bool checkSession(String pageId):}\\ ritorna \texttt{true} solo se l'utente è autenticato; in caso contrario crea e renderizza la pagina di \textit{login};
\item \texttt{+ void load(String resource, String pageId):}\\ crea e aggiunge una vista di tipo \textit{resource} al campo dati \texttt{this.views}, all'indice \textit{pageId};
\item \texttt{+ void changePage(String pageId):}\\ imposta la pagina con id \textit{pageId} come attiva, ed esegue la transizione di cambio pagina.
\end{itemize}
\end{sloppypar}
\item Esito: Superato;
\end{itemize}
\end{flushleft}

\paragraph{Login}
\begin{flushleft}
\begin{itemize}
\item \textbf{Classe utilizzata per i test:} testLogin;
\item \textbf{Descrizione:} Verifica della corretta gestione delle richieste di autenticazione al sistema;
\item \textbf{Verifica dei metodi:}
\begin{sloppypar}
\begin{itemize}
\item \texttt{+ void initialize():}\\ verifica della corretta aggiunta di una pagina \textit{HTML\ped{G}} associata al componente;
\item \texttt{+ void render():}\\ verifica della corretta aggiunta alla pagina \textit{HTML\ped{G}} del \textit{template} campo dati della classe;
\item \texttt{+ void login(Event event):}\\ effettua una richiesta di \textit{login}, utilizzando il campo dati \model{} per comunicare con il \textit{server\ped{G}}.
\end{itemize}
\end{sloppypar}
\item Esito: Superato;
\end{itemize}
\end{flushleft}

\paragraph{MainUser}
\begin{flushleft}
\begin{itemize}
\item \textbf{Classe utilizzata per i test:} testMainUser;
\item \textbf{Descrizione:} verifica della gestione generale della logica delle funzionalità utente;
\item \textbf{Verifica dei metodi:}
\begin{sloppypar}
\begin{itemize}
\item \texttt{+ void initialize():}\\ verifica della corretta aggiunta di una pagina \textit{HTML\ped{G}} associata al componente;
\item \texttt{+ void render():}\\ verifica della corretta aggiunta alla pagina \textit{HTML\ped{G}} del \textit{template} campo dati della classe.
\end{itemize}
\end{sloppypar}
\item Esito: Superato;
\end{itemize}
\end{flushleft}

\paragraph{Register}
\begin{flushleft}
\begin{itemize}
\item \textbf{Classe utilizzata per i test:} testRegister;
\item \textbf{Descrizione:} verifica della richieste di registrazione da parte dell'utente;
\item \textbf{Verifica dei metodi:}
\begin{sloppypar}
\begin{itemize}
\item \texttt{+ void initialize():}\\ verifica della corretta aggiunta di una pagina \textit{HTML\ped{G}} associata al componente;
\item \texttt{+ void render():}\\ verifica della corretta aggiunta alla pagina \textit{HTML\ped{G}} del \textit{template} campo dati della classe;
\item \texttt{+ void register(Event event):}\\ verifica della richiesta di registrazione, utilizzando il campo dati \model{} per comunicare con il \textit{server\ped{G}}.
\end{itemize}
\end{sloppypar}
\item Esito: Superato;
\end{itemize}
\end{flushleft}

\paragraph{UserData}
\begin{flushleft}
\begin{itemize}
\item \textbf{Classe utilizzata per i test:} testUserData;
\item \textbf{Descrizione:} verifica della corretta gestione della visualizzazione e la modifica dei dati dell'utente;
\item \textbf{Verifica dei metodi:}
\begin{sloppypar}
\begin{itemize}
\item \texttt{+ void initialize():}\\ verifica della corretta aggiunta di una pagina \textit{HTML\ped{G}} associata al componente;
\item \texttt{+ void render():}\\ verifica della corretta aggiunta alla pagina \textit{HTML\ped{G}} del \textit{template} campo dati della classe;
\item \texttt{+ void editData():}\\ verifica del corretto salvataggio dei dati modificati dall'utente nel \textit{server\ped{G}}.
\end{itemize}
\end{sloppypar}
\item Esito: Superato;
\end{itemize}
\end{flushleft}

\paragraph{OpenProcess}
\begin{flushleft}
\begin{itemize}
\item \textbf{Classe utilizzata per i test:} testOpenProcess;
\item \textbf{Descrizione:} Classe che ha il compito di selezionare, ricercare e aprire un processo fra quelli eseguibili;
\item \textbf{Verifica dei metodi:}
\begin{sloppypar}
\begin{itemize}
\item \texttt{+ void initialize():}\\ verifica della corretta aggiunta di una pagina \textit{HTML\ped{G}} associata al componente;
\item \texttt{+ void render():}\\ verifica della corretta aggiunta alla pagina \textit{HTML\ped{G}} del \textit{template} campo dati della classe;
\item \texttt{+ void update():}\\ verifica del corretto aggiornamento del campo dati \texttt{collection} e relativa comunicazione con il \textit{server\ped{G}}.
\end{itemize}
\end{sloppypar}
\item Esito: Superato;
\end{itemize}
\end{flushleft}

\paragraph{ManagementProcess}
\begin{flushleft}
\begin{itemize}
\item \textbf{Classe utilizzata per i test:} testManagementProcess;
\item \textbf{Descrizione:} Classe che ha il compito di gestire e accedere alle informazioni relative allo stato del processo selezionato.
\item \textbf{Verifica dei metodi:}
\begin{sloppypar}
\begin{itemize}
\item \texttt{+ void initialize():}\\ verifica della corretta aggiunta di una pagina \textit{HTML\ped{G}} associata al componente;
\item \texttt{+ void render():}\\ verifica della corretta aggiunta alla pagina \textit{HTML\ped{G}} del \textit{template} campo dati della classe;
\item \texttt{+ void update():}\\ verifica del corretto aggiornamento dei campi dati \texttt{process} e \texttt{processData} relativa comunicazione con il \textit{server\ped{G}};
\item \texttt{+ String getParam(String param):}\\ ritorna il valore del parametro \textit{param} se presente nella \textit{URL\ped{G}}.
\end{itemize}
\end{sloppypar}
\item Esito: Superato;
\end{itemize}
\end{flushleft}

\paragraph{PrintReport}
\begin{flushleft}
\begin{itemize}
\item \textbf{Classe utilizzata per i test:} testPrintReport;
\item \textbf{Descrizione:} verifica della corretta creazione del report di fine processo;
\item \textbf{Verifica dei metodi:}
\begin{sloppypar}
\begin{itemize}
\item \texttt{+ void initialize():}\\ verifica della corretta aggiunta di una pagina \textit{HTML\ped{G}} associata al componente;
\item \texttt{+ void render():}\\ verifica della corretta aggiunta alla pagina \textit{HTML\ped{G}} del \textit{template} campo dati della classe;
\end{itemize}
\end{sloppypar}
\item Esito: Superato;
\end{itemize}
\end{flushleft}

\paragraph{SendData}
\begin{flushleft}
\begin{itemize}
\item \textbf{Classe utilizzata per i test:} testSendData;
\item \textbf{Descrizione:} verifica della corretta gestione, inserimento e invio di dati da parte degli utenti, per completare il passo corrente;
\item \textbf{Verifica dei metodi:}
\begin{sloppypar}
\begin{itemize}
\item \texttt{+ void initialize():}\\ verifica della corretta aggiunta di una pagina \textit{HTML\ped{G}} associata al componente;
\item \texttt{+ void render():}\\ verifica della corretta aggiunta alla pagina \textit{HTML\ped{G}} del \textit{template} campo dati della classe;
\item \texttt{+ bool getData():}\\ controlla se i dati inseriti dall'utente sono corretti: se lo sono ritorna \texttt{true} e li aggiunge alla collezione \texttt{processData}, altrimenti ritorna \texttt{false};
\item \texttt{+ bool saveData():}\\ utilizza metodi del campo dati \texttt{processData}, per inviare i dati raccolti al \textit{server\ped{G}}.
\end{itemize}
\end{sloppypar}
\item Esito: Superato;
\end{itemize}
\end{flushleft}

\paragraph{SendText}
\begin{flushleft}
\begin{itemize}
\item \textbf{Classe utilizzata per i test:} testSendText;
\item \textbf{Descrizione:} Classe che permette l'inserimento e il controllo di dati testuali inseriti dagli utenti;
\item \textbf{Verifica dei metodi:}
\begin{sloppypar}
\begin{itemize}
\item \texttt{+ void initialize():}\\ verifica della corretta aggiunta di una pagina \textit{HTML\ped{G}} associata al componente;
\item \texttt{+ void render():}\\ verifica della corretta aggiunta alla pagina \textit{HTML\ped{G}} del \textit{template} campo dati della classe;
\item \texttt{+ bool getData(ProcessDataModel data):}\\ controlla se i dati inseriti dall'utente sono corretti: se lo sono ritorna \texttt{true} e li aggiunge al riferimento \texttt{data}, altrimenti ritorna \texttt{false}.
\end{itemize}
\end{sloppypar}
\item Esito: Superato;
\end{itemize}
\end{flushleft}

\paragraph{SendNumb}
\begin{flushleft}
\begin{itemize}
\item \textbf{Classe utilizzata per i test:} testSendNumb;
\item \textbf{Descrizione:} Classe che ha il compito di permettere l'inserimento e il controllo di dati numerici inseriti dagli utenti;
\item \textbf{Verifica dei metodi:}
\begin{sloppypar}
\begin{itemize}
\item \texttt{+ void initialize():}\\ verifica della corretta aggiunta di una pagina \textit{HTML\ped{G}} associata al componente;
\item \texttt{+ void render():}\\ verifica della corretta aggiunta alla pagina \textit{HTML\ped{G}} del \textit{template} campo dati della classe;
\item \texttt{+ bool getData(ProcessDataModel data):}\\ controlla se i dati inseriti dall'utente sono corretti: se lo sono ritorna \texttt{true} e li aggiunge al riferimento \texttt{data}, altrimenti ritorna \texttt{false}.
\end{itemize}
\end{sloppypar}
\item Esito: Superato;
\end{itemize}
\end{flushleft}

\paragraph{SendImage}
\begin{flushleft}
\begin{itemize}
\item \textbf{Classe utilizzata per i test:} testSendImage;
\item \textbf{Descrizione:} Classe che gestisce l'inserimento e il controllo di immagini inserite dagli degli utenti;
\item \textbf{Verifica dei metodi:}
\begin{sloppypar}
\begin{itemize}
\item \texttt{+ void initialize():}\\ verifica della corretta aggiunta di una pagina \textit{HTML\ped{G}} associata al componente;
\item \texttt{+ void render():}\\ verifica della corretta aggiunta alla pagina \textit{HTML\ped{G}} del \textit{template} campo dati della classe;
\item \texttt{+ bool getData(ProcessDataModel data):}\\ controlla se i dati inseriti dall'utente sono corretti: se lo sono ritorna \texttt{true} e li aggiunge al riferimento \texttt{data}, altrimenti ritorna \texttt{false}.
\end{itemize}
\end{sloppypar}
\item Esito: Superato;
\end{itemize}
\end{flushleft}

\paragraph{SendPosition}
\begin{flushleft}
\begin{itemize}
\item \textbf{Classe utilizzata per i test:} testSendPosition;
\item \textbf{Descrizione:} verifica della corretta gestione della posizione geografica dell'utente;
\item \textbf{Verifica dei metodi:}
\begin{sloppypar}
\begin{itemize}
\item \texttt{+ void initialize():}\\ verifica della corretta aggiunta di una pagina \textit{HTML\ped{G}} associata al componente;
\item \texttt{+ void render():}\\ verifica della corretta aggiunta alla pagina \textit{HTML\ped{G}} del \textit{template} campo dati della classe;
\item \texttt{+ bool getData(ProcessDataModel data):}\\ controlla se i dati inseriti dall'utente sono corretti: se lo sono ritorna \texttt{true} e li aggiunge al riferimento \texttt{data}, altrimenti ritorna \texttt{false}.
\end{itemize}
\end{sloppypar}
\item Esito: Superato;
\end{itemize}
\end{flushleft}


\paragraph{MainProcessOwner}
\begin{flushleft}
\begin{itemize}
\item \textbf{Classe utilizzata per i test:} testMainProcessOwner;
\item \textbf{Descrizione:} verifica della gestione generale della logica delle funzionalità \textit{Process Owner\ped{G}};
\item \textbf{Verifica dei metodi:}
\begin{sloppypar}
\begin{itemize}
\item \texttt{+ void initialize():}\\ verifica della corretta aggiunta di una pagina \textit{HTML\ped{G}} associata al componente;
\item \texttt{+ void render():}\\ verifica della corretta aggiunta alla pagina \textit{HTML\ped{G}} del \textit{template} campo dati della classe;
\end{itemize}
\end{sloppypar}
\item Esito: Superato;
\end{itemize}
\end{flushleft}

\paragraph{OpenProcess}
\begin{flushleft}
\begin{itemize}
\item \textbf{Classe utilizzata per i test:} testOpenProcess;
\item \textbf{Descrizione:} Classe che ha il compito di gestire la ricerca e la selezione di un processo;
\item \textbf{Verifica dei metodi:}
\begin{sloppypar}
\begin{itemize}
\item \texttt{+ void initialize():}\\ verifica della corretta aggiunta di una pagina \textit{HTML\ped{G}} associata al componente;
\item \texttt{+ void render():}\\ verifica della corretta aggiunta alla pagina \textit{HTML\ped{G}} del \textit{template} campo dati della classe;
\item \texttt{+ void update():}\\ aggiorna il campo dati \texttt{collection} comunicando con il \textit{server\ped{G}}.
\end{itemize}
\end{sloppypar}
\item Esito: Superato;
\end{itemize}
\end{flushleft}

\paragraph{NewProcess}
\begin{flushleft}
\begin{itemize}
\item \textbf{Classe utilizzata per i test:} testNewProcess;
\item \textbf{Descrizione:} Classe che ha il compito di gestire la logica della definizione di un nuovo processo;
\item \textbf{Verifica dei metodi:}
\begin{sloppypar}
\begin{itemize}
\item \texttt{+ void initialize():}\\ verifica della corretta aggiunta di una pagina \textit{HTML\ped{G}} associata al componente;
\item \texttt{+ void render():}\\ verifica della corretta aggiunta alla pagina \textit{HTML\ped{G}} del \textit{template} campo dati della classe, e che questo sia compilato con gli eventuali errori(effettuare due prove);
\item \texttt{+ void newStep():}\\ utilizza la classe \texttt{\logicAdmin{}.Add\fshyp{}Step} per definire e aggiungere un nuovo passo al processo \texttt{model};
\item \texttt{+ bool getData():}\\ controlla se i dati inseriti dal \textit{process owner\ped{G}} sono corretti: se lo sono ritorna \texttt{true} e li aggiunge al processo \texttt{model}, altrimenti ritorna \texttt{false};
\item \texttt{+ bool saveProcess():}\\ utilizza metodi del campo dati \texttt{collection}, per inviare il processo \texttt{model} al \textit{server\ped{G}}.
\end{itemize}
\end{sloppypar}
\item \textbf{Esito:} Superato;
\end{itemize}
\end{flushleft}

\paragraph{AddStep}
\begin{flushleft}
\begin{itemize}
\item \textbf{Classe utilizzata per i test:} testAddStep;
\item \textbf{Descrizione:} Classe che ha il compito di gestire la logica di definizione dei passi di un processo;
\item \textbf{Verifica dei metodi:}
\begin{sloppypar}
\begin{itemize}
\item \texttt{+ void initialize():}\\ verifica della corretta aggiunta di una pagina \textit{HTML\ped{G}} associata al componente;
\item \texttt{+ void render():}\\ verifica della corretta aggiunta alla pagina \textit{HTML\ped{G}} del \textit{template} campo dati della classe, e che questo sia compilato con gli eventuali errori(effettuare due prove);
\item \texttt{+ bool getData():}\\ controlla se i dati inseriti dal \textit{process owner\ped{G}} sono corretti: se lo sono ritorna \texttt{true} e li aggiunge al passo \texttt{model}, altrimenti ritorna \texttt{false}.
\end{itemize}
\end{sloppypar}
\item \textbf{Esito:} Superato;
\end{itemize}
\end{flushleft}

\paragraph{ManageProcess}
\begin{flushleft}
\begin{itemize}
\item \textbf{Classe utilizzata per i test:} testManageProcess;
\item \textbf{Descrizione:} Classe che ha il compito di gestire e accedere alle informazioni relative allo stato dei processi e ai dati inviati dagli utenti. Le operazioni di gestione dello stato comprendono la terminazione e l'eliminazione di un processo;
\item \textbf{Verifica dei metodi:}
\begin{sloppypar}
\begin{itemize}
\item \texttt{+ void initialize():}\\ verifica della corretta aggiunta di una pagina \textit{HTML\ped{G}} associata al componente;
\item \texttt{+ void render():}\\ verifica della corretta aggiunta alla pagina \textit{HTML\ped{G}} del \textit{template} campo dati della classe;
\item \texttt{+ void update():}\\ aggiorna i campi dati \texttt{process} e \texttt{processData} comunicando con il \textit{server\ped{G}};
\item \texttt{+ String getParam(String param):}\\ ritorna il valore del parametro \textit{param} se presente nella \textit{URL\ped{G}};
\end{itemize}
\end{sloppypar}
\item \textbf{Esito:} Superato;
\end{itemize}
\end{flushleft}

\paragraph{CheckStep}
\begin{flushleft}
\begin{itemize}
\item \textbf{Classe utilizzata per i test:} testCheckStep;
\item \textbf{Descrizione:} Verifica della logica per l'approvazione di un passo che necessita dell'intervento dell'utente;
\item \textbf{Verifica dei metodi:}
\begin{sloppypar}
\begin{itemize}
\item \texttt{+ void initialize():}\\ verifica della corretta aggiunta di una pagina \textit{HTML\ped{G}} associata al componente;
\item \texttt{+ void render():}\\ verifica della corretta aggiunta alla pagina \textit{HTML\ped{G}} del \textit{template} campo dati della classe;
\item \texttt{+ void update():}\\ aggiorna il campo dati \texttt{processData} comunicando con il \textit{server\ped{G}};
\item \texttt{+ String getParam(String param):}\\ ritorna il valore del parametro \textit{param} se presente nella \textit{URL\ped{G}};
\item \texttt{+ void approveData():}\\ salva nel \textit{server} lo stato "approvato" ai dati della collezione \textit{processData} dei quali il \textit{process owner\ped{G}} ha richiesto l'approvazione;
\item \texttt{+ void rejectData():}\\ salva nel \textit{server} lo stato "approvato" ai dati della collezione \textit{processData} che il \textit{process owner\ped{G}} ha respinto;
\end{itemize}
\end{sloppypar}
\item \textbf{Esito:} Superato;
\end{itemize}
\end{flushleft} da fare



