\section{Pianificazione dei test}
Di seguito elenchiamo tutti i test di validazione, sistema ed integrazione previsti, prevedendo però ad un successivo aggiornamento per i test di unità. Per quanto riguarda le tempistiche di esecuzione dei test si faccia riferimento a \infoPDP.
Il valore \textbf{N.A} è da intendersi come non applicato, in quanto tali test saranno eseguiti successivamente nello svolgimento del progetto.
\subsection{Test di sistema}
In questa sezione vengono descritti i test di sistema che consentiranno a \gruppo ~di verificare il comportamento dinamico del sistema rispetto ai requisiti descritti in \infoAR. I test sotto riportati sono relativi ai requisiti software individuati e meritevoli di test.
\subsubsection{Descrizione dei test di sistema}
\subsubsection{Ambito utente}
\LTXtable{\textwidth}{RequisitiUtente.tex}
\subsubsection{Ambito process owner}
\LTXtable{\textwidth}{RequisitiAmministratore.tex}
\subsection{Requisiti di vincolo}
\LTXtable{\textwidth}{RequisitiVincolo.tex}
\subsection{Test di integrazione}
In questa sezione verranno descritti i test di integrazione, da utilizzare per i vari componenti descritti nella progettazione ad alto livello, che permettono di verificare la corretta integrazione ed il corretto flusso dei dati all’interno del sistema.
Si è deciso di utilizzare una strategia di integrazione incrementale bottom-up che permette di sviluppare e verificare le componenti in parallelo.\\
Assemblando le componenti in modo incrementale i difetti rilevati da un test sono da
attribuirsi, con maggior probabilità, all’ultima parte aggiunta e si rende ogni passo di
integrazione reversibile consentendo di retrocedere verso uno stato noto e sicuro.
In questo modo i componenti base vengono testati più volte riducendo la possibile presenza di errori.\\
Nel diagramma 
\subsubsection{Descrizione dei test di integrazione}
\LTXtable{\textwidth}{Dettagliotestintegrazione.tex}
%da fare
\subsection{Test di unità}
I test di unità saranno definiti in seguito nello sviluppo del prodotto software.
\subsubsection{Descrizione test di unità}%da fare
\LTXtable{\textwidth}{Dettagliotestunita.tex}
\subsection{Test di validazione}%da fare
Di seguito vengono descritti i test di validazione per certificare che il prodotto realizzato sia conforme alle attese. Per ogni test vengono descritti i vari passi da eseguire per testare i requisiti ad esso associati.
\subsubsection{Descrizione dei test di validazione}
\paragraph{Test TV1}
\begin{itemize}
\item loggare utente;
\end{itemize}

