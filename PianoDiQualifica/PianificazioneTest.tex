\section{Pianificazione dei test}
Di seguito elenchiamo tutti i test di validazione, sistema ed integrazione previsti e per questi sono indicati quelli che sono stati superati. 
\subsection{Test di sistema}
In questa sezione vengono descritti i test di sistema che hanno consentito a \gruppo ~di verificare il comportamento dinamico del sistema rispetto ai requisiti descritti in \infoAR. I test sotto riportati sono relativi ai requisiti software individuati e meritevoli di test.
\subsubsection{Descrizione dei test di sistema}
\subsubsection{Ambito utente}
\LTXtable{\textwidth}{RequisitiUtente.tex}
\subsubsection{Ambito process owner}
\LTXtable{\textwidth}{RequisitiAmministratore.tex}
\subsection{Requisiti di vincolo}
\LTXtable{\textwidth}{RequisitiVincolo.tex}
\subsection{Test di integrazione}
In questa sezione sono descritti i test di integrazione, da utilizzare per i vari componenti descritti nella progettazione ad alto livello, che permettono di verificare la corretta integrazione ed il corretto flusso dei dati all'interno del sistema.
Si è deciso di utilizzare una strategia di integrazione incrementale bottom-up che permette di sviluppare e verificare le componenti in parallelo.\\
Assemblando le componenti in modo incrementale i difetti rilevati da un test sono da
attribuirsi, con maggior probabilità, all'ultima parte aggiunta e si rende ogni passo di
integrazione reversibile consentendo di retrocedere verso uno stato noto e sicuro.
\subsubsection{Descrizione dei test di integrazione}
Di seguito sono elencati i test di integrazione relativo ai componenti indicati in \infoST. I test di integrazione valutano l'integrazione dei vari componenti mano a mano che vengono aggregati al progetto, facendo in modo di non perdere le funzionalità acquisite fino a quel momento.
\begin{longtable}{llXr}%p{0.1\textwidth}}
\toprule
\textbf{Test} & \textbf{Descrizione} & \textbf{Componente} & \textbf{Stato}\\
\toprule
TU1&Verificare che il sistema permetta all'utente di registrarsi&mcomponente&N.A.\\
\midrule
TU1&Verificare che il sistema permetta all'utente di registrarsi&componente&N.A.\\
\bottomrule
\caption{Tabella dei test di unità}
\end{longtable}
\subsection{Test di unità}
I test di unità sono effettuati creando una nuova classe basata sulla precedente denominata \textit{"nomeclasseTest"} ed invocando i metodi di test con i rispettivi parametri. 
Non tutte le classi sono state testate in quanto molte classi sono state generate in modo automatico, inoltre ci sono altre classi che sono solamente utilizzate come interfacce. All'interno di queste classi, per la parte server, è presente un metodo che si ripete in ognuna di esse e quindi una volta che ne è stata stabilita la correttezza è stato tralasciato per i test delle successive unità.
Per la definizione delle classi si faccia riferimento a \infoDP.
\subsubsection{Test di unità}
% VIEW

\newcommand{\view}{com.si\fshyp{}ri\fshyp{}us.se\fshyp{}quen\fshyp{}zia\fshyp{}to\fshyp{}re.cli\fshyp{}ent.view}


\newcommand{\viewAdmin}{com.si\fshyp{}ri\fshyp{}us.se\fshyp{}quen\fshyp{}zia\fshyp{}to\fshyp{}re.cli\fshyp{}ent.view.pro\fshyp{}cess\fshyp{}ow\fshyp{}ner}

\newcommand{\viewUser}{com.si\fshyp{}ri\fshyp{}us.se\fshyp{}quen\fshyp{}zia\fshyp{}to\fshyp{}re.cli\fshyp{}ent.view.u\fshyp{}ser}

% PRESENTER

\newcommand{\logic}{com.si\fshyp{}ri\fshyp{}us.se\fshyp{}quen\fshyp{}zia\fshyp{}to\fshyp{}re.cli\fshyp{}ent.pre\fshyp{}sen\fshyp{}ter}

\newcommand{\logicAdmin}{com.si\fshyp{}ri\fshyp{}us.se\fshyp{}quen\fshyp{}zia\fshyp{}to\fshyp{}re.cli\fshyp{}ent.pre\fshyp{}sen\fshyp{}ter.pro\fshyp{}cess\fshyp{}ow\fshyp{}ner}

\newcommand{\logicUser}{com.si\fshyp{}ri\fshyp{}us.se\fshyp{}quen\fshyp{}zia\fshyp{}to\fshyp{}re.cli\fshyp{}ent.pre\fshyp{}sen\fshyp{}ter.u\fshyp{}ser}


% MODEL

\newcommand{\modelAdmin}{com.si\fshyp{}ri\fshyp{}us.se\fshyp{}quen\fshyp{}zia\fshyp{}to\fshyp{}re.cli\fshyp{}ent.mo\fshyp{}del.pro\fshyp{}cess\fshyp{}\_\fshyp{}ow\fshyp{}ner}

\newcommand{\modelUser}{com.si\fshyp{}ri\fshyp{}us.se\fshyp{}quen\fshyp{}zia\fshyp{}to\fshyp{}re.cli\fshyp{}ent.mo\fshyp{}del.u\fshyp{}ser}

\newcommand{\model}{com.si\fshyp{}ri\fshyp{}us.se\fshyp{}quen\fshyp{}zia\fshyp{}to\fshyp{}re.cli\fshyp{}ent.mo\fshyp{}del}

\newcommand{\collection}{com.si\fshyp{}ri\fshyp{}us.se\fshyp{}quen\fshyp{}zia\fshyp{}to\fshyp{}re.cli\fshyp{}ent.mo\fshyp{}del.col\fshyp{}lec\fshyp{}tion}

\newcommand{\collectionu}{com.si\fshyp{}ri\fshyp{}us.se\fshyp{}quen\fshyp{}zia\fshyp{}to\fshyp{}re.cli\fshyp{}ent.mo\fshyp{}del.\fshyp{}user.col\fshyp{}lec\fshyp{}tion}

\newcommand{\collectionp}{com.si\fshyp{}ri\fshyp{}us.se\fshyp{}quen\fshyp{}zia\fshyp{}to\fshyp{}re.cli\fshyp{}ent.mo\fshyp{}del.pro\fshyp{}cess\fshyp{}ow\fshyp{}ner.col\fshyp{}lec\fshyp{}tion}

% SERVER
\newcommand{\sPresenter}{com.si\fshyp{}ri\fshyp{}us.se\fshyp{}quen\fshyp{}zia\fshyp{}to\fshyp{}re.ser\fshyp{}ver.pre\fshyp{}sen\fshyp{}ter}

\newcommand{\sCommunication}{com.si\fshyp{}ri\fshyp{}us.se\fshyp{}quen\fshyp{}zia\fshyp{}to\fshyp{}re.ser\fshyp{}ver.pre\fshyp{}sen\fshyp{}ter.com\fshyp{}mu\fshyp{}ni\fshyp{}ca\fshyp{}tion}

\newcommand{\sLogicUser}{com.si\fshyp{}ri\fshyp{}us.se\fshyp{}quen\fshyp{}zia\fshyp{}to\fshyp{}re.ser\fshyp{}ver.pre\fshyp{}sen\fshyp{}ter.u\fshyp{}ser}

\newcommand{\sLogicAdmin}{com.si\fshyp{}ri\fshyp{}us.se\fshyp{}quen\fshyp{}zia\fshyp{}to\fshyp{}re.ser\fshyp{}ver.pre\fshyp{}sen\fshyp{}ter.pro\fshyp{}cess\fshyp{}ow\fshyp{}ner}

\newcommand{\sModel}{com.si\fshyp{}ri\fshyp{}us.se\fshyp{}quen\fshyp{}zia\fshyp{}to\fshyp{}re.ser\fshyp{}ver.mo\fshyp{}del}

\newcommand{\daoUser}{com.si\fshyp{}ri\fshyp{}us.se\fshyp{}quen\fshyp{}zia\fshyp{}to\fshyp{}re.ser\fshyp{}ver.mo\fshyp{}del.dao\fshyp{}u\fshyp{}ser}

\newcommand{\daoAdmin}{com.si\fshyp{}ri\fshyp{}us.se\fshyp{}quen\fshyp{}zia\fshyp{}to\fshyp{}re.ser\fshyp{}ver.mo\fshyp{}del.dao\fshyp{}pro\fshyp{}cess\fshyp{}ow\fshyp{}ner}

\newcommand{\daoProcess}{com.si\fshyp{}ri\fshyp{}us.se\fshyp{}quen\fshyp{}zia\fshyp{}to\fshyp{}re.ser\fshyp{}ver.mo\fshyp{}del.dao\fshyp{}pro\fshyp{}cess}

\newcommand{\daoStep}{com.si\fshyp{}ri\fshyp{}us.se\fshyp{}quen\fshyp{}zia\fshyp{}to\fshyp{}re.ser\fshyp{}ver.mo\fshyp{}del.dao\fshyp{}step}

\newcommand{\smodel}{com.si\fshyp{}ri\fshyp{}us.se\fshyp{}quen\fshyp{}zia\fshyp{}to\fshyp{}re.ser\fshyp{}ver.mo\fshyp{}del}
Per la \textit{View}, essendo le classi prive di metodi, viene creato un metodo TestView() che riceve come parametro il tipo di dato richiesto e se ne verifica il risultato ottenuto confrontandolo con quello atteso.
Per le classi \textit{Controller} e \textit{Model} vengono verificati i metodi della classe, per controllare che ritornino il valore atteso.
\begin{small}\centering
\begin{tabular}{|c|p{8.0cm}|p{2.0cm}|}
\hline
\textbf{ID Test} & \textbf{Descrizione del test condotto} & \textbf{Esito del test} \\
\hline
\textit{TU - 01} &
\textit{il test verifica la corretta inizializzazione di una processowner:ProcessCollection} & superato \\
\hline
\textit{TU - 02} &
\textit{il test verifica che  processowner:ProcessCollection.fetch estragga correttamente i dati dal server } & superato \\
\hline
\textit{TU - 03} & 
\textit{verifica che processowner:ProcessDataCollection:fecth fetchi correttamente i passi relativi ad un stepid dal server} & superato  \\
\hline
\textit{TU - 04} &
\textit{verifica che processowner:ProcessDataCollection:FetchWaiting fetchi correttamente i dati in attesa di conferma} & superato  \\
\hline
\textit{TU - 05} &
\textit{verifica la corretta inizializzazione di una processowner:ProcessDataCollection } & superato \\
\hline
\textit{TU - 06} &
\textit{verifica la corretta inizializzazione di una processowner:ProcessStepCollection} & superato  \\
\hline
\textit{TU - 07} &
\textit{verifica la corretta inizializzazione di un oggetto user:ProcessCollection} &  superato\\
\hline
\textit{TU - 08} &
\textit{verifica l'avvenuto recupero dal server tramite user:ProcessCollection:Fetch la collezione dei processi eseguibili} & superato \\
\hline
\textit{TU - 09} &
\textit{verifica l'avvenuta terminazione di un processo tramite TerminateProcess} & Test non effettuato \\
\hline
\textit{TU - 10} &
\textit{verifica la corretta inizializzazione di un oggetto user:ProcessDataCollection} &  \\
\hline
\textit{TU - 11} &
\textit{verifica la corretta inizializzazione di un oggetto user:ProcessDataModel} & superato \\
\hline
\textit{TU - 12} &
\textit{verifica che il salvataggio di dati fallisca in user:ProcessDataModel} & superato \\
\hline
\textit{TU - 13} &
\textit{verifica la corretta inizializzazione di un oggetto user:ProcessModelSpec} & superato \\
\hline
\textit{TU - 14} &
\textit{verifica che user:ProcessModel:fetch fetchi le informazioni relative al processmodel testato, dal server} & superato \\
\hline
\end{tabular}\\
\end{small}
\begin{small}\centering
\begin{tabular}{|c|p{8.0cm}|p{2.0cm}|}
\hline
\textit{TU - 15} &
\textit{verifica che user:ProcessModel:subscribe non accetti l'iscrizione al processo} &  superato\\
\hline
\textit{TU - 16} &
\textit{verifica che user:ProcessModel:unsubscribe rimuova l'utente dal processo} &  test non effettuato\\
\hline
\textit{TU - 17} &
\textit{verifica che user:ProcessModel sia correttamente inizializzato} & superato \\
\hline
\textit{TU - 18} &
\textit{verifica che user:ProcessModel:fetch fetchi correttamente i dati dal server } & superato \\
\hline
\textit{TU - 19} &
\textit{verifica che venga inizializzato correttamente un presenter:BaseDispatcher} & superato \\
\hline
\textit{TU - 20} &
\textit{ verifica che venga inserito correttamente un observer nel presenter:BaseDispatcher } & superato \\
\hline
\textit{TU - 21} &
\textit{ verifica che venga correttamente rimosso un observer dal presenter:BaseDispatcher} & superato \\
\hline
\textit{TU - 22} &
\textit{ verifica che fallisca la richiesta di un observer se l'index è superiore agli observer contenuti nel presenter:BaseDispatcher} & superato \\
\hline
\textit{TU - 23} &
\textit{verifica che venga correttamente inizializzato un oggetto presenter:processowner:EventDispatcher } & superato \\
\hline
\textit{TU - 24} &
\textit{verifica che venga rilevato correttamente un aggiornamento della collection in presenter:processowner:EventDispatcher} & superato \\
\hline
\textit{TU - 25} &
\textit{verifica che venga inizializzato correttamente un oggetto presenter:BasePresenter } & superato \\
\hline
\textit{TU - 26} &
\textit{verifica che venga effettuata correttamente la logout } & superato \\
\hline
\textit{TU - 27} &
\textit{verifica l'utente username si trovi nella posizione corretta per completare il passo } & test non effettuato \\
\hline
\textit{TU - 28} &
\textit{presenter:TerminateProcess, verifica che la richiesta di terminazione processo sia gestita } & test non effettuato \\
\hline
\textit{TU - 29} &
\textit{presenter:EventDispatcher, verifica che la notify sia invocata nel qualcaso ci siano nuovi passi in attesa di approvazione } & test non effettuato \\
\hline
\textit{TU - 30} &
\textit{presenter:Update, verifica l'aggiornamento dei dati della pagina recuperandoli dal server} & test non effettuato \\
\hline
\end{tabular}\\
\end{small}
\subsubsection{Descrizione dei test di validazione}
\begin{longtable}{lllll}
\toprule
\textbf{ID Test} & \textbf{Descrizione del test} & \textbf{Stato}\\
\toprule
TV1& Login process owner & superato\\
\midrule
TV2& Creazione processo&superato\\
\midrule
TV3& VisualizzaProcesso&superato\\
\midrule
TV4& Approvazione passo&superato\\
\midrule
TV5& Iscrizione/disiscrizione&superato\\
\midrule
TV6& Gestione dati&superato\\
\midrule
TV7& Esecuzione passo&superato\\
\bottomrule
\caption{Tabella dei test di integrazione}
\end{longtable}



