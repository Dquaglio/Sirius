\section{Pianificazione dei test}
Di seguito elenchiamo tutti i test di validazione, sistema ed integrazione previsti, prevedendo però ad un successivo aggiornamento ed integrazione degli stessi e dei test di unità nella prossima revisione. Per quanto riguarda le tempistiche di esecuzione dei test si faccia riferimento a \infoPDP.
Il valore \textbf{N.A} è da intendersi come non applicato, in quanto tali test saranno eseguiti successivamente nello svolgimento del progetto.
\subsection{Test di sistema}
In questa sezione vengono descritti i test di sistema che consentiranno a \gruppo ~di verificare il comportamento dinamico del sistema rispetto ai requisiti descritti in \infoAR. I test sotto riportati sono relativi ai requisiti software individuati e meritevoli di test.
\subsubsection{Descrizione dei test di sistema}
\subsubsection{Ambito utente}
\LTXtable{\textwidth}{RequisitiUtente.tex}
\subsubsection{Ambito amministratore}
\LTXtable{\textwidth}{RequisitiAmministratore.tex}
\subsection{Requisiti di vincolo}
\LTXtable{\textwidth}{RequisitiVincolo.tex}
\subsection{Test di integrazione}
\subsubsection{Descrizione dei test di integrazione}

\subsection{Test di validazione}
I test di validazione saranno definiti durante la progettazione di dettaglio in modo che possano essere il più completi possibile.

