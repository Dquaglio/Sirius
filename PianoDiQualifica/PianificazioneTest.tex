\section{Pianificazione dei test}
Di seguito elenchiamo tutti i test di validazione, sistema ed integrazione previsti, prevedendo però ad un successivo aggiornamento per i test di unità. Per quanto riguarda le tempistiche di esecuzione dei test si faccia riferimento a \infoPDP.
Il valore \textbf{N.A} è da intendersi come non applicato, in quanto tali test saranno eseguiti successivamente nello svolgimento del progetto.
\subsection{Test di sistema}
In questa sezione vengono descritti i test di sistema che consentiranno a \gruppo ~di verificare il comportamento dinamico del sistema rispetto ai requisiti descritti in \infoAR. I test sotto riportati sono relativi ai requisiti software individuati e meritevoli di test.
\subsubsection{Descrizione dei test di sistema}
\subsubsection{Ambito utente}
\LTXtable{\textwidth}{RequisitiUtente.tex}
\subsubsection{Ambito process owner}
\LTXtable{\textwidth}{RequisitiAmministratore.tex}
\subsection{Requisiti di vincolo}
\LTXtable{\textwidth}{RequisitiVincolo.tex}
\subsection{Test di integrazione}
In questa sezione verranno descritti i test di integrazione, da utilizzare per i vari componenti descritti nella progettazione ad alto livello, che permettono di verificare la corretta integrazione ed il corretto flusso dei dati all’interno del sistema.
Si è deciso di utilizzare una strategia di integrazione incrementale bottom-up che permette di sviluppare e verificare le componenti in parallelo.\\
Assemblando le componenti in modo incrementale i difetti rilevati da un test sono da
attribuirsi, con maggior probabilità, all’ultima parte aggiunta e si rende ogni passo di
integrazione reversibile consentendo di retrocedere verso uno stato noto e sicuro.
In questo modo i componenti base vengono testati più volte riducendo la possibile presenza di errori.\\
Nel diagramma 
\subsubsection{Descrizione dei test di integrazione}
Di seguito sono elencati i test di integrazione relativo ai componenti indicati in \infoST.
\LTXtable{\textwidth}{Dettagliotestintegrazione.tex}
%da fare
\subsection{Test di unità}
I test di unità sulle view sono effettuati creando una nuova classe basata sulla precedente e definita "testnome" ed invocando il metodo test con i rispettivi parametri.
\paragraph{Login}
\begin{itemize}
\item \textbf{Descrizione:} \textit{Template HTML} che permette di gestire l'interfaccia grafica relativa alle richieste di autenticazione al sistema.
\end{itemize}

\paragraph{MainUser}
\begin{itemize}
\item \textbf{Descrizione:} Classe che permette la gestione delle principali componenti dell'interfaccia grafica dell'utente.
\end{itemize}

\paragraph{Register}
\begin{itemize}
\item \textbf{Descrizione:} \textit{Template HTML} che permette di gestire dell'interfaccia grafica relativa alle richieste di registrazione da parte dell'utente.
\end{itemize}

\paragraph{UserData}
\begin{itemize}
\item \textbf{Descrizione:} \textit{Template HTML} che permette la realizzazione dei \textit{widget} che consentono visualizzazione e modifica dei dati dell'utente.
\end{itemize}

\paragraph{OpenProcess}
\begin{itemize}
\item \textbf{Descrizione:} \textit{Template HTML} che permette di realizzare i \textit{widget} per consentire l'apertura di un processo tramite ricerca o selezionandolo da una lista.
\end{itemize}

\paragraph{ManagementProcess}
\begin{itemize}
\item \textbf{Descrizione:} \textit{Template HTML} che permette di realizzare i \textit{widget} per consentire la visualizzazione dello stato del processo selezionato e i vincoli per concludere il passo in corso.
\end{itemize}

\paragraph{SendData}
\begin{itemize}
\item \textbf{Descrizione:} \textit{Template HTML} che permette di realizzare i \textit{widget} per consentire l'invio dei dati richiesti per la conclusione del passo in esecuzione.
\end{itemize}

\paragraph{SendText}
\begin{itemize}
\item \textbf{Descrizione:} \textit{Template HTML} che permette di realizzare i \textit{widget} che consentono di inserire il testo da inviare per concludere il passo in esecuzione.
\end{itemize}

\paragraph{SendNumb}
\begin{itemize}
\item \textbf{Descrizione:} \textit{Template HTML} che permette agli oggetti che la implementano di realizzare i \textit{widget} che consentono di inserire i dati numerici da inviare per concludere il passo in esecuzione.
\end{itemize}

\paragraph{SendPosition}
\begin{itemize}
\item \textbf{Descrizione:} \textit{Template HTML} che permette  di realizzare i \textit{widget} che consentono di inviare la posizione geografica richiesta per la conclusione del passo in esecuzione.
\end{itemize}

\paragraph{SendImage}
\begin{itemize}
\item \textbf{Descrizione:} \textit{Template HTML} che permette di realizzare i \textit{widget} che consentono di inserire le immagini richieste per concludere i passo in esecuzione.
\end{itemize}

\paragraph{PrintProcess}
\begin{itemize}
\item \textbf{Descrizione:} \textit{Template HTML} che permette di realizzare i \textit{widget} che consentono il salvataggio dei \textit{report} sull'esecuzione del processo.
\end{itemize}

\paragraph{MainProcessOwner}
\begin{itemize}
\item \textbf{Descrizione:} Componente che permette la gestione delle principali componenti dell'interfaccia grafica dell'utente \textit{process owner\ped{G}}.
\end{itemize}

\paragraph{NewProcess}
\begin{itemize}
\item \textbf{Descrizione:} \textit{Template HTML} che permette di gestire l'interfaccia grafica che consente di creare nuovi processi.
\end{itemize}

\paragraph{AddStep}
\begin{itemize}
\item \textbf{Descrizione:} \textit{Template HTML} che permette di gestire l'interfaccia grafica che consente di definire un nuovo passo del processo in creazione.
\end{itemize}

\paragraph{OpenProcess}
\begin{itemize}
\item \textbf{Descrizione:} \textit{Template HTML} che permette di realizzare i\textit{widget} che consentono di aprire un processo tramite ricerca o selezionandolo da una lista.
\end{itemize}

\paragraph{ManageProcess}
\begin{itemize}
\item \textbf{Descrizione:} \textit{Template HTML} che permette di realizzare i\textit{widget} che consentono di gestire l'accesso ai dati inviati al\textit{server\ped{G}} dagli utenti.
\end{itemize}

\paragraph{CheckStep}
\begin{itemize}
\item \textbf{Descrizione:} \textit{Template HTML} che permette di realizzare i\textit{widget} che consentono di gestire l'approvazione dei passi che richiedono intervento umano.
\end{itemize}

%\LTXtable{\textwidth}{TestUnitaPresenter.tex}

%\subsection{Test di validazione}%da fare per RA
%Di seguito vengono descritti i test di validazione per certificare che il prodotto realizzato sia conforme alle attese. Per ogni test vengono descritti i vari passi da eseguire per testare i requisiti ad esso associati.
%\subsubsection{Descrizione dei test di validazione}
%\paragraph{Test TV1}
%\begin{itemize}
%\item loggare utente;
%\end{itemize}

