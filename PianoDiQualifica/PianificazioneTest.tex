\section{Pianificazione dei test}
Di seguito elenchiamo tutti i test di validazione, sistema ed integrazione previsti e per questi sono indicati quelli che sono stati superati. 
\subsection{Test di sistema}
In questa sezione vengono descritti i test di sistema che hanno consentito a \gruppo ~di verificare il comportamento dinamico del sistema rispetto ai requisiti descritti in \infoAR. I test sotto riportati sono relativi ai requisiti software individuati e meritevoli di test.
\subsubsection{Descrizione dei test di sistema}
\subsubsection{Ambito utente}
\LTXtable{\textwidth}{RequisitiUtente.tex}
\subsubsection{Ambito process owner}
\LTXtable{\textwidth}{RequisitiAmministratore.tex}
\subsection{Requisiti di vincolo}
\LTXtable{\textwidth}{RequisitiVincolo.tex}
I test di sistema rono risultati utili in quanto hanno evidenziato un bug. La disiscrizione non funzionava in modo corretto. Da un'attenta analisi si è evidenziato che il metodo fetch di \textit{backbone} non resettava correttamente i dati caricati in precedenza. Il bug è stato risolto.
\subsection{Test di integrazione}
In questa sezione sono descritti i test di integrazione, da utilizzare per i vari componenti descritti nella progettazione ad alto livello, che permettono di verificare la corretta integrazione ed il corretto flusso dei dati all'interno del sistema.
Si è deciso di utilizzare una strategia di integrazione incrementale bottom-up che permette di sviluppare e verificare le componenti in parallelo.\\
Assemblando le componenti in modo incrementale i difetti rilevati da un test sono da
attribuirsi, con maggior probabilità, all'ultima parte aggiunta e si rende ogni passo di
integrazione reversibile consentendo di retrocedere verso uno stato noto e sicuro.
\subsubsection{Descrizione dei test di integrazione}
Di seguito sono elencati i test di integrazione relativo ai componenti indicati in \infoST. I test di integrazione valutano l'integrazione dei vari componenti mano a mano che vengono aggregati al progetto, facendo in modo di non perdere le funzionalità acquisite fino a quel momento.
\begin{longtable}{lllllXr}
\toprule
\textbf{Test} & \textbf{Descrizione} & \textbf{Componente} & \textbf{Test} & \textbf{Stato}\\
&&&\textbf{di integrazione}&\\
\toprule
TU1&login utente e login process owner &Login & & Pianificato\\
\midrule
TU2&creazione processo&NuovoProcesso&&N.A.\\
\midrule
TU3&visualizzazione dati processo&VisualizzaProcesso&&N.A.\\
\midrule
TU4&Approvazione passo&ApprovazionePasso&&N.A.\\
\bottomrule
\caption{Tabella dei test di integrazione process owner}
\end{longtable}

\begin{longtable}{lllllXr}
\toprule
\textbf{Test} & \textbf{Descrizione} & \textbf{Componente} & \textbf{Test} & \textbf{Stato}\\
&&&\textbf{di integrazione}&\\
\toprule
TU5&gestione dati&GestioneUser&&N.A.\\
\midrule
TU6&iscrizione/disiscrizione&Registrazione&&N.A.\\
\midrule
TU7&esecuzione passo&EsecuzionePasso&&N.A.\\
\midrule
TU8&Visualizzazione report finale&Stampa&&N.A.\\
\bottomrule
\caption{Tabella dei test di integrazione user}
\end{longtable}
\subsection{Test di unità}
I test di unità sono effettuati creando una nuova classe basata sulla precedente denominata \textit{"nomeclasseTest"} ed invocando i metodi di test con i rispettivi parametri. 
Non tutte le classi sono state testate in quanto molte classi sono state generate in modo automatico, inoltre ci sono altre classi che sono solamente utilizzate come interfacce. All'interno di queste classi, per la parte server, è presente un metodo che si ripete in ognuna di esse e quindi una volta che ne è stata stabilita la correttezza è stato tralasciato per i test delle successive unità.
Per la definizione delle classi si faccia riferimento a \infoDP.
\subsubsection{Test di unità}
% VIEW

\newcommand{\viewAdmin}{se\fshyp{}quen\fshyp{}zia\fshyp{}to\fshyp{}re.cli\fshyp{}ent::view::pro\fshyp{}cess\fshyp{}ow\fshyp{}ner}

\newcommand{\iViewAdmin}{se\fshyp{}quen\fshyp{}zia\fshyp{}to\fshyp{}re::cli\fshyp{}ent::view::i\fshyp{}pro\fshyp{}cess\fshyp{}ow\fshyp{}ner}

\newcommand{\viewUser}{se\fshyp{}quen\fshyp{}zia\fshyp{}to\fshyp{}re::cli\fshyp{}ent::view::u\fshyp{}ser}

\newcommand{\iViewUser}{se\fshyp{}quen\fshyp{}zia\fshyp{}to\fshyp{}re::cli\fshyp{}ent::view::i\fshyp{}U\fshyp{}ser}

% PRESENTER

\newcommand{\iLogicAdmin}{se\fshyp{}quen\fshyp{}zia\fshyp{}to\fshyp{}re::cli\fshyp{}ent::pre\fshyp{}sen\fshyp{}ter::i\fshyp{}pro\fshyp{}cess\fshyp{}ow\fshyp{}ner::i\fshyp{}lo\fshyp{}gic}

\newcommand{\logicAdmin}{se\fshyp{}quen\fshyp{}zia\fshyp{}to\fshyp{}re::cli\fshyp{}ent::pre\fshyp{}sen\fshyp{}ter::pro\fshyp{}cess\fshyp{}ow\fshyp{}ner::lo\fshyp{}gic}

\newcommand{\iLogicUser}{se\fshyp{}quen\fshyp{}zia\fshyp{}to\fshyp{}re::cli\fshyp{}ent::pre\fshyp{}sen\fshyp{}ter::i\fshyp{}u\fshyp{}ser::i\fshyp{}lo\fshyp{}gic}

\newcommand{\logicUser}{se\fshyp{}quen\fshyp{}zia\fshyp{}to\fshyp{}re::cli\fshyp{}ent::pre\fshyp{}sen\fshyp{}ter::u\fshyp{}ser::lo\fshyp{}gic}

\newcommand{\serverCommunication}{se\fshyp{}quen\fshyp{}zia\fshyp{}to\fshyp{}re::cli\fshyp{}ent::pre\fshyp{}sen\fshyp{}ter::ser\fshyp{}ver\fshyp{}com\fshyp{}mu\fshyp{}ni\fshyp{}ca\fshyp{}tion}

\newcommand{\iServerCommunication}{se\fshyp{}quen\fshyp{}zia\fshyp{}to\fshyp{}re::cli\fshyp{}ent::pre\fshyp{}sen\fshyp{}ter::i\fshyp{}ser\fshyp{}ver\fshyp{}com\fshyp{}mu\fshyp{}ni\fshyp{}ca\fshyp{}tion}


% MODEL

\newcommand{\modelAdmin}{se\fshyp{}quen\fshyp{}zia\fshyp{}to\fshyp{}re::cli\fshyp{}ent::mo\fshyp{}del::lo\fshyp{}cal\fshyp{}da\fshyp{}ta\_pro\fshyp{}cess\fshyp{}\_\fshyp{}ow\fshyp{}ner}

\newcommand{\modelUser}{se\fshyp{}quen\fshyp{}zia\fshyp{}to\fshyp{}re::cli\fshyp{}ent::mo\fshyp{}del::lo\fshyp{}cal\fshyp{}da\fshyp{}ta\_u\fshyp{}ser}

\newcommand{\iModelAdmin}{se\fshyp{}quen\fshyp{}zia\fshyp{}to\fshyp{}re::cli\fshyp{}ent::mo\fshyp{}del::i\fshyp{}lo\fshyp{}cal\fshyp{}da\fshyp{}ta\_pro\fshyp{}cess\fshyp{}\_\fshyp{}ow\fshyp{}ner}

\newcommand{\iModelUser}{se\fshyp{}quen\fshyp{}zia\fshyp{}to\fshyp{}re::cli\fshyp{}ent::mo\fshyp{}del::i\fshyp{}lo\fshyp{}cal\fshyp{}da\fshyp{}ta\_u\fshyp{}ser}

\newcommand{\model}{se\fshyp{}quen\fshyp{}zia\fshyp{}to\fshyp{}re::cli\fshyp{}ent::mo\fshyp{}del}
Per la \textit{View}, essendo le classi prive di metodi, viene creato un metodo TestView() che riceve come parametro il tipo di dato richiesto e se ne verifica il risultato ottenuto confrontandolo con quello atteso.
Per le classi \textit{Controller} e \textit{Model} vengono verificati i metodi della classe, per controllare che ritornino il valore atteso.
\begin{small}\centering
\begin{tabular}{|c|p{8.0cm}|p{2.0cm}|}
\hline
\textbf{ID Test} & \textbf{Descrizione del test condotto} & \textbf{Esito del test} \\
\hline
\textit{TU - 01} &
\textit{il test verifica la corretta inizializzazione di una processowner:ProcessCollection} & superato \\
\hline
\textit{TU - 02} &
\textit{il test verifica che  processowner:ProcessCollection.fetch estragga correttamente i dati dal server } & superato \\
\hline
\textit{TU - 03} & 
\textit{verifica che processowner:ProcessDataCollection:fecth fetchi correttamente i passi relativi ad un stepid dal server} & superato  \\
\hline
\textit{TU - 04} &
\textit{verifica che processowner:ProcessDataCollection:FetchWaiting fetchi correttamente i dati in attesa di conferma} & superato  \\
\hline
\textit{TU - 05} &
\textit{verifica la corretta inizializzazione di una processowner:ProcessDataCollection } & superato \\
\hline
\textit{TU - 06} &
\textit{verifica la corretta inizializzazione di una processowner:ProcessStepCollection} & superato  \\
\hline
\textit{TU - 07} &
\textit{verifica la corretta inizializzazione di un oggetto user:ProcessCollection} &  superato\\
\hline
\textit{TU - 08} &
\textit{verifica l'avvenuto recupero dal server tramite user:ProcessCollection:Fetch la collezione dei processi eseguibili} & superato \\
\hline
\textit{TU - 09} &
\textit{verifica l'avvenuta terminazione di un processo tramite TerminateProcess} & Test non effettuato \\
\hline
\textit{TU - 10} &
\textit{verifica la corretta inizializzazione di un oggetto user:ProcessDataCollection} &  \\
\hline
\textit{TU - 11} &
\textit{verifica la corretta inizializzazione di un oggetto user:ProcessDataModel} & superato \\
\hline
\textit{TU - 12} &
\textit{verifica che il salvataggio di dati fallisca in user:ProcessDataModel} & superato \\
\hline
\textit{TU - 13} &
\textit{verifica la corretta inizializzazione di un oggetto user:ProcessModelSpec} & superato \\
\hline
\textit{TU - 14} &
\textit{verifica che user:ProcessModel:fetch fetchi le informazioni relative al processmodel testato, dal server} & superato \\
\hline
\end{tabular}\\
\end{small}

\begin{small}\centering
\begin{tabular}{|c|p{8.0cm}|p{2.0cm}|}
\hline
\textit{TU - 15} &
\textit{verifica che user:ProcessModel:subscribe non accetti l'iscrizione al processo} &  superato\\
\hline
\textit{TU - 16} &
\textit{verifica che user:ProcessModel:unsubscribe rimuova l'utente dal processo} &  test non effettuato\\
\hline
\textit{TU - 17} &
\textit{verifica che user:ProcessModel sia correttamente inizializzato} & superato \\
\hline
\textit{TU - 18} &
\textit{verifica che user:ProcessModel:fetch fetchi correttamente i dati dal server } & superato \\
\hline
\textit{TU - 19} &
\textit{verifica che venga inizializzato correttamente un presenter:BaseDispatcher} & superato \\
\hline
\textit{TU - 20} &
\textit{ verifica che venga inserito correttamente un observer nel presenter:BaseDispatcher } & superato \\
\hline
\textit{TU - 21} &
\textit{ verifica che venga correttamente rimosso un observer dal presenter:BaseDispatcher} & superato \\
\hline
\textit{TU - 22} &
\textit{ verifica che fallisca la richiesta di un observer se l'index è superiore agli observer contenuti nel presenter:BaseDispatcher} & superato \\
\hline
\textit{TU - 23} &
\textit{verifica che venga correttamente inizializzato un oggetto presenter:processowner:EventDispatcher } & superato \\
\hline
\textit{TU - 24} &
\textit{verifica che venga rilevato correttamente un aggiornamento della collection in presenter:processowner:EventDispatcher} & superato \\
\hline
\textit{TU - 25} &
\textit{verifica che venga inizializzato correttamente un oggetto presenter:BasePresenter } & superato \\
\hline
\textit{TU - 26} &
\textit{verifica che venga effettuata correttamente la logout } & superato \\
\hline
\textit{TU - 27} &
\textit{verifica l'utente username si trovi nella posizione corretta per completare il passo } & test non effettuato \\
\hline
\textit{TU - 28} &
\textit{presenter:TerminateProcess, verifica che la richiesta di terminazione processo sia gestita } & test non effettuato \\
\hline
\textit{TU - 29} &
\textit{presenter:EventDispatcher, verifica che la notify sia invocata nel qualcaso ci siano nuovi passi in attesa di approvazione } & test non effettuato \\
\hline
\textit{TU - 30} &
\textit{presenter:Update, verifica l'aggiornamento dei dati della pagina recuperandoli dal server} & test non effettuato \\
\hline
\end{tabular}\\
\end{small}

\begin{small}\centering
\begin{tabular}{|c|p{8.0cm}|p{2.0cm}|}
\hline
\textit{TU - 31} &
\textit{verificato il corretto funzionamento del controller:LoginController} & test superato \\
\hline

\textit{TU - 32} &
\textit{verificato il corretto funzionamento del controller:SignUpController} & test superato \\
\hline

\textit{TU - 33} &
\textit{verificato il corretto funzionamento del controller:ProcessInfoController} & test superato \\
\hline

\textit{TU - 34} &
\textit{verificato il corretto funzionamento del controller:StepInfoController} & test superato \\
\hline

\textit{TU - 35} &
\textit{verificato il corretto funzionamento del controller:ApproveStepController} & test superato \\
\hline

\textit{TU - 36} &
\textit{verificato il corretto funzionamento del controller:StepController} & test superato \\
\hline

\textit{TU - 37} &
\textit{verificato il corretto funzionamento del controller:ProcessController} & test superato \\
\hline

\textit{TU - 38} &
\textit{verificato il corretto funzionamento del controller:ReportController} & test superato \\
\hline

\textit{TU - 39} &
\textit{verificato il corretto funzionamento del controller:UserProcessController} & test superato \\
\hline

\textit{TU - 40} &
\textit{verificato il corretto funzionamento del controller:UserStepController} & test superato \\
\hline

\textit{TU - 41} &
\textit{verificato il corretto funzionamento del service:ApproveStepService} & test superato \\
\hline
\end{tabular}\\
\end{small}

\begin{small}\centering
\begin{tabular}{|c|p{8.0cm}|p{2.0cm}|}
\textit{TU - 42} &
\textit{verificato il corretto funzionamento del service:ProcessInfoService} & test non eseguito \\
\hline

\textit{TU - 43} &
\textit{verificato il corretto funzionamento del service:LoginService} & test superato \\
\hline

\textit{TU - 44} &
\textit{verificato il corretto funzionamento del service:SignUpService} & test superato \\
\hline

\textit{TU - 46} &
\textit{verificato il corretto funzionamento del service:StepInfoService} & test non eseguito \\
\hline

\textit{TU - 47} &
\textit{verificato il corretto funzionamento del service:UserPRocessService} & test non eseguito \\
\hline

\textit{TU - 48} &
\textit{verificato il corretto funzionamento del service:UserStepService} & test non eseguito \\
\hline

\textit{TU - 49} &
\textit{verificato il corretto funzionamento del service:ProcessService} & test non eseguito \\
\hline

\textit{TU - 50} &
\textit{verificato il corretto funzionamento del service:StepService} & test non eseguito \\
\hline

\textit{TU - 51} &
\textit{verificato il corretto funzionamento del service:ReportService} & test non eseguito \\
\hline

\textit{TU - 52} &
\textit{verificato il corretto funzionamento del service:StepInfoService} & test non eseguito \\
\hline
\end{tabular}\\
\end{small}

\begin{small}\centering
\begin{tabular}{|c|p{8.0cm}|p{2.0cm}|}
\textit{TU - 53} &
\textit{verificato il corretto funzionamento del model:IDataAcessObject} & test superato \\
\hline

\textit{TU - 54} &
\textit{verificato il corretto funzionamento del model:ITransferObject} & test superato \\
\hline

\textit{TU - 55} &
\textit{verificato il corretto funzionamento del model:UserDao} & test superato \\
\hline

\textit{TU - 56} &
\textit{verificato il corretto funzionamento del model:ProcessDao} & test superato \\
\hline

\textit{TU - 57} &
\textit{verificato il corretto funzionamento del model:ProcessOwnerDao} & test superato \\
\hline

\textit{TU - 58} &
\textit{verificato il corretto funzionamento del model:StepDao} & test superato \\
\hline

\textit{TU - 59} &
\textit{verificato il corretto funzionamento del model:User} & test superato \\
\hline

\textit{TU - 60} &
\textit{verificato il corretto funzionamento del model:Process} & test superato \\
\hline

\textit{TU - 61} &
\textit{verificato il corretto funzionamento del model:Step} & test superato \\
\hline

\textit{TU - 63} &
\textit{verificato il corretto funzionamento del model:DataSent} & test superato \\
\hline
\end{tabular}\\
\end{small}

\begin{small}\centering
\begin{tabular}{|c|p{8.0cm}|p{2.0cm}|}
\textit{TU - 64} &
\textit{verificato il corretto funzionamento del model:IDataValue} & test superato \\
\hline

\textit{TU - 65} &
\textit{verificato il corretto funzionamento del model:TextualValue} & test superato \\
\hline

\textit{TU - 66} &
\textit{verificato il corretto funzionamento del model:NumericValue} & test superato \\
\hline

\textit{TU - 67} &
\textit{verificato il corretto funzionamento del model:ImageValue} & test superato \\
\hline

\textit{TU - 68} &
\textit{verificato il corretto funzionamento del model:GeographicValue} & test superato \\
\hline

\textit{TU - 69} &
\textit{verificato il corretto funzionamento del model:UserStep} & test superato \\
\hline

\textit{TU - 70} &
\textit{verificato il corretto funzionamento del model:ProcessOwner} & test superato \\
\hline
\end{tabular}\\
\end{small}
\subsubsection{Descrizione dei test di validazione}
\begin{small}\centering
\begin{tabular}{|c|p{8.0cm}|p{2.0cm}|}
\hline
\textbf{ID Test} & \textbf{Descrizione del test} & \textbf{Stato}\\
\hline
TV1& Verifica del corretto funzionamento della parte login per quanto riguarda lo user e il process owner & superato\\
\hline
TV2& Il process owner può creare un processo, visualizzarlo e renderlo visibile agli user&superato\\
\hline
TV3& Il process owner dichiara la validazione di un passo e lo user ne riceve la notifica di avvenuta accettazione&superato\\
\hline
TV4& User deve potersi iscrivere e disiscrivere correttamente, dandone corrispondenza al process owner&superato\\
\hline
TV5& Lo user può effettuare un passo tramite dato geolocalizzazione, immagine o numerico. Verifica con dato corretto e con dato errato. Verifica inoltre che il process owner ne riceva la notifica di passo avvenuto, ed in caso di richiesta di conferma sia possibile accettare o meno il superamento del passo&superato\\
\hline
\end{tabular}\\
\end{small}



