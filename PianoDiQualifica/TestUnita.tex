\begin{small}\centering
\begin{tabular}{|c|p{8.0cm}|p{2.0cm}|}
\hline
\textbf{ID Test} & \textbf{Descrizione del test condotto} & \textbf{Esito del test} \\
\hline
\textit{TU - 01} &
\textit{il test verifica la corretta inizializzazione di una processowner:ProcessCollection} & superato \\
\hline
\textit{TU - 02} &
\textit{il test verifica che  processowner:ProcessCollection.fetch estragga correttamente i dati dal server } & superato \\
\hline
\textit{TU - 03} & 
\textit{verifica che processowner:ProcessDataCollection:fecth fetchi correttamente i passi relativi ad un stepid dal server} & superato  \\
\hline
\textit{TU - 04} &
\textit{verifica che processowner:ProcessDataCollection:FetchWaiting fetchi correttamente i dati in attesa di conferma} & superato  \\
\hline
\textit{TU - 05} &
\textit{verifica la corretta inizializzazione di una processowner:ProcessDataCollection } & superato \\
\hline
\textit{TU - 06} &
\textit{verifica la corretta inizializzazione di una processowner:ProcessStepCollection} & superato  \\
\hline
\textit{TU - 07} &
\textit{verifica la corretta inizializzazione di un oggetto user:ProcessCollection} &  superato\\
\hline
\textit{TU - 08} &
\textit{verifica l'avvenuto recupero dal server tramite user:ProcessCollection:Fetch la collezione dei processi eseguibili} & superato \\
\hline
\textit{TU - 09} &
\textit{verifica l'avvenuta terminazione di un processo tramite TerminateProcess} & Test non effettuato \\
\hline
\textit{TU - 10} &
\textit{verifica la corretta inizializzazione di un oggetto user:ProcessDataCollection} &  \\
\hline
\textit{TU - 11} &
\textit{verifica la corretta inizializzazione di un oggetto user:ProcessDataModel} & superato \\
\hline
\textit{TU - 12} &
\textit{verifica che il salvataggio di dati fallisca in user:ProcessDataModel} & superato \\
\hline
\textit{TU - 13} &
\textit{verifica la corretta inizializzazione di un oggetto user:ProcessModelSpec} & superato \\
\hline
\textit{TU - 14} &
\textit{verifica che user:ProcessModel:fetch fetchi le informazioni relative al processmodel testato, dal server} & superato \\
\hline
\end{tabular}\\
\end{small}
\begin{small}\centering
\begin{tabular}{|c|p{8.0cm}|p{2.0cm}|}
\hline
\textit{TU - 15} &
\textit{verifica che user:ProcessModel:subscribe non accetti l'iscrizione al processo} &  superato\\
\hline
\textit{TU - 16} &
\textit{verifica che user:ProcessModel:unsubscribe rimuova l'utente dal processo} &  test non effettuato\\
\hline
\textit{TU - 17} &
\textit{verifica che user:ProcessModel sia correttamente inizializzato} & superato \\
\hline
\textit{TU - 18} &
\textit{verifica che user:ProcessModel:fetch fetchi correttamente i dati dal server } & superato \\
\hline
\textit{TU - 19} &
\textit{verifica che venga inizializzato correttamente un presenter:BaseDispatcher} & superato \\
\hline
\textit{TU - 20} &
\textit{ verifica che venga inserito correttamente un observer nel presenter:BaseDispatcher } & superato \\
\hline
\textit{TU - 21} &
\textit{ verifica che venga correttamente rimosso un observer dal presenter:BaseDispatcher} & superato \\
\hline
\textit{TU - 22} &
\textit{ verifica che fallisca la richiesta di un observer se l'index è superiore agli observer contenuti nel presenter:BaseDispatcher} & superato \\
\hline
\textit{TU - 23} &
\textit{verifica che venga correttamente inizializzato un oggetto presenter:processowner:EventDispatcher } & superato \\
\hline
\textit{TU - 24} &
\textit{verifica che venga rilevato correttamente un aggiornamento della collection in presenter:processowner:EventDispatcher} & superato \\
\hline
\textit{TU - 25} &
\textit{verifica che venga inizializzato correttamente un oggetto presenter:BasePresenter } & superato \\
\hline
\textit{TU - 26} &
\textit{verifica che venga effettuata correttamente la logout } & superato \\
\hline
\textit{TU - 27} &
\textit{verifica l'utente username si trovi nella posizione corretta per completare il passo } & test non effettuato \\
\hline
\textit{TU - 28} &
\textit{presenter:TerminateProcess, verifica che la richiesta di terminazione processo sia gestita } & test non effettuato \\
\hline
\textit{TU - 29} &
\textit{presenter:EventDispatcher, verifica che la notify sia invocata nel qualcaso ci siano nuovi passi in attesa di approvazione } & test non effettuato \\
\hline
\textit{TU - 30} &
\textit{presenter:Update, verifica l'aggiornamento dei dati della pagina recuperandoli dal server} & test non effettuato \\
\hline
\end{tabular}\\
\end{small}