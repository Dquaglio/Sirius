\section{Descrizione messaggi di errore}
\label{errori}

\subsection{E1: javascript disabilitato}
\label{e1}
Questo errore viene visualizzato quando nel \textit{Browser\ped{G}} utilizzato non è attivato l'utilizzo di \textit{Javascript\ped{G}}.
Per attivare \textit{Javascript} dalle opzioni del \textit{Browser\ped{G}}, è possibile seguire le istruzioni presenti all'indirizzo \url{http://aboutjavascript.com/it.html}.

\subsection{E2: credenziali non corrette}
\label{e2}
Questo errore si verifica se, durante la procedura di autenticazione o di modifica delle credenziali, l'\textit{username} o la \textit{password} inserite non sono corrette.
La guida al corretto utilizzo delle funzionalità di autenticazione  e di gestione delle credenziali, sono presenti nei capitoli \hyperref[autenticazione]{Autenticazione} e \hyperref[autenticazione]{Autenticazione}.

\subsection{E3: errore di connessione}
\label{e3}
Questo errore si verifica se, quando viene effettuata una richiesta al \textit{server\ped{G}}, il dispositivo utilizzato non ha accesso ad alcuna connessione.
Per proseguire con l'utilizzo dell'applicazione è necessario accedere alla rete.

\subsection{E4: username già utilizzato}
\label{e4}
Questo errore si verifica quando si richiede la registrazione al sistema, e l'\textit{username} scelto è già utilizzato da un altro utente.
Per risolvere il problema è sufficiente inserire un \textit{username} diverso, e ritentare la registrazione.

\subsection{E5: dato mancante}
\label{e5}
Questo errore si verifica quando si richiede il salvataggio di un nuovo processo, del quale non sono stati inseriti tutti i dati obbligatori.
Per risolvere il problema è sufficiente inserire il dato segnalato dal messaggio d'errore.
La guida al corretto utilizzo della funzionalità di creazione processo è presente nei seguenti capitoli:
\begin{itemize}
\item \hyperref[creazione]{Creazione di un processo}:
\begin{itemize}
\item \hyperref[definizione]{Definizione di un processo};
\item \hyperref[addstep]{Aggiunta di un passo};
\item \hyperref[vincoli]{Aggiunta di un criterio di superamento}.
\end{itemize}
\end{itemize}

\subsection{E6: email già utilizzata}
\label{e6}
Questo errore si verifica quando si richiede la registrazione al sistema o la modifica della \textit{email}, e l'\textit{email} scelta è già utilizzato da un altro utente.
Per risolvere il problema è sufficiente inserire una \textit{email} diversa, e ritentare la registrazione o la modifica \textit{email}.

\subsection{E7: email non valida}
\label{e7}
Questo errore si verifica quando si richiede la registrazione al sistema o la modifica della \textit{email}, e l'\textit{email} scelta è inesistente.
Per risolvere il problema è sufficiente inserire una \textit{email} diversa, e ritentare la registrazione o la modifica \textit{email}.

\subsection{E8: vincolo non rispettato}
\label{e8}
Questo errore si verifica quando, durante l'esecuzione di un passo, l'utente richiede l'invio dei dati al \textit{server\ped{G}}, e questi dati non soddisfano i vincoli imposti dal creatore del processo.
Per risolvere il problema è sufficiente reinserire il dato segnalato dal messaggio errore, ponendo attenzione ai vincoli segnalati come non rispettati.