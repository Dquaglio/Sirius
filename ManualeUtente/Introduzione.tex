\section{Introduzione}
\label{introduzione}
\subsection{Guida alla lettura del manuale}
Il presente documento ha lo scopo di descrivere le principali funzionalità e il corretto utilizzo del prodotto \textit{software} \progetto{}, ed é suddiviso nelle seguenti sezioni:

\begin{itemize}
\item \hyperref[introduzione]{\textbf{Introduzione:}} descrive lo scopo del documento, come riportare malfunzionamenti, e i requisiti si sistema;
\item \hyperref[istruzioni]{\textbf{Istruzioni per l'uso:}} contiene le istruzioni per l'uso, e illustra e descrive le funzionalità principali del prodotto;
\item \hyperref[errori]{\textbf{Appendice A:}} contiene la descrizione dei possibili messaggi d'errore generati dall'applicazione;
\item \hyperref[glossario]{\textbf{Appendice B:}} contiene la definizione dei termini tecnici, degli acronimi e delle parole che necessitano di essere chiarite, identificabili nel documento dal simbolo ``G'' in pedice.
\end{itemize}


\subsection{Scopo del Prodotto}
Lo scopo del progetto \progetto{}, è di fornire un servizio di gestione di processi definiti da una serie di passi da eseguirsi in sequenza o senza un ordine predefinito, utilizzabile da dispositivi mobili di tipo \textit{smaptphone} o \textit{tablet}.

\subsection{Definizione dell'utente finale}
Il presente documento è destinato all'uso degli utenti base, che potranno iscriversi ai processi resi disponibili dal prodotto \progetto{}, e completarli eseguendone i passi.

\subsection{Come riportare problemi e malfunzionamenti}

In caso di malfunzionamenti inaspettati del prodotto \progetto{}, si prega di inviare una \textit{email} all'indirizzo
\href{mailto:swesirius@gmail.com}{\nolinkurl{swesirius@gmail.com}} indicando possibilmente:

\begin{itemize}
\item \textbf{Browser:} Tipo e versione del \textit{browser\ped{G}} utilizzato;
\item \textbf{Descrizione:} Descrizione dell'azione che si stava eseguendo e dell'errore che si è verificato;
\item \textbf{Errore:} Descrizione degli eventuali messaggi d'errore ricevuti.
\end{itemize}

In \hyperref[errori]{appendice A} viene riportata la descrizione dei possibili messaggi d'errore generati dall'applicazione.

\subsection{Requisiti di sistema}
Per poter usufruire del prodotto \progetto{}, è necessario utilizzare uno tra i \textit{browser\ped{G}} in elenco:
\begin{itemize}
\item \textit{Google Chrome\ped{G}} versione 27 e successive;
\item \textit{Mozilla Firefox\ped{G}} versione 18 e successive;
\item \textit{Internet Explorer\ped{G}} versione 10 e successive;
\item \textit{Opera\ped{G}} versione 12.1 e successive;
\item \textit{Safari\ped{G}} versione 6.1 e successive;
\item \textit{Mobile android\ped{G}} versione 2.3 e successive;
\item \textit{Google Chrome per Android\ped{G}} versione 35 e successive;
\item \textit{Firefox per Android\ped{G}} versione 30 e successive;
\item \textit{mobile Internet Explorer\ped{G}} per \textit{Windows Phone 8G\ped{G}} versione 10 e successive;
\item \textit{Google Chrome per iOS\ped{G}} versione 28 e successive;
\item \textit{mobile Safari\ped{G}} su dispositivi \textit{Apple\ped{G}} che hanno installato \textit{iOS G 3\ped{G}} o superiori;
\end{itemize}

\section{Istruzioni per l'uso}
\label{istruzioni}

\appendix
\section{Descrizione messaggi di errore}
\label{errori}

\section{Glossario}
\label{glossario}