\section{Introduzione}
\label{introduzione}
\subsection{Guida alla lettura del manuale}
Il presente documento ha lo scopo di descrivere le principali funzionalità e il corretto utilizzo del prodotto \textit{software\ped{G}} \progetto{} agli utenti base, ed é suddiviso nelle seguenti sezioni:

\begin{itemize}
\item \hyperref[introduzione]{\textbf{Introduzione:}} descrive lo scopo del documento, come riportare malfunzionamenti, e i requisiti si sistema;
\item \hyperref[istruzioni]{\textbf{Istruzioni per l'uso:}} contiene le istruzioni per l'uso, e illustra e descrive le funzionalità principali del prodotto;
\item \hyperref[errori]{\textbf{Appendice A:}} contiene la descrizione dei possibili messaggi d'errore generati dall'applicazione;
\item \hyperref[glossario]{\textbf{Appendice B:}} contiene la definizione dei termini tecnici, degli acronimi e delle parole che necessitano di essere chiarite, identificabili nel documento dal simbolo ``G'' in pedice.
\end{itemize}


\subsection{Scopo del Prodotto}
Lo scopo del progetto \progetto{}, è di fornire un servizio di gestione di processi definiti da una serie di passi da eseguirsi in sequenza o senza un ordine predefinito, utilizzabile da dispositivi mobili di tipo \textit{smaptphone} o \textit{tablet}.

\subsection{Definizione dell'utente finale}
Il presente documento è destinato all'uso dell'utente base.\\
Tale tipologia d'utenza può iscriversi ai processi disponibili e completarli eseguendone i passi\ped{G}.

\subsection{Come riportare problemi e malfunzionamenti}

In caso di malfunzionamenti inaspettati del prodotto \progetto{}, si prega di inviare una \textit{email} all'indirizzo
\href{mailto:swesirius@gmail.com}{\nolinkurl{swesirius@gmail.com}} indicando possibilmente:

\begin{itemize}
\item \textbf{Browser:} Tipo e versione del \textit{browser\ped{G}} utilizzato;
\item \textbf{Descrizione:} Descrizione dell'azione che si stava eseguendo e dell'errore che si è verificato;
\item \textbf{Errore:} Descrizione degli eventuali messaggi d'errore ricevuti.
\end{itemize}

In \hyperref[errori]{appendice A} viene riportata la descrizione dei possibili messaggi d'errore generati dall'applicazione.

\subsection{Requisiti di sistema}
Per poter usufruire del prodotto \progetto{}, è necessario assicurarsi di disporre uno tra i \textit{browser\ped{G}} riportati nella seguente tabella:

\LTXtable{0.9\textwidth}{Browser.tex}