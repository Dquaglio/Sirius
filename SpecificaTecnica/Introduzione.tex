\section{Introduzione}
\subsection{Scopo del Documento}
Lo scopo di questo documento è la definizione delle specifiche progettuali del prodotto \textit{software} \progetto{}.\\
Viene quindi presentata l'architettura ad alto livello del sistema, e la descrizione delle singole componenti e dei \textit{design pattern\ped{G}} utilizzati.
\subsection{Scopo del Prodotto}
Lo scopo del progetto \progetto{}, è di fornire un servizio di gestione di processi definiti da una serie di passi da eseguirsi in sequenza o senza un ordine predefinito, utilizzabile da dispositivi mobili di tipo \textit{smaptphone} o \textit{tablet}.
\subsection{Glossario}
Al fine di rendere più leggibili e comprensibili i documenti, i termini tecnici, di dominio, gli acronimi e le parole che necessitano di essere chiarite, sono riportate nel documento \Glossario{}.\\
Ciascuna occorrenza dei vocaboli presenti nel \textit{Glossario} è seguita da una ``G'' maiuscola in pedice.
\subsection{Riferimenti}
\subsubsection{Normativi}
\begin{itemize}
\item Norme di Progetto: \NormeDiProgetto{}.
\item Analisi dei Requisiti: \AnalisiDeiRequisiti{}.
\end{itemize}
\subsubsection{Informativi}
\begin{itemize}
\item Design Patterns: Elementi per il riuso di software ad oggetti - Erich Gamma,
Richard Helm, Ralph Johnson e John Vlissides (2002);

\item Learning JavaScript Design Patterns, Addy Osmani, Volume 1.5.2:\\
\url{http://addyosmani.com/resources/essentialjsdesignpatterns/book};

\item Regolamento dei documenti, prof. Vardanega Tullio:\\
\url{http://www.math.unipd.it/~tullio/IS-1/2013/};

\item Dispense di ingegneria del software modulo A:
\begin{itemize}
\item Progettazione software, prof. Vardanega Tullio:\\
\url{http://www.math.unipd.it/~tullio/IS-1/2013/Dispense/P09.pdf};

\item Diagrammi delle classi e degli oggetti, prof. Cardin Riccardo:\\
\url{http://www.math.unipd.it/~tullio/IS-1/2013/Dispense/E02a.pdf};

\item Diagrammi di sequenza, prof. Cardin Riccardo:\\
\url{http://www.math.unipd.it/~tullio/IS-1/2013/Dispense/E03a.pdf};

\item Diagrammi di attività, prof. Cardin Riccardo:\\
\url{http://www.math.unipd.it/~tullio/IS-1/2013/Dispense/E03b.pdf};

\item Introduzione ai design pattern, prof. Cardin Riccardo:\\
\url{http://www.math.unipd.it/~tullio/IS-1/2013/Dispense/E04.pdf};

\item Diagrammi dei package, prof. Cardin Riccardo:\\
\url{http://www.math.unipd.it/~tullio/IS-1/2013/Dispense/E05.pdf};
\end{itemize}


\item Dispense di ingegneria del software modulo B:
\begin{itemize}
\item Design pattern: Model-View-Controller, prof. Cardin Riccardo:\\
\url{http://www.math.unipd.it/~rcardin/pdf/Design%20Pattern%20-%20Model%20View%20Controller_4x4.pdf};

\item Design pattern strutturali, prof. Cardin Riccardo:\\
\url{http://www.math.unipd.it/~rcardin/pdf/Design%20Pattern%20Strutturali_4x4.pdf};

\item Design pattern creazionali, prof. Cardin Riccardo:\\
\url{http://www.math.unipd.it/~rcardin/pdf/Design%20Pattern%20Creazionali_4x4.pdf};

\item Design pattern comportamentali, prof. Cardin Riccardo:\\
\url{http://www.math.unipd.it/~rcardin/pdf/Design%20Pattern%20Comportamentali_4x4.pdf};

\item Esercizi sugli errori rilevati in RP, prof. Cardin Riccardo:\\
\url{http://www.math.unipd.it/~rcardin/pdf/Esercitazione%20-%20Errori%20comuni%20RP_4x4.pdf};
\end{itemize}

\end{itemize}