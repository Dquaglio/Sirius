\section{Descrizione singoli componenti}

\begin{figure}[H] \centering \includegraphics[width=%
\textwidth]
{./pack/POPackage.png} \caption{Diagramma componenti - \textit{process owner}}
\end{figure}

\begin{figure}[H] \centering \includegraphics[width=%
\textwidth]
{./pack/UserMainPackage.png} \caption{Diagramma componenti - \textit{utente}}
\end{figure}


\subsection{Package \view{}}

\begin{figure}[H]
\centering
\includegraphics[trim=0cm 0.8cm 0cm 0cm,clip=true,scale=0.75]%
{./pack/UserViewA.png} \caption{Diagramma principale \textit{view} utente}
\end{figure}

\paragraph{ILogin}
\begin{itemize}
\item \textbf{Nome:} \texttt{ILogin};
\item \textbf{Package:} \texttt{\view{}};
\item \textbf{Descrizione:} Interfaccia che permette di gestire l'interfaccia grafica relativa alle richieste di autenticazione al sistema.
\end{itemize}

\paragraph{Login}
\begin{flushleft}
\begin{itemize}
\item \textbf{Nome:} \texttt{Login};
\item \textbf{Package:} \texttt{\view{}};
\item \textbf{Descrizione:} Componente che permette di gestire l'interfaccia grafica relativa alle richieste di autenticazione al sistema;
\item \textbf{Relazioni con altri componenti:}
\begin{sloppypar}
Il componente implementa l'interfaccia \texttt{\view{}.I\fshyp{}Lo\fshyp{}gin}.
\end{sloppypar}
\end{itemize}
\end{flushleft}

\subsubsection{Package sequenziatore::client::view::user}

\paragraph{MainUser}
\begin{flushleft}
\begin{itemize}
\item \textbf{Nome:} \texttt{MainUser};
\item \textbf{Package:} \texttt{\viewAdmin{}};
\item \textbf{Descrizione:} Classe che permette la gestione delle principali componenti dell'interfaccia grafica dell'utente;
\item \textbf{Relazioni con altri componenti:}
\begin{sloppypar}
La classe implementa l'interfaccia \texttt{\iViewUser{}::I\fshyp{}Main\fshyp{}U\fshyp{}ser} e comunica con il \textit{presenter} utilizzando metodi della classe \texttt{\logicUser{}::Main\fshyp{}Lo\fshyp{}gic}.
\end{sloppypar}
\end{itemize}
\end{flushleft}

\paragraph{UpdateView}
\begin{flushleft}
\begin{itemize}
\item \textbf{Nome:} \texttt{UpdateView};
\item \textbf{Package:} \texttt{\viewAdmin{}};
\item \textbf{Descrizione:} Classe che permette di gestire l’aggiornameno dei \textit{widget\ped{G}} della componente \textit{view};
\item \textbf{Relazioni con altri componenti:}
\begin{sloppypar}
La classe implementa l'interfaccia \texttt{\iViewUser{}::I\fshyp{}Up\fshyp{}da\fshyp{}te\fshyp{}View} e aggiorna le componenti della \textit{view} comunicando con le seguenti classi:
\begin{itemize}
\item \texttt{\viewAdmin{}::MainUser};
\item \texttt{\viewAdmin{}::Login};
\item \texttt{\viewAdmin{}::Register};
\item \texttt{\viewAdmin{}::ViewData};
\item \texttt{\viewAdmin{}::EditData};
\item \texttt{\viewAdmin{}::ChangePassword};
\item \texttt{\viewAdmin{}::OpenProcess};
\item \texttt{\viewAdmin{}::ManagementSelectedProcess};
\item \texttt{\viewAdmin{}::SendData};
\item \texttt{\viewAdmin{}::SendText};
\item \texttt{\viewAdmin{}::SendNumb};
\item \texttt{\viewAdmin{}::SendPosition};
\item \texttt{\viewAdmin{}::SendImage};
\item \texttt{\viewAdmin{}::SendPhoto};
\item \texttt{\viewAdmin{}::EndSelectedProcess};
\item \texttt{\viewAdmin{}::PrintProcess};
\item \texttt{\viewAdmin{}::PreviewProcess}.
\end{itemize}
\end{sloppypar}
\end{itemize}
\end{flushleft}

\paragraph{Login}
\begin{flushleft}
\begin{itemize}
\item \textbf{Nome:} \texttt{Login};
\item \textbf{Package:} \texttt{\viewAdmin{}};
\item \textbf{Descrizione:} Classe che permette di gestire dell'interfaccia grafica relativa alle richieste di autenticazione e chiusura della sessione da parte dell'utente;
\item \textbf{Relazioni con altri componenti:}
\begin{sloppypar}
La classe implementa l'interfaccia \texttt{\iViewUser{}::I\fshyp{}Lo\fshyp{}gin} e comunica con il \textit{presenter} utilizzando metodi della classe \texttt{\logicUser{}::Main\fshyp{}Lo\fshyp{}gic}.
\end{sloppypar}
\end{itemize}
\end{flushleft}

\paragraph{Register}
\begin{flushleft}
\begin{itemize}
\item \textbf{Nome:} \texttt{Register};
\item \textbf{Package:} \texttt{\viewAdmin{}};
\item \textbf{Descrizione:} Classe che permette di gestire dell'interfaccia grafica relativa alle richieste di registrazione da parte dell'utente;
\item \textbf{Relazioni con altri componenti:}
\begin{sloppypar}
La classe implementa l'interfaccia \texttt{\iViewUser{}::I\fshyp{}Re\fshyp{}gis\fshyp{}ter} e comunica con il \textit{presenter} utilizzando metodi della classe \texttt{\logicUser{}::Main\fshyp{}Lo\fshyp{}gic}.
\end{sloppypar}
\end{itemize}
\end{flushleft}

\paragraph{ViewData}
\begin{flushleft}
\begin{itemize}
\item \textbf{Nome:} \texttt{ViewData};
\item \textbf{Package:} \texttt{\viewAdmin{}};
\item \textbf{Descrizione:} Classe che permette la realizzazione dei \textit{widget} che consentono visualizzazione dei dati dell'utente;
\item \textbf{Relazioni con altri componenti:}
\begin{sloppypar}
La classe implementa l'interfaccia \texttt{\iViewUser{}::I\fshyp{}View\fshyp{}Da\fshyp{}ta} e comunica con il \textit{presenter} utilizzando metodi della classe \texttt{\logicUser{}::Main\fshyp{}Lo\fshyp{}gic}.
\end{sloppypar}
\end{itemize}
\end{flushleft}

\paragraph{EditData}
\begin{flushleft}
\begin{itemize}
\item \textbf{Nome:} \texttt{EditData};
\item \textbf{Package:} \texttt{\viewAdmin{}};
\item \textbf{Descrizione:} Classe che permette la realizzazione dei \textit{widget} che consentono la modifica dei dati personali dell'utente;
\item \textbf{Relazioni con altri componenti:}
\begin{sloppypar}
La classe implementa l'interfaccia \texttt{\iViewUser{}::I\fshyp{}Edit\fshyp{}Da\fshyp{}ta} e comunica con il \textit{presenter} utilizzando metodi della classe \texttt{\logicUser{}::Main\fshyp{}Lo\fshyp{}gic}.
\end{sloppypar}
\end{itemize}
\end{flushleft}

\paragraph{ChangePassword}
\begin{flushleft}
\begin{itemize}
\item \textbf{Nome:} \texttt{ChangePassword};
\item \textbf{Package:} \texttt{\viewAdmin{}};
\item \textbf{Descrizione:} Classe che permette la realizzazione dei \textit{widget} che consentono la modifica della \textit{password} dell'utente;
\item \textbf{Relazioni con altri componenti:}
\begin{sloppypar}
La classe implementa l'interfaccia \texttt{\iViewUser{}::I\fshyp{}Change\fshyp{}Pass\fshyp{}word} e comunica con il \textit{presenter} utilizzando metodi della classe \texttt{\logicUser{}::Main\fshyp{}Lo\fshyp{}gic}.
\end{sloppypar}
\end{itemize}
\end{flushleft}

\paragraph{OpenProcess}
\begin{flushleft}
\begin{itemize}
\item \textbf{Nome:} \texttt{OpenProcess};
\item \textbf{Package:} \texttt{\viewAdmin{}};
\item \textbf{Descrizione:} Classe che permette di realizzare i \textit{widget} per consentire l'apertura di un processo tramite ricerca o selezionandolo da una lista;
\item \textbf{Relazioni con altri componenti:}
\begin{sloppypar}
La classe implementa l'interfaccia \texttt{\iViewUser{}::I\fshyp{}O\fshyp{}pen\fshyp{}Pro\fshyp{}cess} e comunica con il \textit{presenter} utilizzando metodi della classe \texttt{\logicUser{}::Main\fshyp{}Lo\fshyp{}gic}.
\end{sloppypar}
\end{itemize}
\end{flushleft}

\paragraph{ManagementSelectedProcess}
\begin{flushleft}
\begin{itemize}
\item \textbf{Nome:} \texttt{ManagementSelectedProcess};
\item \textbf{Package:} \texttt{\viewAdmin{}};
\item \textbf{Descrizione:} Classe che permette di realizzare i \textit{widget} per consentire la visualizzazione dello stato del processo selezionato e i vincoli per concludere il passo in corso;
\item \textbf{Relazioni con altri componenti:}
\begin{sloppypar}
La classe implementa l'interfaccia \texttt{\iViewUser{}::I\fshyp{}Ma\fshyp{}na\fshyp{}ge\fshyp{}ment\fshyp{}Se\fshyp{}lec\fshyp{}ted\fshyp{}Pro\fshyp{}cess} e comunica con il \textit{presenter} utilizzando metodi della classe \texttt{\logicUser{}::Main\fshyp{}Lo\fshyp{}gic}.
\end{sloppypar}
\end{itemize}
\end{flushleft}

\paragraph{SendData}
\begin{flushleft}
\begin{itemize}
\item \textbf{Nome:} \texttt{SendData};
\item \textbf{Package:} \texttt{\viewAdmin{}};
\item \textbf{Descrizione:} Classe che permette di realizzare i \textit{widget} per consentire l'invio dei dati richiesti per la conclusione del passo in esecuzione;
\item \textbf{Relazioni con altri componenti:}
\begin{sloppypar}
La classe implementa l'interfaccia \texttt{\iViewUser{}::I\fshyp{}Send\fshyp{}Da\fshyp{}ta} e comunica con il \textit{presenter} utilizzando metodi della classe \texttt{\logicUser{}::Main\fshyp{}Lo\fshyp{}gic}.
\end{sloppypar}
\end{itemize}
\end{flushleft}

\paragraph{SendText}
\begin{flushleft}
\begin{itemize}
\item \textbf{Nome:} \texttt{SendData};
\item \textbf{Package:} \texttt{\viewAdmin{}};
\item \textbf{Descrizione:} Classe che permette di realizzare i \textit{widget} che consentono di inserire il testo da inviare per concludere il passo in esecuzione;
\item \textbf{Relazioni con altri componenti:}
\begin{sloppypar}
La classe implementa l'interfaccia \texttt{\iViewUser{}::I\fshyp{}Send\fshyp{}Text} e comunica con il \textit{presenter} utilizzando metodi della classe \texttt{\logicUser{}::Main\fshyp{}Lo\fshyp{}gic}.
\end{sloppypar}
\end{itemize}
\end{flushleft}

\paragraph{SendNumb}
\begin{flushleft}
\begin{itemize}
\item \textbf{Nome:} \texttt{SendNumb};
\item \textbf{Package:} \texttt{\viewAdmin{}};
\item \textbf{Descrizione:} Classe che permette agli oggetti che la implementano di realizzare i \textit{widget} che consentono di inserire i dati numerici da inviare per concludere il passo in esecuzione;
\item \textbf{Relazioni con altri componenti:}
\begin{sloppypar}
La classe implementa l'interfaccia \texttt{\iViewUser{}::I\fshyp{}Send\fshyp{}Numb} e comunica con il \textit{presenter} utilizzando metodi della classe \texttt{\logicUser{}::Main\fshyp{}Lo\fshyp{}gic}.
\end{sloppypar}
\end{itemize}
\end{flushleft}

\paragraph{SendPosition}
\begin{flushleft}
\begin{itemize}
\item \textbf{Nome:} \texttt{SendPosition};
\item \textbf{Package:} \texttt{\viewAdmin{}};
\item \textbf{Descrizione:} Classe che permette  di realizzare i \textit{widget} che consentono di inviare la posizione geografica richiesta per la conclusione del passo in esecuzione;
\item \textbf{Relazioni con altri componenti:}
\begin{sloppypar}
La classe implementa l'interfaccia \texttt{\iViewUser{}::I\fshyp{}Send\fshyp{}Po\fshyp{}si\fshyp{}tion} e comunica con il \textit{presenter} utilizzando metodi della classe \texttt{\logicUser{}::Main\fshyp{}Lo\fshyp{}gic}.
\end{sloppypar}
\end{itemize}
\end{flushleft}

\paragraph{SendImage}
\begin{flushleft}
\begin{itemize}
\item \textbf{Nome:} \texttt{SendImage};
\item \textbf{Package:} \texttt{\viewAdmin{}};
\item \textbf{Descrizione:} Classe che permette di realizzare i \textit{widget} che consentono di inserire le immagini richieste per concludere i passo in esecuzione;
\item \textbf{Relazioni con altri componenti:}
\begin{sloppypar}
La classe implementa l'interfaccia \texttt{\iViewUser{}::I\fshyp{}Send\fshyp{}Image} e comunica con il \textit{presenter} utilizzando metodi della classe \texttt{\logicUser{}::Main\fshyp{}Lo\fshyp{}gic}.
\end{sloppypar}
\end{itemize}
\end{flushleft}

\paragraph{SendPhoto}
\begin{flushleft}
\begin{itemize}
\item \textbf{Nome:} \texttt{SendPhoto};
\item \textbf{Package:} \texttt{\viewAdmin{}};
\item \textbf{Descrizione:} Classe che permette di realizzare i \textit{widget} che consentono di scattare le fotografie richieste dal passo in esecuzione e di inviarle;
\item \textbf{Relazioni con altri componenti:}
\begin{sloppypar}
La classe implementa l'interfaccia \texttt{\iViewUser{}::I\fshyp{}Send\fshyp{}Photo} e comunica con il \textit{presenter} utilizzando metodi della classe \texttt{\logicUser{}::Main\fshyp{}Lo\fshyp{}gic}.
\end{sloppypar}
\end{itemize}
\end{flushleft}

\paragraph{EndSelectedProcess}
\begin{flushleft}
\begin{itemize}
\item \textbf{Nome:} \texttt{EndSelectedProcess};
\item \textbf{Package:} \texttt{\viewAdmin{}};
\item \textbf{Descrizione:} Classe che permette di realizzare i \textit{widget} che consentono di visualizzare l'esito del processo e di effettuare le operazioni di conclusione del processo;
\item \textbf{Relazioni con altri componenti:}
\begin{sloppypar}
La classe implementa l'interfaccia \texttt{\iViewUser{}::I\fshyp{}End\fshyp{}Se\fshyp{}lec\fshyp{}ted\fshyp{}Pro\fshyp{}cess} e comunica con il \textit{presenter} utilizzando metodi della classe \texttt{\logicUser{}::Main\fshyp{}Lo\fshyp{}gic}.
\end{sloppypar}
\end{itemize}
\end{flushleft}

\paragraph{PrintProcess}
\begin{flushleft}
\begin{itemize}
\item \textbf{Nome:} \texttt{PrintProcess};
\item \textbf{Package:} \texttt{\viewAdmin{}};
\item \textbf{Descrizione:} Classe che permette di realizzare i \textit{widget} che consentono il salvataggio dei \textit{report} sull'esecuzione del processo;
\item \textbf{Relazioni con altri componenti:}
\begin{sloppypar}
La classe implementa l'interfaccia \texttt{\iViewUser{}::I\fshyp{}Print\fshyp{}Pro\fshyp{}cess} e comunica con il \textit{presenter} utilizzando metodi della classe \texttt{\logicUser{}::Main\fshyp{}Lo\fshyp{}gic}.
\end{sloppypar}
\end{itemize}
\end{flushleft}

\paragraph{PreviewProcess}
\begin{flushleft}
\begin{itemize}
\item \textbf{Nome:} \texttt{PreviewProcess};
\item \textbf{Package:} \texttt{\viewAdmin{}};
\item \textbf{Descrizione:} Classe che permette agli oggetti che la implementano di realizzare i \textit{widget} per consentire la visualizzazione dei \textit{report} sull'esecuzione del processo;
\item \textbf{Relazioni con altri componenti:}
\begin{sloppypar}
La classe implementa l'interfaccia \texttt{\iViewUser{}::I\fshyp{}Pre\fshyp{}view\fshyp{}Pro\fshyp{}cess} e comunica con il \textit{presenter} utilizzando metodi della classe \texttt{\logicUser{}::Main\fshyp{}Lo\fshyp{}gic}.
\end{sloppypar}
\end{itemize}
\end{flushleft}
\subsubsection{Package sequenziatore.client.view}
\paragraph{Package sequenziatore.client.view.admin}
Interfacce \\
\begin{description}
	\item[imainAdmin] 
  	\hfill \\
  	Nome: imainAdmin\\
  	Tipo: interface\\
	Package: sequenziatore.client.view.admin\\
	Descrizione: interfaccia che permette la gestione delle varie componenti 	dell'interfaccia grafica dell'utente amministratore
\end{description}

\begin{description}
	\item[ilogin] 
  	\hfill \\
  	Nome: ilogin\\
  	Tipo: interface\\
	Package: sequenziatore.client.view.admin\\
	Descrizione: interfaccia che permette di richiedere l'accesso oppure terminazione al sistema Sequenziatore da parte dell’utente amministratore
\end{description}

\begin{description}
	\item[inewProcess] 
  	\hfill \\
  	Nome: inewProcess\\
  	Tipo: interface\\
	Package: sequenziatore.client.view.admin\\
	Descrizione: interfaccia che permette la realizzazione su richiesta del presenter del widget\ped{G} per la creazione di nuovi processi
\end{description}

\begin{description}
	\item[isetProcess] 
  	\hfill \\
  	Nome: isetProcess\\
  	Tipo: interface\\
	Package: sequenziatore.client.view.admin\\
	Descrizione: interfaccia che permette la realizzazione su richiesta del presenter del widget per la definizione del nuovo Processcesso da creare
\end{description}

\begin{description}
	\item[iaddStep] 
  	\hfill \\
  	Nome: iaddStep\\
  	Tipo: interface\\
	Package: sequenziatore.client.view.admin\\
	Descrizione: interfaccia che permette la realizzazione su richiesta del presenter del widget per l'aggiunta di passi per il nuovo Processcesso da creare
\end{description}

\begin{description}
	\item[ipreviewProcess] 
  	\hfill \\
  	Nome: ipreviewProcess\\
  	Tipo: interface\\
	Package: sequenziatore.client.view.admin\\
	Descrizione: interfaccia che permette agli oggetti che la implementano di realizzare il widget per la visualizzazione dell'anteprima del Processcesso in creazione
\end{description}

\begin{description}
	\item[iopenProcess] 
  	\hfill \\
  	Nome: iopenProcess\\
  	Tipo: interface\\
	Package: sequenziatore.client.view.admin\\
	Descrizione: interfaccia che permette agli oggetti che la implementano di realizzare il widget, su richiesta del presenter, per la visualizzazione dei Processcessi creati, per la loro selezione e ricerca per nome
\end{description}

\begin{description}
	\item[imanagmentselectedProcess] 
  	\hfill \\
  	Nome: imanagmentselectedProcess\\
  	Tipo: interface\\
	Package: sequenziatore.client.view.admin\\
	Descrizione: interfaccia realizzare il widget per la gestione del Processcesso selezionato nel widget che implementa iopenProcess
\end{description}

\begin{description}
	\item[icheckStep] 
  	\hfill \\
  	Nome: icheckStep\\
  	Tipo: interface\\
	Package: sequenziatore.client.view.admin\\
	Descrizione: interfaccia che permette agli oggetti che la implementano di realizzare il widget per la gestione delle conferme da dare ai passi cui richiedono la conferma da parte dell'utente amministratore
\end{description}

\begin{description}
	\item[istatitics] 
  	\hfill \\
  	Nome: istatitics\\
  	Tipo: interface\\
	Package: sequenziatore.client.view.admin\\
	Descrizione: interfaccia che permette di realizzare il widget per la visualizzazione delle statistiche del Processcesso selezionato dall'utente amministratore nel widget di selezione Processcesso
\end{description}
\begin{description}
	\item[iinviteUser] 
  	\hfill \\
  	Nome: iinviteUser\\
  	Tipo: interface\\
	Package: sequenziatore.client.view.admin\\
	Descrizione: interfaccia che realizza il widget per invitare utenti al Processcesso selezionato dall'utente amministratore
\end{description}
