\subsection{Package sequenziatore::server::presenter}
\subsubsection{Package sequenziatore::server::presenter::icommunication}
\paragraph{IHttpCommunication}
	\begin{itemize}
		\item \textbf{Nome:} IHttpCommunication;
		\item \textbf{Tipo:} Interface;
		\item \textbf{Package:} sequenziatore::server::presenter::icommunication
		\item \textbf{Descrizione:} interfaccia che gestisce le comunicazioni con il presenter lato \textit{client} tramite richieste http;
	\end{itemize}
%----------------------------------------------------------------------------------------------%
\paragraph{IWebsocketCommunication}
	\begin{itemize}
		\item \textbf{Nome:} IWebsocketCommunication;
		\item \textbf{Tipo:} Interface;
		\item \textbf{Package:} sequenziatore::server::presenter::icommunication
		\item \textbf{Descrizione:} interfaccia che gestisce le comunicazioni con il presenter lato \textit{client} tramite richieste http;
	\end{itemize}
%----------------------------------------------------------------------------------------------%
\paragraph{IDataFormatter}
	\begin{itemize}
		\item \textbf{Nome:} IDataFormatter;
		\item \textbf{Tipo:} Interface;
		\item \textbf{Package:} sequenziatore::server::presenter::icommunication
		\item \textbf{Descrizione:} interfaccia che gestisce le comunicazioni con il presenter lato \textit{client} tramite richieste http;
	\end{itemize}

%----------------------------------------------------------------------------------------------%
\paragraph{IChooser}
	\begin{itemize}
		\item \textbf{Nome:} IChooser;
		\item \textbf{Tipo:} Interface;
		\item \textbf{Package:} sequenziatore::server::presenter::icommunication
		\item \textbf{Descrizione:} interfaccia che gestisce le comunicazioni con il presenter lato \textit{client} tramite richieste http;
	\end{itemize}
%0000000000000000000000000000000000000000000000000000000000000000000000000000000000000000000000000%
\subsubsection{Package sequenziatore::server::presenter::communication}
\paragraph{HttpCommunication}
	\begin{itemize}
		\item \textbf{Nome:} HttpCommunication;
		\item \textbf{Tipo:} Class;
		\item \textbf{Package:} sequenziatore::server::presenter::communication
		\item \textbf{Descrizione:} classe responsabile della gestione delle comunicazioni con il presenter lato \textit{client} tramite richieste http;
		\item \textbf{Relazione con altre componenti:}
	\end{itemize}
%----------------------------------------------------------------------------------------------%
\paragraph{WebsocketCommunication}
	\begin{itemize}
		\item \textbf{Nome:} WebsocketCommunication;
		\item \textbf{Tipo:} Class;
		\item \textbf{Package:} sequenziatore::server::presenter::communication
		\item \textbf{Descrizione:} classe responsabile della gestione delle comunicazioni con il presenter lato \textit{client} tramite WebSocket;
		\item \textbf{Relazione con altre componenti:}
	\end{itemize}
%----------------------------------------------------------------------------------------------%
\paragraph{XXXX}
	\begin{itemize}
		\item \textbf{Nome:} XXX;
		\item \textbf{Tipo:} Class;
		\item \textbf{Package:} sequenziatore::server::presenter::communication
		\item \textbf{Descrizione:} XXX;
		\item \textbf{Relazione con altre componenti:}
	\end{itemize}

%----------------------------------------------------------------------------------------------%
\paragraph{Chooser}
	\begin{itemize}
		\item \textbf{Nome:} Chooser;
		\item \textbf{Tipo:} Class;
		\item \textbf{Package:} sequenziatore::server::presenter::communication
		\item \textbf{Descrizione:} classe che decide in base alla richiesta ricevuta, se tale richiesta è stata ricevuta da un \textit{ProcessOwner} o da un \textit{User} e a chi assegnarne l' elaborazione di conseguenza;
		\item \textbf{Relazione con altre componenti:}
	\end{itemize}