\subsubsection{Package sequenziatore::server::presenter::processowner}
\paragraph{LoginController}
	\begin{itemize}
		\item \textbf{Nome:} \texttt{LoginController};
		\item \textbf{Package:} sequenziatore::server::presenter::processowner
		\item \textbf{Descrizione:} Classe che permette la gestione della login del \textit{process owner};
		\item \textbf{Relazione con altre componenti:} la classe invoca i metodi delle classi:
		\begin{itemize}
			\item BOTTER;
		\end{itemize}
	\end{itemize}

\paragraph{ProcessController}
	\begin{itemize}
		\item \textbf{Nome:} \texttt{ProcessController};
		\item \textbf{Package:} sequenziatore::server::presenter::processowner
		\item \textbf{Descrizione:} Classe che riceve le richieste da parte del \textit{process owner} per la gestione dei processi come ad esempio la creazione, la modifica e la eliminazione degli stessi;
		\item \textbf{Relazione con altre componenti:} la classe invoca i metodi delle classi:
		\begin{itemize}
			\item BOTTER;
		\end{itemize}
	\end{itemize}
%----------------------------------------------------------------------------------------------%
\paragraph{StepController}
	\begin{itemize}
		\item \textbf{Nome:} \texttt{StepController};
		\item \textbf{Package:} sequenziatore::server::presenter::processowner
		\item \textbf{Descrizione:} Classe che permette l' elaborazione delle richieste del \textit{process owner} per quanto concerne la creazione,la rimozione e la modifica di singoli passi, per esempio permetterà di aggiungere o rimuovere dei campi richiesti;
		\item \textbf{Relazione con altre componenti:} la classe invoca i metodi delle classi:
		\begin{itemize}
			\item BOTTER;
		\end{itemize}
	\end{itemize}
	%----------------------------------------------------------------------------------------------%
\paragraph{UserController}
	\begin{itemize}
		\item \textbf{Nome:} \texttt{UserController};
		\item \textbf{Package:} sequenziatore::server::presenter::processowner
		\item \textbf{Descrizione:} Classe che permette la gestione degli utenti iscritti alla piattaforma, permettendogli di rimuovere utenti da processi o fornire le informazioni di un singolo utente quando richiesto;
		\item \textbf{Relazione con altre componenti:} la classe invoca i metodi delle classi:
		\begin{itemize}
			\item sequenziatore::server::model::DaoFacade;
		\end{itemize}
	\end{itemize}

%----------------------------------------------------------------------------------------------%
\paragraph{ReportController}
	\begin{itemize}
		\item \textbf{Nome:} \texttt{ReportController};
		\item \textbf{Package:} sequenziatore::server::presenter::processowner
		\item \textbf{Descrizione:} Classe che permette la gestione delle richieste dei report al \textit{process owner}, permettendogli di visualizzare i risultati raggiunti in un processo;
		\item \textbf{Relazione con altre componenti:} la classe invoca i metodi delle classi:
		\begin{itemize}
			\item sequenziatore::server::model::DaoFacade;
		\end{itemize}
	\end{itemize}
