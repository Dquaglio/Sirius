\subsubsection{Package sequenziatore::server::presenter::iprocessowner}
\paragraph{ProcessOwnerFacade}
	\begin{itemize}
		\item \textbf{Nome:} ProcessOwnerFacade;
		\item \textbf{Tipo:} abstract;
		\item \textbf{Package:} sequenziatore::server::presenter::iprocessowner
		\item \textbf{Descrizione:} classe che decide in base alla richiesta ricevuta a che classe assegnarne l' elaborazione di tale richiesta all' interno del package processowner
		\item \textbf{Relazione con altre componenti:} la classe richiama metodi delle classi:
		\begin{itemize}
			\item sequenziatore::server::presenter::processowner::Login tramite l' interfaccia sequenziatore::server::presenter::processowner::ILogin;
			\item sequenziatore::server::presenter::processowner::ProcessManager tramite l' interfaccia sequenziatore::server::presenter::processowner::IProcessManager;
			\item sequenziatore::server::presenter::processowner::StepManager tramite l' interfaccia sequenziatore::server::presenter::processowner::IStepManager;
			\item sequenziatore::server::presenter::processowner::UserManager tramite l' interfaccia sequenziatore::server::presenter::processowner::IUserManager;
			\item sequenziatore::server::presenter::processowner::Report tramite l' interfaccia
sequenziatore::server::presenter::processowner::IReport;
		\end{itemize}
	\end{itemize}
%-----------------------------------------------------------------------------------------------%
\paragraph{IProcessManager}
	\begin{itemize}
		\item \textbf{Nome:} \texttt{IProcessManager};
		\item \textbf{Package:} sequenziatore::server::presenter::iprocessowner
		\item \textbf{Descrizione:} Interfaccia che permette la gestione dei processi al \textit{process owner};
	\end{itemize}

%----------------------------------------------------------------------------------------------%
\paragraph{ILogin}
	\begin{itemize}
		\item \textbf{Nome:} \texttt{ILogin};
		\item \textbf{Package:} sequenziatore::server::presenter::iprocessowner
		\item \textbf{Descrizione:} Interfaccia che permette la gestione della login del \textit{process owner};
	\end{itemize}

%----------------------------------------------------------------------------------------------%
\paragraph{IStepManager}
	\begin{itemize}
		\item \textbf{Nome:} \texttt{IStepManager};
		\item \textbf{Package:} sequenziatore::server::presenter::iprocessowner
		\item \textbf{Descrizione:} Interfaccia che permette la gestione dei passi al \textit{process owner};
	\end{itemize}

%----------------------------------------------------------------------------------------------%
\paragraph{IUserManager}
	\begin{itemize}
		\item \textbf{Nome:} \texttt{IUserManager};
		\item \textbf{Package:} sequenziatore::server::presenter::iprocessowner
		\item \textbf{Descrizione:} Interfaccia per la gestione degli utenti iscritti ai processi;
	\end{itemize}
%----------------------------------------------------------------------------------------------%

\paragraph{IReport}
	\begin{itemize}
		\item \textbf{Nome:} \texttt{IReport};
		\item \textbf{Package:} sequenziatore::server::presenter::iprocessowner
		\item \textbf{Descrizione:} Interfaccia che permette la gestione dei report al \textit{process owner};
	\end{itemize}
%0000000000000000000000000000000000000000000000000000000000000000000000000000000000000000000000000%
\subsubsection{Package sequenziatore::server::presenter::processowner}
\paragraph{Login}
	\begin{itemize}
		\item \textbf{Nome:} \texttt{Login};
		\item \textbf{Package:} sequenziatore::server::presenter::iprocessowner
		\item \textbf{Descrizione:} Classe che permette la gestione della login del \textit{process owner};
		\item \textbf{Relazione con altre componenti:} la classe implementa l' intefaccia sequenziatore::server::presenter::iprocessowner::ILogin ed invoca i metodi delle classi:
		\begin{itemize}
			\item sequenziatore::server::model::DaoFacade;
		\end{itemize}
	\end{itemize}

\paragraph{ProcessManager}
	\begin{itemize}
		\item \textbf{Nome:} \texttt{ProcessManager};
		\item \textbf{Package:} sequenziatore::server::presenter::processowner
		\item \textbf{Descrizione:} Classe che riceve le richieste del \textit{process owner} per la gestione dei processi come creazione,modifica e eliminazione degli stessi;
		\item \textbf{Relazione con altre componenti:} la classe implementa l' intefaccia sequenziatore::server::presenter::iprocessowner::IProcessManager ed invoca i metodi delle classi:
		\begin{itemize}
			\item sequenziatore::server::model::DaoFacade;
		\end{itemize}
	\end{itemize}
%----------------------------------------------------------------------------------------------%
\paragraph{StepManager}
	\begin{itemize}
		\item \textbf{Nome:} \texttt{StepManager};
		\item \textbf{Package:} sequenziatore::server::presenter::processowner
		\item \textbf{Descrizione:} Classe che permette l' elaborazione delle richieste del \textit{process owner} per quanto concerne la creazione,la rimozione e la modifica di passi;
		\item \textbf{Relazione con altre componenti:} la classe implementa l' intefaccia sequenziatore::server::presenter::iprocessowner::IStepManager ed invoca i metodi delle classi:
		\begin{itemize}
			\item sequenziatore::server::model::DaoFacade;
		\end{itemize}
	\end{itemize}
	%----------------------------------------------------------------------------------------------%
\paragraph{UserManager}
	\begin{itemize}
		\item \textbf{Nome:} \texttt{UserManager};
		\item \textbf{Package:} sequenziatore::server::presenter::iprocessowner
		\item \textbf{Descrizione:} Classe che permette la gestione degli utenti iscritti alla piattaforma, permettendogli di rimuovere utenti da processi,;
		\item \textbf{Relazione con altre componenti:} la classe implementa l' intefaccia sequenziatore::server::presenter::iprocessowner::IUserManager ed invoca i metodi delle classi:
		\begin{itemize}
			\item sequenziatore::server::model::DaoFacade;
		\end{itemize}
	\end{itemize}

%----------------------------------------------------------------------------------------------%
\paragraph{Report}
	\begin{itemize}
		\item \textbf{Nome:} \texttt{Report};
		\item \textbf{Package:} sequenziatore::server::presenter::iprocessowner
		\item \textbf{Descrizione:} Classe che permette la gestione delle richieste dei report al \textit{process owner}, permettendogli di visualizzare i risultati raggiunti in un processo;
		\item \textbf{Relazione con altre componenti:} la classe implementa l' intefaccia sequenziatore::server::presenter::iprocessowner::IReport ed invoca i metodi delle classi:
		\begin{itemize}
			\item sequenziatore::server::model::DaoFacade;
		\end{itemize}
	\end{itemize}
