\subsubsection{Package com.sirius.sequenziatore.server.presenter.user}
\paragraph{AccountController}
	\begin{itemize}
		\item \textbf{Nome:} \texttt{AccountController};
		\item \textbf{Package:} com.sirius.sequenziatore.server.presenter.user
		\item \textbf{Descrizione:} classe che permette la modifica dei dati di un utente come password o altre informazioni inerenti ai dettagli personali di un utente;
		\item \textbf{Relazione con altre componenti:} la classe richiama i metodi della classe:
		\begin{itemize}
			\item com.sirius.sequenziatore.server.model.IDataAccessObject;
		\end{itemize}
	\end{itemize}
%-----------------------------------------------------------------------------------------------%
\paragraph{UserProcessController}
	\begin{itemize}
		\item \textbf{Nome:} \texttt{UserProcessController};
		\item \textbf{Package:} com.sirius.sequenziatore.server.presenter.user
		\item \textbf{Descrizione:} classe che restituisce all' utente i dati di uno o più processi, può inoltre permettere l' inoltro della richiesta di un utente a iscriversi o disiscriversi a un processo;
		\item \textbf{Relazione con altre componenti:} la classe richiama i metodi della classe:
		\begin{itemize}
			\item com.sirius.sequenziatore.server.model.IDataAccessObject;
		\end{itemize}
	\end{itemize}
%-----------------------------------------------------------------------------------------------%
\paragraph{UserStepController}
	\begin{itemize}
		\item \textbf{Nome:} \texttt{UserStepController};
		\item \textbf{Package:} com.sirius.sequenziatore.server.presenter.user
		\item \textbf{Descrizione:} Gestisce l' esecuzione di un passo da parte di un utente inoltrando la richiesta di inserire i dati nel \textit{database} e in caso sia richiesto, notifica l' amministratore che deve controllare se il passo è stato completato, inoltre è incaricato di restituire i dati inseriti di un passo quando richiesto da un utente;
		\item \textbf{Relazione con altre componenti:} la classe richiama i metodi della classe:
		\begin{itemize}
			\item com.sirius.sequenziatore.server.model.IDataAccessObject;
		\end{itemize}
	\end{itemize}
%-----------------------------------------------------------------------------------------------%
\paragraph{ReportController}
	\begin{itemize}
		\item \textbf{Nome:} \texttt{ReportController};
		\item \textbf{Package:} com.sirius.sequenziatore.server.presenter.user
		\item \textbf{Descrizione:} Classe che fornisce i dati per generare il report dell' utente riferito al processo richiesto;
		\item \textbf{Relazione con altre componenti:} la classe implementa l' interfaccia sequenziatore.server.presenter.iuser.IReport e richiama i metodi della classe:
		\begin{itemize}
			\item com.sirius.sequenziatore.server.model.IDataAccessObject;
		\end{itemize}
	\end{itemize}
%-----------------------------------------------------------------------------------------------%