\subsubsection{Package sequenziatore::server::presenter::iuser}
\paragraph{IAccountManager}
	\begin{itemize}
		\item \textbf{Nome:} \texttt{IAccountManager};
		\item \textbf{Package:} sequenziatore::server::presenter::iuser
		\item \textbf{Descrizione:} Interfaccia che permette la gestione del proprio account all' utente.
	\end{itemize}
%-----------------------------------------------------------------------------------------------%
\paragraph{IUserProcessManager}
	\begin{itemize}
		\item \textbf{Nome:} \texttt{IUserProcessManager};
		\item \textbf{Package:} sequenziatore::server::presenter::iuser
		\item \textbf{Descrizione:} Interfaccia che permette la gestione dei processi di un utente;
	\end{itemize}
%-----------------------------------------------------------------------------------------------%
\paragraph{IUserStepManager}
	\begin{itemize}
		\item \textbf{Nome:} \texttt{IUserStepManager};
		\item \textbf{Package:} sequenziatore::server::presenter::iuser
		\item \textbf{Descrizione:} Interfaccia che gestisce l' esecuzione di un passo da parte di un utente.
	\end{itemize}
%-----------------------------------------------------------------------------------------------%
\paragraph{IReport}
	\begin{itemize}
		\item \textbf{Nome:} \texttt{IReport};
		\item \textbf{Package:} sequenziatore::server::presenter::iuser
		\item \textbf{Descrizione:} Interfaccia che permette la creazione del report per l' utente.
	\end{itemize}
%000000000000000000000000000000000000000000000000000000000000000000000000000000000000000000000000000000000%
\subsubsection{Package sequenziatore::server::presenter::user}
\paragraph{AccountManager}
	\begin{itemize}
		\item \textbf{Nome:} \texttt{AccountManager};
		\item \textbf{Package:} sequenziatore::server::presenter::user
		\item \textbf{Descrizione:} classe che permette la modifica dei dati e il controllo del \textit{log in} di un utente;
		\item \textbf{Relazione con altre componenti:} la classe implementa l' interfaccia sequenziatore::server::presenter::iuser::IAccountManager e richiama i metodi delle classi:
		\begin{itemize}
			\item sequenziatore::server::model::dao::daouser::ObjectTransfer tramite l' interfaccia sequenziatore::server::model::dao::daouser::IObjectTransfer
			\item sequenziatore::server::model::dao::daouser::DataAccessObject tramite l' interfaccia sequenziatore::server::model::dao::daouser::IDataAccessObject
		\end{itemize}
	\end{itemize}
%-----------------------------------------------------------------------------------------------%
\paragraph{UserProcessManager}
	\begin{itemize}
		\item \textbf{Nome:} \texttt{UserProcessManager};
		\item \textbf{Package:} sequenziatore::server::presenter::user
		\item \textbf{Descrizione:} classe che permette l' inoltro della richiesta di un utente a iscriversi o disiscriversi a un processo;
		\item \textbf{Relazione con altre componenti:} la classe implementa l' interfaccia sequenziatore::server::presenter::iuser::IUserProcessManager e richiama i metodi delle classi:
		\begin{itemize}
			\item sequenziatore::server::model::dao::daoprocess::ObjectTransfer tramite l' interfaccia sequenziatore::server::model::dao::daoprocess::IObjectTransfer
			\item sequenziatore::server::model::dao::daoprocess::DataAccessObject tramite l' interfaccia sequenziatore::server::model::dao::daoprocess::IDataAccessObject
			\item sequenziatore::server::model::dao::daouser::ObjectTransfer tramite l' interfaccia sequenziatore::server::model::dao::daouser::IObjectTransfer
			\item sequenziatore::server::model::dao::daouser::DataAccessObject tramite l' interfaccia sequenziatore::server::model::dao::daouser::IDataAccessObject
		\end{itemize}
	\end{itemize}
%-----------------------------------------------------------------------------------------------%
\paragraph{UserStepManager}
	\begin{itemize}
		\item \textbf{Nome:} \texttt{UserStepManager};
		\item \textbf{Package:} sequenziatore::server::presenter::user
		\item \textbf{Descrizione:} Gestisce l' esecuzione di un passo da parte di un utente inoltrando la richiesta di inserire i dati nel \textit{database} e in caso sia richiesto notifica l' amministratore che deve controllare se il passo è stato completato;
		\item \textbf{Relazione con altre componenti:} la classe implementa l' interfaccia sequenziatore::server::presenter::iuser::IUserStepManager e richiama i metodi delle classi:
		\begin{itemize}
			\item sequenziatore::server::model::dao::daostep::ObjectTransfer tramite l' interfaccia sequenziatore::server::model::dao::daostep::IObjectTransfer
			\item sequenziatore::server::model::dao::daostep::DataAccessObject tramite l' interfaccia sequenziatore::server::model::dao::daostep::IDataAccessObject
			\item sequenziatore::server::model::dao::daouser::ObjectTransfer tramite l' interfaccia sequenziatore::server::model::dao::daouser::IObjectTransfer
			\item sequenziatore::server::model::dao::daouser::DataAccessObject tramite l' interfaccia sequenziatore::server::model::dao::daouser::IDataAccessObject
		\end{itemize}
	\end{itemize}
%-----------------------------------------------------------------------------------------------%
\paragraph{Report}
	\begin{itemize}
		\item \textbf{Nome:} \texttt{Report};
		\item \textbf{Package:} sequenziatore::server::presenter::user
		\item \textbf{Descrizione:} Classe che genera il report dell' utente riferito al processo richiesto;
		\item \textbf{Relazione con altre componenti:} la classe implementa l' interfaccia sequenziatore::server::presenter::iuser::IReport e richiama i metodi delle classi:
		\begin{itemize}
			\item sequenziatore::server::model::dao::daostep::ObjectTransfer tramite l' interfaccia sequenziatore::server::model::dao::daostep::IObjectTransfer
			\item sequenziatore::server::model::dao::daostep::DataAccessObject tramite l' interfaccia sequenziatore::server::model::dao::daostep::IDataAccessObject
			\item sequenziatore::server::model::dao::daouser::ObjectTransfer tramite l' interfaccia sequenziatore::server::model::dao::daouser::IObjectTransfer
			\item sequenziatore::server::model::dao::daouser::DataAccessObject tramite l' interfaccia sequenziatore::server::model::dao::daouser::IDataAccessObject
			\item sequenziatore::server::model::dao::daoprocess::ObjectTransfer tramite l' interfaccia sequenziatore::server::model::dao::daoprocess::IObjectTransfer
			\item sequenziatore::server::model::dao::daoprocess::DataAccessObject tramite l' interfaccia sequenziatore::server::model::dao::daoprocess::IDataAccessObject
		\end{itemize}
	\end{itemize}
%-----------------------------------------------------------------------------------------------%