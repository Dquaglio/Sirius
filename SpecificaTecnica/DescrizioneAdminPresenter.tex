\subsubsection{Package \logicAdmin{}}

\begin{figure}[H]
\centering
\includegraphics[trim=0cm 0.8cm 0cm 0cm,clip=true,scale=0.75]%
{./pack/POPresenter.png} \caption{Diagramma \textit{presenter Process Owner} }
\end{figure}

\paragraph{MainProcessOwner}
\begin{flushleft}
\begin{itemize}
\item \textbf{Nome:} \texttt{MainProcessOwner};
\item \textbf{Package:} \texttt{\logicAdmin{}};
\item \textbf{Descrizione:} Classe che ha il compito della gestione generale della logica delle funzionalità \textit{Process Owner\ped{G}};
\item \textbf{Relazioni con altri componenti:}
\begin{sloppypar}
La classe comunica con l'interfaccia \texttt{\viewAdmin{}.I\fshyp{}Main\fshyp{}Pro\fshyp{}cess\fshyp{}Ow\fshyp{}ner} per la realizzazione dell'interfaccia grafica.
\end{sloppypar}
\end{itemize}
\end{flushleft}

\paragraph{OpenProcess}
\begin{flushleft}
\begin{itemize}
\item \textbf{Nome:} \texttt{OpenProcess};
\item \textbf{Package:} \texttt{\logicAdmin{}};
\item \textbf{Descrizione:} Classe che ha il compito di gestire la ricerca e la selezione di un processo;
\item \textbf{Relazioni con altri componenti:}
\begin{sloppypar}
La classe comunica con l'interfaccia \texttt{\viewAdmin{}.I\fshyp{}O\fshyp{}pen\fshyp{}Pro\fshyp{}cess} per la realizzazione dell'interfaccia grafica, e con la classe \texttt{\collection{}.Pro\fshyp{}cess\fshyp{}Col\fshyp{}lec\fshyp{}tion} per gestire e ottenere i dati dal \textit{server\ped{G}}.
\end{sloppypar}
\end{itemize}
\end{flushleft}

\paragraph{NewProcess}
\begin{flushleft}
\begin{itemize}
\item \textbf{Nome:} \texttt{NewProcess};
\item \textbf{Package:} \texttt{\logicAdmin{}};
\item \textbf{Descrizione:} Classe che ha il compito di gestire la logica della definizione di un nuovo processo;
\item \textbf{Relazioni con altri componenti:}
\begin{sloppypar}
La classe comunica con l'interfaccia \texttt{\viewAdmin{}.I\fshyp{}New\fshyp{}pro\fshyp{}cess} per la realizzazione dell'interfaccia grafica, con la classe \texttt{\collection{}.Pro\fshyp{}cess\fshyp{}Col\fshyp{}lec\fshyp{}tion} comunicare con il \textit{server\ped{G}}, e con la classe \texttt{\logicAdmin{}Add\fshyp{}Step};.
\end{sloppypar}
\end{itemize}
\end{flushleft}

\paragraph{AddStep}
\begin{flushleft}
\begin{itemize}
\item \textbf{Nome:} \texttt{AddStep};
\item \textbf{Package:} \texttt{\logicAdmin{}};
\item \textbf{Descrizione:} Classe che ha il compito di gestire la logica di definizione dei passi di un processo;
\item \textbf{Relazioni con altri componenti:}
\begin{sloppypar}
La classe comunica con l'interfaccia \texttt{\viewAdmin{}.I\fshyp{}Add\fshyp{}Step} per la realizzazione dell'interfaccia grafica e utilizza la classe \texttt{\model{}Step} per salvare i dati del passo in creazione.
\end{sloppypar}
\end{itemize}
\end{flushleft}

\paragraph{ManageProcess}
\begin{flushleft}
\begin{itemize}
\item \textbf{Nome:} \texttt{ManageProcess};
\item \textbf{Package:} \texttt{\logicAdmin{}};
\item \textbf{Descrizione:} Classe che ha il compito di gestire e accedere alle informazioni relative allo stato dei processi e ai dati inviati dagli utenti. Le operazioni di gestione dello stato comprendono la terminazione e l'eliminazione di un processo;
\item \textbf{Relazioni con altri componenti:}
\begin{sloppypar}
La classe comunica con l'interfaccia \texttt{\viewAdmin{}.I\fshyp{}Ma\fshyp{}na\fshyp{}ge\fshyp{}Pro\fshyp{}cess} per la realizzazione dell'interfaccia grafica, e con le classi \texttt{\collection{}.Pro\fshyp{}cess\fshyp{}Da\fshyp{}ta\fshyp{}Col\fshyp{}lec\fshyp{}tion} e \texttt{\model{}Pro\fshyp{}cess\fshyp{}Mo\fshyp{}del} per gestire e ottenere i dati dal \textit{server\ped{G}}.
\end{sloppypar}
\end{itemize}
\end{flushleft}

\paragraph{CheckStep}
\begin{flushleft}
\begin{itemize}
\item \textbf{Nome:} \texttt{CheckStep};
\item \textbf{Package:} \texttt{\logicAdmin{}};
\item \textbf{Descrizione:} Classe che ha il compito di definire la logica del controllo di un passo che richiede intervento umano per essere approvato;
\item \textbf{Relazioni con altri componenti:}
\begin{sloppypar}
La classe comunica con l'interfaccia \texttt{\viewAdmin{}.I\fshyp{}Check\fshyp{}Step} per la realizzazione dell'interfaccia grafica, e con le classi \texttt{\collection{}.Pro\fshyp{}cess\fshyp{}Da\fshyp{}ta\fshyp{}Col\fshyp{}lec\fshyp{}tion} e \texttt{\model{}Pro\fshyp{}cess\fshyp{}Mo\fshyp{}del} per gestire e ottenere i dati dal \textit{server\ped{G}}.
\end{sloppypar}
\end{itemize}
\end{flushleft}