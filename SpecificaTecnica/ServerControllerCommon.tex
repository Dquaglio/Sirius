\subsection{Package com.sirius.sequenziatore.server.controller}
\begin{figure}[H] \centering \includegraphics[width=%
\textwidth]
{./pack/servercontroller.png} \caption{Diagramma package controller del server}
\end{figure}
\subsubsection{Package com.sirius.sequenziatore.server.controller.common}
Questo \textit{package} contiene le classi che effettuano operazioni generali oppure comuni tra \textit{Process Owner} e Utenti.
\paragraph{LoginController}
	\begin{itemize}
		\item \textbf{Nome:} \texttt{LoginController};
		\item \textbf{Package:} \texttt{com.sirius.sequenziatore.server.controller.common}
		\item \textbf{Descrizione:} Classe che riceve la richiesta di login di un utilizzatore del sistema, e ritorna l' esito dell' elaborazione del \textit{service} avvisando se l' utente loggato è un \textit{process owner} o un utente normale;
		\item \textbf{Relazione con altre componenti:} la classe invoca i metodi della classe:
		\begin{itemize}
			\item \texttt{com.sirius.sequenziatore.server.service.LoginService};
		\end{itemize}
	\end{itemize}
	
\paragraph{SignUpConroller}
	\begin{itemize}
		\item \textbf{Nome:} \texttt{SignUpController};
		\item \textbf{Package:} \texttt{com.sirius.sequenziatore.server.controller.common}
		\item \textbf{Descrizione:} Classe che permette la gestione della registrazione di un nuovo utente nel sistema, nonostante la correttezza dei dati inseriti venga controllata dalla parte client, per sicurezza verrà effettuato un nuovo controllo anche sulla parte server prima di inserire un utente nel sistema;
		\item \textbf{Relazione con altre componenti:} la classe invoca i metodi della classe:
		\begin{itemize}
			\item \texttt{com.sirius.sequenziatore.server.service.SignUpService};
		\end{itemize}
	\end{itemize}
\paragraph{StepInfoController}
	\begin{itemize}
		\item \textbf{Nome:} \texttt{StepInfoController};
		\item \textbf{Package:} \texttt{com.sirius.sequenziatore.server.controller.common}
		\item \textbf{Descrizione:} Classe che fornisce a chi lo richiede lo scheletro di un passo, dopo averlo richiesto al service;
		\item \textbf{Relazione con altre componenti:} la classe invoca i metodi delle classi:
		\begin{itemize}
			\item \texttt{com.sirius.sequenziatore.server.service.StepInfoService}
		\end{itemize}
	\end{itemize}
\paragraph{ProcessInfoController}
	\begin{itemize}
		\item \textbf{Nome:} \texttt{ProcessInfoController};
		\item \textbf{Package:} \texttt{com.sirius.sequenziatore.server.controller.common}
		\item \textbf{Descrizione:} Classe incaricata di fornire a chi lo richieda lo scheletro di un processo;
		\item \textbf{Relazione con altre componenti:} la classe invoca i metodi della classe:
		\begin{itemize}
			\item \texttt{com.sirius.sequenziatore.server.service.ProcessInfoService}
		\end{itemize}
	\end{itemize}

