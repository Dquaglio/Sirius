\subsubsection{Package \logicUser}

\begin{figure}[H]
\centering
\includegraphics[trim=0cm 0.8cm 0cm 0cm,clip=true,scale=0.75]%
{./pack/UserPresenterA.png} \caption{Diagramma principale \textit{presenter} utente}
\end{figure}

\begin{figure}[H]
\centering
\includegraphics[trim=0cm 0.8cm 0cm 0cm,clip=true,scale=0.75]%
{./pack/UserPresenterB.png} \caption{Diagramma \textit{presenter} utente - gestione processi}
\end{figure}

\paragraph{EventDispatcher}
\begin{flushleft}
\begin{itemize}
\item \textbf{Nome:} \texttt{EventDispatcher};
\item \textbf{Package:} \texttt{\logicUser};
\item \textbf{Descrizione:} Classe che gestisce il dispatching degli eventi per la parte \texttt{user};
\item \textbf{Relazioni con altri componenti:}
\begin{sloppypar}
La classe estende la classe \texttt{\logic{}.Ba\fshyp{}se\fshyp{}Do\fshyp{}spat\fshyp{}cher}.
\end{sloppypar}
\end{itemize}
\end{flushleft}

\paragraph{MainUser}
\begin{flushleft}
\begin{itemize}
\item \textbf{Nome:} \texttt{MainUser};
\item \textbf{Package:} \texttt{\logicUser};
\item \textbf{Descrizione:} Classe che ha il compito della gestione generale della logica delle funzionalità utente;
\item \textbf{Relazioni con altri componenti:}
\begin{sloppypar}
La classe comunica con l'interfaccia \texttt{\viewUser{}.I\fshyp{}Main\fshyp{}U\fshyp{}ser} per la realizzazione dell'interfaccia grafica.
La classe estende la classe \texttt{\logic{}.Ba\fshyp{}se\fshyp{}Pre\fshyp{}sen\fshyp{}ter}.
\end{sloppypar}
\end{itemize}
\end{flushleft}


\paragraph{Register}
\begin{flushleft}
\begin{itemize}
\item \textbf{Nome:} \texttt{Register};
\item \textbf{Package:} \texttt{\logicUser};
\item \textbf{Descrizione:} Classe che ha il compito di gestire le richieste di registrazione da parte dell'utente;
\item \textbf{Relazioni con altri componenti:}
\begin{sloppypar}
La classe comunica con l'interfaccia \texttt{\viewUser{}.I\fshyp{}Re\fshyp{}gi\fshyp{}ster} per la realizzazione dei \textit{widget} per la registrazione, e con la classe \texttt{\model{}.U\fshyp{}ser\fshyp{}Mo\fshyp{}del} per comunicare col il \textit{server\ped{G}}.
La classe estende la classe \texttt{\logic{}.Ba\fshyp{}se\fshyp{}Pre\fshyp{}sen\fshyp{}ter}.\end{sloppypar}
\end{itemize}
\end{flushleft}

\paragraph{UserData}
\begin{flushleft}
\begin{itemize}
\item \textbf{Nome:} \texttt{UserData};
\item \textbf{Package:} \texttt{\logicUser};
\item \textbf{Descrizione:} Classe che ha il compito di gestire la visualizzazione e la modifica dei dati dell'utente;
\item \textbf{Relazioni con altri componenti:}
\begin{sloppypar}
La classe comunica con l'interfaccia \texttt{\viewUser{}.I\fshyp{}Us\fshyp{}er\fshyp{}Da\fshyp{}ta} per realizzare il \textit{widget} preposto alla visualizzazione e modifica dei dati dell'utente, e con la classe \texttt{\model{}.U\fshyp{}ser\fshyp{}Mo\fshyp{}del} per comunicare col il \textit{server\ped{G}}.
La classe estende la classe \texttt{\logic{}.Ba\fshyp{}se\fshyp{}Pre\fshyp{}sen\fshyp{}ter}.\end{sloppypar}
\end{itemize}
\end{flushleft}

\paragraph{OpenProcess}
\begin{flushleft}
\begin{itemize}
\item \textbf{Nome:} \texttt{OpenProcess};
\item \textbf{Package:} \texttt{\logicUser};
\item \textbf{Descrizione:} Classe che ha il compito di selezionare, ricercare e aprire un processo fra quelli eseguibili;
\item \textbf{Relazioni con altri componenti:}
\begin{sloppypar}
La classe realizza e modifica l'opportuno \textit{widget} mediante l'interfaccia \texttt{\viewUser{}.I\fshyp{}Op\fshyp{}en\fshyp{}Pro\fshyp{}cess} e utilizza la classe \texttt{\collectionu{}.Pro\fshyp{}cess\fshyp{}Col\fshyp{}lec\fshyp{}tion} per gestire e ottenere i dati dal \textit{server\ped{G}}.
La classe estende la classe \texttt{\logic{}.Ba\fshyp{}se\fshyp{}Pre\fshyp{}sen\fshyp{}ter}.
\end{sloppypar}
\end{itemize}
\end{flushleft}

\paragraph{ManagmentProcess}
\begin{flushleft}
\begin{itemize}
\item \textbf{Nome:} \texttt{ManagmentProcess};
\item \textbf{Package:} \texttt{\logicUser};
\item \textbf{Descrizione:} Classe che ha il compito di gestire e accedere alle informazioni relative allo stato del processo selezionato.;
\item \textbf{Relazioni con altri componenti:}
\begin{sloppypar}
La classe estende la classe \texttt{\viewUser{}.Ba\fshyp{}se\fshyp{}Pre\fshyp{}sen\fshyp{}ter}.
La classe inoltre comunica con l'interfaccia \texttt{\viewUser{}.I\fshyp{}Ma\fshyp{}na\fshyp{}gment\fshyp{}Pro\fshyp{}cess} per realizzare il \textit{widget} che permette la gestione del processo selezionato, utilizza la classe \texttt{\model{}.Pro\fshyp{}cess\fshyp{}Mo\fshyp{}del} per gestire e ottenere i dati dal \textit{server\ped{G}}, e provvede ad invocare le seguenti classi in base alle decisioni dell'utente:
\begin{itemize}
\item \texttt{\logicUser{}.Print\fshyp{}Re\fshyp{}port};
\item \texttt{\logicUser{}.Send\fshyp{}Da\fshyp{}ta}.
\end{itemize}
\end{sloppypar}
\end{itemize}
\end{flushleft}

\paragraph{PrintReport}
\begin{flushleft}
\begin{itemize}
\item \textbf{Nome:} \texttt{PrintReport};
\item \textbf{Package:} \texttt{\logicUser};
\item \textbf{Descrizione:} Classe che ha il compito di gestire la creazione del report di fine processo;
\item \textbf{Relazioni con altri componenti:}
\begin{sloppypar}
La classe comunica con l'interfaccia \texttt{\viewUser{}.I\fshyp{}Print\fshyp{}Re\fshyp{}port} per realizzare il \textit{widget} per creare il report di fine processo, e utilizza la classe \texttt{\collectionu{}.Pro\fshyp{}cess\fshyp{}Data\fshyp{}Col\fshyp{}lec\fshyp{}tion} per gestire e ottenere i dati dal \textit{server\ped{G}}.
\end{sloppypar}
\end{itemize}
\end{flushleft}

\paragraph{SendData}
\begin{flushleft}
\begin{itemize}
\item \textbf{Nome:} \texttt{SendData};
\item \textbf{Package:} \texttt{\logicUser};
\item \textbf{Descrizione:} Classe che ha il compito di gestire l'inserimento e l'invio di dati da parte degli utenti, per completare il passo corrente;
\item \textbf{Relazioni con altri componenti:}
\begin{sloppypar}
La classe comunica con l'interfaccia \texttt{\viewUser{}.I\fshyp{}Send\fshyp{}Da\fshyp{}ta} per creare il \textit{widget} che consente di inviare i dati, utilizza la classe \texttt{\collectionu{}.Pro\fshyp{}cess\fshyp{}Data\fshyp{}Col\fshyp{}lec\fshyp{}tion} per gestire e ottenere i dati dal \textit{server\ped{G}}, e infine invoca le seguenti classi che gestiscono l'invio di un tipo di dato specifico:
\begin{itemize}
	\item \texttt{\logicUser{}.SendText};
	\item \texttt{\logicUser{}.SendNumb};
	\item \texttt{\logicUser{}.SendImage};
	\item \texttt{\logicUser{}.SendPosition}.
\end{itemize}
\end{sloppypar}
\end{itemize}
\end{flushleft}

\paragraph{SendText}
\label{sendText}
\begin{flushleft}
\begin{itemize}
\item \textbf{Descrizione:} Classe che permette l'inserimento e il controllo di dati testuali inseriti dagli utenti;
\item \textbf{Relazioni con altri componenti:}
\begin{sloppypar}
La classe, mediante l'interfaccia \texttt{\viewUser{}.I\fshyp{}Send\fshyp{}Text}, realizza e aggiorna l'opportuno \textit{widget}.
\end{sloppypar}
\end{itemize}
\end{flushleft}

\paragraph{SendNumb}
\label{sendNumb}
\begin{flushleft}
\begin{itemize}
\item \textbf{Descrizione:} Classe che ha il compito di permettere l'inserimento e il controllo di dati numerici inseriti dagli utenti;
\item \textbf{Relazioni con altri componenti:}
\begin{sloppypar}
La classe, mediante l'interfaccia \texttt{\viewUser{}.I\fshyp{}Send\fshyp{}Numb}, realizza e aggiorna l'opportuno \textit{widget}.
\end{sloppypar}
\end{itemize}
\end{flushleft}

\paragraph{SendImage}
\label{sendImage}
\begin{flushleft}
\begin{itemize}
\item \textbf{Descrizione:} Classe che gestisce l'inserimento e il controllo di immagini inserite dagli degli utenti;
\item \textbf{Relazioni con altri componenti:}
\begin{sloppypar}
La classe, mediante l'interfaccia \texttt{\viewUser{}.I\fshyp{}Send\fshyp{}Ima\fshyp{}ge}, realizza e aggiorna l'opportuno \textit{widget}.
\end{sloppypar}
\end{itemize}
\end{flushleft}

\paragraph{SendPosition}
\label{sendPosition}
\begin{flushleft}
\begin{itemize}
\item \textbf{Descrizione:} Classe che ha il compito di gestire il calcolo e il controllo della posizione geografica dell'utente;
\item \textbf{Relazioni con altri componenti:}
\begin{sloppypar}
La classe, mediante l'interfaccia \texttt{\viewUser{}.I\fshyp{}Send\fshyp{}Po\fshyp{}si\fshyp{}tion}, realizza e aggiorna l'opportuno \textit{widget}.
\end{sloppypar}
\end{itemize}
\end{flushleft}