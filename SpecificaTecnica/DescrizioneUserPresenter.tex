\subsubsection{Package sequenziatore.client.presenter.user.views}

\paragraph{UserRouter}
\begin{flushleft}
\begin{itemize}
\item \textbf{Nome:} \texttt{UserRouter};
\item \textbf{Descrizione:} Classe che permette la modifica dello stato dell'interfaccia grafica in base all'evolversi dell'interazione fra utente e applicazione}
\begin{sloppypar}
La classe cambia lo stato dell'interfaccia richiamando le seguente classi, le quali permetteranno la realizzazione del widget desiderato implementa l'interfaccia:
\end{sloppypar}
\begin{itemize}
\item \texttt{Lo\fshyp{}gin};
\item \texttt{Re\fshyp{}gis\fshyp{}ter};
\item \texttt{Ma\fshyp{}in\fshyp{}Us\fshyp{}er};
\item \texttt{Us\fshyp{}er\fshyp{}Da\fshyp{}ta};
\item \texttt{Op\fshyp{}en\fshyp{}Pro\fshyp{}cess\fshyp{}gic};
\item \texttt{Ma\fshyp{}nag\fshyp{}ment\fshyp{}Pro\fshyp{}cess};
\end{itemize}
\end{itemize}
\end{flushleft}

\paragraph{MainLogic}
\begin{flushleft}
\begin{itemize}
\item \textbf{Nome:} \texttt{MainLogic};
\item \textbf{Package:} \texttt{\logicUser{}};
\item \textbf{Descrizione:} Classe che permette di gestire gli eventi generati dalla componente \textit{View};
\item \textbf{Relazioni con altri componenti:}
\begin{sloppypar}
La classe implementa l'interfaccia \texttt{\iLogicUser{}::I\fshyp{}Main\fshyp{}Lo\fshyp{}gic} e delega la gestione della logica di dettaglio alle seguenti classi:
\end{sloppypar}
\begin{itemize}
\item \texttt{\logicUser{}::Lo\fshyp{}gin\fshyp{}Lo\fshyp{}gic};
\item \texttt{\logicUser{}::Re\fshyp{}gis\fshyp{}ter\fshyp{}Lo\fshyp{}gic};
\item \texttt{\logicUser{}::Ma\fshyp{}nag\fshyp{}ment\fshyp{}Da\fshyp{}ta\fshyp{}Lo\fshyp{}gic};
\item \texttt{\logicUser{}::Ma\fshyp{}nag\fshyp{}ment\fshyp{}Pro\fshyp{}cess\fshyp{}Lo\fshyp{}gic};
\item \texttt{\logicUser{}::Send\fshyp{}Da\fshyp{}ta\fshyp{}Lo\fshyp{}gic};
\item \texttt{\logicUser{}::Re\fshyp{}port\fshyp{}Lo\fshyp{}gic};
\end{itemize}
\end{itemize}
\end{flushleft}

\paragraph{Login}
\begin{flushleft}
\begin{itemize}
\item \textbf{Nome:} \texttt{LoginLogic};
\item \textbf{Descrizione:} Classe che ha il compito di gestire le richieste di autenticazione e chiusura della sessione da parte dell'utente e realizzare l'opportuna interfaccia grafica;
\item \textbf{Relazioni con altri componenti:}
\begin{sloppypar}
La classe, comunicando con l'interfaccia \texttt{I\fshyp{}Lo\fshyp{}gin}, realizza il widget per la l'autenticazione.
\end{sloppypar}
\end{itemize}
\end{flushleft}

\paragraph{Register}
\begin{flushleft}
\begin{itemize}
\item \textbf{Nome:} \texttt{Register};
\item \textbf{Descrizione:} Classe che ha il compito di gestire le richieste di registrazione da parte dell'utente;
\item \textbf{Relazioni con altri componenti:}
\begin{sloppypar}
La classe, comunicando attraverso l'interfaccia \texttt{I\fshyp{}Re\fshyp{}gi\fshyp{}ster} realizza il widget per la registrazione.
\end{sloppypar}
\end{itemize}
\end{flushleft}

\paragraph{UserData}
\begin{flushleft}
\begin{itemize}
\item \textbf{Nome:} \texttt{UserData};
\item \textbf{Descrizione:} Classe che ha il compito di gestire la visualizzazione e la modifica dei dati dell'utente;
\item \textbf{Relazioni con altri componenti:}
\begin{sloppypar}
La classe, per mezzo dell'interfaccia\texttt{I\fshyp{}Us\fshyp{}er\fshyp{}Da\fshyp{}ta}, realizza il widget preposto alla visualizzazione e modifica dei dati dell'utente.
\end{sloppypar}
\end{itemize}
\end{flushleft}

\paragraph{OpenProcess}
\begin{flushleft}
\begin{itemize}
\item \textbf{Nome:} \texttt{OpenProcess};
\item \textbf{Descrizione:} Classe che ha il compito di selezionare, ricercare e aprire un processo fra quelli eseguibili;
\item \textbf{Relazioni con altri componenti:}
\begin{sloppypar}
La classe realizza e modifica l'opportuno widget mediante l'interfaccia \texttt{I\fshyp{}Op\fshyp{}en\fshyp{}Pro\fshyp{}cess}.
\end{sloppypar}
\end{itemize}
\end{flushleft}

\paragraph{ManagmentProcess}
\begin{flushleft}
\begin{itemize}
\item \textbf{Nome:} \texttt{ManagmentProcess};
\item \textbf{Descrizione:} Classe che ha il compito di gestire e accedere alle informazioni relative allo stato del processo selezionato.;
\item \textbf{Relazioni con altri componenti:}
\begin{sloppypar}
La classe, mediante l'interfaccia \texttt{I\fshyp{}Ma\fshyp{}na\fshyp{}gment\fshyp{}Pro\fshyp{}cess}, realizza e aggiorna il widget che permette la gestione del processo selezionato. Inoltre provvede a creare e chiamare le seguenti classi in base alle decisioni dell'utente:
\begin{itemize}
\item \texttt{PrintReport}
\item \texttt{SendData}
\end{itemize}
\end{sloppypar}
\end{itemize}
\end{flushleft}

\paragraph{PrintReport}
\begin{flushleft}
\begin{itemize}
\item \textbf{Nome:} \texttt{PrintReport};
\item \textbf{Descrizione:} Classe che ha il compito di gestire la creazione di report sull'andamento dei processi in esecuzione;
\item \textbf{Relazioni con altri componenti:}
\begin{sloppypar}
La classe chiama l'interfaccia \texttt{I\fshyp{}Print\fshyp{}Re\fshyp{}port} per poter realizzare il widget per poter creare il report.
\end{sloppypar}
\end{itemize}
\end{flushleft}

\paragraph{SendData}
\begin{flushleft}
\begin{itemize}
\item \textbf{Nome:} \texttt{SendData};
\item \textbf{Descrizione:} Classe che ha il compito di gestire l'inserimento e l'invio di dati da parte degli utenti, per completare il passo corrente;
\item \textbf{Relazioni con altri componenti:}
\begin{sloppypar}
La classe,mediante l'interfaccia \texttt{I\fshyp{}Send\fshyp{}Da\fshyp{}ta}, crea ed aggiorna il widget preposto per la selezione del vario tipo di dato da inviare. Crea ed invoca le seguenti classi preposte alla realizzazione e gestione dei widget per inviare i vari tipo di dati:
\begin{itemize}
	\item \texttt{SendText}
	\item \texttt{SendNumb}
	\item \texttt{SendImage}
	\item \texttt{SendPosition}
\end{itemize}
\end{sloppypar}
\end{itemize}
\end{flushleft}

\paragraph{SendText}
\begin{flushleft}
\begin{itemize}
\item \textbf{Nome:} \texttt{SendText};
\item \textbf{Descrizione:} Classe che permette l'inserimento e l'invio di dati testuali da parte degli utenti;
\item \textbf{Relazioni con altri componenti:}
\begin{sloppypar}
La classe,mediante l'interfaccia \texttt{I\fshyp{}Send\fshyp{}Text}, realizza e aggiorna l'opportuno widget.
\end{sloppypar}
\end{itemize}
\end{flushleft}

\paragraph{SendNumb}
\begin{flushleft}
\begin{itemize}
\item \textbf{Nome:} \texttt{SendNumb};
\item \textbf{Descrizione:} Classe che ha il compito di permettere l'inserimento e l'invio di dati numerici da parte degli utenti, per completare il passo corrente;
\item \textbf{Relazioni con altri componenti:}
\begin{sloppypar}
La classe,mediante l'interfaccia \texttt{I\fshyp{}Send\fshyp{}Numb}, realizza e aggiorna l'opportuno widget.
\end{sloppypar}
\end{itemize}
\end{flushleft}

\paragraph{SendImage}
\begin{flushleft}
\begin{itemize}
\item \textbf{Nome:} \texttt{SendImage};
\item \textbf{Descrizione:} Classe che gestisce l'inserimento e l'invio di immagini da parte degli utenti, richieste per completare il passo corrente;
\item \textbf{Relazioni con altri componenti:}
\begin{sloppypar}
La classe,mediante l'interfaccia \texttt{I\fshyp{}Send\fshyp{}Ima\fshyp{}ge}, realizza e aggiorna l'opportuno widget.
\end{sloppypar}
\end{itemize}
\end{flushleft}

\paragraph{SendPosition}
\begin{flushleft}
\begin{itemize}
\item \textbf{Nome:} \texttt{SendPosition};
\item \textbf{Descrizione:} Classe che ha il compito di gestire il calcolo e l'invio della posizione geografica dell'utente;
\item \textbf{Relazioni con altri componenti:}
\begin{sloppypar}
La classe,mediante l'interfaccia \texttt{I\fshyp{}Send\fshyp{}Po\fshyp{}si\fshyp{}tion}, realizza e aggiorna l'opportuno widget.
\end{sloppypar}
\end{itemize}
\end{flushleft}

