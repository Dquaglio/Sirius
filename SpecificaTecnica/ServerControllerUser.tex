\subsubsection{Package com.sirius.sequenziatore.server.controller.user}
%\paragraph{AccountController}
%	\begin{itemize}
	%	\item \textbf{Nome:} \texttt{AccountController};
	%	\item \textbf{Package:} com.sirius.sequenziatore.server.controller.user
	%	\item \textbf{Descrizione:} classe che permette la modifica dei dati di un utente come password o altre informazioni inerenti ai dettagli personali di un utente;
	%	\item \textbf{Relazione con altre componenti:} la classe richiama i metodi della classe:
	%	\begin{itemize}
	%		\item com.sirius.sequenziatore.server.model.IDataAccessObject;
	%	\end{itemize}
	%\end{itemize}
%-----------------------------------------------------------------------------------------------%
\paragraph{UserProcessController}
	\begin{itemize}
		\item \textbf{Nome:} \texttt{UserProcessController};
		\item \textbf{Package:} \texttt{com.sirius.sequenziatore.server.controller.user}
		\item \textbf{Descrizione:} classe che restituisce all' utente i dati di uno o più processi, l' esito della richiesta di un utente a iscriversi o disiscriversi a un processo e lo stato di un utente per un processo;
		\item \textbf{Relazione con altre componenti:} la classe richiama i metodi della classe:
		\begin{itemize}
			\item \texttt{com.sirius.sequenziatore.server.service.UserProcessService;}
		\end{itemize}
	\end{itemize}
%-----------------------------------------------------------------------------------------------%
\paragraph{UserStepController}
	\begin{itemize}
		\item \textbf{Nome:} \texttt{UserStepController};
		\item \textbf{Package:} \texttt{com.sirius.sequenziatore.server.controller.user}
		\item \textbf{Descrizione:} riceve la richiesta da parte del client per il salvataggio dei dati di un passo da parte di un utente;
		\item \textbf{Relazione con altre componenti:} la classe richiama i metodi della classe:
		\begin{itemize}
			\item \texttt{com.sirius.sequenziatore.server.service.UserStepService;}
		\end{itemize}
	\end{itemize}
%-----------------------------------------------------------------------------------------------%
\paragraph{ReportController}
	\begin{itemize}
		\item \textbf{Nome:} \texttt{ReportController};
		\item \textbf{Package:} com.sirius.sequenziatore.server.controller.user
		\item \textbf{Descrizione:} Classe riceve la richiesta da parte di un utente che vuole ottenere i dati per generare il report di processo richiesto;
		\item \textbf{Relazione con altre componenti:} la classe richiama i metodi della classe:
		\begin{itemize}
			\item com.sirius.sequenziatore.server.service.ReportService;
		\end{itemize}
	\end{itemize}
%-----------------------------------------------------------------------------------------------%