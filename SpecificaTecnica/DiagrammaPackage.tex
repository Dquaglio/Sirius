\subsection{Diagrammi dei package}
Il seguente diagramma descrive le dipendenze intercorse fra i vari package\ped{G} del sistema Sequenziatore.
I diagrammi dei package\ped{G} descrivono le dipendenze che intercorrono tra i vari
package\ped{G} che compongono il sistema.
Figura 3: Diagramma dei package del prodotto MyTalk.
Il sistema Sequenziatore è composto da due macro package\ped{G}:
\begin{enumerate}
	\item sequenziatore.client: le componenti di questo package\ped{G} realizzano la parte front-end\ped{G} del sistema Sequenziatore 
	\item sequenziatore.server: le componenti di questo package\ped{G} realizzano la parte back-end\ped{G} del sistema Sequenziatore 
\end{enumerate}
Il package\ped{G} sequenziatore.client è composto dai seguenti package\ped{G}:
\begin{itemize}
	\item sequenziatore.client.view;
	\item sequenziatore.client.presenter;
	\item sequenziatore.client.model.
\end{itemize}
Come è facilmente intuibile, la struttura del package\ped{G} sequenziatore.client si basa sulla struttura del design patter
architetturale Model View Presenter, scelto dal team Sirius per poter separare la logica di presentazione dei dati dalla logica di business.\\
I package\ped{G} che compongono il package\ped{G} sequenziatore::server sono:
\begin{itemize}
	\item sequenziatore.server.presenter;
	\item sequenziatore.server.model.
\end{itemize}
\subsubsection{Package sequenziatore.client.view}
Il package\ped{G} sequenziatore.client.view è composto da i seguenti package\ped{G}:
\begin{itemize}
	\item sequenziatore.client.view.processowner: contiene le componenti template necessarie per la realizzazione dell’interfaccia grafica del process owner.
	\item sequenziatore.client.view.user: contiene le componenti template necessarie per la realizzazione dell’interfaccia grafica dell'user.
\end{itemize}
\subsubsection{Package sequenziatore.client.presenter}
Il package\ped{G} sequenziatore.client.presenter contiene tutte le componenti del Presenter della parte client\ped{G} del sistema Sequenziatore; ed è composto da i seguenti package\ped{G}:
\begin{itemize}
	\item sequenziatore.client.presenter.processowner: contiene le componenti che costituiscono la componente Presenter
per il process owner, il package\ped{G} sequenziatore.client.presenter.processowner è diviso ulteriormente nei sotto-package\ped{G}:
	\begin{itemize}
		\item sequenziatore.client.presenter.processowner.views: contiene le classi necessarie per realizzare e gestire l'aggiornamento della parte grafica, usando i template presenti nel package\ped{g} sequenziatore.client.view.processowner, e alla gestione degli eventi generati dall'interazione da parte del process owner con l'interfaccia grafica, gestendo inoltre la logica di business dell'applicazione;
	\end{itemize}
	\item sequenziatore.client.presenter.user: contiene le componenti che realizzano la componente Presenter per l'utente autenticato; i sotto-package\ped{G} di sequenziatore.client.presenter.user sono i seguenti:
	\begin{itemize}
		\item sequenziatore.client.presenter.user.views: contiene le classi necessarie per realizzare, mediante i template presenti nel package\ped{g} sequenziatore.client.view.user, l'interfaccia utente e gestirne l'interazione con l'utente autenticato, gestendo inoltre la logica di business dell'applicazione;
	\end{itemize}
\end{itemize}
\subsubsection{Package sequenziatore.client.model}
Il package\ped{G} sequenziatore.client.model contiene tutte le classi della componente Model. 
Il package\ped{G} sequenziatore.client.model contiene inoltre:
\begin{itemize}
	\item sequenziatore.client.model.collections contiene le varie collezioni di dati contenuti nel package model; il nome del package ricalca inoltre il nome del supertipo di tutte le collezioni di strutture date usate in un sistema sviluppato usando il framework\ped{G} Backbone.js\ped{G}
\end{itemize}
