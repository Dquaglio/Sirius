\subsection{Diagrammi dei package}
Il seguente diagramma descrive le dipendenze intercorse fra i vari package\ped{G} del sistema Sequenziatore.
I diagrammi dei package |g| descrivono le dipendenze che intercorrono tra i vari
package\ped{G} che compongono il sistema.
Figura 3: Diagramma dei package del prodotto MyTalk.
Il sistema Sequenziatore è composto da due macro package\ped{G}:
\begin{enumerate}
	\item sequenziatore.client: le componenti di questo package\ped{G} realizzano la parte front-end\ped{G} del sistema Sequenziatore 
	\item sequenziatore.server: le componenti di questo package\ped{G} realizzano la parte back-end\ped{G} del sistema Sequenziatore 
\end{enumerate}
Il package\ped{G} sequenziatore.client è composto dai seguenti package\ped{G}:
\begin{itemize}
	\item sequenziatore.client.view;
	\item sequenziatore.client.presenter;
	\item sequenziatore.client.model.
\end{itemize}
Come è facilmente intuibile, la struttura del package\ped{G} sequenziatore.client si basa sulla struttura del design patter
architetturale Model View Presenter, scelto dal team Sirius per poter separare la logica di presentazione dei dati dalla logica di business.\\
I package\ped{G} che compongono il package\ped{G} sequenziatore.server sono:
\begin{itemize}
	\item sequenziatore.server.presenter;
	\item sequenziatore.server.model.
\end{itemize}
\subsubsection{Package sequenziatore.client.view}
Il package\ped{G} sequenziatore.client.view è composto da i seguenti package\ped{G}:
\begin{itemize}
	\item sequenziatore.client.view.admin: contiene le classi e interfacce necessarie a gestire 
l’interfaccia grafica e a generare gli eventi della parte grafica dell'utente amministratore .
	\item sequenziatore.client.view.user: contiene le classi e interfacce necessarie a gestire l’interfaccia
grafica e a generare gli eventi della parte grafica dell’utente.
\end{itemize}
\subsubsection{Package sequenziatore.client.presenter}
Il package\ped{G} sequenziatore.client.presenter contiene tutte le classi e interfacce del Presenter della 
parte client\ped{G} del sistema Sequenziatore; ed è composto da i seguenti package\ped{G}:
\begin{itemize}
	\item sequenziatore.client.presenter.admin: contiene le classi che costituiscono la componente Presenter
per l’utente amministratore, il package\ped{G} sequenziatore.client.presenter.admin è diviso ulteriormente nei sotto-package\ped{G}:
	\begin{itemize}
		\item sequenziatore.client.presenter.admin.logicGest: gestisce gli eventi generati dalle componenti del package\ped{g}
sequenziatore.client.view.admin e aggiorna la parte grafica dell'utente amministratore;
		\item sequenziatore.client.presenter.admin.servnication: contiene le componenti necessarie per interfacciarsi alla parte server\ped{G} del sistema Sequenziatore e gestire la comunicazione con quest'ultima.
	\end{itemize}
	\item sequenziatore.client.presenter.user: contiene tutti i package\ped{G} e le classi che compongono la componente Presenter
per l’utente generico e loggato, i sotto-package\ped{G} di sequenziatore.client.presenter.user sono i seguenti:
	\begin{itemize}
		\item sequenziatore.client.presenter.user.logicGest: gestisce gli eventi generati dalle componenti del package\ped{g}
sequenziatore.client.view.user e aggiorna la parte grafica per l'utente generico e loggato;
		\item sequenziatore.client.presenter.user.servnication: contiene le componenti necessarie, per la parte user, per interfacciarsi alla parte server\ped{G} del sistema Sequenziatore e gestire la comunicazione con quest'ultima.
	\end{itemize}
\end{itemize}
\subsubsection{Package sequenziatore.client.model}
Il package\ped{G} sequenziatore.client.model contiene tutte le classi della componente Model. 
Il package\ped{G} è suddiviso in:
\begin{itemize}
	\item sequenziatore.client.model.localDataAdmin: è composto dalle relative informazioni dell'utente amministratore autenticato al sistema Sequenziatore;
	\item sequenziatore.client.model.localDataUser: è composto dalle relative informazioni dell'utente autenticato al sistema Sequenziatore.
\end{itemize}
