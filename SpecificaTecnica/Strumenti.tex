\appendix
\section{Tecnologie utilizzate}
\subsection{Spring Framework}
Spring è stato scelto per rendere più facile lo sviluppo della nostra applicazione lato server, infatti non è necessario configurare e realizzare le servlet perchè vengono automaticamente gestite e realizzate dal framework. Inoltre rende facile e chiara la separazione delle componenti grazie al suo sistema di annotazioni e anche un codice più pulito, come nel caso di @RequestBody, che grazie alla libreria JacksonJson trasforma automaticamente l' oggetto JSON ricevuto dal controller nell' oggetto Java voluto.
\subsection{HTML5}
\textit{HTML5\ped{G}}, richiesto espressamente dal proponente all'interno del capitolato d'appalto, verrà utilizzato per la struttura base della pagine, inoltre ci permetterà di utilizzare il controllo della geolocalizzazione, fondamentale nello sviluppo del nostro sistema, oltre alle altre novità che introduce.
\subsection{CSS3}
\textit{CSS3} è un linguaggio \textit{style sheet} verrà utilizzato per la rappresentazione grafica delle pagine, in modo da separarla dai contenuti. In questo modo verranno migliorate comprensione, manutenibilità e portabilità.
\subsection{Javascript}
L' utilizzo di \textit{Javascript\ped{G}} è stato richiesto espressamente dal proponente, è un linguaggio di \textit{scripting} non compilato ma interpretato direttamente dal \textit{browser\ped{G}}. Nello sviluppo del nostro progetto ci permetterà di sviluppare l'applicazione lato lato \textit{client\ped{G}} e di comunicare con il \textit{server\ped{G}}.
\subsection{Backbone.js}
\textit{Backbone.js} è un \textit{framework} basato sul paradigma \textit{model-view-presenter}, utilizzata per lo sviluppo dell'applicazione \progetto{}.
Il \textit{framework} è particolarmente leggero e necessita come unica dipendenza della libreria \textit{Underscore.js}.
\textit{Backbone.js} è creato per sviluppare applicazioni \textit{web} di tipo \textit{single page}, e consente di strutturare il codice, grazie alle classi \textit{Model, View} e \textit{Router} estendibili dal programmatore.
Il gruppo \gruppo{} ha scelto questo \textit{framework}, in quanto si presta alle esigenze architetturali del progetto, e inoltre è molto ben documentato.
\subsection{Underscore.js}
\textit{Underscore.js} è una libreria necessaria al \textit{framework Backbone.js}. Viene utilizzata in particolare per gestire la comunicazione tra \textit{Backbone} e i \textit{template} utilizzati nel \textit{package view}.
\subsection{Require.js}
\textit{Require.js} è una libreria utilizzata per gestire le dipendenze tra le componenti e le librerie, e per implementare il \textit{pattern Asynchronous Module Definition
}. La libreria è stata scelta per l'ottima compatibilità con il \textit{framework Backbone.js}.
\subsection{JQuery}
\textit{Jquery} è una libreria \textit{Javascript} per applicazioni web.
La libreria consente di interagire con gli elementi \textit{DOM}, di gestire eventi e implementare funzionalità \textit{AJAX}.
\subsection{JQueryMobile}
Questa libreria verrà usata per lo sviluppo di un \textit{front-end} per dispositivi di tipo \textit{responsive}, accessibili da \textit{smartphne}, \textit{tablet} e computer. La scelta del team di questa libreria è data dal fatto che è affermata nel mondo del \textit{web development}.
\subsection{JAVA 7}
\textit{Java\ped{G}} è un linguaggio orientato agli oggetti che permette di essere quanto più indipendenti possibili dalla piattaforma di esecuzione. Nello sviluppo del nostro sistema verrà utilizzato nella la creazione del \textit{back-end}, in particolare per la creazione delle \textit{Servlet}.
\subsection{JSON}
\textit{JavaScript Object Notation} è il formato scelto per lo scambio dati tra \textit{client\ped{G}} e \textit{server\ped{G}}, è molto facile da utilizzare e si integra bene con la programmazione in AJAX e il suo uso con \textit{Javascript}. Il \textit{parsing} di tale tipo di dato viene effettuato con la semplice chiamata ad un metodo.
\subsection{JDBC}
\textit{Java DataBase Connectivity} è un connettore per \textit{database} in grado di consentire l'accesso alle basi di dati da un programma scritto in \textit{Java}. Fornisce i metodi per interrogare e modificare i dati nella base di dati.
\subsection{MySQL}
\textit{MySQL\ped{G}} è un Relational database management system(RDBMS). Il team ha scelto questo tipo di base di dati in quanto di semplice utilizzo e già utilizzata da tutti i membri del gruppo.
\subsection{Apache Tomcat}
\textit{Apache Tomcat} è un contenitore \textit{servlet} \textit{open source} che offre una piattaforma per l'esecuzione di applicazioni web sviluppate in \textit{Java}. La versione 4.x comprende \textit{Catalina} e \textit{Coyote}, rispettivamente il contenitore \textit{servlet} e il connettore \textit{HTTP}.