\section{Analisi dei rischi}
Dal momento che il gruppo Sirius ha dovuto iniziare successivamente rispetto agli altri gruppi, con un rispettivo calo di tempo per la realizzazione del progetto e vista la conseguente riduzione del tempo disponibile per la realizzazione del progetto, si \'e prestata molta attenzione alla stesura di questa parte del documento adottando una strategia di gestione dei rischi preventiva.\\
Per poter effettuare al meglio l' analisi dei rischi si sono seguiti i seguenti passi:
\begin{enumerate}
	\item \textbf{Identificazione: }Innanzitutto si \'e intrapresa con un' attenta identificazione dei rischi coinvolgendo quando possibile gli \textit{stakeholders}. Questi rischi saranno in seguito divisi secondo i seguenti livelli: tecnologico, del personale, organizzativo, dei requisiti e di valutazione dei costi.
\item \textbf{Analisi: }A questo punto si \'e eseguita un' analisi dei rischi identificati, valutandone la probabilit\'a del verificarsi e il rispettivo impatto fornendo una tabella riassuntiva di questi punti.
\item \textbf{Pianificazione: }Infine si \'e deciso come evitare ogni rischio e, dove non \'e stato possibile, di mitigarne gli effetti negativi.
\end{enumerate}
\begin{center}
	\begin{tabular}{| >{\centering\arraybackslash}m{1in} | >{\centering\arraybackslash}m{1in} | >{\centering\arraybackslash}m{1in} | >{\centering\arraybackslash}m{1in} |}
		\hline
		\textbf{Livello} & \textbf{Tipo} & \textbf{Probabilit\'a del verificarsi}& \textbf{Grado di pericolosit\'a} \\
		\hline
		Tecnologico & Tecnologie adottate & Media & Alto\\
		\hline
		& Inesperienza del team & Alta & Medio \\		
		\cline{2-4}
		 Del personale   & Team non completo & Certa & Medio \\
		 \cline{2-4}
		& Presenza di studente lavoratore & Certa & Medio \\
		 \cline{2-4}
		 & Problemi tra componenti del gruppo & Media & Basso \\
		 \hline
		 & Tempi delle ativit\'a & Media & Medio \\
		 \cline{2-4}
		Organizzativo  & Errata stima delle risorse necessarie & Media & Medio \\
		 \hline
		 Dei requisiti & Capitolato vago & Bassa & Medio \\
		 \hline
		 Valutazione dei costi & Calcolo dei costi errato & Media & Alto \\
		 \hline
	\end{tabular}
	\\
	Tabella 2: Tabella riassuntiva dei rischi.
\end{center}
\subsection{Livello tecnologico}
\subsubsection{Tecnologie adottate}
	\begin{itemize}
	\item \textbf{Analisi: }Molte delle tecnologie che il team \gruppo ha scelto e che andr\'a a scegliere saranno, con molta probabilit\'a , nuove per la maggior parte dei componenti del gruppo e ci\'o porter\'a a un rallentamento dei processi lavorativi.
	\item \textbf{Probabilit\'a di occorrenza:}Media;
	\item \textbf{Grado di pericolosit\'a:}Alto;
	\item \textbf{Pianificazione: }L' amministratore del gruppo dovr\'a fornire delle guide che ogni membro dovr\'a seguire per la propria formazione. Inoltre, qualora il responsabile lo ritenga necessario, l' amministratore dovr\'a spiegare al resto del gruppo il funzionamento di alcune tecnologie.
	\end{itemize}
\subsection{Livello del personale}
\subsubsection{Inesperienza dei componenti del team}
	\begin{itemize}
	\item \textbf{Analisi: }Dal momento che ogni membro del gruppo dovr\'a aver rivestito almeno una volta  ognuno dei ruoli descritti nelle \infoNDP~ durante lo svolgimento del progetto, \'e possibile che un componente del team si ritrovi a dover svolgere una mansione a lui poco affine per competenze. Ci\'o pu\'o portare ad un' inefficienza di tale membro che potrebbe causare il rallentamento dello svolgimento delle attivit\'a del team.
	\item \textbf{Probabilit\'a di occorrenza:}Alta;
	\item \textbf{Grado di pericolosit\'a:}Medio;
	\item \textbf{Pianificazione: }Per evitare che ogni membro del gruppo debba ogni volta informarsi sul proprio ruolo e su come coloro che lo hanno preceduto hanno lavorato, ogni membro \'e  tenuto a riportare in documenti riposti su google drive dei consigli per coloro che dovranno rivestire tale ruolo successivamente.
	\end{itemize}
\subsubsection{Team non completo}
	\begin{itemize}
	\item \textbf{Analisi: }Il gruppo non \'e stato in grado di raccogliere sette persone per creare un gruppo completo, e nonostante sia possibile formare gruppi da sei persone, la mancanza di un elemento rende il carico di lavoro per il resto dei componenti del team maggiore.
	\item \textbf{Probabilit\'a di occorrenza:}Certa;
	\item \textbf{Grado di pericolosit\'a:}Medio;
	\item \textbf{Pianificazione: }Grazie a un' attenta analisi del carico di lavoro si cercher\'a per quanto possibile di rendere uniforme la divisione dei compiti all' interno dei vari periodi di lavoro per diminuire lo stress di ogni componente del gruppo.
	\end{itemize}
\subsubsection{Presenza di studente lavoratore}
	\begin{itemize}
	\item \textbf{Analisi: }All' interno del team \gruppo \'e presente uno studente-lavoratore il quale, non potendo lavorare al progetto a discapito dei giorni lavorativi, potr\'a rendere l' organizzazione del lavoro pi\'u complessa.
	\item \textbf{Probabilit\'a di occorrenza:} Certa;
	\item \textbf{Grado di pericolosit\'a:} Medio;
	\item \textbf{Pianificazione: }Il membro del team, preso atto della difficolt\'a e dello stress che ci\'o pu\'o comportare, si \'e reso disponibile un giorno alla settimana, il marted\'i, per dedicarsi interamente al progetto e impegnandosi a lavorare almeno per un' ora e mezza nel corso delle serate infrasettimanali e per due ore nei giorni del fine settimana.	
	\end{itemize}
\subsubsection{Problemi tra componenti del gruppo}
	\begin{itemize}
	\item \textbf{Analisi: }Essendo il team \gruppo formato da persone completamente estranee tra loro, \'e possibile che si vengano a creare contrasti tra i membri in quanto l' interazione tra essi \'e meno spontanea e non si conoscono a pieno i rispettivi caratteri dei vari membri.
	\item \textbf{Probabilit\'a di occorrenza:} Media;
	\item \textbf{Grado di pericolosit\'a:} Basso;
	\item \textbf{Pianificazione: }Sar\'a ruolo del responsabile di progetto evitare di far lavorare insieme personalit\'a troppo contrastanti che possono entrare in contrasto e che andrebbero ad incidere negativamente sul tempo di svolgimento di un compito.
	\end{itemize}
\subsection{Livello organizzativo}
\subsubsection{Tempi delle attivit\'a}
	\begin{itemize}
	\item \textbf{Analisi: }Data la grande esperienza che si deve possedere per poter svolgere al meglio il ruolo di responsabile, e considerata la mancanza di tale conoscienza tra i membri del gruppo, \'e possibile che i tempi per lo svolgimento delle attivit\'a vengano calcolati in modo non corretto. Se ci\'o dovesse accadere per attivit\'a di grande rilievo, e si calcoli un tempo minore di quello veramente necessario, si provocherebbe un aumento dei costi e un ritardo nel completamento del compito.
	\item \textbf{Probabilit\'a di occorrenza:} Media;
	\item \textbf{Grado di pericolosit\'a:} Medio;
	\item \textbf{Pianificazione: }Si \'e deciso che per ogni attivit\'a la cui terminazione in ritardo crei molti disagi e un aumento eccessivo dei costi si aggiunger\'a un periodo di slack\ped{G}.
	\end{itemize}
\subsubsection{Errata stima delle risorse necessarie}
	\begin{itemize}
	\item \textbf{Analisi: }In caso di errata pianificazione dell' assegnazione delle ore lavorative ai vari membri del team si potrebbe avere uno o pi\'u componenti del gruppo con troppe o troppe poche ore di lavoro svolte. In tal caso sar\'a necessario un intervento del \textit{Responsabile} per riassegnare le varie attivit\'a~, portando ad un aumento dei costi.
	\item \textbf{Probabilit\'a di occorrenza:} Media;
	\item \textbf{Grado di pericolosit\'a:} Medio;
	\item \textbf{Pianificazione: }Si \'e cercato di realizzare una suddivisione dei compiti ottimale, rispettando gli impegni dei vari componenti, in modo da diminuire la probabilit\'a di occorrenza di tale problema.
	\end{itemize}
\subsection{Livello dei requisiti}
\subsubsection{Capitolato vago}
	\begin{itemize}
	\item \textbf{Analisi: }Essendo vago il capitolato nell' esprimere i requisiti del sistema c'\'e la possibilit\'a che le idee del gruppo non vengano apprezzate dal proponente, deludendo le sue aspettative e portando a un possibile fallimento del progetto.
	\item \textbf{Probabilit\'a di occorrenza:} Bassa;
	\item \textbf{Grado di pericolosit\'a:} Medio;
	\item \textbf{Pianificazione: }Per evitare tutto ci\'o~ il gruppo \gruppo ha richiesto un' incontro con il proponente cercando di chiarire i dubbi sul capitolato e definire in modo adeguato i requisiti del sistema. L' esito dell' incontro con il proponente e le decisioni prese in seguito ad esso vengono trattate nel \VerbaleB.
	\end{itemize}
\subsection{Livello di valutazione dei costi}
\subsubsection{Calcolo dei costi errato}
\label{subsubsec:rischiocosti}
	\begin{itemize}
	\item \textbf{Analisi: }\'e possibile, data l' inesperienza del gruppo, che i costi per lo svolgimento del progetto contengano diversi errori e quindi, nel caso peggiore di una sottostima nei tempi di esecuzione di alcune attivit\'a , si otterrebbe un' aumento dei costi portando il gruppo a sforare il preventivo proposto andando cos\'i in perdita.
	\item \textbf{Probabilit\'a di occorrenza:} Media;
	\item \textbf{Grado di pericolosit\'a:} Alto;
	\item \textbf{Pianificazione: } Per cercare di controbilanciare gli errori prodotti dall' inesperienza del gruppo, si \'e deciso che, oltre ai tempi di slack\ped{G} aggiunti a certe  attivit\'a che in caso di sovrastima dei tempi di esecuzione porterebbero a un risparmio rispetto a quanto preventivato, si aggiunger\'a al costo totale del preventivo un 10\% come margine.
	\end{itemize}