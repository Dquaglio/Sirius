\section{Analisi dei rischi}
Dal momento che il gruppo \gruppo~ ha dovuto iniziare successivamente rispetto agli altri gruppi, con un rispettivo calo di tempo per la realizzazione del progetto e vista la conseguente riduzione del tempo disponibile per la realizzazione del progetto, si è prestata molta attenzione alla stesura di questa parte del documento adottando una strategia di gestione dei rischi preventiva.\\
Per poter effettuare al meglio l' analisi dei rischi si sono seguiti i seguenti passi:
\begin{enumerate}
	\item \textbf{Identificazione: }Innanzitutto si è intrapresa con un' attenta identificazione dei rischi coinvolgendo quando possibile gli \textit{stakeholders}. Questi rischi saranno in seguito divisi secondo i seguenti livelli: tecnologico, del personale, organizzativo, dei requisiti e di valutazione dei costi.
\item \textbf{Analisi: }A questo punto si è eseguita un' analisi dei rischi identificati, valutandone la probabilità del verificarsi e il rispettivo impatto fornendo una tabella riassuntiva di questi punti.
\item \textbf{Strategia di mitigazione: }Infine si è deciso come evitare ogni rischio e, dove non è stato possibile, di mitigarne gli effetti negativi.
\item \textbf{Riscontro ed effetti:} Al termine dei periodi, qualora si sia dovuti affrontare un rischio identificato, si riporteranno gli effetti, e se presenti i cambiamenti effettuati alla strategia di mitigazione.
\end{enumerate}
\begin{center}
	\begin{tabular}{| >{\centering\arraybackslash}m{1in} | >{\centering\arraybackslash}m{1in} | >{\centering\arraybackslash}m{1in} | >{\centering\arraybackslash}m{1in} |}
		\hline
		\textbf{Livello} & \textbf{Tipo} & \textbf{Probabilità del verificarsi}& \textbf{Grado di pericolosità} \\
		\hline
		Tecnologico & Tecnologie adottate & Media & Alto\\
		\hline
		& Inesperienza del team & Alta & Medio \\		
		\cline{2-4}
		 Del personale   & Team non completo & Certa & Medio \\
		 \cline{2-4}
		& Presenza di studente lavoratore & Certa & Medio \\
		 \cline{2-4}
		 & Problemi tra componenti del gruppo & Alta & Medio \\
		 \hline
		 & Tempi delle attività & Media & Medio \\
		 \cline{2-4}
		Organizzativo  & Errata stima delle risorse necessarie & Media & Medio \\
		 \hline
		 Dei requisiti & Capitolato vago & Bassa & Medio \\
		 \hline
		 Valutazione dei costi & Calcolo dei costi errato & Media & Alto \\
		 \hline
	\end{tabular}
	\\
	Tabella 2: Tabella riassuntiva dei rischi.
\end{center}
\subsection{Livello tecnologico}
\subsubsection{Tecnologie adottate}
	\begin{itemize}
	\item \textbf{Analisi: }Molte delle tecnologie che il team \gruppo ~ha scelto e che andrà a scegliere saranno, con molta probabilità , nuove per la maggior parte dei componenti del gruppo e ciò porterà a un rallentamento dei processi lavorativi.
	\item \textbf{Probabilità di occorrenza:}Media;
	\item \textbf{Grado di pericolosità:}Alto;
	\item \textbf{Strategia di mitigazione: }L' amministratore del gruppo dovrà fornire delle guide che ogni membro dovrà seguire per la propria formazione. Inoltre, qualora il responsabile lo ritenga necessario, l' amministratore dovrà spiegare al resto del gruppo il funzionamento di alcune tecnologie.
	\item \textbf{Riscontro:} Durante il periodo di progettazione architetturale, nella stesura della architettura generale, spesso era necessaria una lunga attività di "auto formazione" sulle varie attività che spesso ha portato ad allungare i tempi, già stretti, della progettazione, ciò ha avuto un impatto molto rilevante che ha affermato il grado di pericolosità ad un livello alto.
	\end{itemize}
\subsection{Livello del personale}
\subsubsection{Inesperienza dei componenti del team}
	\begin{itemize}
	\item \textbf{Analisi: }Dal momento che ogni membro del gruppo dovrà aver rivestito almeno una volta  ognuno dei ruoli descritti nelle \infoNDP~ durante lo svolgimento del progetto, è possibile che un componente del team si ritrovi a dover svolgere una mansione a lui poco affine per competenze. Ciò può portare ad un' inefficienza di tale membro che potrebbe causare il rallentamento dello svolgimento delle attività del team.
	\item \textbf{Probabilità di occorrenza:}Alta;
	\item \textbf{Grado di pericolosità:}Medio;
	\item \textbf{Strategia di mitigazione: }Per evitare che ogni membro del gruppo debba ogni volta informarsi sul proprio ruolo e su come coloro che lo hanno preceduto hanno lavorato, ogni membro è  tenuto a riportare in documenti riposti su google drive dei consigli per coloro che dovranno rivestire tale ruolo successivamente.
	\end{itemize}
\subsubsection{Team non completo}
	\begin{itemize}
	\item \textbf{Analisi: }Il gruppo non è stato in grado di raccogliere sette persone per creare un gruppo completo, e nonostante sia possibile formare gruppi da sei persone, la mancanza di un elemento rende il carico di lavoro per il resto dei componenti del team maggiore.
	\item \textbf{Probabilità di occorrenza:}Certa;
	\item \textbf{Grado di pericolosità:}Medio;
	\item \textbf{Strategia di mitigazione: }Grazie a un' attenta analisi del carico di lavoro si cercherà per quanto possibile di rendere uniforme la divisione dei compiti all' interno dei vari periodi di lavoro per diminuire lo stress di ogni componente del gruppo.
	\end{itemize}
\subsubsection{Presenza di studente lavoratore}
	\begin{itemize}
	\item \textbf{Analisi: }All' interno del team \gruppo ~è presente uno studente-lavoratore il quale, non potendo lavorare al progetto a discapito dei giorni lavorativi, potrà rendere l' organizzazione del lavoro più complessa.
	\item \textbf{Probabilità di occorrenza:} Certa;
	\item \textbf{Grado di pericolosità:} Medio;
	\item \textbf{Strategia di mitigazione: }Il membro del team, preso atto della difficoltà e dello stress che ciò può comportare, si è reso disponibile un giorno alla settimana, il martedì, per dedicarsi interamente al progetto e impegnandosi a lavorare almeno per un' ora e mezza nel corso delle serate infrasettimanali e per due ore nei giorni del fine settimana.	
	\end{itemize}
\subsubsection{Problemi tra componenti del gruppo}
	\begin{itemize}
	\item \textbf{Analisi: }Essendo il team \gruppo ~formato da persone completamente estranee tra loro, è possibile che si vengano a creare contrasti tra i membri in quanto l' interazione tra essi è meno spontanea e non si conoscono a pieno i rispettivi caratteri dei vari membri.
	\item \textbf{Probabilità di occorrenza:} Alta;
	\item \textbf{Grado di pericolosità:} Medio;
	\item \textbf{Strategia di mitigazione: }Sarà ruolo del responsabile di progetto evitare di far lavorare insieme personalità troppo contrastanti che possono entrare in contrasto e che andrebbero ad incidere negativamente sul tempo di svolgimento di un compito.
	\item \textbf{Riscontro:} durante i periodi di Analisi e Progettazione Architetturale si sono avuti diversi episodi di contrasti tra alcuni membri del gruppo, i quali nonostante non siano stati di grande impatto effettivo, hanno sicuramente generato un clima all' interno del gruppo poco socievole, per tale motivo la probabilità del verificarsi è stata portata a alta e il grado di pericolosità a medio.
	\end{itemize}
\subsection{Livello organizzativo}
\subsubsection{Tempi delle attività}
	\begin{itemize}
	\item \textbf{Analisi: }Data la grande esperienza che si deve possedere per poter svolgere al meglio il ruolo di responsabile, e considerata la mancanza di tale conoscenza tra i membri del gruppo, è possibile che i tempi per lo svolgimento delle attività vengano calcolati in modo non corretto. Se ciò dovesse accadere per attività di grande rilievo, e si calcoli un tempo minore di quello veramente necessario, si provocherebbe un aumento dei costi e un ritardo nel completamento del compito.
	\item \textbf{Probabilità di occorrenza:} Media;
	\item \textbf{Grado di pericolosità:} Medio;
	\item \textbf{Strategia di mitigazione: }Si è deciso che per ogni attività la cui terminazione in ritardo crei molti disagi e un aumento eccessivo dei costi si aggiungerà un periodo di \textit{slack}\ped{G}.
	\end{itemize}
\subsubsection{Errata stima delle risorse necessarie}
	\begin{itemize}
	\item \textbf{Analisi: }In caso di errata pianificazione dell' assegnazione delle ore lavorative ai vari membri del team si potrebbe avere uno o più componenti del gruppo con troppe o troppe poche ore di lavoro svolte. In tal caso sarà necessario un intervento del \textit{Responsabile} per riassegnare le varie attività, portando ad un aumento dei costi.
	\item \textbf{Probabilità di occorrenza:} Media;
	\item \textbf{Grado di pericolosità:} Medio;
	\item \textbf{Strategia di mitigazione: }Si è cercato di realizzare una suddivisione dei compiti ottimale, rispettando gli impegni dei vari componenti, in modo da diminuire la probabilità di occorrenza di tale problema.
	\end{itemize}
\subsection{Livello dei requisiti}
\subsubsection{Capitolato vago}
	\begin{itemize}
	\item \textbf{Analisi: }Essendo vago il capitolato nell' esprimere i requisiti del sistema c'è la possibilità che le idee del gruppo non vengano apprezzate dal proponente, deludendo le sue aspettative e portando a un possibile fallimento del progetto.
	\item \textbf{Probabilità di occorrenza:} Bassa;
	\item \textbf{Grado di pericolosità:} Medio;
	\item \textbf{Strategia di mitigazione: }Per evitare tutto ciò il gruppo \gruppo~ ha richiesto un' incontro con il proponente cercando di chiarire i dubbi sul capitolato e definire in modo adeguato i requisiti del sistema. L' esito dell' incontro con il proponente e le decisioni prese in seguito ad esso vengono trattate nel \VerbaleB.
	\end{itemize}
\subsection{Livello di valutazione dei costi}
\subsubsection{Calcolo dei costi errato}
	\begin{itemize}
	\item \textbf{Analisi: }è possibile, data l' inesperienza del gruppo, che i costi per lo svolgimento del progetto contengano diversi errori e quindi, nel caso peggiore di una sottostima nei tempi di esecuzione di alcune attività , si otterrebbe un' aumento dei costi portando il gruppo a sforare il preventivo proposto andando così in perdita.
	\item \textbf{Probabilità di occorrenza:} Media;
	\item \textbf{Grado di pericolosità:} Alto;
	\item \textbf{Strategia di mitigazione: } Per cercare di controbilanciare gli errori prodotti dall' inesperienza del gruppo si aggiungeranno dei tempi di slack\ped{G} aggiunti a quelle attività che il responsabile ritiene di non essere stato in grado di stimare in modo corretto, in caso di sovrastima dei tempi di esecuzione si otterrebbe un risparmio rispetto a quanto preventivato.
	\end{itemize}