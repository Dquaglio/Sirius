\section{Consuntivo}
In questa sezione verrà trattato il rendiconto dei risultati dei costi sostenuti in ogni periodo di attività. Per migliorare la leggibilità dei dati verranno riportate, per ogni periodo, delle tabelle riportanti i vari ruoli, le ore preventivate, quelle consuntivate e i relativi costi mettendo in evidenza se il bilancio del periodo è in \textbf{positivo}, in \textbf{negativo} o in \textbf{pari}. 
\subsection{Analisi}
Si è deciso di riportare il consuntivo di questo periodo, nonostante il costo prodotto in questo lasso di tempo non venga posto a carico del proponente, per utilità interna al gruppo; infatti grazie ad essa si è potuto constatare come le misure messe in atto per contrastare l' inesperienza del gruppo in materia di pianificazione e di controllo dei costi siano state efficaci ed abbiano permesso di concludere il periodo di Analisi in \textbf{positivo} di 70\euro .
\begin{center}
\begin{tabular}{| l | c | c | c | c |}
\hline
\multicolumn{1}{| c |}{Ruoli} & \multicolumn{2}{c}{Preventivo} & \multicolumn{2}{ c|}{Consuntivo}\\
\cline{2-5}
& Ore & Costo (\euro) & Ore & Costo(\euro) \\
\hline
Responsabile & 30 & 900 & 28 & 840 \\
Amministratore & 34 & 680 & 32 & 640\\
Analista & 68 & 1700 & 68 & 1700 \\
Progettista & 0 & 0 & 0 & 0 \\
Verificatore & 26 & 390 & 28 & 420 \\
Programmatore & 0 & 0 & 0 & 0 \\
\hline
\textbf{Totale} & 158 & 3670 & 156 & 3600 \\
\hline
\end{tabular}
\\
Tabella 13: Tabella di confronto tra preventivo e consuntivo per il periodo di AN.
\end{center}
Dalla tabella 13 si nota come i ruoli di \textit{Responsabile} e \textit{Amministratore} siano stati coinvolti in modo minore di quanto pianificato, mentre c'è stata necessità di una maggiore attività di verifica.
\subsection{Progettazione Architetturale}
Qui di seguito verrà riportata la tabella del consuntivo riguardante il periodo di progettazione architetturale.
\begin{center}
\begin{tabular}{| l | c | c | c | c |}
\hline
\multicolumn{1}{| c |}{Ruoli} & \multicolumn{2}{c}{Preventivo} & \multicolumn{2}{ c|}{Consuntivo}\\
\cline{2-5}
& Ore & Costo (\euro) & Ore & Costo(\euro) \\
\hline
Responsabile & 38 & 1140 & 21 & 630 \\
Amministratore & 6 & 120 & 29 & 580 \\
Analista & 48 & 67 & 20 & 500 \\
Progettista & 67 & 1474 & 91 & 2002 \\
Verificatore & 22 & 330 & 26 & 390 \\
Programmatore & 0 & 0 & 0 & 0 \\
\hline
\textbf{Totale} & 181 & 4264 & 187 & 4102 \\
\hline
\end{tabular}
\\
Tabella 14: Tabella di confronto tra preventivo e consuntivo per il periodo di PA.
\end{center}
Dalla tabella 14 si nota come i ruoli di \textit{Responsabile} e \textit{Analista}, quest'ultimo in particolare, siano stati coinvolti in modo minore di quanto pianificato, per quanto riguarda l' analista ciò è dato dal fatto che si era immaginato che i requisiti e l'analisi in generale avessero bisogno di una mole maggiore di correzioni.\\
Le ore di \textit{Progettista} sono aumentate rispetto al preventivo, questo perché il team si è ritrovato durante la progettazione dell'architettura ad effettuare dei cambiamenti dati dall'inesperienza del team in materia di progettazione.\\
Al termine di questo periodo il team \gruppo è riuscito a concludere in positivo di  162\euro .
\subsection{Progettazione di Dettaglio e Codifica}
Qui di seguito viene riportata la tabella del consuntivo riguardante il periodo di progettazione di dettaglio e codifica.
\begin{center}
\begin{tabular}{| l | c | c | c | c |}
\hline
\multicolumn{1}{| c |}{Ruoli} & \multicolumn{2}{c}{Preventivo} & \multicolumn{2}{ c|}{Consuntivo}\\
\cline{2-5}
& Ore & Costo (\euro) & Ore & Costo(\euro) \\
\hline
Responsabile & 35 & 1050 & 6 & 180 \\
Amministratore & 8 & 160 & 6 & 120 \\
Analista & 5 & 125 & 15 & 375 \\
Progettista & 81 & 1781 & 167 & 3674 \\
Verificatore & 93 & 1395 & 40 & 600 \\
Programmatore & 64 & 960 & 48 & 720 \\
\hline
\textbf{Totale} & 286 & 5472 & 282 & 5669 \\
\hline
\end{tabular}
\\
Tabella 14: Tabella di confronto tra preventivo e consuntivo per il periodo di PDC.
\end{center}
Al termine di questo periodo il team \emph{Sirius} ha speso di più di quanto pianificato andando in negativo di 197\euro  ciò è dato dai fattori che verranno trattati qui di seguito. 
Innanzitutto dalla tabella si nota subito che il ruolo del \emph{progettista} è stato il più richiesto dell' intero periodo e il suo numero di ore effettive del periodo si distacca largamente da quanto pianificato; questo perchè è stato  necessario modificare la specifica tecnica in vari punti in quanto non erano stati considerati molti fattori legati all' utilizzo dei \textit{framework} \emph{Spring} e \emph{Backbone}, infatti a causa di tutto ciò è stato  necessario apportare grandi modifiche ai presenter lato client e server, nate anche dalle considerazioni riportate nella correzione del periodo di progettazione architetturale.
Il ruolo di \emph{responsabile} non è stato coinvolto come quanto pianificato, infatti la sua necessità è stata sovrastimata durante la pianificazione delle attività in quanto si pensava ci sarebbe stato maggior bisogno del suo intervento.
Anche il ruolo di \emph{verificatore} è stato poco utilizzato, ma questo non è dovuto a un errore di sovrastima, infatti c'è ancora una grande necessità di eseguire attività di verifica, ma in quanto è stato necessario rivedere l' architettura del sistema in modo quanto più ottimale possibile, si è deciso di utilizzare maggiormente le risorse nella stesura dei documenti di \DefinizioneDiProdotto  e di \SpecificaTecnica  sperando di ridurre così nel periodo di Validazione la necessità di \emph{progettisti}, andando a utilizzare le risorse risparmiate in \emph{verificatori}.
\subsection{Validazione}
Qui di seguito viene riportata la tabella del consuntivo riguardante il periodo di validazione.
\begin{center}
\begin{tabular}{| l | c | c | c | c |}
\hline
\multicolumn{1}{| c |}{Ruoli} & \multicolumn{2}{c}{Preventivo} & \multicolumn{2}{ c|}{Consuntivo}\\
\cline{2-5}
& Ore & Costo (\euro) & Ore & Costo(\euro) \\
\hline
Responsabile & 33 & 990 & 2 & 60 \\
Amministratore & 20 & 400 & 2 & 40 \\
Analista & 0 & 0 & 0 & 0 \\
Progettista & 24 & 528 & 4 & 88 \\
Verificatore & 77 & 1155 & 125 & 1875 \\
Programmatore & 15 & 225 & 26 & 390 \\
\hline
\textbf{Totale} & 169 & 3298 & 159 & 2453 \\
\hline
\end{tabular}
\\
Tabella 15: Tabella di confronto tra preventivo e consuntivo per il periodo di VV.
\end{center}
\subsection{Preventivo a finire}
Il gruppo \emph{Sirius} nei precedenti periodi ha risparmiato 192\euro , aggiungendo a tale cifra i 197\euro di negativo di questo periodo si ottiene che il budget è in negativo di 5\euro , anche se tale cifra non è elevata, è comunque un indice negativo e i soldi risparmiati in precedenza non possono essere più usati per attività di verifica e validazione ulteriori sul prodotto; ad ogni modo avendo realizzato un' architettura di dettaglio quanto più ottimale possibile, non dovrebbe essere necessario nel prossimo periodo apportarci grandi modifiche, quindi sarà possibile incentrare le risorse del team in attività di verifica e validazione andando così a migliorare la qualità del software realizzato.