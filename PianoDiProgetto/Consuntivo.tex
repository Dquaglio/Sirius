\section{Consuntivo}
In questa sezione verr\'a trattato il rendiconto dei risultati dei costi sostenuti in ogni periodo di attivit\'a~. Per migliorare la leggibilit\'a dei dati verranno riportate, per ogni periodo, delle tabelle riportanti i vari ruoli, le ore preventivate, quelle consuntivate e i relativi costi mettendo in evidenza se il bilancio del periodo \'e in \textbf{positivo}, in \textbf{negativo} o in \textbf{pari}. 
\subsection{Analisi}
Si \'e deciso di riportare il consuntivo di questo periodo, nonostante il costo prodotto in questo lasso di tempo non venga posto a carico del proponente, per utilit\'a interna al gruppo; infatti grazie ad essa si \'e potuto constatare come le misure messe in atto per contrastare l' inesperienza del gruppo in materia di pianificazione e di controllo dei costi siano state efficaci ed abbiano permesso di concludere il periodo di Analisi in \textbf{positivo} di 70\euro.
\begin{center}
\begin{tabular}{| l | c | c | c | c |}
\hline
\multicolumn{1}{| c |}{Ruoli} & \multicolumn{2}{c}{Preventivo} & \multicolumn{2}{ c|}{Consuntivo}\\
\cline{2-5}
& Ore & Costo (\euro) & Ore & Costo(\euro) \\
\hline
Responsabile & 30 & 900 & 28 & 840 \\
Amministratore & 34 & 680 & 32 & 640\\
Analista & 68 & 1700 & 68 & 1700 \\
Progettista & 0 & 0 & 0 & 0 \\
Verificatore & 26 & 390 & 28 & 420 \\
Programmatore & 0 & 0 & 0 & 0 \\
\hline
\textbf{Totale} & 158 & 3670 & 156 & 3600 \\
\hline
\end{tabular}
\\
Tabella 13: Tabella di confronto tra preventivo e consuntivo.
\end{center}
Dalla tabbella 13 si nota come i ruoli di \textit{Responsabile} e \textit{Amministratore} siano stati coinvolti in modo minore di quanto pianificato, mentre c'\'e stata necessit\'a di una maggiore attivit\'a di verifica.