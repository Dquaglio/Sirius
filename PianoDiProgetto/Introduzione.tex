\section{Introduzione}
\subsection{Scopo del documento}
Questo documento \'e stato steso per definire l'organizzazione e l'approccio usato dal team \gruppo per portare a termine il progetto \progetto. \\
Gli scopi principali di questo documento sono:
\begin{itemize}
  \item Mostrare l'organigramma del gruppo;
  \item Organizzare le attivit\'a in m
odo da produrre risultati utili per 
valutare con efficacia il grado di avanzamento del lavoro;
  \item	Esporre la pianificazione delle attivi\'a e il modello di ciclo di vita adottato;
  \item Analizzare i rischi;
  \item Presentare un prospetto economico.
\end{itemize}

\subsection{Scopo del prodotto}
Il prodotto che il team \gruppo intende realizzare \'e un sistema composto di un programma server dotato di un' interfaccia per la creazione di processi composti da uno o pi\'u passi. Questo sistema dovr\'a poi gestire l' esecuzione di questi passi da parte di utenti dotati di smartphone valutandone la correttezza.\\

\subsection{Riferimenti}

\subsubsection{Normativi}
\begin{itemize}
	\item \textbf{Norme di progetto:} \infoNDP ;
	\item \textbf{Capitolato d' appalto C4:} Sequenziatore\\ \capitolato .
\end{itemize}
\subsubsection{Informativi}
\begin{itemize}
\item \textbf{Software Engineering- Ian Sommerville}
	\begin{itemize}
	\item \textit{Chapter} 2.1
	\end{itemize}
\item \textbf{Mitre}
	\begin{itemize} 
	\item 5.0 \textit{Collaboration and Individual Characteristics} \url{http://www.mitre.org/sites/default/files/publications/10_0678_presentation.pdf}
	\end{itemize}
\end{itemize}
\subsection{Ciclo di vita}
\label{subsec:ciclodivita}
Il modello che il gruppo \gruppo ha scelto di utilizzare per rappresentare i vari processi \'e il \textbf{modello incrementale} il quale si basa sull' idea di sviluppare inizialmente una prima implementazione ed esporla all' utente ed evolverla attraverso diverse versioni per giungere infine alla produzione di un adeguato sistema. Le motivazioni che hanno spinto ad usare questo modello sono:
\begin{itemize}
	\item La disponibilit\'a del proponente ad incontri, per ascoltare i suoi \textit{feedback} e mostrargli quanto \'e stato implementato aiuta a ridurre i rischi di fallimento;
	\item Rispetto al modello a cascata, introdurre cambiamenti richiesti dal proponente costa meno e la quantit\'a di analisi e documentazione che necessita di essere rifatta \'e minore.
\end{itemize}
Questo modello inizialmente utilizzer\'a le risorse per produrre una versione del sistema che implementa le pi\'u importanti o urgenti funzionalit\'a richieste. In questo modo il proponente potr\'a valutare il sistema in uno stadio iniziale e potr\'a in seguito esaminare se rispetta le proprie aspettative.

\subsection{Organigramma}
\subsubsection{Redazione}
\begin{center}
\begin{tabular}{l l l}
\hline
Nominativo & Data di redazione & Firma\\
\hline
Quaglio Davide & 09/02/2014 & \\ %\includegraphics[scale=0.9]{Pics/dq.png} 
\hline
\\
\end{tabular}
\end{center}
\subsubsection{Approvazione}
\begin{center}
\begin{tabular}{l l l}
\hline
Nominativo & Data & Firma\\
\hline
Quaglio Davide & 09/02/2014 &  \\ %\includegraphics[scale=0.9]{Pics/dq.png}\\
\hline
\end{tabular}
\end{center}
\subsubsection{Accettazione dei componenti}
\begin{center}
\begin{tabular}{l l l}
\hline
Nominativo & Data di accettazione & Firma\\
\hline
Botter Marco & 09/02/2014 & \\ %\includegraphics[scale=0.9]{Pics/luca.png}\\
\hline
Giachin Vanni & 09/02/2014 & \\ %\includegraphics[scale=0.9]{Pics/antonio.png}\\
\hline
Marcomin Gabriele & 09/02/2014 & \\ %\includegraphics[scale=0.9]{Pics/emanuele.png}\\
\hline
Quaglio Davide & 09/02/2014 &  \\ %\includegraphics[scale=0.9]{Pics/dq.png}\\
\hline
Santangelo Davide & 09/02/2014 & \\ %\includegraphics[scale=0.9]{Pics/jorge.png}\\
\hline
Seresin Davide & 09/02/2014 & \\ %\includegraphics[scale=0.9]{Pics/fabio.png}\\
\hline
\\
\end{tabular}
\end{center}
\subsubsection{Componenti}
\begin{center}
\begin{tabular}{l l l}
\hline
Nominativo & Matricola & Firma\\
\hline
Botter Marco & 561940 & \\ %\includegraphics[scale=0.9]{Pics/luca.png}\\
\hline
Giachin Vanni & 1005519 & \\ %\includegraphics[scale=0.9]{Pics/antonio.png}\\
\hline
Marcomin Gabriele & 1008916 & \\ %\includegraphics[scale=0.9]{Pics/emanuele.png}\\
\hline
Quaglio Davide & 1026451 &  \\ %\includegraphics[scale=0.9]{Pics/dq.png}\\
\hline
Santangelo Davide & 1004037 & \\ %\includegraphics[scale=0.9]{Pics/jorge.png}\\
\hline
Seresin Davide & 611100 & \\ %\includegraphics[scale=0.9]{Pics/fabio.png}\\
\hline
\\
\end{tabular}
\end{center}
\subsection{Definizione dei ruoli}

\subsection{Scadenze}
\label{subsec:Scadenze}
Qui di seguito vengono riportate le date delle scadenze che il gruppo \gruppo intende rispettare per le consegne del progetto e sulle quali pianificher\'a le proprie attivit\'a :
\begin{itemize}
	\item \textit{Revisione dei requisiti: }2014-03-05
	\item \textit{Revisione di progettazione: }2014-03-29
	\item \textit{Revisione di qualifica: }2014-06-28
	\item \textit{Revisione di accettazione: }2014-07-18
\end{itemize}
