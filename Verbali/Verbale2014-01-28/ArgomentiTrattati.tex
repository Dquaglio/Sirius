\section{Argomenti trattati}
\subsection{Prossimi incontri}
Seresin Davide richiede che sia pianificata una riunione ordinaria di gruppo con cadenza settimanale, nel quale raggruppare tutte i lavori svolti individualmente e pianificare i lavori futuri.\\
Visti gli impegni di ogni componente del gruppo si è deciso di pianificare la riunione ordinaria al martedì presso il laboratorio del Paolotti alle ore 14:00.\\
L'eventuale indisponibilità di un membro deve essere comunicata il prima possibile tramite mail a tutti i componenti del gruppo, che valuteranno se tenere o meno l'incontro.\\
In caso si necessitasse di ulteriore incontro precedente a quello ordinario, questo verrà indetto con congruo anticipo (2gg) e comunicato via mail a tutti i componenti del gruppo.\\
\subsection{Definizione del capitolato}
Durante la fase di analisi che ci stiamo approntando a sviluppare, abbiamo già riscontrato qualche incomprensione riguardo alle specifiche richieste; per questo motivo abbiamo deciso di contattare direttamente il proponente \emph{Zucchetti s.p.a.} per richiedere un incontro. Vanni Giachin si occuperà di questo.\\
\subsection{Norme di progetto}
Vanni Giachin si è reso disponibile a riunire in un unico documento \LaTeX{} le norme di progetto stese fin'ora da lui, Seresin e Santangelo.\\
Una volta ultimata la prima fase di stesura, il documento passerà a Seresin per un'analisi della parte da lui scritta e poi Marcomin e Botter verificheranno. Il tutto entro Martedì 2014-02-04.
\subsection{Studio fattibilità}
Seresin si prende in carico il documento e lo redigerà entro venerdì 2014-01-31.\\
Santangelo lo verificherà con scadenza 2014-02-04.\\
\subsection{Repository}
Quaglio e Botter si stanno occupando della fase operativa del repository e hanno caricato i primi documenti.\\
\subsection{Documenti}
Su consiglio di Quaglio si redigeranno tutti i documenti \LaTeX{} con divisione in capitoli.\\
Questo per fare in modo che sul repository si possa lavorare con due branch diversi sullo stesso documento, senza ostacolare uno il lavoro dell'altro.
Dopo uan breve discussione la proposta è stata accettata.\\
Quaglio si occuperà di redigere una breve guida all'interno delle \emph{Norme di Progetto} per l'utilizzo della suddetta organizzazione.\\
\subsection{Analisi requisiti}
Marcomin inizia con lo stendere lo scheletro dell'\emph{Analisi dei requisiti}.
Si occuperà anche della ricerca delle macro-classi di vincoli.\\
Analizzando il documento in oggetto, è sorta la discussione circa la scelta di un applicativo per la creazione e gestione dei diagrammi dei casi d'uso; Marcomin prone l'utilizzo di Astra.\\
Botter si occuperà di ricercare e saper utilizzare il programma entro martedì 2014-02-04.\\
