\section{Argomenti trattati}

A seguito dell'incontro avvenuto con il rappresentante dell'azienda  \textit{Zucchetti S.p.a}, sono venute alla luce mancanze e incomprensioni da parte del team \textit{Sirius}. L'incontro si può considerare prolifico e sono state identificate migliorie da portare al prodotto finale.

\subsection{Struttura di un processo}

Il proponente ha apprezzato il fatto che il nostro prototipo sia completamente utilizzabile da dispositivo mobile. Ci ha dato varie direttive per proseguire in questo senso anche per quanto riguarda la funzionalità di creazione di un processo.
Ha consigliato di suddividere i processi in blocchi, cioè insiemi di passi.
I blocchi possono essere di due tipi:
\begin{itemize}
	\item blocchi di passi ordinati (sequenziali);
	\item blocchi di passi non ordinati.
\end{itemize}
I blocchi non ordinati possono richiedere il completamento di uno, alcuni, o tutti i passi di quest'ultimo.
Per gestire i collegamenti tra i passi, ci è stato consigliato di utilizzare la possibilità di effettuare il trascinamento degli elementi  \textit{HTML\ped{G}}.


\subsection{Usabilità}
Per migliorare l'usabilità del prodotto finale, sono stati suggerite le seguenti caratteristiche:
\begin{itemize}
	\item l'utente \textit{process owner}\ped{G}, per ogni processo, deve poter ricercare i dati filtrando per passo o per utente;
	\item la navigazione nei passi da controllare deve essere espansa;
	\item aggiungere help per comprendere tutte le funzionalità;
	\item aggiungere colori per suddividere logicamente le tipologie di blocchi e passi;
	\item espandere le possibilità di navigazione dell'utente \textit{process owner}\ped{G} nella gestione dei processi.
\end{itemize}

\subsection{Notifiche}
L'utente \textit{process owner}\ped{G} deve poter essere avvisato della presenza di nuovi processi che richiedono intervento umano. È apprezzato l'utilizzo di avvisi grafici come colorazioni del pulsante di \textit{controllo passi}, e la segnalazione del numero di passi in attesa.

\subsection{Requisiti opzionali}
É apprezzato il soddisfacimento dei seguenti requisiti suggeriti durante l'incontro:
\begin{itemize}
	\item Sarebbe apprezzata la funzionalità di generazione di report in formato PDF\ped{G}. A riguardo il 	proponente ha consigliato l'utilizzo della libreria JSPDF\ped{G};
	\item La possibilità di eseguire passi anche in assenza di connessione è apprezzata dal proponente che, tuttavia, lascia al gruppo la scelta di implementare o meno tale funzionalità.
\end{itemize}


\subsection{Requisiti tecnologici}
Per quanto riguarda la funzionalità di ricezione dei processi in attesa di approvazione da parte dell'utente \textit{process owner}\ped{G}, è stato consigliato di effettuare le chiamate al server tramite la tecnica \textit{polling}\ped{G} con un intervallo di circa 30 secondi, piuttosto che tramite \textit{Web Socket}\ped{G}. É apprezzato  anche l'aggiunta di un pulsante \textit{aggiorna} per forzare il recupero dei dati dal \textit{server}\ped{G}.
