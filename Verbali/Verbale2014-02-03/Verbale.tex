\section{Argomenti trattati}
\subsection{Distinzione utenti}
A seguito di una discussione interna il team \emph{Sirius }ha ritenuto necessario chiedere al rappresantante della Zucchetti \textit{S.P.A} delucidazioni in merito alla distinzione delle tipologie di \emph{users }del \progetto.
Ne è stato ricavato che:
\begin{itemize}
\item Gli utenti devono essere nettamente divisi tra admin_G e non-admin;
\item Solo gli utenti admin possono creare workflows_G;
\item Gli utenti non-admin possono partecipare a workflows ma non crearli;
\end{itemize}

\subsection{Definizione workflow}
L'analisi del capitolato aveva lasciato dei dubbi riguardo il concetto stesso di workflow, tali dubbi grazie alla discussione sono stati risolti.
Qui vengono riportati i punti salienti:

\begin{itemize}
\item Gli amministratori possono decidere il tipo di workflow sotto ogni aspetto, quantità di passi, sequenzialità, partecipanti;
\item Si possono definire worflows composti da un singolo step ripetuto e scandito tramite archi di tempo;
\item Determinati workflows per avanzare di passo possono richiedere una verifica umana (in genere grazie all'intervento dell'amministratore)
\item I workflows possono essere pubblici, cioè accessibili da tutti gli utenti, o privati, cioè accessibili solo tramite intervento dell'amministratore di quel determinato workflow.
\end{itemize}

\subsection{Tecnologie}
Infine è stato discusso l'argomento tecnologie utilizzabili, in quanto era sorto un dubbio circa l'implementazione del software.
\textit{Zucchetti} ha chiarito che la scelta di sviluppare il progetto come applicazione Android_G o applicazione HTML5 era indifferente, tuttavia ha sottolineato che sviluppare un applicazione Android avrebbe determinato la perdita di alcuni utenti all'interno del mercato mobile_G.

