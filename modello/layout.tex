\documentclass[11pt,a4paper]{article}
\usepackage[a4paper,portrait,top=3.5cm,bottom=3.5cm,left=3cm,right=3cm,bindingoffset=5mm]{geometry}


\usepackage[italian]{babel}
\usepackage{ucs} %unicode sistema gli accenti
\usepackage[utf8x]{inputenc} %unicode sistema gli accenti
\usepackage{fancyhdr}
\usepackage{subfigure} % per figure affiancate
\usepackage{hyperref}
\usepackage{float} % per far bene le figures
\usepackage{indentfirst}
\usepackage{color}
\usepackage{colortbl}
\usepackage{rotating}
\usepackage[table]{xcolor}
\usepackage{wrapfig}
\usepackage{array}
\usepackage{eurosym}
\usepackage{graphicx}
\usepackage{breakurl}
\usepackage{lastpage} % total page count
\usepackage{chngpage}
\usepackage{amsfonts}
\usepackage{listings}
\usepackage{enumitem}
\usepackage{booktabs}% funzionalità aggiuntive per tabelle
\usepackage{bookmark}% consente la creazione di segnalibri
\usepackage{caption}% funzionalità aggiuntive didascalie
\usepackage{tabularx}% funzionalità aggiuntive per tabelle
\usepackage{longtable}% funzionalità aggiuntive per tabelle
\usepackage{ltxtable}% funzionalità aggiuntive per tabelle


%\graphicspath{{./Pics}} % cartella di salvataggio immagini

% Per l'indice analitico
%\usepackage{makeidx}
%\makeindex

%pagestyle{fancy}
%\renewcommand{\chaptermark}[1]{\markboth{\thechapter.\ #1}{}}
%\renewcommand{\sectionmark}[1]{\markright{\thesection.\ \ #1}{}}
%\fancyhead{}
%comandi dell header
%\fancyhead[EL]{\slshape \leftmark}
%\fancyhead[OR]{\slshape \rightmark}
%\fancyfoot[EC,OC]{\slshape \thepage}

\pagestyle{fancy}
%\newcommand{\license}{\href{http://creativecommons.org/licenses/by/3.0/}{Some rights reserved}}
\newcommand{\groupname}{Sirius - Sequenziatore}

\newcommand{\subscript}[1]{\raisebox{-0.6ex}{\scriptsize #1}}
%\newcommand{\subscript}[1]{\ensuremath{_{\textrm{#1}}}}
%\renewcommand{\sectionmark}[1]{\markright{\thesection.\ #1}}
%\lhead{\nouppercase{\rightmark}}
%\rhead{\nouppercase{\leftmark}}
%\renewcommand{\chaptermark}[1]{\markboth{\thechapter.\ #1}{}}


\fancypagestyle{plain}{%
	\chead{}
	\lfoot{\info}
	\cfoot{}
	\rfoot{\thepage\ / \pageref{LastPage}}
	\renewcommand{\headrulewidth}{0.3pt}
	\renewcommand{\footrulewidth}{0.3pt}
}
	\lhead{\setlength{\unitlength}{1mm}
        \begin{picture}(0,0)
                \put(5,0){\includegraphics[scale=0.001]{img/Sirius.png}}
        \end{picture}}
	\rhead{\groupname}
	\chead{}
	\lfoot{\info}
	\cfoot{}
%	\rfoot{\thepage}
	\rfoot{\thepage\ / \pageref{LastPage}}
	\renewcommand{\headrulewidth}{0.3pt}
	\renewcommand{\footrulewidth}{0.3pt}
\linespread{1.2}	% valore interlinea


\fancypagestyle{romano}{
	\lhead{\setlength{\unitlength}{1mm}
        \begin{picture}(0,0)
                \put(5,0){\includegraphics[scale=0.03]{../modello/img/sirius.png}}
        \end{picture}}
	\chead{}
	\rhead{\groupname}
	\lfoot{\info}
	\cfoot{}
	\rfoot{\thepage}
	\renewcommand{\headrulewidth}{0.3pt}
	\renewcommand{\footrulewidth}{0.3pt}
}



\hypersetup{
    colorlinks=true,linkcolor=[rgb]{0.11,0.55,0.83
    },          % colore link interni
    urlcolor=cyan           %colore link esterni
}
\definecolor{err}{rgb}{0.9,0.1,0.1}
\definecolor{rt}{rgb}{0.1,0.6,0.8}
\definecolor{grey}{rgb}{0.4,0.3,0.4}
\definecolor{mycolor}{rgb}{0.67,1,0.18}

\bibliographystyle{plain_ita}%bibliografia stile italiano

\pagenumbering{Roman}
\setlength\parindent{0pt} % sempre senza indentatura
% fine layout